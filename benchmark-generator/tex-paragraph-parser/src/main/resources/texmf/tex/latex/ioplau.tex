%%%%%%%%%%%%%%%%%%%%%%%%%%%%%%%%%%%%%%%%%%%%%%%%%%%%%%%%%%%%%%%%%%%%%%%%%%%
%%%%%%%%%%%%%%%%%%%%%%%%%%%%%%%%%%%%%%%%%%%%%%%%%%%%%%%%%%%%%%%%%%%%%%%%%%%%
%%%    INSTITUTE OF PHYSICS PUBLISHING                                   %%%
%%%                                                                      %%%
%%%   `Preparing an article for publication in an Institute of Physics   %%%
%%%    Publishing journal using LaTeX'                                   %%%
%%%                                                                      %%%
%%%    LaTeX source code `ioplau.tex' used to generate `author           %%%
%%%    guidelines', the documentation explaining and demonstrating use   %%%
%%%    of the Institute of Physics Publishing LaTeX preprint macro file  %%%
%%%    `ioplppt.sty'.                                                    %%%
%%%                                                                      %%%
%%%    `ioplau.tex' itself uses LaTeX with `ioplppt.sty'                 %%%
%%%    an optional file iopfts.sty can also be loaded to allow the       %%%
%%%    use of the AMS extension fonts msam and msbm with the IOP         %%%
%%%    preprint style.                                                   %%%
%%%                                                                      %%%
%%%%%%%%%%%%%%%%%%%%%%%%%%%%%%%%%%%%%%%%%%%%%%%%%%%%%%%%%%%%%%%%%%%%%%%%%%%%
%%%
%%%
%%% First we have a character check
%%%
%%% ! exclamation mark    " double quote
%%% # hash                ` opening quote (grave)
%%% & ampersand           ' closing quote (acute)
%%% $ dollar              % percent
%%% ( open parenthesis    ) close paren.
%%% - hyphen              = equals sign
%%% | vertical bar        ~ tilde
%%% @ at sign             _ underscore
%%% { open curly brace    } close curly
%%% [ open square         ] close square bracket
%%% + plus sign           ; semi-colon
%%% * asterisk            : colon
%%% < open angle bracket  > close angle
%%% , comma               . full stop
%%% ? question mark       / forward slash
%%% \ backslash           ^ circumflex
%%%
%%% ABCDEFGHIJKLMNOPQRSTUVWXYZ
%%% abcdefghijklmnopqrstuvwxyz
%%% 1234567890
%%%
%%%%%%%%%%%%%%%%%%%%%%%%%%%%%%%%%%%%%%%%%%%%%%%%%%%%%%%%%%%%%%%%%%%%%
%%%
%\documentstyle{ioplppt}                % use this for journal style
\documentstyle[12pt]{ioplppt}          % use this for preprint style
\def\cqo#1#2{\cos[q\Omega^{\rm #1}_S(b_f,R^s_{#2})]}
%
\begin{document}
\title{Preparing an article for publication in an Institute of Physics
Publishing journal using \LaTeX}[Author guidelines for IOP Journals]

\author{A J Cox\dag, Mark Telford\ddag\ and Al Troyano\ddag\ftnote{3}{To
whom correspondence should be addressed.}}

\address{\dag\ Electronic Production Manager, Institute of Physics
Publishing, Techno
House, Redcliffe Way, Bristol BS1 6NX, UK}

\address{\ddag\ Production Department, Institute of Physics Publishing,
Techno House,
Redcliffe Way, Bristol BS1 6NX, UK}


\begin{abstract}
This document describes the  preparation of an article in \LaTeX\ using
\verb"ioplppt.sty" (the IOP \LaTeX\ preprint style file)
for any of the journals published by Institute of Physics
Publishing.  This style file is designed to help
the author by simplifying the production of an article in a form that
can be refereed and copy edited and that can subsequently be swiftly
converted into the normal IOP journal styles by changing the
style file.
Authors submitting to both the single-column B5 and
double-column A4 format journals
should follow the guidelines set out here.
The source code will be converted to
the appropriate journal format at Institute of Physics Publishing.
For the printed version, Times fonts (and Helvetica in double-column
journals) will be used instead of Computer Modern.
\end{abstract}

%
%  Uncomment out if preprint format required
%
%\pacs{00.00, 20.00, 42.10}
%\maketitle

\section{Introduction}
Institute of Physics Publishing
(IOP) publishes a wide range of research and review
journals and magazines. It is wholly owned by The Institute of Physics,
the United Kingdom professional body for physicists.
For many of the journals authors regularly use \TeX\ or \LaTeX\ to
produce their typescripts
and at IOP we can use author's \TeX\ or \LaTeX\ source code
to produce the printed version;  this gives
more rapid publication with a smaller chance of typographical
error.

This document gives the procedures and specific requirements for
the preparation and presentation of text and illustrations for
articles in \LaTeX\null. It has been prepared using the
\verb"ioplppt.sty" style file to illustrate its use. The
style file is available to all authors and copies can be
obtained by contacting the Electronic Production Manager,
Institute of Physics
Publishing, Techno House, Redcliffe Way, Bristol BS1~6NX, UK. The file
is available as hard copy, on 5$1\over 4$ or 3$1\over 2$ inch IBM
compatible (MS-DOS) discs,  or via e-mail (JANET:
prod2@uk.co.ioppublishing;
please put `\LaTeX\ Style file request' as the subject of the message).
Foreign authors requesting copies
of the files by e-mail may need to reverse the host name, i.e.\
ioppublishing.co.uk, and are asked to acknowledge receipt of the files.

The output given by this `IOP preprint' style file is intended for A4 size
paper, but the output also fits on American 8.5\;in $\times$ 11\;in
paper.
There are ten and twelve point versions of the style and an option to load
additional fonts from the AMS extension fonts. The twelve point version
of the `preprint' style gives a page width and type size
1.2 times larger than that for normal single-column journals
with extra spacing between lines. The page depth is less than 1.2 times
the normal page depth so that articles will fit on the page on both
A4 and 8.5\;in $\times$ 11\;in paper.
This form is the one required for
refereeing and copy editing.
The ten point version has the same page dimensions and type sizes as a
single-column journal and shows approximately
how the text would appear in print. It can also be used to produce
camera-ready
copy for journal special issues.
Other \LaTeX\ optional files may be used where appropriate.
Authors writing for double-column journals may use the IOP preprint
macros. Conversion from the single-column format to the double-column
output required for printing will be done at Institute of Physics
Publishing.


Authors are asked to prepare their articles using the preprint style
and submit three printed copies of the article for refereeing as
normal.  They should inform the journal's Managing Editor that the
\LaTeX\ source
code is available and how they can send it, but should {\bf not} send
the file until the article has received favourable referee reports.
The Managing Editor of the journal concerned
will notify the author when to send the \LaTeX\ files.

Most authors send their \LaTeX\ files to us via e-mail; this is
perfectly acceptable. However, problems can occur to files sent via
e-mail: long lines may be truncated and certain ASCII characters may
be corrupted. To guard against this, authors using e-mail are asked to
ensure that no line in their source code exceeds 75 characters in
length, and that a simple character-check table is sent with their
paper (like the one at the beginning of this file and
\verb"ioplppt.sty").


There is also an equivalent file for Plain \TeX\ \verb"iopppt.tex" which
is available for authors who prefer to use Plain \TeX, but,
although the use of one of these style files is recommended,
submission of \TeX\ files is not restricted to files using them.
Articles prepared using almost any  version of \TeX\ can be
handled (${\cal AMS}$\TeX, L${\cal AMS}$TeX, PHYZZX, etc)
and authors not using the IOP
style files
can submit their source code in the way described
above. Alterations to the source code will be made in-house, in order
to bring it in line with IOP style.

We aim to maintain our normal standards
for articles published from \TeX\ files so we reserve
the right to make small alterations  to clarify and improve the
English where necessary and to put the article into IOP house style.


\section{Preparing your article}
Using \LaTeX\ with the \verb"ioplppt" style file
provides a simple way of
producing an article in a form suitable for publication in one
of the IOP journals. Authors may add their own macros
at the start of an article
provided they do not overwrite existing definitions and
that they send copies of their new macros with their text file.
\verb"ioplppt" can be used with other optional files such
as those loading the AMS extension fonts
\verb"msam" and \verb"msbm" (these fonts provide the
blackboard bold alphabet
and various extra maths symbols as well as symbols useful in figure
captions). If you have these fonts available and wish to use
them an optional style file \verb"iopfts.sty"
is included to access the AMS fonts and extra bold italic and bold sans
serif fonts. Details of the features of this optional file are
given in the file \verb"iopfts.tex".

In preparing your article you are
requested to follow these guidelines as closely as possible; this will
minimize the amount of copy editing required and will hasten the
production process. This is particularly important with regard to the
reference list.

The file name can be up to eight characters long with the suffix \verb".tex".
Please use files names that are likely to be unique, and include
commented material to identify the journal, author and reference number if
known. The first non-commented line of the file should be
\verb"\documentstyle[12pt]{ioplppt}"  to load the preprint style
file. Other standard optional files can be included in square brackets;
copies of any non-standard options must be sent in with the source code.
Omitting \verb"[12pt]" produces an article in the normal journal
page and type
sizes. Macros for the individual paper not included in a style file
should be inserted in the preamble to the paper with comments to
describe any complex or non-obvious ones.
Authors of long articles may find it convenient to separate their article
into a series of files each containing a section, each of which is called
in turn by the primary file. This is also quite acceptable.
The start of the document is then signalled with the usual \LaTeX\
\verb"\begin{document}".
The
journal to which the article is to be submitted may be chosen with the
command \verb"\jl{#1}", where \verb"#1" is the journal reference number
given in
table~\ref{jlns}. This information is used if the command \verb"\maketitle"
is used (see later) but is not essential to the running of the
file.

\begin{table}
\caption{The reference numbers for all the journals for which the
IOP macros can be used.}
\label{jlns}
\lineup
\begin{indented}
\item[]\begin{tabular}{@{}ll}
\br
No&{\rm Journal}\\
\mr
\01&Journal of Physics A: Mathematical and General\\
\02&Journal of Physics B: Atomic, Molecular and Optical Physics\\
\03&Journal of Physics: Condensed Matter\\
\04&Journal Physics G: Nuclear and Particle Physics\\
\05&Inverse Problems\\
\06&Classical and Quantum Gravity\\
\07&Network\\
\08&Nonlinearity\\
\09&Quantum Optics\\
10&Waves in Random Media\\
11&Pure and Applied Optics\\
12&Physics in Medecine and Biology\\
13&Modelling and Simulation in Materials Science and Engineering\\
14&Plasma Physics and Controlled Fusion\\
15&Physiological Measurement\\
16&Soviet Lightwave Communications\\
17&High Performance Polymers\\
18&Journal of Hard Materials\\
19&Journal of Physics D: Applied Physics\\
20&Superconductor Science and Technology\\
21&Semiconductor Science and Technology\\
22&Nanotechnology\\
23&Measurement Science and Technology\\
24&Plasma Sources Science and Technology\\
25&Smart Materials Structure\\
26&Journal of Micromechanics and Microengineering\\
27&Distributed Systems Engineering\\
\br
\end{tabular}
\end{indented}
\end{table}


\section{The title and abstract page}
The title is set in bold unjustified type using the command
\verb"\title{#1}[#2]", where \verb"#1" is the title of the article. The
first letter
of the title should be capitalized with the rest in lower case.
Mathematical expressions within the title may be left in light-face type
rather than bold because the Computer Modern bold maths and symbol
fonts are often not available at the size required for the title.
The final
printed version will have bold mathematical expressions
in the title. \verb"[#2]" is
an optional argument defining a short title to be used as
the running head on odd-numbered pages of the printed version,
it is only necessary when the title itself is too long to
be used
as the short title. If the
\verb"\maketitle" command is used, the short title is printed at the bottom
of the title page otherwise it is just stored.


For articles other than papers the IOP preprint style includes
a generic heading \verb"\article{#1}{#2}[#3]" and the specific
definitions given in table~\ref{arttype}. The optional argument where
present is the short title. For Letters no short title is required as
the running head is automatically defined to be {\it Letter to the Editor}.
The generic heading could be used for
articles such as those presented at a conference or workshop, e.g.
\begin{verbatim}
\article{WORKSHOP ON HIGH-ENERGY PHYSICS}{Title}[Short title]
\end{verbatim}


\begin{table}
\caption{Types of article defined in IOP macros.\label{arttype}}
\footnotesize\rm
\begin{tabular*}{\textwidth}{@{}l*{15}{@{\extracolsep{0pt plus12pt}}l}}
\br
Command&Type&Heading on first page\\
\mr
\verb"\title{#1}[#2]"&Paper&---\\
\verb"\review{#1}[#2]"&Review&REVIEW\\
\verb"\topical{#1}[#2]"&Topical review&TOPICAL REVIEW\\
\verb"\comment{#1}[#2]"&Comment&COMMENT\\
\verb"\note{#1}[#2]"&Note&NOTE\\
\verb"\paper{#1}[#2]"&Paper&---\\
\verb"\prelim{#1}[#2]"&Preliminary communication&PRELIMINARY COMMUNICATION\\
\verb"\letter{#1}"&Letter&LETTER TO THE EDITOR\\
\verb"\article{#1}{#2}[#3]"&Other articles&Whatever is entered as {\tt
\#1}\\
\br
\end{tabular*}
\end{table}


The next information required is the list of authors' names and their
affiliations. For the authors names type \verb"\author{#1}",
where \verb"#1" is the
list of all authors' names. The style for the names is initials then
surname, with a comma after all but the last
two names, which are separated by `and'. Initials should {\it not} have
full stops. Christian names may be used if
desired. If the authors are at different addresses one of the symbols
\dag, \ddag, \S, $\Vert$, \P, $^+$, *, $\sharp$ should be used after each
surname to reference an author to his/her address
and any footnotes. The symbols should be used in the order given.  If
an author has additional information to appear as a footnote, such as
a permanent address, and the
footnote symbols are not being used to identify an address,
the footnote should be entered after the surname
as a normal \LaTeX\ footnote, without
specifying a sign. Where footnote symbols are being used to indicate
which address the author is at, the
symbol used for a footnote should be the next one from the list given above
and has to be selected individually using the command
\verb"\ftnote{<num>}{Text of footnote}", where \verb"<num>" is a
number representing the position of the desired symbol in the list above,
i.e.\ for 1 for \dag, 2 for \ddag, etc.
.

The addresses follow the list of authors. Each address is called by
\verb"\address{#1}" with the address as the single parameter in braces.
If there is more
than one address then the appropriate symbol should come at the start of
the address.

The abstract follows the addresses and
should give readers concise information about the content
of the article and indicate the main results obtained and conclusions
drawn. It should be complete in itself with no table numbers, figure
numbers or references included and should not normally exceed 200
words.
To indicate the start
of the abstract type \verb"\begin{abstract}" followed by the text of the
abstract (not in braces).  The abstract should normally be restricted
to a single paragraph and is terminated by the command
\verb"\end{abstract}"

Following the abstract come any
Physics and Astronomy Classification System (PACS) codes
or American Mathematical Society
(AMS) classification scheme numbers.
The command
\verb"\pacs{#1}", with the subject classification numbers from the
Physics and Astronomy Classification Scheme as the parameter, defines
the subject area of the paper (or for a single number \verb"\pacno{#1}").
If PACS numbers are not readily
available, {\it Physics Abstracts\/} classification scheme numbers can be
given instead. If this command  is omitted the
classification numbers for indexing
will be allocated by IOP staff. It is
unnecessary~to supply PACS numbers for {\it Inverse
Problems} and {\it Physics in Medicine and Biology}.
AMS classification numbers may be given as well as or instead of PACS
numbers for mathematical articles, they are specified using the
command \verb"\ams{#1}".

The command \verb"\maketitle" prints the short title,
the journal to which the article has been submitted and the date. It then
forces a page break. If \verb"\maketitle" is omitted the text of the
article will
start immediately after the abstract.


The code for the start of a title page of a typical paper might read:
\begin{verbatim}
\documentstyle{ioplppt}
\begin{document}
\jl{4}
\title{The anomalous magnetic moment of the neutrino and its
relation to the solar neutrino problem}[The anomalous magnetic
moment of the neutrino]

\author{P J Smith\dag, T M Collins\ddag,
R J Jones\ddag\ftnote{3}{Present address:
Department of Physics, University of Bristol, Tyndalls Park Road,
Bristol BS8 1TS, UK.} and Janet Williams\P}

\address{\dag\ Mathematics Faculty, Open University,
Milton Keynes MK7~6AA, UK}
\address{\ddag\ Department of Mathematics,
Imperial College, Prince Consort Road, London SW7~2BZ, UK}
\address{\P\ Department of Computer Science,
University College London, Gower Street, London WC1E~6BT, UK}

\begin{abstract}
...
\end{abstract}

\pacs{1315, 9440T}
\maketitle
\end{verbatim}

\section{The text}
\subsection{Sections, subsections and subsubsections}
The text of papers and reviews, but not comments or letters, should be
divided into sections, subsections and, where necessary,
subsubsections. To start a new section end the previous paragraph and
then include \verb"\section" followed by the section heading within braces.
Numbering of sections is done {\it automatically} in the headings:
sections will be numbered 1, 2, 3, etc, subsections will be numbered
2.1, 2.2,  3.1, etc, and subsubsections will be numbered 2.3.1, 2.3.2,
etc.  Cross references to other sections in the text can be made using
labels (see section~\ref{xrefs}) or can
be made manually. Subsections and subsubsections are similar to sections but
the commands are \verb"\subsection" and \verb"\subsubsection" respectively.
Sections have a bold heading, subsections an italic heading and
subsubsections an italic heading with the text following on directly.
\begin{verbatim}
\section{This is the section title}
\subsection{This is the subsection title}
\end{verbatim}


The first section is normally an introduction,  which should state clearly
the object of the work, its scope and the main advances reported, with
brief references to relevant results by other workers. In long papers it is
helpful to indicate the way in which the paper is arranged and the results
presented.

For articles not divided into sections, precede the
start of the text with the command \verb"\nosections", which provides the
appropriate space and causes the paragraph indentation to be cancelled
for the first paragraph.

Footnotes should be avoided whenever possible. If required they should be
used only for brief notes that do not fit conveniently into the text. The
standard \LaTeX\ macro \verb"\footnote" should be used and will normally
give an appropriate symbol; if a footnote sign needs to be specified
directly \verb"\ftnote{<num>}{Text}" can be used instead.


\subsection{Appendices}
Technical detail that it is necessary to include, but that interrupts
the flow of the article, may be consigned to an appendix. If there are
two or more appendices they will be called Appendix A, Appendix B, etc.
Numbered equations should be in the form (A1), (A2), etc,
unless there are two or more appendices and
equation numbering is by section, in which case the numbering (A1),
(A2), (B1), (B2) or
(A1.2), (A2.1), etc will be used.

The command \verb"\appendix" is used to signify the start of the
appendixes. Thereafter \verb"\section", \verb"\subsection", etc, will
give headings appropriate for an appendix. To obtain a simple heading of
`Appendix' use the code \verb"\section*{Appendix}".


\subsection{Acknowledgments}
Authors wishing to acknowledge assistance or encouragement from
colleagues, special work by technical staff or financial support from
organizations should do so in an unnumbered Acknowledgments section
immediately following the last numbered section of the paper. The
command \verb"\ack" sets the acknowledgments heading as an unnumbered
section. For Letters
\verb"\ack" does not set a heading but leaves a line space and does not
indent the next paragraph.


\subsection{Some matters of style}
The main elements of IOP house style are presented in the booklet
{\it Notes for Authors} (available upon request from Institute of Physics
Publishing,
Techno House, Redcliffe Way, Bristol, BS1 6NX, UK). Some points to note,
however, are the following.

(i) It is IOP house style to use `-ize' spellings (diagonalize,
renormalization, minimization, etc). Common exceptions are devise,
promise and advise.

(ii) English spellings are preferred (colour, flavour, behaviour,
tunnelling, artefact, focused, focusing, fibre, etc). We write of a
computer program on disk; otherwise, we use `programme' and `disc'.

(iv) Compound words beginning `non-' or `self-' are hyphenated
(non-existent, self-consistent, etc).

(v) The words table, figure, equation and reference should be written
in full and {\bf not} contracted to Tab., fig., eq. and ref.

(vi) The contractions `i.e.' and `e.g.' should appear Roman, {\bf not}
italic.

(vi) Data {\it are} \dots ({\bf not} data is \dots).


\section{Mathematics}
\subsection{Two-line constructions}
The great advantage of \TeX\ and \LaTeX\
over other text processing systems is their
ability to handle mathematics to almost any degree of complexity. However,
in order to produce an article suitable for publication within a journal,
authors should exercise some restraint on the constructions used.
For instance, displayed equations should generally be restricted to a
height of two lines, e.g.\ constructions such as
\[
P={{\displaystyle{a\over b}+{c\over d}+{b\over c}}\over
(a^2+b^2)(c^2+d^2)}
\]
should not be used but converted to one of the
equivalent two-line forms
\[
P={a/b+c/d+b/c\over (a^2+b^2)(c^2+d^2)}
\qquad P={ab^{-1}+cd^{-1}+bc^{-1} \over
(a^2+b^2)(c^2+d^2)}
\]
or
\[
P=\left({a\over b}+{c\over d}+{b\over c}\right)
[(a^2+b^2)(c^2+d^2)]^{-1}.
\]
For simple fractions in the text the solidus \verb"/", as in
$\lambda/2\pi$, should be used instead of \verb"\frac" or \verb"\over",
care
being taken to use parentheses where necessary to avoid ambiguity, for
example to distinguish between $1/(n-1)$ and $1/n-1$. Exceptions to
this are the proper fractions $\frac12$, $\frac13$, $\frac34$,
etc, which are better left in this form. In displayed equations
horizontal lines are preferable to solidi provided the equation is
kept within a height of two lines. A two-line solidus should not be
used; the construction $(\ldots)^{-1}$ should be used instead.

\subsection{Roman and italic in mathematics}
In mathematics mode \LaTeX\ automatically sets variables in an italic
font. In most cases authors should accept this italicization. However,
there are some cases where it is necessary to use a Roman font; for
instance, IOP journals use a Roman d for a differential d, a Roman e
for an exponential e and a Roman i for the square root of $-1$. To
accommodate this and to simplify the  typing of equations we have
modified the plain \TeX\ definitions \verb"\d" and \verb"\i"
and added a macro
\verb"\e". \verb"\d" now gives a Roman d for use in equations,
e.g.\ $\d x/\d y$
is obtained by typing \verb"$\d x/\d y$", and the macro for a dot under has
been redefined to be \verb"\du". \verb"\e" gives a Roman e for use in simple
exponentials such as $\e^x$. The macro \verb"\i" has also been amended and
will give a Roman i ($\i$) in maths mode; \verb"\ii" gives a dotless i
(\ii) in normal text.

Certain other common mathematical functions, such as cos, sin, det and
ker, should appear in Roman type. \LaTeX\ provides macros for
most of these functions
(in the cases above, \verb"\cos", \verb"\sin", \verb"\det" and \verb"\ker"
respectively), we provide additional definitions for $\Tr$, $\tr$ and
$\Or$ (\verb"\Tr", \verb"\tr" and \verb"\Or", respectively).
The author should take care to use these macros.

Subscripts and superscripts should be in Roman type if they are labels
rather than variables or characters that take values. For example in the
equation
\[
\epsilon_m=-g\mu_{\rm n}Bm
\]
$m$, the $z$ component of the nuclear spin, is italic because it can have
different values whereas n is Roman because it
is a label meaning nuclear ($\mu_{\rm n}$
is the nuclear magneton).



\subsection{Alignment of mathematics}
\subsubsection{Alignment on the secondary margin}
IOP style for displayed mathematics is not to centre equations,
as \LaTeX\ normally does, but to have each equation indented to a
secondary margin a fixed
distance of five picas from the primary left-hand margin, except for long
equations that will just fit on one line, or need to be continued on
subsequent lines, which start full left.
Any continuation lines are indented the fixed amount (five picas in the
printed form).
The macros in the IOP preprint style automatically include the
\LaTeX\ \verb"fleqn"
option and line equations up on the
secondary margin unless they are set within double dollar signs.
Thus double dollar signs should not be used and the alternative
\verb"\[ ... \]" should be used instead.
It is thus only necessary to indicate which lines
should start full left and this is done by including \verb"\fl" (full
left) at the start of these lines.
Thus the equations:
\begin{eqnarray}
\phi_{k}(\vec{r})=(2\pi)^{-3/2} \exp(\i\vec{k}\cdot\vec{r}) \\
N^+=\exp(\case12\pi\nu)\Gamma(1-\i\nu).
\end{eqnarray}
are set with the code
\begin{verbatim}
\begin{eqnarray}
\phi_{k}(\vec{r})=(2\pi)^{-3/2} \exp(\i\vec{k}\bdot\vec{r}) \\
N^+=\exp(\case12\pi\nu)\Gamma(1-\i\nu).
\end{eqnarray}
\end{verbatim}
Where an equation will not fit on a line if indented but would if it
were not, then the equation is started full left and this is achieved
simply by adding \verb"\fl" to the start of the line. For example
the equation
\begin{equation}
\fl R_{\rho l m,\rho'l'm'}(E)=\frac{1}{r_0}\sum_{i,j}\;\langle\rho l m r_0
\mid\Phi_i\rangle\; [(H_{\Omega}+B)-ES_{\Omega}]^{-1}_{i,j}\;\langle\Phi_j
\mid\rho'l'm'r_0\rangle.
\end{equation}
does not fit on the line if indented to the secondary margin but fits in
comfortably when full left.


For equations which do not fit on one line, even if started full left,
the first line should be set full left with the turnover lines at the
secondary margin. This is achieved simply by adding \verb"\fl" at the start
of the first line and \verb"\\" at the end of each line (the \verb"\\"
at the
end of the last line is optional). Equations should be split at
mathematically sound points, often at =, + or $-$ signs or between
terms multiplied together. The connecting signs are not repeated and
appear only at the beginning of the turned-over line. A multiplication
sign should be added to the start of turned-over lines where the break
is between two multiplied terms. Where an equation is broken at an
equals sign (or similar, i.e.\ $\equiv$, $\le$, $\sim$, etc) the sign
is made more prominent by aligning it to the left of the secondary
margin; where it is a +, $-$ or $\times$ the sign goes to the right.
Alignment to the left of the secondary margin is achieved by adding
\verb"\lo" in front of the sign (and enclosing the sign within braces if it
requires more than one character to control sequence, e.g.\
\verb"\lo{:=}"). An example demonstrating these features is:
\begin{eqnarray}
\fl\langle\cos(q\Omega_s)\rangle=\frac12\int^\infty_0
\frac{k_s(b)}{k^{\rm tot}_s}\{\cqo{o}{x}+\cqo{i}{x}\}
2\pi b\, \d b\nonumber\\
\lo=\sum_c{(\mu^s_c)^2/\vert\Delta V'_s(R^s_{\rm c})\vert \over \sum_n
(R^s_n\mu^s_n)^2(1-V^s_n/E)^{1/2}/\vert\Delta'_s(R^s_n)\vert}\nonumber\\
\times \frac12\int^{b_{\rm max}}_0 \{\cqo{o}{\rm c}+\cqo{i}{\rm c}\}
b\, \d b/v_s(b,R^s_{\rm c}).
\end{eqnarray}
where a simplified version of the
code used is:
\begin{verbatim}
\begin{eqnarray}
\fl    <first line>  \nonumber\\
\lo=   <second line> \nonumber\\
\times <third line>
\end{eqnarray}
\end{verbatim}
Some more examples of
turned-over equations and their code are given in the sample paper.

Note that alignment at the secondary margin normally takes
precedence over aligning equals signs so there is usually no need
to include any ampersands within the
\verb"eqnarray" environment.


\subsubsection{Secondary alignment}
While the primary alignment either on the secondary
margin or full left is adequate in most cases
there are examples where additional
alignment is desirable. Firstly, for repeated series of short
equations, secondly for
equations with attached conditions and thirdly for connected
series of equations with a short left-hand side which together
occupy more than a full line but where each individual
part is short. In these cases the \verb"eqnarray" environment
should be used; there will still be alignment at the
secondary margin but ampersands should be positioned to
provide the secondary alignment. For equations with conditions the
space separating the longest part from its condition is provided by
\verb"\qquad". Examples of equations requiring secondary alignment are:
\begin{eqnarray}
A^{(3/2)}=A^{(+)}-A^{(-)}&(I=\case32)\\
A^{(1/2)}=A^{(+)}+2A^{(-)}\qquad&(I=\case12)\\
A^{(0)}&({\rm otherwise}).
\end{eqnarray}
which is obtained with the code
\begin{verbatim}
\begin{eqnarray}
A^{(3/2)}=A^{(+)}-A^{(-)}&(I=\case32)\\
A^{(1/2)}=A^{(+)}+2A^{(-)}\qquad&(I=\case12)\\
A^{(0)}&({\rm otherwise}).
\end{eqnarray}
\end{verbatim}
and
\begin{eqnarray}
C(12)&=[\vec\pi(x)\cdot\vec\phi(x+r)]\nonumber\\
&\simeq 1-{\rm const}{r^2\over L^2}\int^L_r{x\, \d x\over
x^2}+\cdots\\
&\simeq 1-{\rm const}{r^2\over L^2}\ln\left({L\over r}\right)+\cdots.
\end{eqnarray}
for which the code is
\begin{verbatim}
\begin{eqnarray}
C(12)&=[\vec\pi(x)\cdot\vec\phi(x+r)]\nonumber\\
&\simeq 1-{\rm const}{r^2\over L^2}\int^L_r{x\, \d x\over
x^2}+\cdots\\
&\simeq 1-{\rm const}{r^2\over L^2}\ln\left({L\over r}\right)+\cdots.
\end{eqnarray}
\end{verbatim}

\subsection{Special characters for mathematics}
Bold Greek capital characters are obtained with the commands
\verb"\bGamma" \dots\ \verb"\bOmega". Bold lowercase Greek characters can
be obtained with commands \verb"\balpha" \dots\ \verb"\bomega"
provided the AMS extension font set is available and the
style option \verb"iopfts" has been loaded. For more details of
other special characters available from the AMS extension fonts
see \verb"iopfts.tex". When the AMS fonts are  not available vectors
should be indicated with an over arrow, e.g.\ \verb"\vec{r}", and matrices
by a double underline using the command \verb"\mat{#1}" which gives, for
example, $\mat{A}$.


Calligraphic letters are obtained with \verb"{\cal ABC}" or
\verb"\Cal{ABC}".
If the AMS fonts are available and the optional style
\verb"iopfts" has been loaded, bold calligraphic
are obtained with \verb"{\bcal ABC}" or
\verb"\bCal{ABC}" and open face characters
(often called `blackboard bold')
are obtained  with the code \verb"\Bbb{#1}".
In addition to the AMS characters some
extra maths characters are also defined with the \verb"iopfts" option; see
\verb"iopfts.tex" for details.

Table~\ref{math-tab2} lists some other macros for use in
mathematics with a brief description of their purpose.
Both \verb"\ms" (medium space) and \verb"\bs" (big space) can be used to
provide extra spacing between lines of a displayed equation or table.
This space may be necessary when several separate equations are within the
same equation environment. \verb"\dsty", \verb"\tsty", \verb"\ssty" and
\verb"\sssty" are simply shortened forms of standard commands solely to
save time in typing.

\begin{table}
\caption{Other macros defined in IOP macros for use in maths.
\label{math-tab2}}
\begin{indented}
\item[]\begin{tabular*}{\indentedwidth}{@{}l*{15}{@{\extracolsep{0pt plus
12pt}}l}}
\br
Macro&Result&Description\\
\mr
Spaces\\
\verb"\fl"&&Start line of equation full left\\
\verb"\ms"&&Openout lines in displayed equations slightly ($\sim$3pt)\\
\verb"\bs"&&Bigger space ($\sim$6pt) to separate lines in displays\\
\verb"\ns"&&Small negative space between lines in displays\\
\bs
\multispan{3}{For symbols to left of 5 pica indent\hfill}&\\
\verb"\lo{#1}"&$\lo{\#1}$&Any symbol overhanging to left\\
\verb"\eql"&$\eql$&Left overhanging equals sign\\
\verb"\lsim"&$\lsim$&Left overhanging tilde\\
\verb"\lsime"q&$\lsimeq$&Left overhanging approximately equals\\
\verb"\lequiv"&$\lequiv$&Left overhanging equivalent sign\\
\bs
Miscellaneous\\
\verb"\case#1#2"&$\case{\#1}{\#2}$&Text style fraction in display\\
\verb"\dsty"&&Display style\\
\verb"\tsty"&&Text style\\
\verb"\ssty"&&Script style\\
\verb"\sssty"&&Scriptscript style\\
\br
\end{tabular*}
\end{indented}
\end{table}

\subsection{Miscellaneous points}
Exponential expressions, especially those containing subscripts or
superscripts, are clearer if the notation $\exp(\ldots)$ is used except for
simple examples. For instance $\exp[\i(kx-\omega t)]$ and $\exp(z^2)$ are
preferred to $\e^{\i(kx-\omega t)}$ and $\e^{z^2}$, but
$\e^2$
is acceptable. Similarly the square root sign $\sqrt{\phantom{b}}$ should
only be used with relatively
simple expressions, e.g.\ $\sqrt2$ and $\sqrt{a^2+b^2}$;
in other cases the
power $1/2$ should be used.

It is important to distinguish between ln $= \log_\e$ and lg
$=\log_{10}$. Braces, brackets and parentheses should be used in the
following order: $\{[(\;)]\}$. The same ordering of brackets should be
used within each size. However, this ordering is to be ignored if the
brackets have a
special meaning (e.g.\ if they denote an average or a function).  Decimal
fractions should
always be preceded by a zero: for example 0.123 {\bf not} .123. For long
numbers commas are not inserted but instead a thin space is added after
every third character away from the position of the decimal point unless
this leaves a single separated character: e.g.\ $60\,000$, $0.123\,456\,78$
but 4321 and 0.7325.

Equations that are referred to in the text should be numbered with
the number on the right-hand side.


\subsection{Equation numbering\label{eqnum}}
\LaTeX\ provides the facilities for automatically numbering equations
and these should be used where possible. Sequential numbering (1), (2), etc,
is the default numbering system although if the command
\verb"eqnobysec" is included in the preamble equation numbering
by section is obtained, e.g.\
(2.1), (2.2), etc. Equation numbering by section {\it must}
be used for {\it Reports on Progress in
Physics}. When referring to an equation in the text, either put
the equation number, in brackets, e.g.\ `as in (2)', or spell out the
word equation in full, e.g.\ `if equation (2) is factorized'; do not
use abbreviations such as eqn or eq.
When cross-referencing is used, \verb"\ref{<label>}"
 will produce `(\verb"<eqnum>")',
\verb"\eref{<label>}" produces `equation (\verb"<eqnum>")' and
\verb"Eref{<label>}" produces `Equation (\verb"<eqnum>")',
where \verb"<label>"
is the
label to produce equation number \verb"<eqnum>".


If an equation number is centred between lines then the
command \verb"\eqalign{...}"
can be used as within the `equation' environment.
After \verb"\begin{equation}" enclosed the lines over
which the number is
to be centred
within \verb"\eqalign{...}"  with
\verb"\\" or \verb"\cr"
at the end of each line. Ampersands are unnecessary within the
\verb"\eqalign" but can be used for secondary alignment if necessary.
The code
\begin{verbatim}
\begin{equation}
\eqalign{T_{11}&=(1+P_\e)I_{\uparrow\uparrow}-(1-P_\e)
I_{\uparrow\downarrow}\\
T_{-1-1}&=(1+P_\e)I_{\downarrow\downarrow}-(1-P_\e)
I_{\uparrow\downarrow}\\
S_{11}&=(3+P_\e)I_{\downarrow\uparrow}-(3-P_e)I_{\uparrow\uparrow}\\
S_{-1-1}&=(3+P_\e)I_{\uparrow\downarrow}-(3-P_\e)
I_{\downarrow\downarrow}}
\end{equation}
\end{verbatim}
gives four equations with a centred
number:
\begin{equation}
\eqalign{T_{11}&=(1+P_\e)I_{\uparrow\uparrow}-(1-P_\e)
I_{\uparrow\downarrow}\\
T_{-1-1}&=(1+P_\e)I_{\downarrow\downarrow}-(1-P_\e)I_{\uparrow\downarrow}\\
S_{11}&=(3+P_\e)I_{\downarrow\uparrow}-(3-P_e)I_{\uparrow\uparrow}\\
S_{-1-1}&=(3+P_\e)I_{\uparrow\downarrow}-(3-P_\e)
I_{\downarrow\downarrow}}
\end{equation}
Note that the secondary alignment at the equals signs would not normally
be necessary but is included here for demonstration purposes.


Sometimes it is useful to number equations as parts of the same
basic equation. This can be accomplished by inserting the
commands \verb"\numparts" before the equations concerned and
\verb"\endnumparts" when reverting to the normal sequential numbering.
The equations below show the previous equations numbered as separate parts
using \verb"\numparts ... \endnumparts" and the \verb"eqnarray"
environment
\numparts
\begin{eqnarray}
T_{11}&=(1+P_\e)I_{\uparrow\uparrow}-(1-P_\e)
I_{\uparrow\downarrow}\\
T_{-1-1}&=(1+P_\e)I_{\downarrow\downarrow}-(1-P_\e)I_{\uparrow\downarrow}\\
S_{11}&=(3+P_\e)I_{\downarrow\uparrow}-(3-P_e)I_{\uparrow\uparrow}\\
S_{-1-1}&=(3+P_\e)I_{\uparrow\downarrow}-(3-P_\e)
I_{\downarrow\downarrow}
\end{eqnarray}
\endnumparts


\subsection{Miscellaneous extra commands for displayed equations}
The \verb"\cases" command of Plain \TeX\ is available
for use with \LaTeX\ but has been amended slightly to
increase the space between the equation and the condition.
\Eref{cases}
demonstrates simply the output from the \verb"\cases" command
\begin{equation}
\label{cases}
X=\cases{1&for $x \ge 0$\\
-1&for $x<0$\\}
\end{equation}
The code used was:
\begin{verbatim}
\begin{equation}
\label{cases}
X=\cases{1&for $x \ge 0$\\
-1&for $x<0$\\}
\end{equation}
\end{verbatim}

To obtain text style fractions within displayed maths the command
\verb"\case#1#2" can be used (see equations (2) and (5)) instead
of the usual \verb"\frac#1#2" command or \verb"#1 \over #2".

When two or more short equations are on the same line they should be
separated by a `qquad space' (\verb"\qquad"), rather than
\verb"\quad" or any combination of \verb"\,", \verb"\>", \verb"\;"
and \verb"\ ".



\section{Referencing}
Two different styles of referencing are in common use: the Harvard
alphabetical system and the Vancouver
numerical system. All the IOP journals allow
the use of the Harvard system but the numerical system should {\bf not} be
used in {\it Physics in Medicine and Biology}.
Brief descriptions of the use of the two
referencing systems are given below.

\subsection{Harvard system}
In the Harvard system the name of the author appears in the text together
with the year of publication. As appropriate, either the date or the name
and date are included within parentheses. Where there are only two authors
both names should be given in the text; if there are more than two
authors only the first name should appear followed by `{\it et al}'
(which can be obtained by
typing \verb"\etal"). When two or
more references to work by one author or group of authors occur for the
same year they should be identified by including a, b, etc after the date
(e.g.\ 1986a). If several references to different pages of the same article
occur the appropriate page number may be given in the text, e.g.\ Kitchen
(1982, p 39).

The reference list at the end of an article consists of an
unnumbered section containing an
alphabetical listing by authors' names and in date order for each
author or group of identical authors. The reference list in the
preprint style is started by including the command
\verb"\section*{References}". and then
\verb"\begin{harvard}".
There will be two basic types of
entries within the reference list: (i) those to journal articles and
(ii) those to books, conference proceedings and reports. For both of
these types of references \verb"\item[]"
is required before the start of an individual reference.
The reference list is completed with \verb"\end{harvard}".
There is also a shortened form of the coding; \verb"\section*{References}"
and \verb"\begin{harvard}" can be replaced by the single command
\verb"\References" and \verb"\end{harvard}" can be shortened to
\verb"\endrefs".


\subsubsection{References to journal articles}
A normal reference to a journal article contains three changes of
font:
the authors and date appear in Roman type, the journal title in
italic, the volume number in bold and the page numbers in Roman again.
A typical journal entry would be:

\smallskip
\begin{harvard}
\item[] Cisneros A 1971 {\it Astrophys.\ Space Sci.} {\bf 10} 87
\end{harvard}
\smallskip

\noindent which would be obtained by typing, within the references
environment
\begin{verbatim}
\item[] Cisneros A 1971 {\it Astrophys. Space Sci.} {\bf 10} 87
\end{verbatim}

Features to note are the following.

(i) The authors should be in the form surname (with only the first
letter capitalized) {\bf followed} by the initials with {\bf no}
periods after the initials. Authors should be separated by a comma
except for the last two which should be separated by `and' with no
comma preceding it. For journals that accept titles of articles in the
reference list,  the title should be in Roman (upright)
lower case letters, except for an initial
capital, and should follow the date.

(ii) The journal is in italic and is abbreviated. \ref{jlabs}
gives a list of
macros that will give the correct abbreviation for
many of the common journals. If a journal has several parts denoted by
different letters the part letter
should be inserted after the journal in Roman type, e.g.\
{\PR\ \rm A}. An exception to this is {\it Physics Letters} where
the part letter is included in the volume number.

(iii) The volume number is bold; the page number is Roman.
 Both the initial and final page
numbers should be given where possible. The final page number should be in
the shortest possible form and separated from the initial page number by an
en rule (\verb"--"), e.g.\ 1203--14.

(iv) Where there are two or more references with identical authors,
the authors' names should not be repeated but should be replaced by
\verb"\dash" on the second and following occasions. Thus
\begin{verbatim}
\item[]Davis R, Mann A K and Wolfenstein L 1989  {\it Ann. Rev. Nucl.
Part. Sci.} {\bf 39} 467
\item[]\dash 1990 Private communication
\end{verbatim}



\subsubsection{References to books, conference proceedings and reports}
References to books, proceedings and reports are similar, but have only two
changes of font. The authors and date of publication are in Roman, the
title of the book is in italic, and the editors, publisher,
town of publication
and page number are in Roman. A typical reference to a book and a
conference paper might be

\smallskip
\begin{harvard}
\item[] Dorman L I 1975 {\it Variations of Galactic Cosmic Rays}
(Moscow: Moscow State University Press) p~103
\item[] Caplar R and Kulisic P 1973 {\it Proc.\
Int.\ Conf.\ on Nuclear Physics (Munich)} vol~1 (Amsterdam:
North-Holland/American Elsevier) p~517
\end{harvard}
\smallskip

\noindent which would be obtained by typing
\begin{verbatim}
\item[] Dorman L I 1975 {\it Variations of Galactic Cosmic Rays}
(Moscow: Moscow State University Press) p~103
\item[] Caplar R and Kulisic P 1973 {\it Proc. Int. Conf. on Nuclear
Physics (Munich)} vol~1 (Amsterdam: North-Holland/American
Elsevier) p~517
\end{verbatim}
\noindent respectively.


Features to note are the following.

(i) Book titles are in italic and should be spelt out in full with
initial capital letters for all except minor words. Words such as
Proceedings, Symposium, International, Conference, Second, etc should
be abbreviated to Proc., Symp., Int., Conf., 2nd,
respectively, but the rest of the title should be given in full,
followed by the town or city where the conference was held. For
Laboratory Reports the Laboratory should be spelt out wherever
possible, e.g.\ {\it Argonne National Laboratory Report}.

(ii) The volume number as, for example, vol~2, should be followed by
the editors, if any, in a form such as ed~A~J~Smith and P~R~Jones. Use
\etal if there are more than two editors. Next comes the town of
publication and publisher, within brackets and separated by a colon,
and finally the page numbers preceded by p if only one number is given
or pp if both the initial and final numbers are given.

Cross referencing between the text and the
reference list is not necessary for alphabetic referencing
in the Harvard system as adding or deleting a reference
does not normally change any of the other references.

\subsection{Numerical system}
In the numerical system references are numbered sequentially
throughout the text. The numbers occur within square brackets and one
number can be used to designate several references. A numerical
reference list in the preprint style is started by including the
command \verb"\section*{References}" and then
\verb"\begin{thebibliography}{<num>}", where \verb"<num>" is the largest
number in the reference list (or any other number with the same number
of digits).  The
reference list gives the references in
numerical, not alphabetical, order and is completed by
\verb"\end{thebibliography}". Short forms of the commands are again
available \verb"\Bibliography{<num>}" can be used at the start of the
references section and \verb"\endbib" at the end.
(Note that footnotes should not be
part of a numerical reference system, but should be included in the
text using the symbols \dag, \ddag, etc.)

References to journals and books are similar to those in the Harvard
system, except that two or more references with identical authors are
spelt out in full, i.e.\ they are {\bf not} replaced with \verb"\dash".
When one number covers two or more separate references \verb"\nonum"
or \verb"\par\item[]" should be included at
the start of each reference in a group after the first.
A typical numerical reference list might begin

\smallskip

\numrefs{1}
\item Dorman L I 1975 {\it Variations of Galactic Cosmic Rays} (Moscow:
Moscow State University Press) p~103
\item Caplar R and Kulisic P 1973 {\it Proc.\ Int.\ Conf.\
on Nuclear Physics (Munich)} vol~1 (Amsterdam:
North-Holland/American Elsevier) p~517
\item Cisneros A 1971 {\it Astrophys. Space Sci.} {\bf 10} 87
\endnumrefs
\smallskip

\noindent which would be obtained by typing

\begin{verbatim}
\item Dorman L I 1975 {\it Variations of Galactic Cosmic Rays}
(Moscow: Moscow State University Press) p~103
\item Caplar R and Kulisic P 1973 {\it Proc. Int. Conf.
on Nuclear Physics (Munich)} vol~1 (Amsterdam:
North-Holland/American Elsevier) p~517
\item Cisneros A 1971 {\it Astrophys. Space Sci.} {\bf 10} 87
\end{verbatim}

The point to note is that this is identical to the entries in the
Harvard system except that square brackets following
\verb"\item" are no longer required.

\subsection{Reference lists}
A complete reference should provide the reader with enough information to
locate the article concerned and should consist of: name(s) and initials,
date published, title of journal or book, volume number, editors, if any,
and town of publication and publisher in parentheses for books,
and finally the
page numbers. Titles of journal articles may also be included.
Up to twenty authors may be given in a particular reference; where
there are more than twenty only the first should be given followed by
`{\it et al}'. Abbreviations of the names of periodicals used by Institute
of Physics Publishing are usually the same as those
given in British Standard
BS 4148: 1985. If an author is unsure of an abbreviation and the
journal is not given in Appendix C, it is best to leave the title in
full. The terms {\it loc.\ cit.\ }and {\it ibid.\ }should not be used.
Unpublished conferences and reports should generally not be included
in the reference list and articles in the course of publication should
be entered only if the journal of publication is known. References to
preprints should give the title of the preprint and/or preprint number
(if relevant). A thesis submitted for a higher degree may be included
in the reference list if it has not been superseded by a published
paper and is available through a library; sufficient information
should be given for it to be traced readily.

\section{Cross referencing\label{xrefs}}
The facility to cross reference items in the text is very useful when
composing articles the precise form of which is uncertain at the start
and where revisions and amendments may subsequently be made. When using
cross referencing labels are given to elements in the text, for
instance sections, figures, tables or equations and the elements may
be referred to elsewhere
in the text by using the label. When the article is
first processed the labels are read in and assigned, e.g.\ 2.1 for a
subsection or (4) for an equation number. When the article is
processed a second or subsequent time the labels assignments are read
in at the start of the file and the correct values given in the text.
\LaTeX\ provides excellent facilities for doing cross-referencing
and these can be very useful in preparing articles.

\subsection{References}
Cross referencing is useful for numeric reference lists because, if it
is used, adding
another reference to the list does not then involve renumbering all
subsequent references. It is not necessary for referencing
in the Harvard system where the final reference list is alphabetical
and normally no other changes are necessary when a reference is added or
deleted.
Two passes are necessary initially to get the cross references right
but once they are correct a single run is usually sufficient provided an
\verb".aux" file is available and the file
is run to the end each time.
\verb"\cite" and \verb"\bibitem" are used to link citations in the text
with their entry in the reference list;
if the
reference list contains an entry \verb"\bibitem{label}",
then \verb"\bibitem{label}"
will produce the correct number in the reference list and
\verb"\cite{label}" will produce the number within square brackets in the
text. \verb"label" may contain alphabetic letters,
or punctuation characters but must not contain spaces or commas. It is also
recommended that the underscore character \_{} is not used in cross
referencing.
Thus labels for the form
\verb"eq:partial", \verb"fig:run1", \verb"eq:dy'",
etc, may be used. When several
references occur together in the text \verb"\cite" may be used with
multiple labels with commas but no spaces separating them;
the output will be the
numbers within a single pair of square brackets with a comma and a
thin space separating the numbers. Thus \verb"\cite{label1,label2,label4}"
would give [1,\,2,\,4]. Note that no attempt is made to sort the
labels and no shortening of groups of consecutive numbers is done.
Authors should therefore try to use multiple labels in the correct
order.

The numbers for the cross referencing are generated in the order the
references appear in the reference list, so that if the entries in the
list are not in the order in which the references appear in the text
then the
numbering within the text will not be sequential. To correct this
change the ordering of the entries in the reference list and then
rerun {\it twice}.

\subsection{Equation numbers, sections, subsections, figures and
tables}
Cross references can be made to equation numbers, sections,
subsections, figures and tables or
any numbered environment
and this is a very useful feature when
writing a document as its final structure is often not fully defined
at the start. Thus a later section can be referred to by a label
before its precise number is known and when it is defined there is no
need to search back through the document to insert the correct value
manually. For this reason the use of cross referencing
can save considerable time.

Labels for equation numbers, sections, subsections, figures and tables
are all defined with the \verb"\label{label}" command and cross references
to them are made with the \verb"\ref{label}" command. The \verb"\label"
macro
identifies the type of environment it is used in and converts \verb"label"
into the correct form for that type of environment, thus \verb"\ref{label}"
might give (2.3) for an equation number but 3.1 for a subsection and 2
for a figure or table number.

Any section, subsection, subsubsection, appendix or subappendix
command defines a section type label, e.g. 1, 2.2, A2, A1.2 depending
on context. A typical article might have in the code of its introduction
`The results are discussed in section\verb"~\ref{disc}".' and
the heading for the discussion section would be:
\begin{verbatim}
\section{Results\label{disc}}
\end{verbatim}
Labels to sections, etc, may occur anywhere within that section except
within another numbered environment.
Within a maths environment labels can be used to tag equations which are
referred to within the text.
An example of an equation with a label and a reference to it
is:
\begin{verbatim}
\begin{equation}
X=a\cos\theta+ b\sin\phi. \label{cossin}
\end{equation}
Equation (\ref{cossin}) ...
\end{verbatim}
which produces
\begin{equation}
X=a\cos\theta+ b\sin\phi. \label{cossin}
\end{equation}
Equation (\ref{cossin}) ...

In addition to the standard \verb"\ref{<label>}" the abbreviated
forms given in the \tref{abrefs}
are available for reference to standard parts of the text

\Table{Alternatives to the normal references $\backslash${\tt ref}
and the text generated by
them. Note it is not normally necessary to include the word equation
before an equation number except where the number starts a sentence. The
versions producing an initial capital should only be used at the start of
sentences.\label{abrefs}}
\br
Reference&Text produced\\
\mr
\verb"\eref{<label>}"&(\verb"<num>")\\
\verb"\Eref{<label>}"&Equation (\verb"<num>")\\
\verb"\fref{<label>}"&figure \verb"<num>"\\
\verb"\Fref{<label>}"&Figure \verb"<num>"\\
\verb"\sref{<label>}"&section \verb"<num>"\\
\verb"\Sref{<label>}"&Section \verb"<num>"\\
\verb"\tref{<label>}"&table \verb"<num>"\\
\verb"\Tref{<label>}"&Table \verb"<num>"\\
\br
\endTable

\section{Tables and table captions}
Tables are numbered serially and referred to in the text
by number (table 1, etc, {\bf not} tab. 1). Each table should have an
explanatory caption which should be as concise as possible. If a table
is divided into parts these should be labelled \pt(a), \pt(b),
\pt(c), etc but there should be only one caption for the whole
table, not separate ones for each part.

In the preprint style the tables may be included in the text
or listed separately after the reference list
starting on a new page.

\subsection{The basic table format}
The standard form for a table is:
\begin{verbatim}
\begin{table}
\caption{Table caption.}
\begin{indented}
\item[]\begin{tabular}{@{}llll}
\br
Head 1&Head 2&Head 3&Head 4\\
\mr
1.1&1.2&1.3&1.4\\
2.1&2.2&2.3&2.4\\
\br
\end{tabular}
\end{indented}
\end{table}
\end{verbatim}

Points to note are:
\begin{enumerate}
\item The caption comes before the table. It should have a full stop at
the end.

\item Tables are normally set in a smaller type than the text.
The normal style is for tables to be indented in the same way as
equations. This is accomplished
by using \verb"\begin{indented}" \dots\ \verb"\end{indented}"
and putting \verb"\item[]" before the start of the tabular environment.
Omit these
commands for any tables which will not fit on the page when indented
and add \verb"\footnotesize\rm" immediately after the caption.

\item The default alignment of columns should be aligned left and
adding \verb"@{}" omits the extra space before the first column.

\item Tables have only horizontal rules and no vertical ones. The rules at
the top and bottom are thicker than internal rules and are set with
\verb"\br" (bold rule).
The rule separating the headings from the entries is set with
\verb"\mr" (medium rule).

\item Numbers in columns should be aligned on the decimal point;
to help do this a control sequence \verb"\lineup" has been defined
which sets \verb"\0" equal to a space the size of a digit, \verb"\m"
to be a space the width of a minus sign, and \verb"\-" to be a left
overlapping minus sign. \verb"\-" is for use in text mode while the other
two commands may be used in maths or text.
(\verb"\lineup" should only be used within a table
environment after the caption so that \verb"\-" has its normal meaning
elsewhere.) See table~\ref{tabone} for an example of a table where
\verb"\lineup" has been used.
\end{enumerate}

\begin{table}
\caption{A simple example produced using the standard table commands
and $\backslash${\tt lineup} to assist in aligning columns on the
decimal point. The width of the
table and rules is set automatically by the
preamble.\label{tabone}}

\begin{indented}
\lineup
\item[]\begin{tabular}{@{}*{7}{l}}
\br
$\0\0A$&$B$&$C$&\m$D$&\m$E$&$F$&$\0G$\cr
\mr
\0\023.5&60  &0.53&$-20.2$&$-0.22$ &\01.7&\014.5\cr
\0\039.7&\-60&0.74&$-51.9$&$-0.208$&47.2 &146\cr
\0123.7 &\00 &0.75&$-57.2$&\m---   &---  &---\cr
3241.56 &60  &0.60&$-48.1$&$-0.29$ &41   &\015\cr
\br
\end{tabular}
\end{indented}
\end{table}

\subsection{Simplified coding and extra features for tables}
The basic coding format can be simplified using extra commands in
the \verb"ioplppt" style. The first four lines of the example above
can be replaced by
\begin{verbatim}
\Table{Table caption}
\end{verbatim}
and this also activates the definitions within \verb"lineup".
The final three lines can also be reduced to \verb"\endTable" or
\verb"\endtab". Similarly for a table which does not fit in when indented
\verb"\fulltable{caption}" \dots\ \verb"\endfulltable" or \verb"\endtab"
can be used. \LaTeX\ optional positional parameters can be added after
\verb"\Table{caption}" and \verb"\fulltable{caption}" if desired.


\verb"\centre{#1}{#2}" can be used to centre a heading
\verb"#2" over \verb"#1"
columns and \verb"\crule{#1}" puts a rule across
\verb"#1" columns. A negative
space \verb"\ns" is usually useful to reduce the space between a centred
heading and a centred rule. \verb"\ns" should occur immediately after the
\verb"\\" of the row containing the centred heading (see code for
\tref{tabl3}). A small space can be
inserted between rows of the table
with \verb"\ms" and a half line space with \verb"\bs"
(both must follow a \verb"\\" but should not have a
\verb"\\" following them).

\Table{A table with headings spanning two columns and containing notes.
To improve the
visual effect a negative skip ($\backslash${\tt ns})
has been put in between the lines of the
headings. Commands set-up by $\backslash${\tt lineup} are used to aid
alignment in columns. $\backslash${\tt lineup} is defined within
the $\backslash${\tt Table} definition.\label{tabl3}}
\br
&&&\centre{2}{Separation energies}\\
\ns
&Thickness&&\crule{2}\\
Nucleus&(mg cm$^{-2}$)&Composition&$\gamma$, n (MeV)&$\gamma$, 2n (MeV)\\
\mr
$^{181}$Ta&$19.3\0\pm 0.1^{\rm a}$&Natural&7.6&14.2\\
$^{208}$Pb&$\03.8\0\pm 0.8^{\rm b}$&99\%\ enriched&7.4&14.1\\
$^{209}$Bi&$\02.86\pm 0.01^{\rm b}$&Natural&7.5&14.4\\
\br
\tabnotes
\item[] $^{\rm a}$ Self-supporting.
\item[] $^{\rm b}$ Deposited over Al backing.
\endtabnotes

Units should not normally be given within the body of a table but
given in brackets following the column heading; however, they can be
included in the caption for long column headings or complicated units.
Where possible tables should not be broken over pages.
If a table has related notes these should appear directly below the table
rather than at the bottom of the page. Notes can be designated with
footnote symbols (preferable when there are only a few notes) or
superscripted small roman letters. The notes are set to the same width as
the table and in normal tables follow after \verb"\end{tabular}", each
note preceded by \verb"\item[]". For a full width table \verb"\noindent"
should precede the note rather than \verb"\item[]". To simplify the coding
\verb"\tabnotes" can, if desired, replace \verb"\end{tabular}" and
\verb"\endtabnotes" replaces
\verb"\end{indented}\end{table}".

If all the tables are grouped at the end of a document
the command \verb"\Tables" is used to start a new page and
set a heading `Tables and table captions'.

\section{Figures and figure captions}
We do not yet have the facilities for handling figures electronically
other than those generated within the \LaTeX\ picture environment
or Pic\TeX.
Authors should send the normal fair copies of their figures with their
submission and attached lettered copies of them to the back of the
printed version. The fair copies should be in black
Indian ink or printing on tracing paper, plastic or white card or
paper,
or glossy photographs.

Each figure should have a brief caption describing it and, if
necessary, interpreting the various lines and symbols on the figure.
As much lettering as possible should be removed from the figure and
included in the caption. If a figure has parts, these should be
labelled ($a$), ($b$), ($c$), etc.

Unless \LaTeX\ figures are used the captions should not be
included with the text but listed at the end of the article.
The command \verb"\Figures" starts a new page and an unnumbered section
with the heading `Figure captions'.
The captions should then be set with the commands:
\begin{verbatim}
\begin{figure}
\caption{Figure caption.}
\end{figure}
\end{verbatim}
or more simply
\begin{verbatim}
\Figure{Figure caption.}
\end{verbatim}
The caption should finish with a full stop and the printed version will be
indented as in Institute of Physics Publishing single-column journals.
\Tref{blobs} gives the definitions for describing symbols and lines often
used within figure captions (more symbols are defined using characters
from the AMS extension fonts in \verb"iopfts.sty").

\begin{table}
\caption{Control sequences to describe lines and symbols in figure
captions.\label{blobs}}
\begin{indented}
\item[]\begin{tabular}{@{}lllll}
\br
Control sequence&Output&&Control sequence&Output\\
\mr
\verb"\dotted"&\dotted        &&\verb"\opencirc"&\opencirc    \\
\verb"\dashed"&\dashed        &&\verb"\fullcirc"&\fullcirc    \\
\verb"\broken"&\broken        &&\verb"\opensqr"&\opensqr      \\
\verb"\longbroken"&\longbroken&&\verb"\fullsqr"&\fullsqr      \\
\verb"\chain"&\chain          &&\verb"$\triangle$"&$\triangle$\\
\verb"\dashddot"&\dashddot    &&\verb"$\Diamond$"&$\Diamond$  \\
\verb"\full"&\full            &&                              \\
\br
\end{tabular}
\end{indented}
\end{table}

\clearpage

\appendix
\section{List of macros for formatting text, figures and tables}


\begin{table}
\caption{Macros available for use in text. Parameters in square brackets
are optional.}
\footnotesize\rm
\begin{tabular}{@{}*{7}{l}}
\br
Macro name&Purpose\\
\mr
\verb"\title{#1}[#2]"&Title of article and short title\\
\verb"\paper{#1}[#2]"&Title of paper and short title\\
\verb"\letter{#1}"&Title of Letter to the Editor\\
\verb"\comment{#1}[#2]"&Title of Comment and short title\\
\verb"\topical{#1}[#2]"&Title of Topical Review and short title\\
\verb"\review{#1}[#2]"&Title of review article\\
\verb"\note{#1}[#2]"&Title of Note and short title\\
\verb"\prelim{#1}[#2]"&Title of Preliminary Communication \& short title\\
\verb"\author{#1}"&List of all authors\\
\verb"\article{#1}{#2}[#3]"&Type and title of other articles and
short title\\
\verb"\address{#1}"&Address of author\\
\verb"\pacs{#1}"&PACS classification codes\\
\verb"\pacno{#1}"&Single PACS classification code\\
\verb"\ams{#1}"&American Mathematical Society classification code\\
\verb"\jl{#1}"&Number of journal article submitted to\\
\verb"\submitted"&Submitted to message\\
\verb"\maketitle"&Creates title page\\
\verb"\begin{abstract}"&Start of abstract\\
\verb"\end{abstract}"&End of abstract\\
\verb"\nosections"&Space before text when no sections\\
\verb"\section{#1}"&Section heading\\
\verb"\subsection{#1}"&Subsection heading\\
\verb"\subsubsection{#1}"&Subsubsection heading\\
\verb"\appendix"&Start of appendixes\\
\verb"\ack"&Acknowledgments heading\\
\verb"\References"&Heading for reference list\\
\verb"\begin{harvard}"&Start of alphabetic reference list\\
\verb"\end{harvard}"&End of alphabetic reference list\\
\verb"\begin{thebibliography}{#1}"&Start of numeric reference list\\
\verb"\end{thebibliography}{#1}"&End of numeric reference list\\
\verb"\etal"&\etal for text and reference lists\\
\verb"\dash"&Rule for repeated authors in alphabetical reference list\\
\verb"\nonum"&Unnumbered entry in numerical reference list\\
\br
\end{tabular}
\end{table}


\Table{Macros defined within {\tt ioplppt.sty}
for use with figures and tables.}
\br
Macro name&Purpose\\
\mr
\verb"\Figures"&Heading for list of figure captions\\
\verb"\Figure{#1}"&Figure caption\\
\verb"\Tables"&Heading for tables and table captions\\
\verb"\Table{#1}"&Table caption\\
\verb"\fulltable{#1}"&Table caption for full width table\\
\verb"\endTable"&End of table created with \verb"\Table"\\
\verb"\endfulltab"&End of table created with \verb"\fulltable"\\
\verb"\endtab"&End of table\\
\verb"\br"&Bold rule for tables\\
\verb"\mr"&Medium rule for tables\\
\verb"\ns"&Small negative space for use in table\\
\verb"\centre{#1}{#2}"&Centre heading over columns\\
\verb"\crule{#1}"&Centre rule over columns\\
\verb"\lineup"&Set macros for alignment in columns\\
\verb"\m"&Space equal to width of minus sign\\
\verb"\-"&Left overhanging minus sign\\
\verb"\0"&Space equal to width of a digit\\
\br
\endtab


\clearpage
\section{Control sequences for journal
abbreviations\label{jlabs}}



\begin{table}
\caption{Abbreviations for the IOP journals.}
\begin{indented}
\item[]
\begin{tabular}{@{}lll}
\br
Macro name&{\rm Short form of journal title}&Years relevant\\
\mr
\verb"\CQG"&Class. Quantum Grav.\\
\verb"\IP"&Inverse Problems\\
\verb"\JPA"&J. Phys. A: Math. Gen.\\
\verb"\JPB"&J. Phys. B: At. Mol. Phys.&1968--1987\\
\verb"\jpb"&J. Phys. B: At. Mol. Opt. Phys.&1988 and onwards\\
\verb"\JPC"&J. Phys. C: Solid State Phys.&1968--1988\\
\verb"\JPCM"&J. Phys: Condens. Matter&1989 and onwards\\
\verb"\JPD"&J. Phys. D: Appl. Phys.\\
\verb"\JPE"&J. Phys. E: Sci. Instrum.&1968--1989\\
\verb"\JPF"&J. Phys. F: Met. Phys.\\
\verb"\JPG"&J. Phys. G: Nucl. Phys.&1975--1988\\
\verb"\jpg"&J. Phys. G: Nucl. Part. Phys.&1989 and onwards\\
\verb"\MSMSE"&Modelling Simul. Mater. Sci. Eng.\\
\verb"\MST"&Meas. Sci. Technol.&1990 and onwards\\
\verb"\NET"&Network\\
\verb"\NL"&Nonlinearity\\
\verb"\NT"&Nanotechnology\\
\verb"\PAO"&Pure and Applied Optics\\
\verb"\PMB"&Phys. Med. Biol.\\
\verb"\PSST"&Plasma Sources Sci. Technol.\\
\verb"\QO"&Quantum Opt.\\
\verb"\RPP"&Rep. Prog. Phys.\\
\verb"\SST"&Semicond. Sci. Technol.\\
\verb"\SUST"&Supercond. Sci. Technol.\\
\verb"\WRM"&Waves in Random Media\\
\br
\end{tabular}
\end{indented}
\end{table}



\begin{table}
\caption{Abbreviations for some more common non-IOP journals.}
\begin{indented}
\item[]\begin{tabular}{@{}ll}
\br
Macro name&{\rm Short form of journal}\\
\mr
\verb"\AC"&Acta Crystallogr.\\
\verb"\AM"&Acta Metall.\\
\verb"\AP"&Ann. Phys., Lpz\\
\verb"\APNY"&Ann. Phys., NY\\
\verb"\APP"&Ann. Phys., Paris\\
\verb"\CJP"&Can. J. Phys.\\
\verb"\GRG"&Gen. Rel. Grav.\\
\verb"\JAP"&J. Appl. Phys.\\
\verb"\JCP"&J. Chem. Phys.\\
\verb"\JJAP"&Japan. J. Appl. Phys.\\
\verb"\JMMM"&J. Magn. Magn. Mater.\\
\verb"\JMP"&J. Math. Phys.\\
\verb"\JOSA"&J. Opt. Soc. Am.\\
\verb"\JP"&J. Physique\\
\verb"\JPhCh"&J. Phys. Chem.\\
\verb"\JPSJ"&J. Phys. Soc. Japan\\
\verb"\JQSRT"&J. Quant. Spectrosc. Radiat. Transfer\\
\verb"\NC"&Nuovo Cimento\\
\verb"\NIM"&Nucl. Instrum. Methods\\
\verb"\NP"&Nucl. Phys.\\
\verb"\PF"&Phys. Fluids\\
\verb"\PL"&Phys. Lett.\\
\verb"\PR"&Phys. Rev.\\
\verb"\PRL"&Phys. Rev. Lett.\\
\verb"\PRS"&Proc. R. Soc.\\
\verb"\PS"&Phys. Scr.\\
\verb"\PSS"&Phys. Status Solidi\\
\verb"\PTRS"&Phil. Trans. R. Soc.\\
\verb"\RMP"&Rev. Mod. Phys.\\
\verb"\RSI"&Rev. Sci. Instrum.\\
\verb"\SSC"&Solid State Commun.\\
\verb"\SPJ"&Sov. Phys.--JETP\\
\verb"\ZP"&Z. Phys.\\
\br
\end{tabular}
\end{indented}
\end{table}

\end{document}

%
%
% end ioplau.tex

