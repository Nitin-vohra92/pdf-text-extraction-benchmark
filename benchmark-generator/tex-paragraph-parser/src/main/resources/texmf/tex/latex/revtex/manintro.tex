%%% ======================================================================
%%%  @LaTeX-file{
%%%     filename        = "manintro.tex",
%%%     version         = "3.0",
%%%     date            = "November 10, 1992",
%%%     ISO-date        = "1992.11.10",
%%%     time            = "15:41:54.18 EST",
%%%     author          = "American Physical Society,
%%%                        and
%%%                        American Institute of Physics,
%%%                        and
%%%                        Optical Society of America",
%%%     contact         = "Christopher B. Hamlin",
%%%     address         = "APS Publications Liaison Office
%%%                        500 Sunnyside Blvd.
%%%                        Woodbury, NY 11797",
%%%     telephone       = "(516) 576-2390",
%%%     FAX             = "(516) 349-7817",
%%%     email           = "mis@aps.org (Internet)",
%%%     supported       = "yes",
%%%     archived        = "pinet.aip.org/pub/revtex,
%%%                        Niord.SHSU.edu:[FILESERV.REVTEX]",
%%%     keywords        = "REVTeX, version 3.0, REVTeX Input Guide,
%%%                        Introduction",
%%%     codetable       = "ISO/ASCII",
%%%     checksum        = "61923 433 2196 16404",
%%%     docstring       = "This is the introduction to the input guide
%%%                        for REVTeX 3.0.
%%%
%%%                        The checksum field above contains a CRC-16
%%%                        checksum as the first value, followed by the
%%%                        equivalent of the standard UNIX wc (word
%%%                        count) utility output of lines, words, and
%%%                        characters.  This is produced by Robert
%%%                        Solovay's checksum utility."
%%% }
%%% ======================================================================
%
%   This file is part of the files in the REVTeX 3.0 distribution.
%   Version 3.0 of REVTeX, November 10, 1992.
%
%   Copyright (c) 1992 American Physical Society, Optical Society of America,
%   American Institute of Physics.
%
%   See the REVTeX 3.0 README file for restrictions and more information.
%
%
\documentstyle[aps]{revtex}
\def\REVTeX{REV\TeX}
\begin{document}
\flushbottom
% run page numbers by "chapter"
\def\thepage{\roman{page}}
\title{\vspace*{1.5in}THE \REVTeX{} INPUT GUIDE}
\author{A joint effort of\\The American Physical Society\\The Optical
Society of America\\The American Institute of Physics}

\address{Woodbury, New York/Washington, DC\\ November 10, 1992}

\maketitle

\makeatletter
\global\@specialpagefalse
\def\@oddhead{\REVTeX{} 3.0\hfill Released November 10, 1992}
\let\@evenhead\@oddhead
% run page numbers by "chapter", with copyright for first page
\def\@oddfoot{\reset@font\rm\hfill \thepage \hfill
\ifnum\c@page=1
  \llap{\protect\copyright{}~1992~~%
  $\vcenter{\baselineskip10pt
    \hbox{American Physical Society}
    \hbox{Optical Society of America}
    \hbox{American Institute of Physics}
  }$}%
\fi
} \let\@evenfoot\@oddfoot
\makeatother

\newpage


\vspace*{.4in}

\section*{Table of contents}
{ \leftskip1in\rightskip\leftskip


\vspace{.4in}
{\parindent0pt\parfillskip0pt\baselineskip2\baselineskip

Preface\leaders\hbox to1em{\hfil.\hfil}\hfill iii

Introduction \leaders\hbox to1em{\hfil.\hfil}\hfill iv--vi

\REVTeX{} Information for APS Authors\leaders\hbox to1em{\hfil.\hfil}\hfill
1-1--1-19

\REVTeX{} Information for OSA Authors\leaders\hbox to1em{\hfil.\hfil}\hfill
2-1--2-37

\REVTeX{} Information for AIP Authors\leaders\hbox to1em{\hfil.\hfil}\hfill
3-1--3-4

Appendix A: Character Set Listing\leaders\hbox to1em{\hfil.\hfil}\hfill
A1--A4

Appendix B: Command List\leaders\hbox to1em{\hfil.\hfil}\hfill
B1--B3

\vspace{.4in}

}

This input guide is for \REVTeX{} 3.0. The README file for \REVTeX{}
3.0 should be consulted before the \REVTeX{} macros are used. Important
copyright information is contained in the README file.

\vspace{\baselineskip}

This table of contents is for the {\em \REVTeX{} Input Guide\/} as a whole.
Individual society chapters of the guide may include their own tables of
contents. The sections of the guide are produced by the following files:

Preface and Introduction: manintro.tex.

\REVTeX{} Information for APS Authors: manaps.tex.

\REVTeX{} Information for OSA Authors: manosa.tex.

\REVTeX{} Information for AIP Authors: manaip.tex.

Appendixes A and B: manend.tex.

\vfill

\begin{center}
\leftskip0pt plus1fill \rightskip\leftskip
\TeX{} is a trademark of the American Mathematical Society.\par
\end{center}

}

\newpage

{\leftskip1in\rightskip\leftskip

\section*{PREFACE}

      1986: In response to author requests and market research, the
American Physical Society (APS) launched a new compuscript development
effort. The focus was on the \TeX\ typesetting program, and the goal was to
expand the capabilities of the Society's  electronic submission program to
include \TeX-prepared compuscripts, using those submissions to produce
final pages for Physical Review.

      Evaluating the available \TeX-based tools, including LaTeX, APS
recognized that the \TeX\ physics community did not have all the necessary
tools to create a physics manuscript.  At that point the APS staff designed
the macro package and style guide called \REVTeX{} to help meet these
requirements and standardize compuscript submission files.

   The idea for a \REVTeX{} macro set was born under the following premises.

   1. Ease of reformatting.  Files were to be reformatted from preprints to
   final pages.  This reformatting would have to be easily accomplished with
   minimal modification to the file.

   2. Ease of use.  If possible, build on the knowledge and documentation
   already available in the community.  Also, the package must provide for
   any and all acceptable styling situations.  In this way authors would be
   encouraged to follow the rules in manuscript preparation.

   3. Built-in allowance for growth.  The macro set must easily allow for
   more journal formats.

   4. Electronic submissions.  The macro set must contain preprint styles to
   facilitate original manuscript submissions to editorial offices.

   5. Author compliance.  Develop procedures to encourage author compliance,
   thereby minimizing handling of compuscripts, resulting in decreased
   proofreading for both the author and staff.

   ``\REVTeX{},'' so named for the {\em Physical Review} journals, was
     released in 1988.  Version 2 responded to author feedback and
     also addressed certain in-house production concerns.
     Version 2 was released in March 1990.

      The Optical Society of America (OSA), the American Astronomical
Society (AAS), and the American Institute of Physics (AIP) expressed
interest in this project and were kept informed of progress.  The APS has
enjoyed continued success with their compuscript program, and as of July
1992 were receiving 20\% of their published pages as compuscripts.




\subsection*{The Joint Society Task Force}

      In 1990 the American Institute of Physics and a number of its member
Societies organized a Joint Society Task Force (JSTF) on Electronic
Publishing for the purpose of long-range strategic planning on electronic
communication and publishing.  The Executive Directors and Publishing
Directors meet regularly to discuss common directions and goals.  One of
these goals involves the establishment of a standard tool for compuscript
submission.

      As a result, a \REVTeX{} Working Group was established to explore
the issues surrounding the use of \REVTeX{} as this standard.  Under the
leadership of Janice Fleming, Chairperson (OSA), and with representatives
from each organization, the group worked to resolve issues and to ensure
that \REVTeX{} version 3.0 would meet the needs of the participating
organizations.

\newpage


}


\twocolumn[ \section*{THE \REVTeX{} INPUT GUIDE: INTRODUCTION}\vspace*{4pt}]


\section{What Is \REVTeX{}?}

\REVTeX{} is an electronic publishing product of the American Physical
Society.  It is a set of \LaTeX-based tags that can be used to prepare a
physics manuscript for submission. If \REVTeX{} tags are used consistently
throughout a manuscript and published guidelines are followed exactly, it
is hoped that the \TeX\ manuscript file (called a {\it compuscript}) can be
used by APS, OSA, AIP, and other participating member societies in
production of  author proofs.

\REVTeX{} is physically several {\it style files}. \LaTeX\ accesses these
style files in order to recognize allowable tags and to get information
about how these tags should be interpreted for a specific document. These
\REVTeX{} style files have been developed by the aforementioned groups to
meet individual society needs as well as the needs of the users.

\section{What do you need in order to use \REVTeX{}?}

You need \TeX\ and \LaTeX\ to use \REVTeX{}. \REVTeX's software and
hardware requirements are especially significant.  It is the \TeX\ program
and the \LaTeX\ macros that really define \REVTeX's needs.

\subsection{Software requirements for all computers}

\begin{enumerate}

\item The \REVTeX{} style files.

\item \TeX\ program (comes with Computer Modern fonts; must have \TeX\
version that is compatible with your version of operating system).

\item \LaTeX\ macros.

\item A text editor or word processor (you'll be happier with one that is
quick and easy to get in/out of and saves files as ASCII text as a
default).

\item A printer {\it driver} (software needed to print).

\item Documentation for \REVTeX{} and \LaTeX.
\end{enumerate}

\subsection{Hardware requirements}

\TeX\ can be run on a variety of computers and operating systems.  Users
must obtain the proper version of \TeX\ for their computer and operating
system; in other words, it is not possible to take a colleague's
\TeX\ program that runs on a PC and have that run on a VAX.

\subsubsection{For the personal computer}

The most widely used computer for manuscript preparation in the physics
community is the personal computer. Implementations of \TeX\ vary in
their requirements.   Below is a representative hardware configuration
for a PC.

\begin{itemize}
\item A hard disk (will need 10--20 megabytes space).

\item A CGA, EGA, VGA, Super VGA, or Hercules monitor.

\item A PC/XT/AT or compatible with a minimum of 640K RAM.

\item A printer (lasers give the best output, but even some dot matrix
printers are fine).
\end{itemize}

Your \TeX\ vendor or \TeX\ provider will provide documentation
that specifies exact hardware requirements.

\subsubsection{For other computers}

Needs should be reviewed with \TeX\ site coordinators.  These
are people that the \TeX\ User's Group ``contracts'' with to distribute
\TeX\ software and to keep software current. Coordinators are not
available to help install software or debug.

Call the \TeX\ User's Group for a referral to the proper coordinator.


\section{How do you get what you need?}

For current users of \TeX , \REVTeX{} can be obtained by contacting
representatives listed in the society-specific chapters that
follow this Introduction.

If you are not yet a \TeX\ user, the other software listed above must be
obtained from different sources, depending on the hardware that will be
used. As noted above, the \TeX\ User's Group has established Site
Coordinators who distribute \TeX\ for the various computers.  Software
obtained in this manner will be very inexpensive, but will require some
computer aptitude to install.  TUG can also advise on commercial sources
for the software. These companies or consultants charge a fee for assisting
with installation and/or providing technical assistance.

\section{How does \REVTeX{} benefit authors?}
\begin{itemize}
\item \REVTeX{} provides all the elements needed for preparation of a
physics manuscript.   Users will not need to develop special tags (macros)
to meet their needs.

\item \REVTeX{} formats are designed to be acceptable for manuscript
submission.  Users will not need to be concerned about proper format for
editorial offices (double spacing, margin requirements, etc.).

\item \REVTeX{} macros accommodate many presubmission distribution needs.
For example, users can assign preprint numbers to manuscripts  and can
easily change to single-spaced copy to save paper before submission to
editorial offices.

\item \REVTeX{} macros are recognized by numerous physics organizations as
a \TeX\ standard for manuscript preparation.

\item \REVTeX{} compuscript files can be used by a variety of publishers to
create author proofs.  Depending on the composition method used and the
consistency of tagging, this can result in less proofreading for the
author, accelerated production schedules, and/or reduced cost-per-page.
\end{itemize}

\section{How to get started?}


\begin{itemize}
\item Get \TeX\ and \LaTeX\ operational. Become familiar with creating
simple \TeX\ and \LaTeX\ documents with your screen editor or word
processor.  Learn how to print \TeX\ documents and/or how to view them on
the screen.

\item Get the compuscript toolbox.  This contains the following:

  General files: README, revtex.sty, manintro.tex (this file), and manend.tex.

  Society specific files: Listed in society-specific chapters of this
  document.

\item {\it Read\/} the README file thoroughly!

\item Follow installation instructions in README file.

\item Read this document thoroughly. Scan the appendixes; they are provided
for reference purposes, but if you look at them now you'll know where to
find the information later.

\penalty-10000

\item Read thoroughly the society-specific chapter of this document that
is of the most interest to you,  but try to at least scan {\it all\/} the
society-specific  chapters.

\item Now focus on one society-specific chapter. Copy the template file for
that society into a junk file for some experimenting.

\item Edit the junk file to create an experimental manuscript. Maybe try to
prepare one of the sample documents that have been provided; this way, if
you get ``stuck,'' you can consult the documentation for help.

\item  Try to create a new document.  If you are a new \TeX\ user, you may
want to try to recreate an old document rather than a current one.

\item When you have a document that is ready for submission to the
editorial office of interest, review the policies for submission in the
society-specific chapter.   Individual editorial offices have differing
requirements for when you should provide the \REVTeX{} file.  For example,
most editorial offices currently require  that hard copy (paper copy) be
provided for original submission, but APS offices encourage original
manuscript submissions to be electronic. See the Reminder below.

\item Reminder:  The society-specific chapters are provided because each
organization has different policies and processing procedures for
compuscripts. Please make sure that you review the appropriate chapter even
if and when you are an expert \REVTeX{} user!
\end{itemize}
\section{Can you create your own macros?}

The use of author-created macros is strongly discouraged. The \REVTeX{}
macro set is intended to represent completely the tagging needs for a
physics manuscript. Non-standard tags will probably disqualify a manuscript
file from a compuscript program because most journal compositors cannot
translate them cost-efficiently.

This point frequently requires further clarification: usually it is not
technically impossible to use or to translate author-created macros.
However,

\begin{quote}
{\bf the time and effort required to massage compuscript files that do not
follow standard formatting guidelines will eliminate any cost savings
and/or time savings in production, and will eliminate any proofreading time
savings for the author.}
\end{quote}


Therefore, authors who use custom macros to save keystrokes in preparing
their manuscript should remove them before generating the manuscript output
for submission.  Most screen editors and word processors have tools that
will facilitate this.


\section{What else can I read?}

The following are the standard references, by the authors of \LaTeX{} and
\TeX{}, respectively.


\begin{references}
\bibitem{latexbook}Leslie Lamport, {\it \LaTeX: A Document Preparation
System} (Addison-Wesley, Reading, MA, 1986).
\bibitem{texbook}Donald Knuth, {\it The \TeX book} (Addison-Wesley,
Reading, MA, 1984).
\end{references}
\end{document}
