% This is cp-aa.ini, the INITEX file to create a format file for
% the Springer journal Astronomy & Astrophysics, based on plain TeX,
% using Computer Modern fonts                           version 3.0
%                                (earlier versions were called aa.cmm)
%%%%%%%%%%%%%%%%%%%%%%%%%%%%%%%%%%%%%%%%%%%%%%%%%%%%%%%%%%%%%%%%%%%%%%
\catcode`\{=1 % left brace is begin-group character
\catcode`\}=2 % right brace is end-group character
\catcode`\#=6 % hash mark is macro parameter character
%%% removed from cp-aa.ini:
%\let\x=\input \def\input#1 {\let\input=\x \let\x=\undefined}
%
\let\qayb=\font
\def\font#1=#2 {\def\qayba{#1}\def\qaybb{#2}
                \def\qaybf{\preloaded}
                \def\qaybg{\tenrm}\def\qaybh{\tenex}
                \def\qaybu{cmr7}\def\qaybv{cmtt10}\def\qaybw{cmssbx10}
                \ifx\qayba\qaybg \qayb#1=#2
                \else\ifx\qayba\qaybh \qayb#1=#2
                \else\ifx\qayba\qaybf
                  \ifx\qaybb\qaybu \qayb#1=#2
                  \else\ifx\qaybb\qaybv \qayb#1=#2
                  \else\ifx\qaybb\qaybw \qayb#1=#2 \fi\fi\fi
                \fi\fi\fi}
%
\let\qayc=\skewchar \def\skewchar#1=#2 {\relax}
\let\qayd=\textfont \def\textfont#1=#2{\relax}
\let\qaye=\scriptfont \def\scriptfont#1=#2{\relax}
\let\qayf=\scriptscriptfont \def\scriptscriptfont#1=#2{\relax}
%
%\x plain                                        %%% removed from cp-aa.ini
%
\let\font=\qayb \let\qayb=\undefined
\let\skewchar=\qayc  \let\qayc=\undefined
\let\textfont=\qayd  \let\qayd=\undefined
\let\scriptfont=\qaye  \let\qaye=\undefined
\let\scriptscriptfont=\qayf  \let\qayf=\undefined
%
\def\fonttype{cm}              
%\input hyphen                                   %%% removed from cp-aa.ini
\def\Initexing{}               
\font\sevenrm=cmr7
\font\fiverm=cmr5
\font\teni=cmmi10 % math italic
\font\seveni=cmmi7
\font\fivei=cmmi5
\font\tensy=cmsy10 % math symbols
\font\sevensy=cmsy7
\font\fivesy=cmsy5
\font\tenbf=cmbx10 % boldface extended
\font\sevenbf=cmbx7
\font\fivebf=cmbx5
\font\tentt=cmtt10 % typewriter
\font\tenit=cmti10 % text italic

\skewchar\teni='177 \skewchar\seveni='177 \skewchar\fivei='177
\skewchar\tensy='60 \skewchar\sevensy='60 \skewchar\fivesy='60

\textfont0=\tenrm \scriptfont0=\sevenrm \scriptscriptfont0=\fiverm
\textfont1=\teni \scriptfont1=\seveni \scriptscriptfont1=\fivei
\textfont2=\tensy \scriptfont2=\sevensy \scriptscriptfont2=\fivesy
\textfont3=\tenex \scriptfont3=\tenex \scriptscriptfont3=\tenex
\textfont\itfam=\tenit
\textfont\bffam=\tenbf \scriptfont\bffam=\sevenbf
\scriptscriptfont\bffam=\fivebf
\textfont\ttfam=\tentt
\rm
% This is values.aa
% i.e. Values of package AA
% it contains sizes and measures specific for this macro package
% indention of equations
\newskip\mathindent      \mathindent=0pt
% \titlea
\newskip\tabefore \tabefore=20pt plus 10pt minus 5pt      % space above
\newskip\taafter  \taafter=10pt                           % space below
\newskip\tabaselineskip
\tabaselineskip=20pt
% \titleb
\newskip\tbbeforeback    \tbbeforeback=-20pt              % corrective space to a \titlea
\newskip\tbbefore        \tbbefore=17pt plus 7pt minus3pt % spaceabove
\newskip\tbafter         \tbafter=8pt                     % space below
\newskip\tbbaselineskip
\tbbaselineskip=17pt
% \titlec
\newskip\tcbeforeback    \tcbeforeback=-3pt               % corrective space to a \titleb
\advance\tcbeforeback by -10pt                            % corrective space to a \titleb
\newskip\tcbefore        \tcbefore=10pt plus 5pt minus 1pt% space above
\newskip\tcafter         \tcafter=6pt                     % space below
% \titled
\newskip\tdbeforeback    \tdbeforeback=-3pt                  % corrective space to a \titlec
\advance\tdbeforeback by -10pt                               % corrective space to a \titlec
\newskip\tdbefore        \tdbefore=10pt plus 4pt minus 1pt   % space above
% \petit
\newskip\petitsurround
\petitsurround=6pt\relax
% \ack
\newskip\ackbefore      \ackbefore=10pt plus 5pt             % space above
\newskip\ackafter       \ackafter=6pt                        % space below
% indention of lists
\newdimen\itemindent    \newdimen\itemitemindent
\itemindent=1.5em       \itemitemindent=2\itemindent
% This is cmlayout.tex
% it sets up some measures and defines macros needed by the font
% selection
\catcode`@=11    % use @ as a normal character
\normallineskip=1pt
\normallineskiplimit=0pt
\newskip\ttglue%
\def\ifundefin@d#1#2{%
\expandafter\ifx\csname#1#2\endcsname\relax}
%
\def\getf@nt#1#2#3#4{%
\ifundefin@d{#1}{#2}%
\global\expandafter\font\csname#1#2\endcsname=#3#4%
\fi\relax
}
\newfam\sffam
\newfam\scfam
\def\makesize#1#2#3#4#5#6#7{%
%%% these fonts cannot be loaded dynamically, because there
%%% is no macro we can overload
 \getf@nt{rm}{#1}{cmr}{#2}%
 \getf@nt{rm}{#3}{cmr}{#4}%
 \getf@nt{rm}{#5}{cmr}{#6}%
 \getf@nt{mi}{#1}{cmmi}{#2}%
 \getf@nt{mi}{#3}{cmmi}{#4}%
 \getf@nt{mi}{#5}{cmmi}{#6}%
 \getf@nt{sy}{#1}{cmsy}{#2}%
 \getf@nt{sy}{#3}{cmsy}{#4}%
 \getf@nt{sy}{#5}{cmsy}{#6}%

%%% skewchar
 \skewchar\csname mi#1\endcsname ='177
 \skewchar\csname mi#3\endcsname ='177
 \skewchar\csname mi#5\endcsname ='177
 \skewchar\csname sy#1\endcsname ='60
 \skewchar\csname sy#3\endcsname ='60
 \skewchar\csname sy#5\endcsname ='60

\expandafter\def\csname#1size\endcsname{%
 \normalbaselineskip=#7
 \normalbaselines
 \setbox\strutbox=\hbox{\vrule height0.75\normalbaselineskip%
    depth0.25\normalbaselineskip width0pt}%
 %
%%% "roman"
 \textfont0=\csname rm#1\endcsname
 \scriptfont0=\csname rm#3\endcsname
 \scriptscriptfont0=\csname rm#5\endcsname
    \def\oldstyle{\fam1\csname mi#1\endcsname}%
%%% "mathit"
 \textfont1=\csname mi#1\endcsname
 \scriptfont1=\csname mi#3\endcsname
 \scriptscriptfont1=\csname mi#5\endcsname
%%% "mathsy"
 \textfont2=\csname sy#1\endcsname
 \scriptfont2=\csname sy#3\endcsname
 \scriptscriptfont2=\csname sy#5\endcsname
%%% "mathex"
 \textfont3=\tenex\scriptfont3=\tenex\scriptscriptfont3=\tenex
   \def\rm{%
 \fam0\csname rm#1\endcsname%
   }%
   \def\it{%
 \getf@nt{it}{#1}{cmti}{#2}%
 \textfont\itfam=\csname it#1\endcsname
 \fam\itfam\csname it#1\endcsname
   }%
   \def\sl{%
 \getf@nt{sl}{#1}{cmsl}{#2}%
 \textfont\slfam=\csname sl#1\endcsname
 \fam\slfam\csname sl#1\endcsname}%
   \def\bf{%
 \getf@nt{bf}{#1}{cmbx}{#2}%
 \getf@nt{bf}{#3}{cmbx}{#4}%
 \getf@nt{bf}{#5}{cmbx}{#6}%
 \textfont\bffam=\csname bf#1\endcsname
 \scriptfont\bffam=\csname bf#3\endcsname
 \scriptscriptfont\bffam=\csname bf#5\endcsname
 \fam\bffam\csname bf#1\endcsname}%
   \def\tt{%
 \getf@nt{tt}{#1}{cmtt}{#2}%
 \textfont\ttfam=\csname tt#1\endcsname
 \fam\ttfam\csname tt#1\endcsname
 \ttglue=.5em plus.25em minus.15em
   }%
  \def\sf{%
\getf@nt{sf}{#1}{cmss}{10 at #2pt}%
\textfont\sffam=\csname sf#1\endcsname
\fam\sffam\csname sf#1\endcsname}%
   \def\sc{%
 \getf@nt{sc}{#1}{cmcsc}{10 at #2pt}%
 \textfont\scfam=\csname sc#1\endcsname
 \fam\scfam\csname sc#1\endcsname}%
\rm }}
\makesize{Xf}{10}{VIIf}{7}{Vf}{5}{12pt}
\newfam\mibfam
\def\normalsize{\Xfsize
\def\sf{%
   \getf@nt{sf}{Xf}{cmss}{10}%
   \getf@nt{sf}{VIIf}{cmss}{10 at 7pt}%
   \getf@nt{sf}{Vf}{cmss}{10 at 5pt}%
   \textfont\sffam=\csname sfXf\endcsname
   \scriptfont\sffam=\csname sfVIIf\endcsname
   \scriptscriptfont\sffam=\csname sfVf\endcsname
   \fam\sffam\csname sfXf\endcsname}%
\def\mib{%
   \getf@nt{mib}{Xf}{cmmib}{10}%
   \getf@nt{mib}{VIIf}{cmmib}{10 at7pt}%
   \getf@nt{mib}{Vf}{cmmib}{10 at5pt}%
   \textfont\mibfam=\csname mibXf\endcsname
   \scriptfont\mibfam=\csname mibVIIf\endcsname
   \scriptscriptfont\mibfam=\csname mibVf\endcsname
   \fam\mibfam\csname mibXf\endcsname}%
\def\boldmath{\textfont1=\mibXf \scriptfont1=\mibVIIf
\scriptscriptfont1=\mibVf}%
\if Y\REFEREE \normalbaselineskip=2\normalbaselineskip
\normallineskip=0.1\normalbaselineskip\normalbaselines
\fi
\rm}
\Xfsize
\it\bf\tt\rm
\def\tenrm{\rmXf}
\def\fiverm{\rmVf}
\def\teni{\miXf}
\def\fivei{\miVf}
\def\tensy{\syXf}
\def\fivesy{\syVf}
\def\tenbf{\bfXf}
\def\fivebf{\bfVf}
\def\tentt{\ttXf}
\def\tensl{\slXf}
\def\tenit{\itXf}
\let\REFEREE=N
\normalsize
\it\bf\tt\sf\mib\rm
\catcode`@=12 % reset catcode
% This is layout.aa
% it sets up some measures and defines miscellaneous macros
\newdimen\fullhsize
\newcount\verybad \verybad=1010
\let\lr=L%
\baselineskip=12pt
\vsize=56\baselineskip
\hoffset=-1true cm
\voffset=-1true cm
\fullhsize=180mm
\newdimen\halfsize
\halfsize=88mm
\hsize=\halfsize
\def\fullline{\hbox to\fullhsize}
\def\makefootline{\baselineskip=12pt \fullline{\the\footline}}
\def\makeheadline{\vbox to 0pt{\vskip-22.5pt
            \fullline{\vbox to 8.5pt{}\the\headline}\vss}\nointerlineskip}
\hfuzz=2pt
\vfuzz=2pt
\tolerance=1000
\abovedisplayskip=3 mm plus6pt minus 4pt
\belowdisplayskip=3 mm plus6pt minus 4pt
\abovedisplayshortskip=0mm plus6pt
\belowdisplayshortskip=2 mm plus4pt minus 4pt
\parindent=1.5em
\newdimen\stdparindent\stdparindent\parindent
\frenchspacing
% Paginierung
\nopagenumbers
%
\predisplaypenalty=600        % Make a page break before a display harder
\displaywidowpenalty=2000     % and even harder for a widow display.
%
\def\widowsandclubs#1{\global\verybad=#1
   \global\widowpenalty=\the\verybad1      % default: 10101
   \global\clubpenalty=\the\verybad2  }    % default: 10102
\widowsandclubs{1010}
% This is runhead.aa
% it sets up the page layout for the first page of an article
% and enables the running head and its changing mechanism
%
\catcode`@=11 % use @ as a normal character
\let\firstpage=Y%
\def\paglay{\headline={\normalsize
\ifx Y\firstpage
 \global\let\firstpage=N%
 \let\REFEREE=N%
 \vbox{\hsize=.75\fullhsize
       \hrule
       \line{\vrule\kern3pt
             \vbox{\kern3pt
                   \hbox{\bf A\&A manuscript no.}
                   \hbox{(will be inserted by hand later)}
                   \kern3pt
                   \hrule
                   \kern3pt
                   \hbox{\bf Your thesaurus codes are:}
                   \hbox{\rightskip=0pt plus3em\advance\hsize by-7pt
                         \vbox{\bf\noindent\ignorespaces\the\THESAURUS}}%
                   \kern3pt}%
             \hfil
             \kern3pt
             \vrule}%
       \hrule}%
 \hfil\llap{\AALogo}%
\else
   \koltitle
\fi}}
\def\koltest{\setbox0=\hbox{\qquad\koltext}%
   \ifdim\wd0>\fullhsize
      \infuser{^^JThe running head built automatically from
               \string\AUTHOR\space and \string\MAINTITLE
               ^^Jexceeds the pagewidth, supply a shorter form
               using \string\AUTHORRUNNINGHEAD\string{...\string}
               ^^Jand/or \string\MAINTITLERUNNINGHEAD\string{...\string}
               after the \string\maketitle-command.}%
      \global\AUTHOR={Please give a shorter version with}%
      \global\MAINTITLE={{\tt\string\AUTHORRUNNINGHEAD\ }and/or%
                         {\tt\ \string\MAINTITLERUNNINGHEAD}}%
   \fi
   \global\let\kolTest=\relax}
%
\let\kolTest=\koltest
%
{\catcode`@=\active
\gdef\koltitle{%
   \petit
   \catcode`\@=\active
   \def@##1{}%
   \def\FOOTNOTE##1{}%
   \kolTest
   \ifodd\pageno
      \hfil\enspace
      \koltext
      \enspace\hfil\llap{\folio}%
   \else
      \rlap{\folio}%
      \hfil\enspace
      \koltext
      \enspace\hfil
   \fi}}
\def\koltext{\ignorespaces\the\AUTHOR\unskip: \ignorespaces
\the\MAINTITLE\unskip}
%
\def\kolTEST{\setbox0=\hbox{\qquad\koltext}%
   \ifdim\wd0>\fullhsize
      \infuser{^^JThe running head you supplied with
               \string\AUTHORRUNNINGHEAD\space and/or
               ^^J\string\MAINTITLERUNNINGHEAD\space exceeds the
               pagewidth, give a shorter form
               ^^Jusing \string\AUTHORRUNNINGHEAD\string{...\string}
               and/or \string\MAINTITLERUNNINGHEAD\string{...\string}
               ^^Jafter the \string\maketitle-command.}%
      \global\AUTHOR={Please give a shorter running head with}%
      \global\MAINTITLE={{\tt\string\AUTHORRUNNINGHEAD\ }and/or%
                         {\tt\ \string\MAINTITLERUNNINGHEAD}}%
   \fi
   \global\let\kolTest=\relax}
%
\def\MAINTITLERUNNINGHEAD#1{\global\MAINTITLE={#1}%
\infuser{^^JMAINTITLE part of running head has been changed}%
\let\kolTest=\kolTEST}
\def\AUTHORRUNNINGHEAD#1{\global\AUTHOR={#1}%
\infuser{^^JAUTHOR part of running head has been changed}%
\let\kolTest=\kolTEST}
\catcode`@=12 % reset catcode
% This is aacpetit.aa
% here the small print is defined
\catcode`@=11    % use @ as a normal character
\makesize{IXf}{9}{VIf}{6}{Vf}{5}{11pt}
\IXfsize\it\bf\tt\rm
\normalsize
\def\petit{\IXfsize
   \def\sf{%
      \getf@nt{sf}{IXf}{cmss}{9}%
      \getf@nt{sf}{VIf}{cmss}{10 at 6pt}%
      \getf@nt{sf}{Vf}{cmss}{10 at 5pt}%
      \textfont\sffam=\csname sfIXf\endcsname
      \scriptfont\sffam=\csname sfVIf\endcsname
      \scriptscriptfont\sffam=\csname sfVf\endcsname
      \fam\sffam\csname sfIXf\endcsname
}%
\def\mib{%
   \getf@nt{mib}{IXf}{cmmib}{10 at 9pt}%
   \getf@nt{mib}{VIf}{cmmib}{10 at 6pt}%
   \getf@nt{mib}{Vf}{cmmib}{10 at 5pt}%
   \textfont\mibfam=\csname mibIXf\endcsname
   \scriptfont\mibfam=\csname mibVIf\endcsname
   \scriptscriptfont\mibfam=\csname mibVf\endcsname
   \fam\mibfam\csname mibIXf\endcsname}%
\def\boldmath{\textfont1=\mibIXf\scriptfont1=\mibVIf
\scriptscriptfont1=\mibVf}%
 \if Y\REFEREE \normalbaselineskip=2\normalbaselineskip
 \normallineskip=0.1\normalbaselineskip\fi
 \setbox\strutbox=\hbox{\vrule height9pt depth2pt width0pt}%
 \normalbaselines\rm}%
%--------------------------------------------------------------------
\def\begpet{\vskip\petitsurround
\bgroup\petit}%  Beginn eines Paragraphen in petit
\def\endpet{\vskip\petitsurround
\egroup}%  Ende eines Paragraphen in petit
\petit\sf\mib
\normalsize
\catcode`@=12 % reset catcode
% This is aacmfont.aa
% It defines the fonts used for the cm version of the a&a package
% and connects the names to fonts already defined
%
% Fonts for \MAINTITLE
 \font \tamt            = cmmib10 scaled \magstep3
 \font \tams            = cmmib10 scaled \magstep1
%\font \tamss           = cmmib10  = \mibXf
 \let  \tamss=\mibXf
 \font \tast            = cmsy10 scaled \magstep3
 \font \tass            = cmsy10 scaled \magstep1
%\font \tasss           = cmsy10   = \syXf
 \let  \tasss=\syXf
 \font \tatt            = cmbx10 scaled \magstep3
 \font \tats            = cmbx10 scaled \magstep1
%\font \tatss           = cmbx10   = \bfXf
 \let  \tatss=\bfXf
% Fonts for \SUBTITLE
 \font \tbmt            = cmmib10 scaled \magstep2
%\font \tbms            = cmmib10  = \mibXf
 \let  \tbms=\mibXf
%\font \tbmss           = cmmib10 at 9pt = \mibIXf
 \let  \tbmss=\mibIXf
 \font \tbst            = cmsy10 scaled \magstep2
%\font \tbss            = cmsy10   = \syXf
 \let  \tbss=\syXf
%\font \tbsss           = cmsy9    = \syIXf
 \let  \tbsss=\syIXf
 \font \tbtt            = cmbx10 scaled \magstep2
%\font \tbts            = cmbx10   = \bfXf
 \let  \tbts=\bfXf
%\font \tbtss           = cmbx9    = \bfIXf
 \let  \tbtss=\bfIXf
 \font \headnotefont    = cmti10 scaled \magstep3
% This is ucgreek.aa
% the definition of versal greek characters
\mathchardef\Gamma  ="0000
\mathchardef\Delta  ="0001
\mathchardef\Theta  ="0002
\mathchardef\Lambda ="0003
\mathchardef\Xi     ="0004
\mathchardef\Pi     ="0005
\mathchardef\Sigma  ="0006
\mathchardef\Upsilon="0007
\mathchardef\Phi    ="0008
\mathchardef\Psi    ="0009
\mathchardef\Omega  ="000A
% This is aamisc.tex
% here miscellaneous macros are defined
%
% Makros to communicate with the user
\catcode`@=11 % use @ as a normal character
\newwrite\@info
\def\infuser#1{\immediate\write\@info{#1}}
\catcode`@=12 % at signs are no longer letters
%
\newlinechar=`\^^J % Zeilenumbruchhilfe fuer TeX-Meldungen am Bildschirm
\def\newline{\hfill\break}% makes a new line in the text :)
%-------------------------------------------------------------------
% draws a frame of height #1 cm times \hsize
\def\rahmen#1{\vbox{\hrule\line{\vrule\vbox to#1true
cm{\vfil}\hfil\vrule}\vfil\hrule}}
%-------------------------------------------------------------------
% shortcuts
\let\ts=\thinspace
\def\,{\relax\ifmmode\mskip\thinmuskip\else\thinspace\fi}
%-------------------------------------------------------------------
\def\unvskip{%
   \ifvmode
      \ifdim\lastskip=0pt
      \else
         \vskip-\lastskip
      \fi
   \fi}
% This is fleqn.tex
% it convinces TeX not to center the displayed formulas
\catcode`@=11    % use @ as a normal character
%
% allocate token registers for the equation and its number
\newtoks\eq\newtoks\eqn
%
% the horizontal size for a displayed formula is the actual \hsize
% minus the indention of the formula depending on the package
\newdimen\mathhsize
\def\calcmathhsize{\mathhsize=\hsize
\advance\mathhsize by-\mathindent}
\calcmathhsize
%
% define \eqalign@new, \displaylines@new, \(l)eqalignno@new to use this
% size but remember the old macros to make switching possible
\let\eqalign@old=\eqalign
\let\displaylines@old=\displaylines
\let\eqalignno@old=\eqalignno
\let\leqalignno@old=\leqalignno
\def\eqalign@new#1{\null\vcenter{\openup\jot\m@th
  \ialign{\strut\hfil$\displaystyle{##}$&$\displaystyle{{}##}$\hfil
      \crcr#1\crcr}}}
\def\displaylines@new#1{{}$\displ@y
\hbox{\vbox{\halign{$\@lign\hfil\displaystyle##\hfil$\crcr
    #1\crcr}}}${}}
\def\eqalignno@new#1{{}$\displ@y
  \hbox{\vbox{\halign
to\mathhsize{\hfil$\@lign\displaystyle{##}$\tabskip\z@skip
    &$\@lign\displaystyle{{}##}$\hfil\tabskip\centering
    &\llap{$\@lign##$}\tabskip\z@skip\crcr
    #1\crcr}}}${}}
\def\leqalignno@new#1{{}$\displ@y
\hbox{\vbox{\halign
to\mathhsize{\qquad\hfil$\@lign\displaystyle{##}$\tabskip\z@skip
    &$\@lign\displaystyle{{}##}$\hfil\tabskip\centering
    &\kern-\mathhsize\rlap{$\@lign##$}\tabskip\hsize\crcr
    #1\crcr}}}${}}
% main macro borrowed from the dirty tricks of the TeXbook
\def\generaldisplay{%
\ifeqno
       \ifleqno\leftline{$\displaystyle\the\eqn\quad\the\eq$}%
       \else\noindent\kern\mathindent\hbox to\mathhsize{$\displaystyle
             \the\eq\hfill\the\eqn$}%
       \fi
\else
       \kern\mathindent
       \hbox to\mathhsize{$\displaystyle\the\eq$\hss}%
\fi
\global\eq={}\global\eqn={}}%
%
% flags and defaults
\newif\ifeqno\newif\ifleqno
% check for the "$$"s
\def\displaysetup#1$${\displaytest#1\eqno\eqno\displaytest}
% look for equation numbers
\def\displaytest#1\eqno#2\eqno#3\displaytest{%
\if!#3!\ldisplaytest#1\leqno\leqno\ldisplaytest
\else\eqnotrue\leqnofalse\eqn={#2}\eq={#1}\fi
\generaldisplay$$}
\def\ldisplaytest#1\leqno#2\leqno#3\ldisplaytest{\eq={#1}%
\if!#3!\eqnofalse\else\eqnotrue\leqnotrue\eqn={#2}\fi}
%
% establish a switching between normal and flush left equations
%
\def\flushleftequations{%
   \infuser{^^JEquations are \ifx\Initexing\undefined now \fi
            typeset flush left.}%
   \everydisplay={\displaysetup}%
   \let\eqalign=\eqalign@new
   \let\displaylines=\displaylines@new
   \let\eqalignno=\eqalignno@new
   \let\leqalignno=\leqalignno@new}
%
\def\centeredequations{%
   \infuser{^^JEquations are \ifx\Initexing\undefined now \fi
            centered.}%
   \everydisplay={}%
   \let\eqalign=\eqalign@old
   \let\displaylines=\displaylines@old
   \let\eqalignno=\eqalignno@old
   \let\leqalignno=\leqalignno@old}
%
% set the default to flush left equations
\flushleftequations
%
\catcode`@=12 % at signs are no longer letters
% This is autnum.tex
% it defines a macro that steps and typesets a counter for equations
\newcount\eqnum\eqnum=0% register
%
% use: $$<formula>\eqno\autnum$$
%  or: $$\eqalignno{a&=b&\autnum\cr}$$
\def\autnum{\global\advance\eqnum by 1\relax{\rm(\the\eqnum)}}
% This is item.tex
% is defines a new \litem(item) with a left aligned argument
% and redefines plain \item(item)
%
\catcode`@=11    % use @ as a normal character
\def\litem#1{\item{#1\hfill}}
%
% \ch@ckitem(item)mark measures the width of the \item(item)s mark
% and issues a warning if it will not fit in the space provided
\def\ch@ckitemmark#1{\setbox0=\hbox{\enspace#1}%
\ifdim\wd0>\itemindent
   \infuser{^^J\string\item: Your mark `\string#1' is too wide. }%
\fi}
\def\ch@ckitemitemmark#1{\setbox0=\hbox{\kern\itemindent\enspace#1}%
\ifdim\wd0>\itemitemindent
   \infuser{^^J\string\itemitem: Your mark `\string#1' is too wide. }%
\fi}
%
% \set@item@mark is used to produce the \item's mark
\def\set@item@mark#1{\ch@ckitemmark{#1}%
\hbox to\itemindent{#1\hss}\ignorespaces}
% \set@itemitem@mark is used to produce the \itemitem's mark
\def\set@itemitem@mark#1{\ch@ckitemitemmark{#1}%
\dimen0=\itemitemindent
\advance\dimen0 by-\itemindent
\kern\itemindent\hbox to\dimen0{#1\hss}\ignorespaces}
%
% \setitem(item)indent takes its argument as the widest mark
% of an \item(item) and changes the \item(item)indent accordingly
\def\setitemindent#1{\setbox0=\hbox{\ignorespaces#1\unskip\enspace}%
\itemindent=\wd0\relax
\ifx\quiet\undefined
\infuser{^^J\string\setitemindent: Mark width modified to hold
         ^^J`\string#1' plus an \string\enspace\space gap. }%
\fi
}
\def\setitemitemindent#1{\setbox0=\hbox{\ignorespaces#1\unskip\enspace}%
\itemitemindent=\wd0\relax
\ifx\quiet\undefined
\infuser{^^J\string\setitemitemindent: Mark width modified to hold
         ^^J`\string#1' plus an \string\enspace\space gap. }%
\fi
\advance\itemitemindent by\itemindent}
%
% \item is redefined to produce a right aligned mark with
% a fixed gap, the hanging indentation has the width \itemindent.
% \itemitem is redefined to produce a left aligned mark with indention
% \itemindent, the hanging indentation has the width \itemitemindent.
% If there are flush left equations (\mathhsize is defined)
% that size is also corrected for use inside an \item(item).
\ifx\undefined\mathhsize
   \def\item{\par\noindent
   \hangindent\itemindent\hangafter=1\relax
   \set@item@mark}
   %
   \def\itemitem{\par\noindent
   \hangindent\itemitemindent\hangafter=1\relax
   \set@itemitem@mark}
\else
   \def\item{\par\noindent\advance\mathhsize by-\itemindent
   \hangindent\itemindent\hangafter=1\relax
   \everypar={\global\mathhsize=\hsize
   \global\advance\mathhsize by-\mathindent
   \global\everypar={}}\set@item@mark}
   %
   \def\itemitem{\par\noindent\advance\mathhsize by-\itemitemindent
   \hangindent\itemitemindent\hangafter=1\relax
   \everypar={\global\mathhsize=\hsize
   \global\advance\mathhsize by-\mathindent
   \global\everypar={}}\set@itemitem@mark}
\fi
\catcode`@=12 % at signs are no longer letters
% This is aatypset.tex
% it defines the final actions of the macro package
% i.e. printing of the identifying message and the
% characters invented by the author
\catcode`@=11    % use @ as a normal character
\newcount\the@end \global\the@end=0
\newbox\springer@macro \setbox\springer@macro=\vbox{}
\def\typeset{\setbox\springer@macro=\vbox{\begpet\noindent
   This article was processed by the author using
   Sprin\-ger-Ver\-lag \TeX{} A\&A macro package 1992.\par
   \egroup}\global\the@end=1}
% \show@special lists the contents of the (user defined) control
% sequence "specialn" where n is the lowercase roman numeral
% representation of the counter \footcount and then steps that counter.
\def\show@special{%
   \smallskip
   \noindent special character \#\number\sterne:
   \csname special\romannumeral\sterne\endcsname
   \advance\sterne by 1\relax
   \testspeci@l}
%
% \testspeci@l checks whether the control sequence "specialn" is
% defined, causing to print it and recursing to n+1 if yes,
% and aborting otherwise.
\def\testspeci@l{%
   \expandafter
   \ifx
      \csname special\romannumeral\sterne\endcsname
      \relax
      \let\next\relax
   \else
      \let\next\show@special
   \fi
   \next}
%
\outer\def\bye{\bigskip\typeset
\sterne=1 \testspeci@l
\if R\lr\null\fi\vfill\supereject\end}
\catcode`@=12    % reset catcode
% This is aalogo.tex
% it defines the A&A logo for the first page of an article
\def\AALogo{\setbox254=\hbox{ ASTROPHYSICS }%
\vbox{\baselineskip=10pt\hrule\hbox{\vrule\vbox{\kern3pt
\hbox to\wd254{\hfil ASTRONOMY\hfil}
\hbox to\wd254{\hfil AND\hfil}\copy254
\hbox to\wd254{\hfil\number\day.\number\month.\number\year\hfil}
\kern3pt}\vrule}\hrule}}
% This is aalegend.tex
% it defines the macro to produce figure captions
\def\figure#1#2{\medskip\noindent{\petit{\bf Fig.\ts#1.\
}\ignorespaces#2\par}}
% This is tabcap.tex
% it defines the macro to produce the table heads
% Tabellenkoepfe
\def\tabcap#1#2{\smallskip\noindent{\bf Table\ts\ignorespaces
#1\unskip.\ }\ignorespaces #2\vskip3mm}
% This is counters.tex
% it defines and initializes the counters needed
\catcode`@=11    % use @ as a normal character
\expandafter \newcount \csname c@Tl\endcsname
    \csname c@Tl\endcsname=0
\expandafter \newcount \csname c@Tm\endcsname
    \csname c@Tm\endcsname=0
\expandafter \newcount \csname c@Tn\endcsname
    \csname c@Tn\endcsname=0
\expandafter \newcount \csname c@To\endcsname
    \csname c@To\endcsname=0
\expandafter \newcount \csname c@Tp\endcsname
    \csname c@Tp\endcsname=0
\expandafter \newcount \csname c@fn\endcsname
    \csname c@fn\endcsname=0
\def \stepc#1    {\global
    \expandafter
    \advance
    \csname c@#1\endcsname by 1}
\def \resetcount#1    {\global
    \csname c@#1\endcsname=0}
\def\@nameuse#1{\csname #1\endcsname}
\def\arabic#1{\@arabic{\@nameuse{c@#1}}}
\def\@arabic#1{\ifnum #1>0 \number #1\fi}
\catcode`@=12    % reset catcode
% This is maintitl.aa
% it defines the macros to start and end the main title
\catcode`@=11 % use @ as a normal character
\def \aTa  { \goodbreak
\bgroup
\par
\let\bf=\relax\let\boldmath=\relax\let\bffam=\z@
\textfont0=\tatt \scriptfont0=\tats \scriptscriptfont0=\tatss
\textfont1=\tamt \scriptfont1=\tams \scriptscriptfont1=\tamss
\textfont2=\tast \scriptfont2=\tass \scriptscriptfont2=\tasss
\ifx\grfam\undefined\else
   \textfont\grfam=\tagrt \scriptfont\grfam=\tagrs
   \scriptscriptfont\grfam=\tagrss
\fi
\baselineskip=\tabaselineskip
\lineskiplimit=0pt\lineskip=0pt
\rightskip=0pt plus4cm
\pretolerance=10000
\noindent
\tatt}
% --------------------------------------------------------------
% Ende Ueberschrift
%
\def\eTa{\vskip10pt\egroup
         \noindent
         \ignorespaces}
\catcode`@=12 % at signs are no longer letters
% This is subtitl.tex
% it defines the macros to start and end the subtitle
\catcode`@=11 % use @ as a normal character
\def\aTb{\goodbreak
\bgroup
\par
\let\bf=\relax\let\boldmath=\relax\let\bffam=\z@
\textfont0=\tbtt \scriptfont0=\tbts \scriptscriptfont0=\tbtss
\textfont1=\tbmt \scriptfont1=\tbms \scriptscriptfont1=\tbmss
\textfont2=\tbst \scriptfont2=\tbss \scriptscriptfont2=\tbsss
\ifx\grfam\undefined\else
   \textfont\grfam=\tbgrt \scriptfont\grfam=\tbgrs
   \scriptscriptfont\grfam=\tbgrss
\fi
\baselineskip=\tbbaselineskip
\lineskip=0pt\lineskiplimit=0pt
\rightskip=0pt plus4cm
\pretolerance=10000
\noindent
\tbtt}
% ---------------------------------------------------------------
% Ende Ueberschrift 2. Ordnung
%
\def\eTb{\vskip10pt
    \egroup
    \noindent
    \ignorespaces}
\catcode`@=12 % at signs are no longer letters
% This is aatita.tex
% it defines the heading of first order that is numbered automatically
\catcode`@=11    % use @ as a normal character
\newcount\section@penalty  \section@penalty=0
\newcount\subsection@penalty  \subsection@penalty=0
\newcount\subsubsection@penalty  \subsubsection@penalty=0
%
\def\titlea#1{\par\stepc{Tl}
    \resetcount{Tm}
    \bgroup
       \normalsize
       \bf \rightskip 0pt plus4em
       \pretolerance=20000
       \boldmath
       \setbox0=\vbox{\vskip\tabefore
          \noindent
          \arabic{Tl}.\
          \ignorespaces#1
          \vskip\taafter}
       \dimen0=\ht0\advance\dimen0 by\dp0
       \advance\dimen0 by 2\baselineskip
       \advance\dimen0 by\pagetotal
       \ifdim\dimen0>\pagegoal
          \ifdim\pagetotal>\pagegoal
          \else\eject\fi\fi
       \vskip\tabefore
       \penalty\section@penalty \global\section@penalty=-200
       \global\subsection@penalty=10007
       \noindent
       \arabic{Tl}.\
       \ignorespaces#1\par
    \egroup
    \nobreak
    \vskip\taafter
    \parindent=0pt
    \let\lasttitle=A%
\everypar={\parindent=\stdparindent
    \penalty\z@\let\lasttitle=N\everypar={}}%
    \ignorespaces}
\catcode`@=12    % reset catcode
% This is aatitb.tex
% it defines the heading of second order that is numbered automatically
\catcode`@=11    % use @ as a normal character
\def\titleb#1{\par\stepc{Tm}
    \resetcount{Tn}
    \if N\lasttitle\else\vskip\tbbeforeback\fi
    \bgroup
       \normalsize
       \raggedright
       \pretolerance=10000
       \it
       \setbox0=\vbox{\vskip\tbbefore
          \normalsize
          \raggedright
          \pretolerance=10000
          \noindent \it \arabic{Tl}.\arabic{Tm}.\ \ignorespaces#1
          \vskip\tbafter}
       \dimen0=\ht0\advance\dimen0 by\dp0\advance\dimen0 by 2\baselineskip
       \advance\dimen0 by\pagetotal
       \ifdim\dimen0>\pagegoal
          \ifdim\pagetotal>\pagegoal
          \else \if N\lasttitle\eject\fi \fi\fi
       \vskip\tbbefore
       \if N\lasttitle \penalty\subsection@penalty \fi
       \global\subsection@penalty=-100
       \global\subsubsection@penalty=10007
       \noindent \arabic{Tl}.\arabic{Tm}.\ \ignorespaces#1\par
    \egroup
    \nobreak
    \vskip\tbafter
    \let\lasttitle=B%
    \parindent=0pt
    \everypar={\parindent=\stdparindent
       \penalty\z@\let\lasttitle=N\everypar={}}%
       \ignorespaces}
\catcode`@=12    % reset catcode
% This is aatitc.tex
% it defines the heading of third order that is numbered automatically
\catcode`@=11    % use @ as a normal character
\def\titlec#1{\par\stepc{Tn}
    \resetcount{To}
    \if N\lasttitle\else\vskip\tcbeforeback\fi
    \bgroup
       \normalsize
       \raggedright
       \pretolerance=10000
       \setbox0=\vbox{\vskip\tcbefore
          \noindent
          \arabic{Tl}.\arabic{Tm}.\arabic{Tn}.\
          \ignorespaces#1\vskip\tcafter}
       \dimen0=\ht0\advance\dimen0 by\dp0\advance\dimen0 by 2\baselineskip
       \advance\dimen0 by\pagetotal
       \ifdim\dimen0>\pagegoal
           \ifdim\pagetotal>\pagegoal
           \else \if N\lasttitle\eject\fi \fi\fi
       \vskip\tcbefore
       \if N\lasttitle \penalty\subsubsection@penalty \fi
       \global\subsubsection@penalty=-50
       \noindent
       \arabic{Tl}.\arabic{Tm}.\arabic{Tn}.\
       \ignorespaces#1\par
    \egroup
    \nobreak
    \vskip\tcafter
    \let\lasttitle=C%
    \parindent=0pt
    \everypar={\parindent=\stdparindent
       \penalty\z@\let\lasttitle=N\everypar={}}%
       \ignorespaces}
\catcode`@=12    % reset catcode
% This is aatitd.tex
% it defines the heading of fourth order that is numbered automatically
\def\titled#1{\par\stepc{To}
    \resetcount{Tp}
    \if N\lasttitle\else\vskip\tdbeforeback\fi
    \vskip\tdbefore
    \bgroup
       \normalsize
       \if N\lasttitle \penalty-50 \fi
       \it \noindent \ignorespaces#1\unskip\
    \egroup\ignorespaces}
% This is referenc.tex
% it defines the macros needed to produce the references
%
\def\begref#1{\par
   \unvskip
   \goodbreak\vskip\tabefore
   {\noindent\bf\ignorespaces#1%
   \par\vskip\taafter}\nobreak\let\INS=N}
\def\ref{\if N\INS\let\INS=Y\else\goodbreak\fi
   \hangindent\stdparindent\hangafter=1\noindent\ignorespaces}
\def\endref{\goodbreak}% Ende der Referenzen
% This is acknow.tex
% it prints the acknowledgement text
\def\acknow#1{\par
   \unvskip
   \vskip\tcbefore
   \noindent{\it Acknowledgements\/}. %
   \ignorespaces#1\par
   \vskip\tcafter}
% This is appendix.aa
% it starts an appendix section without automatic numeration
\def\appendix#1{\vskip\tabefore
    \vbox{\noindent{\bf Appendix #1}\vskip\taafter}%
    \global\eqnum=0\relax
    \nobreak\noindent\ignorespaces}
% This is referee.tex
% it allows the author to produce a referees copy
% single column wider line spacing
\newbox\refereebox
\setbox\refereebox=\vbox
to0pt{\vskip0.5cm\fullline{\hrulefill\tentt\lower0.5ex
\hbox{\kern5pt referee's copy\kern5pt}\hrulefill}\vss}%
\def\refereelayout{\let\REFEREE=Y\footline={\copy\refereebox}%
    \infuser{^^JA referee's copy will be produced}\par
    \if N\lr\else\onecolumn\topskip=10pt\fi
    \normalsize
    \calcmathhsize}
% This is symbols.tex
% the symbols not available in plain TeX are constructed
% by overprinting some characters
\def\sun{\hbox{$\odot$}}
\def\la{\mathrel{\mathchoice {\vcenter{\offinterlineskip\halign{\hfil
$\displaystyle##$\hfil\cr<\cr\noalign{\vskip1.5pt}\sim\cr}}}
{\vcenter{\offinterlineskip\halign{\hfil$\textstyle##$\hfil\cr<\cr
\noalign{\vskip1.0pt}\sim\cr}}}
{\vcenter{\offinterlineskip\halign{\hfil$\scriptstyle##$\hfil\cr<\cr
\noalign{\vskip0.5pt}\sim\cr}}}
{\vcenter{\offinterlineskip\halign{\hfil$\scriptscriptstyle##$\hfil
\cr<\cr\noalign{\vskip0.5pt}\sim\cr}}}}}
\def\ga{\mathrel{\mathchoice {\vcenter{\offinterlineskip\halign{\hfil
$\displaystyle##$\hfil\cr>\cr\noalign{\vskip1.5pt}\sim\cr}}}
{\vcenter{\offinterlineskip\halign{\hfil$\textstyle##$\hfil\cr>\cr
\noalign{\vskip1.0pt}\sim\cr}}}
{\vcenter{\offinterlineskip\halign{\hfil$\scriptstyle##$\hfil\cr>\cr
\noalign{\vskip0.5pt}\sim\cr}}}
{\vcenter{\offinterlineskip\halign{\hfil$\scriptscriptstyle##$\hfil
\cr>\cr\noalign{\vskip0.5pt}\sim\cr}}}}}
\def\sq{\hbox{\rlap{$\sqcap$}$\sqcup$}}
\def\degr{\hbox{$^\circ$}}
\def\arcmin{\hbox{$^\prime$}}
\def\arcsec{\hbox{$^{\prime\prime}$}}
\def\utw{\smash{\rlap{\lower5pt\hbox{$\sim$}}}}
\def\udtw{\smash{\rlap{\lower6pt\hbox{$\approx$}}}}
\def\fd{\hbox{$.\!\!^{\rm d}$}}
\def\fh{\hbox{$.\!\!^{\rm h}$}}
\def\fm{\hbox{$.\!\!^{\rm m}$}}
\def\fs{\hbox{$.\!\!^{\rm s}$}}
\def\fdg{\hbox{$.\!\!^\circ$}}
\def\farcm{\hbox{$.\mkern-4mu^\prime$}}
\def\farcs{\hbox{$.\!\!^{\prime\prime}$}}
\def\fp{\hbox{$.\!\!^{\scriptscriptstyle\rm p}$}}
\def\getsto{\mathrel{\mathchoice {\vcenter{\offinterlineskip
\halign{\hfil$\displaystyle##$\hfil\cr\gets\cr\to\cr}}}
{\vcenter{\offinterlineskip\halign{\hfil$\textstyle##$\hfil\cr
\gets\cr\to\cr}}}
{\vcenter{\offinterlineskip\halign{\hfil$\scriptstyle##$\hfil\cr
\gets\cr\to\cr}}}
{\vcenter{\offinterlineskip\halign{\hfil$\scriptscriptstyle##$\hfil\cr
\gets\cr\to\cr}}}}}
\def\cor{\mathrel{\mathchoice {\hbox{$\widehat=$}}{\hbox{$\widehat=$}}
{\hbox{$\scriptstyle\hat=$}}
{\hbox{$\scriptscriptstyle\hat=$}}}}
\def\lid{\mathrel{\mathchoice {\vcenter{\offinterlineskip\halign{\hfil
$\displaystyle##$\hfil\cr<\cr\noalign{\vskip1.5pt}=\cr}}}
{\vcenter{\offinterlineskip\halign{\hfil$\textstyle##$\hfil\cr<\cr
\noalign{\vskip1pt}=\cr}}}
{\vcenter{\offinterlineskip\halign{\hfil$\scriptstyle##$\hfil\cr<\cr
\noalign{\vskip0.5pt}=\cr}}}
{\vcenter{\offinterlineskip\halign{\hfil$\scriptscriptstyle##$\hfil\cr
<\cr\noalign{\vskip0.5pt}=\cr}}}}}
\def\gid{\mathrel{\mathchoice {\vcenter{\offinterlineskip\halign{\hfil
$\displaystyle##$\hfil\cr>\cr\noalign{\vskip1.5pt}=\cr}}}
{\vcenter{\offinterlineskip\halign{\hfil$\textstyle##$\hfil\cr>\cr
\noalign{\vskip1pt}=\cr}}}
{\vcenter{\offinterlineskip\halign{\hfil$\scriptstyle##$\hfil\cr>\cr
\noalign{\vskip0.5pt}=\cr}}}
{\vcenter{\offinterlineskip\halign{\hfil$\scriptscriptstyle##$\hfil\cr
>\cr\noalign{\vskip0.5pt}=\cr}}}}}
\def\sol{\mathrel{\mathchoice {\vcenter{\offinterlineskip\halign{\hfil
$\displaystyle##$\hfil\cr\sim\cr\noalign{\vskip-0.2mm}<\cr}}}
{\vcenter{\offinterlineskip
\halign{\hfil$\textstyle##$\hfil\cr\sim\cr<\cr}}}
{\vcenter{\offinterlineskip
\halign{\hfil$\scriptstyle##$\hfil\cr\sim\cr<\cr}}}
{\vcenter{\offinterlineskip
\halign{\hfil$\scriptscriptstyle##$\hfil\cr\sim\cr<\cr}}}}}
\def\sog{\mathrel{\mathchoice {\vcenter{\offinterlineskip\halign{\hfil
$\displaystyle##$\hfil\cr\sim\cr\noalign{\vskip-0.2mm}>\cr}}}
{\vcenter{\offinterlineskip
\halign{\hfil$\textstyle##$\hfil\cr\sim\cr>\cr}}}
{\vcenter{\offinterlineskip
\halign{\hfil$\scriptstyle##$\hfil\cr\sim\cr>\cr}}}
{\vcenter{\offinterlineskip
\halign{\hfil$\scriptscriptstyle##$\hfil\cr\sim\cr>\cr}}}}}
\def\lse{\mathrel{\mathchoice {\vcenter{\offinterlineskip\halign{\hfil
$\displaystyle##$\hfil\cr<\cr\noalign{\vskip1.5pt}\simeq\cr}}}
{\vcenter{\offinterlineskip\halign{\hfil$\textstyle##$\hfil\cr<\cr
\noalign{\vskip1pt}\simeq\cr}}}
{\vcenter{\offinterlineskip\halign{\hfil$\scriptstyle##$\hfil\cr<\cr
\noalign{\vskip0.5pt}\simeq\cr}}}
{\vcenter{\offinterlineskip
\halign{\hfil$\scriptscriptstyle##$\hfil\cr<\cr
\noalign{\vskip0.5pt}\simeq\cr}}}}}
\def\gse{\mathrel{\mathchoice {\vcenter{\offinterlineskip\halign{\hfil
$\displaystyle##$\hfil\cr>\cr\noalign{\vskip1.5pt}\simeq\cr}}}
{\vcenter{\offinterlineskip\halign{\hfil$\textstyle##$\hfil\cr>\cr
\noalign{\vskip1.0pt}\simeq\cr}}}
{\vcenter{\offinterlineskip\halign{\hfil$\scriptstyle##$\hfil\cr>\cr
\noalign{\vskip0.5pt}\simeq\cr}}}
{\vcenter{\offinterlineskip
\halign{\hfil$\scriptscriptstyle##$\hfil\cr>\cr
\noalign{\vskip0.5pt}\simeq\cr}}}}}
\def\grole{\mathrel{\mathchoice {\vcenter{\offinterlineskip\halign{\hfil
$\displaystyle##$\hfil\cr>\cr\noalign{\vskip-1.5pt}<\cr}}}
{\vcenter{\offinterlineskip\halign{\hfil$\textstyle##$\hfil\cr
>\cr\noalign{\vskip-1.5pt}<\cr}}}
{\vcenter{\offinterlineskip\halign{\hfil$\scriptstyle##$\hfil\cr
>\cr\noalign{\vskip-1pt}<\cr}}}
{\vcenter{\offinterlineskip\halign{\hfil$\scriptscriptstyle##$\hfil\cr
>\cr\noalign{\vskip-0.5pt}<\cr}}}}}
\def\leogr{\mathrel{\mathchoice {\vcenter{\offinterlineskip\halign{\hfil
$\displaystyle##$\hfil\cr<\cr\noalign{\vskip-1.5pt}>\cr}}}
{\vcenter{\offinterlineskip\halign{\hfil$\textstyle##$\hfil\cr
<\cr\noalign{\vskip-1.5pt}>\cr}}}
{\vcenter{\offinterlineskip\halign{\hfil$\scriptstyle##$\hfil\cr
<\cr\noalign{\vskip-1pt}>\cr}}}
{\vcenter{\offinterlineskip\halign{\hfil$\scriptscriptstyle##$\hfil\cr
<\cr\noalign{\vskip-0.5pt}>\cr}}}}}
\def\loa{\mathrel{\mathchoice {\vcenter{\offinterlineskip\halign{\hfil
$\displaystyle##$\hfil\cr<\cr\noalign{\vskip1.5pt}\approx\cr}}}
{\vcenter{\offinterlineskip\halign{\hfil$\textstyle##$\hfil\cr<\cr
\noalign{\vskip1.0pt}\approx\cr}}}
{\vcenter{\offinterlineskip\halign{\hfil$\scriptstyle##$\hfil\cr<\cr
\noalign{\vskip0.5pt}\approx\cr}}}
{\vcenter{\offinterlineskip\halign{\hfil$\scriptscriptstyle##$\hfil\cr
<\cr\noalign{\vskip0.5pt}\approx\cr}}}}}
\def\goa{\mathrel{\mathchoice {\vcenter{\offinterlineskip\halign{\hfil
$\displaystyle##$\hfil\cr>\cr\noalign{\vskip1.5pt}\approx\cr}}}
{\vcenter{\offinterlineskip\halign{\hfil$\textstyle##$\hfil\cr>\cr
\noalign{\vskip1.0pt}\approx\cr}}}
{\vcenter{\offinterlineskip\halign{\hfil$\scriptstyle##$\hfil\cr>\cr
\noalign{\vskip0.5pt}\approx\cr}}}
{\vcenter{\offinterlineskip\halign{\hfil$\scriptscriptstyle##$\hfil\cr
>\cr\noalign{\vskip0.5pt}\approx\cr}}}}}
\def\bbbr{{\rm I\!R}} %reelle Zahlen
\def\bbbn{{\rm I\!N}} %natuerliche Zahlen
\def\bbbm{{\rm I\!M}}
\def\bbbh{{\rm I\!H}}
\def\bbbf{{\rm I\!F}}
\def\bbbk{{\rm I\!K}}
\def\bbbp{{\rm I\!P}}
\def\bbbone{{\mathchoice {\rm 1\mskip-4mu l} {\rm 1\mskip-4mu l}
{\rm 1\mskip-4.5mu l} {\rm 1\mskip-5mu l}}}
\def\bbbc{{\mathchoice {\setbox0=\hbox{$\displaystyle\rm C$}\hbox{\hbox
to0pt{\kern0.4\wd0\vrule height0.9\ht0\hss}\box0}}
{\setbox0=\hbox{$\textstyle\rm C$}\hbox{\hbox
to0pt{\kern0.4\wd0\vrule height0.9\ht0\hss}\box0}}
{\setbox0=\hbox{$\scriptstyle\rm C$}\hbox{\hbox
to0pt{\kern0.4\wd0\vrule height0.9\ht0\hss}\box0}}
{\setbox0=\hbox{$\scriptscriptstyle\rm C$}\hbox{\hbox
to0pt{\kern0.4\wd0\vrule height0.9\ht0\hss}\box0}}}}
\def\bbbq{{\mathchoice {\setbox0=\hbox{$\displaystyle\rm Q$}\hbox{\raise
0.15\ht0\hbox to0pt{\kern0.4\wd0\vrule height0.8\ht0\hss}\box0}}
{\setbox0=\hbox{$\textstyle\rm Q$}\hbox{\raise
0.15\ht0\hbox to0pt{\kern0.4\wd0\vrule height0.8\ht0\hss}\box0}}
{\setbox0=\hbox{$\scriptstyle\rm Q$}\hbox{\raise
0.15\ht0\hbox to0pt{\kern0.4\wd0\vrule height0.7\ht0\hss}\box0}}
{\setbox0=\hbox{$\scriptscriptstyle\rm Q$}\hbox{\raise
0.15\ht0\hbox to0pt{\kern0.4\wd0\vrule height0.7\ht0\hss}\box0}}}}
\def\bbbt{{\mathchoice {\setbox0=\hbox{$\displaystyle\rm
T$}\hbox{\hbox to0pt{\kern0.3\wd0\vrule height0.9\ht0\hss}\box0}}
{\setbox0=\hbox{$\textstyle\rm T$}\hbox{\hbox
to0pt{\kern0.3\wd0\vrule height0.9\ht0\hss}\box0}}
{\setbox0=\hbox{$\scriptstyle\rm T$}\hbox{\hbox
to0pt{\kern0.3\wd0\vrule height0.9\ht0\hss}\box0}}
{\setbox0=\hbox{$\scriptscriptstyle\rm T$}\hbox{\hbox
to0pt{\kern0.3\wd0\vrule height0.9\ht0\hss}\box0}}}}
\def\bbbs{{\mathchoice
{\setbox0=\hbox{$\displaystyle     \rm S$}\hbox{\raise0.5\ht0\hbox
to0pt{\kern0.35\wd0\vrule height0.45\ht0\hss}\hbox
to0pt{\kern0.55\wd0\vrule height0.5\ht0\hss}\box0}}
{\setbox0=\hbox{$\textstyle        \rm S$}\hbox{\raise0.5\ht0\hbox
to0pt{\kern0.35\wd0\vrule height0.45\ht0\hss}\hbox
to0pt{\kern0.55\wd0\vrule height0.5\ht0\hss}\box0}}
{\setbox0=\hbox{$\scriptstyle      \rm S$}\hbox{\raise0.5\ht0\hbox
to0pt{\kern0.35\wd0\vrule height0.45\ht0\hss}\raise0.05\ht0\hbox
to0pt{\kern0.5\wd0\vrule height0.45\ht0\hss}\box0}}
{\setbox0=\hbox{$\scriptscriptstyle\rm S$}\hbox{\raise0.5\ht0\hbox
to0pt{\kern0.4\wd0\vrule height0.45\ht0\hss}\raise0.05\ht0\hbox
to0pt{\kern0.55\wd0\vrule height0.45\ht0\hss}\box0}}}}
\def\bbbz{{\mathchoice {\hbox{$\sf\textstyle Z\kern-0.4em Z$}}
{\hbox{$\sf\textstyle Z\kern-0.4em Z$}}
{\hbox{$\sf\scriptstyle Z\kern-0.3em Z$}}
{\hbox{$\sf\scriptscriptstyle Z\kern-0.2em Z$}}}}
\def\diameter{{\ifmmode\oslash\else$\oslash$\fi}}
%
% redefine \Re and \Im
\def\Re{{\rm Re}}
\def\Im{{\rm Im}}
%
\def\diff{{\rm d}}
\def\eul{{\rm e}}
\def\imag{{\rm i}}
% This is vec.tex
% it defines vectors in boldface math and tensors in sans serif style
%
\catcode`@=11 % use @ as a normal character
\def\vec#1{{\boldmath\bf
\textfont\z@=\textfont\bffam\scriptfont\z@=\scriptfont\bffam
\scriptscriptfont\z@=\scriptscriptfont\bffam
\ifmmode
\mathchoice{\hbox{$\displaystyle#1$}}{\hbox{$\textstyle#1$}}
{\hbox{$\scriptstyle#1$}}{\hbox{$\scriptscriptstyle#1$}}\else
$#1$\fi}}
%
\def\tens#1{\relax\ifmmode
\mathchoice{\hbox{$\displaystyle\sf#1$}}{\hbox{$\textstyle\sf#1$}}
{\hbox{$\scriptstyle\sf#1$}}{\hbox{$\scriptscriptstyle\sf#1$}}\else
$\sf#1$\fi}
\catcode`@=12 % at signs are no longer letters
% This is titsetup.tex
% it defines variables and macros for the title page
% Variablenvereinbarung fuer Titelseitenautomatik
\newcount\sterne \sterne=0
\newdimen\fullhead
{\catcode`@=11    % use @ as a normal character
\def\newtoks{\alloc@5\toks\toksdef\@cclvi}
\outer\gdef\makenewtoks#1{\newtoks#1#1={ ????? }}}
\makenewtoks\DATE
\makenewtoks\MAINTITLE
\makenewtoks\SUBTITLE
\makenewtoks\AUTHOR
\makenewtoks\INSTITUTE
\makenewtoks\ABSTRACT
\makenewtoks\KEYWORDS
\makenewtoks\THESAURUS
\makenewtoks\OFFPRINTS
\makenewtoks\HEADNOTE
% Klammeraffe(@-Zeichen) wird Schluesselbuchstabe fuer affiliations
% solange Titel-page aktiv ist (Benutzung in AUTHOR und INSTITUTE)
\let\INS=N%
{\catcode`\@=\active
% Aktionen, die bei Antreffen des @-Zeichens zu machen sind;
% zwei Faelle a) @ bei AUTHOR, b) @ bei INSTITUTE
\gdef@#1{%
\if N\INS
   $^{#1}$%
\else
   \let\INS=Y%
   \par
   \noindent\hbox to0.5\stdparindent{$^{#1}$\hfil}\ignorespaces
\fi}%
}% @ wieder normal
% Automatische Fussnotennumerierung mit wachsender Anzahl Sterne
\def\mehrsterne{\global\advance\sterne by1\relax}%
% This is footnote.tex
% it defines all footnote macros
\def\footnoterule{\kern-3pt\hrule width 2true cm\kern2.6pt}% Trennlinie
%-------------------------------------------------------------------------------
\def\makeOFFPRINTS#1{\bgroup\let\REFEREE=N \normalsize
       \if N\lr\hsize=\fullhsize\else\hsize=\halfsize\fi
       \lineskiplimit=0pt\lineskip=0pt
       \def\textindent##1{\noindent{\it Send offprint
          requests to\/}: }\relax
       \vfootnote{nix}{\ignorespaces#1}\egroup}
%-------------------------------------------------------------------------------
% Fussnote mit stern(*) gekennzeichnet
\def\makesterne{\count254=0\loop\ifnum\count254<\sterne
\advance\count254 by1\star\repeat}
\def\FOOTNOTE#1{\bgroup\let\REFEREE=N
       \ifhmode\unskip\fi
       \mehrsterne$^{\makesterne}$\relax
       \normalsize
       \if N\lr\hsize=\fullhsize\else\hsize=\halfsize\fi
       \lineskiplimit=0pt\lineskip=0pt
       \def\textindent##1{\noindent\hbox
       to\stdparindent{##1\hss}}\relax
       \vfootnote{$^{\makesterne}$}{\ignorespaces#1}\egroup}
%
\def\PRESADD#1{\FOOTNOTE{Present address: \ignorespaces#1}}
%
% Automatisch numerierte Fussnote
\def\fonote#1{\relax\ifhmode\unskip\fi
       \mehrsterne$^{\the\sterne}$\bgroup
       \normalsize
       \if N\lr\hsize=\fullhsize\else\hsize=\halfsize\fi
       \def\textindent##1{\noindent\hbox
       to\stdparindent{##1\hss}}\relax
       \vfootnote{$^{\the\sterne}$}{\ignorespaces#1}\egroup}
% This is maketitl.tex
% it defines all macros to construct the title page
%
\def\missmsg#1{\infuser{^^JMissing #1 }}
%% test if missing:
%
\def\tstmiss#1#2#3#4#5{%
\edef\test{\the #1}%
\ifx\test\missing%
  #2\relax%  infuser
  #3%   action if missing
\else
  \ifx\test\missingi%
    #2\relax%  infuser
    #3%   action if missing
  \else #4%  action if existing
  \fi
\fi
#5%   action at any rate
}%
\def\strich{\par
\vbox to0pt{\hrule width\hsize\vss}\vskip-1.2\baselineskip
\vskip0pt plus3\baselineskip\relax}%
%
\def\at{\char64 }
%
% Hauptmacro fuer automatische Titelseite
\def\maketitle{\paglay%
\def\missing{ ????? }% Schablone zur Erkennung von nicht ausgefuellten
\def\missingi{ }% Schablone zur Erkennung von nicht ausgefuellten
% Feldern
%
% Methode zum Test auf Leerfelder: Variable wird mit \missing=" ????? "
% verglichen; Fallunterscheidung.
% \edef\test{\the\VARIABLE} macht die Variable dem Vergleich zugaenglich
% \ifx\test = \missing    d.h. steht in Variable der Text: " ????? "?
% !!! Aktion fuer Fall ja, hier wurde also nichts eingetragen !!!
% \else !!! Fall fuer ausgefuellte Variable!!! \fi
{\parskip=0pt\relax
\setbox0=\vbox{\hsize=\fullhsize\null\vskip2truecm
%
\tstmiss%
  {\HEADNOTE}%
  {}%
  {}%
  {%   write HEADNOTE:
   \noindent{\headnotefont\ignorespaces\the\HEADNOTE}\vskip8mm
   }%
  {}%
%
\tstmiss%
  {\MAINTITLE}%
  {}%
  {\global\MAINTITLE={MAINTITLE should be given}}%
  {}%
  {%   write MAINTITLE:
   \aTa\ignorespaces\the\MAINTITLE\eTa}%
%
\tstmiss%
  {\SUBTITLE}%
  {}%
  {}%
  {%   write SUBTITLE:
   \aTb\ignorespaces\the\SUBTITLE\eTb}%
  {}%
%
\tstmiss%
  {\AUTHOR}%
  {}%
  {\AUTHOR={Name(s) and initial(s) of author(s) should be given}}
  {}%
  {%   write AUTHOR:
\noindent{\bf\ignorespaces\the\AUTHOR\vskip4pt}}%
%
\tstmiss%
  {\INSTITUTE}%
  {}%
  {\INSTITUTE={Address(es) of author(s) should be given.}}%
  {}%
  {%   write INSTITUTE:
   \let\INS=E
\noindent\ignorespaces\the\INSTITUTE\vskip10pt}%
%
\tstmiss%
  {\DATE}%
  {}%
  {\DATE={$[$the date of receipt and acceptance should be inserted
later$]$}}%
  {}%
  {%   write DATE:
{\noindent\ignorespaces\the\DATE\vskip21pt}\bf A}%
%
}%
%
\global\fullhead=\ht0\global\advance\fullhead by\dp0
\global\advance\fullhead by10pt\global\sterne=0
%
{\hsize=\halfsize\null\vskip2truecm
\tstmiss%
  {\OFFPRINTS}%
  {}%
  {}%
  {\makeOFFPRINTS{\the\OFFPRINTS}}%
  {}%
%
\hsize=\fullhsize
%
\tstmiss%
  {\HEADNOTE}%
  {}%
  {}%
  {%   write HEADNOTE:
   \noindent{\headnotefont\ignorespaces\the\HEADNOTE}\vskip8mm
   }%
  {}%
%
\tstmiss%
  {\MAINTITLE}%
  {\missmsg{MAINTITLE}}%
  {\global\MAINTITLE={MAINTITLE should be given}}%
  {}%
  {%   write MAINTITLE:
   \aTa\ignorespaces\the\MAINTITLE\eTa}%
%
\tstmiss%
  {\SUBTITLE}%
  {}%
  {}%
  {%   write SUBTITLE:
   \aTb\ignorespaces\the\SUBTITLE\eTb}%
  {}%
%
\tstmiss%
  {\AUTHOR}%
  {\missmsg{name(s) and initial(s) of author(s)}}%
  {\AUTHOR={Name(s) and initial(s) of author(s) should be given}}
  {}%
  {%   write AUTHOR:
\noindent{\bf\ignorespaces\the\AUTHOR\vskip4pt}}%
%
\tstmiss%
  {\INSTITUTE}%
  {\missmsg{address(es) of author(s)}}%
  {\INSTITUTE={Address(es) of author(s) should be given.}}%
  {}%
  {%   write INSTITUTE:
   \let\INS=E
\noindent\ignorespaces\the\INSTITUTE\vskip10pt}%
%
\tstmiss%
  {\DATE}%
  {\infuser{^^JThe date of receipt and acceptance should be inserted
later.}}%
  {\DATE={$[$the date of receipt and acceptance should be inserted
later$]$}}%
  {}%
  {%   write DATE:
{\noindent\ignorespaces\the\DATE\vskip21pt}}%
}%
%
\tstmiss%
  {\THESAURUS}%
  {\infuser{^^JThesaurus codes are not given.}}%
  {\global\THESAURUS={missing; you have not inserted them}}%
  {}%
  {}%
%
\normalsize
%
\tstmiss%
  {\ABSTRACT}%
  {\missmsg{ABSTRACT}}%
  {\ABSTRACT={Not yet given.}}%
  {}%
  {\noindent{\bf Abstract. }\ignorespaces\the\ABSTRACT\vskip0.5true cm}%
%
\tstmiss%
  {\KEYWORDS}%
  {\missmsg{KEYWORDS}}%
  {\KEYWORDS={Not yet given.}}%
  {}%
  {\noindent{\bf Key words: }\ignorespaces\the\KEYWORDS
  \strich}%
%
\global\sterne=0
\global\catcode`\@=12
}\normalsize}%Ende von maketitle
% @ ist wieder ein ganz normales Zeichen und wird auch als solches
% gedruckt.
%-------------------------------------------------------------------
% This is twocolout.tex
% This file gives plain TeX the ability of single and double
% column output with floating insertions at the top which span
% both columns in double column format.
% The default format is double column format. Switching between
% single and double column format is provided but inserts
% an immediate page break and ejects all floating insertions.
% Double column floating insertions defined in text in a
% right column are held back until the beginning of the next page.
% -----------------------------------------------------------------
%
% Macros which the user should know about:
%
% \setuplr#1#2#3   : Should be called before any output
%                    #1 = O (one column) or T (two columns,default)
%                    #2 <dimen> = text width (both columns)
%                    #3 <dimen> = column width
%
% \newpage         : Fill a whole page and start a new one
% \newcolumn       : Start a new column
% \onecolumn       : Break the current page and switch to one column
% \twocolumns      : Break the current page and switch to two columns
%
% \bothtopinsert   : Beginning of a floating insertion over the
%                    full text width
% \endbothinsert   : End of such an insertion (analog to \endinsert)
% \specialpage     : Insert a special page following the current page
%
% All macros with a '@' in their names are for internal usage only.
%
% Modified macros for A&A:
%    \output \typeset \refereelayout
%    \titlea \titleb \titlec \titled
%    \begref \ref \endref \ack \app \figure
%    \begfig \begfigwid \begtab \begtabfullwid \begtabemptywid
% Original A&A macros using one of the modified macros:
%    \bye (\typeset)
% New macros for A&A:
%    \begfigpage \begtabempty \begtabfull \begtabpage
%    \puthere \putattop \putatbottom
% -----------------------------------------------------------------
%
\catcode`@=11    % use @ as a normal character
%
\newdimen\@txtwd  \@txtwd=\hsize
\newdimen\@txtht  \@txtht=\vsize
\newdimen\@colht  \@colht=\vsize
\newdimen\@colwd  \@colwd=-1pt
\newdimen\@colsavwd
%%%
%%% =========== Macros for initializing the whole thing ==========
%%%
\newcount\in@t \in@t=0
\def\initlr{\if N\lr \ifdim\@colwd<0pt \global\@colwd=\hsize \fi
   \else\global\let\lr=L\ifdim\@colwd<0pt \global\@colwd=\hsize
      \global\divide\@colwd\tw@ \global\advance\@colwd by -10pt
   \fi\fi\global\advance\in@t by 1}
\def\setuplr#1#2#3{\let\lr=O \ifx#1\lr\global\let\lr=N
      \else\global\let\lr=L\fi
   \@txtht=\vsize \@colht=\vsize \@txtwd=#2 \@colwd=#3
   \if N\lr \else\multiply\@colwd\tw@ \fi
   \ifdim\@colwd>\@txtwd\if N\lr
        \errmessage{The text width is less than the column width}%
      \else
        \errmessage{The text width is less the two times the column width}%
      \fi \global\@colwd=\@txtwd
      \if N\lr\divide\@colwd by 2\fi
   \else \global\@colwd=#3 \fi \initlr \@colsavwd=#3
   \global\@insmx=\@txtht
   \global\hsize=\@colwd}
%% ----------- switching between one and two column output ------
\def\twocolumns{\@fillpage\eject\global\let\lr=L \@makecolht
   \global\@colwd=\@colsavwd \global\hsize=\@colwd}
\def\onecolumn{\@fillpage\eject\global\let\lr=N \@makecolht
   \global\@colwd=\@txtwd \global\hsize=\@colwd}
\def\newpage{\@fillpage\eject}
\def\@fillpage{\vfill\supereject\if R\lr \null\vfill\eject\fi}
\def\newcolumn{\vfill\supereject}
%% ----------------------------------------------------------------
%%%
%%% =============== Macros used by the output routine ===============
%%%
\newbox\@leftcolumn
\newbox\@rightcolumn
\newbox\@outputbox
\newbox\@tempboxa
\newbox\@keepboxa
\newbox\@keepboxb
\newbox\@bothcolumns
\newbox\@savetopins
\newbox\@savetopright
\newcount\verybad \verybad=1010
%% -------- \@makecolumn puts the current column in a box ---------
\def\@makecolumn{\ifnum \in@t<1\initlr\fi
   \ifnum\outputpenalty=\the\verybad1  %%% i.e. 10101 if \verybad=1010
      \if L\lr\else\advance\pageno by1\fi
      \infuser{Warning: There is a 'widow' line
      at the top of page \the\pageno\if R\lr (left)\fi.
      This is unacceptable.} \if L\lr\else\advance\pageno by-1\fi \fi
   \ifnum\outputpenalty=\the\verybad2
      \infuser{Warning: There is a 'club' line
      at the bottom of page \the\pageno\if L\lr(left)\fi.
      This is unacceptable.} \fi
   \if L\lr \ifvoid\@savetopins\else\@colht=\@txtht\fi \fi
   \if R\lr \ifvoid\@bothcolumns \ifvoid\@savetopright
       \else\@colht=\@txtht\fi\fi\fi
   \global\setbox\@outputbox
   \vbox to\@colht{\boxmaxdepth\maxdepth
   % One-column top insertions are held back if there is already a
   % two-column floating insertion and the one-column top insertion
   % doesn't fit entirely in the column.
   \if L\lr \ifvoid\@savetopins\else\unvbox\@savetopins\fi \fi
   \if R\lr \ifvoid\@bothcolumns \ifvoid\@savetopright\else
       \unvbox\@savetopright\fi\fi\fi
   \ifvoid\topins\else\ifnum\count\topins>0
         \ifdim\ht\topins>\@colht
            \infuser{^^JError: Too many or too large single column
            box(es) on this page.}\fi
         \unvbox\topins
      \else
         \global\setbox\@savetopins=\vbox{\ifvoid\@savetopins\else
         \unvbox\@savetopins\penalty-500\fi \unvbox\topins} \fi\fi
   \dimen@=\dp\@cclv \unvbox\@cclv % open up \box255
   \ifvoid\bottomins\else\unvbox\bottomins\fi
   \ifvoid\footins\else % footnote info is present
     \vskip\skip\footins
     \footnoterule
     \unvbox\footins\fi
   \ifr@ggedbottom \kern-\dimen@ \vfil \fi}%
}
%% --------- \@outputpage puts the columns and the top insertions
%% --------- together and puts them out
\def\@outputpage{\@dooutput{\lr}}
\def\@colbox#1{\hbox to\@colwd{\box#1\hss}}
\def\@dooutput#1{\global\topskip=10pt
  \ifdim\ht\@bothcolumns>\@txtht
    \if #1N
       \unvbox\@outputbox
    \else
       \unvbox\@leftcolumn\unvbox\@outputbox
    \fi
    \global\setbox\@tempboxa\vbox{\hsize=\@txtwd\makeheadline
       \vsplit\@bothcolumns to\@txtht
       \makefootline\hsize=\@colwd}%
    \infuser{^^JError: Too many double column boxes on this page.}%
    \shipout\box\@tempboxa\advancepageno
    \unvbox255 \penalty\outputpenalty
  \else
    \global\setbox\@tempboxa\vbox{\hsize=\@txtwd\makeheadline
       \ifvoid\@bothcolumns\else\unvbox\@bothcolumns\fi
       \hsize=\@colwd
       \if #1N
          \hbox to\@txtwd{\@colbox{\@outputbox}\hfil}%
       \else
          \hbox to\@txtwd{\@colbox{\@leftcolumn}\hfil\@colbox{\@outputbox}}%
       \fi
       \hsize=\@txtwd\makefootline\hsize=\@colwd}%
    \shipout\box\@tempboxa\advancepageno
  \fi
  \ifnum \special@pages>0 \s@count=100 \page@command
      \xdef\page@command{}\global\special@pages=0 \fi
  }
%% -------- \balance@right@left balances the columns on the last
%% -------- page of text.
\def\balance@right@left{\dimen@=\ht\@leftcolumn
    \advance\dimen@ by\ht\@outputbox
    \advance\dimen@ by\ht\springer@macro
    \dimen2=\z@ \global\the@end=0
    % put both columns together and compensate \vfill at the end
    \ifdim\dimen@>70pt\setbox\z@=\vbox{\unvbox\@leftcolumn
          \unvbox\@outputbox}%
       \loop
          \dimen@=\ht\z@
          \advance\dimen@ by0.5\topskip
          \advance\dimen@ by\baselineskip
          \advance\dimen@ by\ht\springer@macro
          \advance\dimen@ by\dimen2
          \divide\dimen@ by2
          \splittopskip=\topskip
          % Now split it to two parts of about the same height
          {\vbadness=10000
             \global\setbox3=\copy\z@
             \global\setbox1=\vsplit3 to \dimen@}%
          \dimen1=\ht3 \advance\dimen1 by\ht\springer@macro
       \ifdim\dimen1>\ht1 \advance\dimen2 by\baselineskip\repeat
       \dimen@=\ht1
       % Restore the column boxes and adjust
       \global\setbox\@leftcolumn
          \hbox to\@colwd{\vbox to\@colht{\vbox to\dimen@{\unvbox1}\vfil}}%
       \global\setbox\@outputbox
          \hbox to\@colwd{\vbox to\@colht{\vbox to\dimen@{\unvbox3
             \vfill\box\springer@macro}\vfil}}%
    \else
       \setbox\@leftcolumn=\vbox{unvbox\@leftcolumn\bigskip
          \box\springer@macro}%
    \fi}
%
%%%
%%% ================== Insertion routines ======================
%%%
%%% This follows loosely the definition of \topins by Knuth but without
%%% the need to distinguish between 3 different kinds of insertions.
%%% See the TeXBook p.363.
%%% Insertions in right columns are first saved in a box (\rightins)
%%% and inserted to \bothins after this column has been shipped out.
%
\newinsert\bothins
\newbox\rightins
\skip\bothins=\z@skip
\count\bothins=1000
\dimen\bothins=\@txtht \advance\dimen\bothins by -\bigskipamount
\def\bothtopinsert{\par\begingroup\setbox\z@\vbox\bgroup
    \hsize=\@txtwd\parskip=0pt\par\noindent\bgroup}
\def\endbothinsert{\egroup\egroup
  \if R\lr
    \right@nsert
  \else    % L\lr or N\lr
    % If double column insertions don't fit into the current columm
    % keep them until the next page starts.
    \dimen@=\ht\z@ \advance\dimen@ by\dp\z@ \advance\dimen@ by\pagetotal
    \advance\dimen@ by \bigskipamount \advance\dimen@ by \topskip
    \advance\dimen@ by\ht\topins \advance\dimen@ by\dp\topins
    \advance\dimen@ by\ht\bottomins \advance\dimen@ by\dp\bottomins
    \advance\dimen@ by\ht\@savetopins \advance\dimen@ by\dp\@savetopins
    \ifdim\dimen@>\@colht\right@nsert\else\left@nsert\fi
  \fi  \endgroup}
\def\right@nsert{\global\setbox\rightins\vbox{\ifvoid\rightins
    \else\unvbox\rightins\fi\penalty100
    \splittopskip=\topskip
    \splitmaxdepth\maxdimen \floatingpenalty200
    \dimen@\ht\z@ \advance\dimen@\dp\z@
    \box\z@\nobreak\bigskip}}
\def\left@nsert{\insert\bothins{\penalty100
    \splittopskip=\topskip
    \splitmaxdepth\maxdimen \floatingpenalty200
    \box\z@\nobreak\bigskip}
    \@makecolht}
\newdimen\@insht    \@insht=\z@
\newdimen\@insmx    \@insmx=\vsize
%% ------ \@makecolht computes the available height of the current column.
\def\@makecolht{\global\@colht=\@txtht \@compinsht
    \global\advance\@colht by -\@insht \global\vsize=\@colht
    \global\dimen\topins=\@colht}
\def\@compinsht{\if R\lr
       \dimen@=\ht\@bothcolumns \advance\dimen@ by\dp\@bothcolumns
       \ifvoid\@bothcolumns \advance\dimen@ by\ht\@savetopright
          \advance\dimen@ by\dp\@savetopright \fi
    \else
       \dimen@=\ht\bothins \advance\dimen@ by\dp\bothins
       \advance\dimen@ by\ht\@savetopins \advance\dimen@ by\dp\@savetopins
    \fi
    \ifdim\dimen@>\@insmx
       \global\@insht=\dimen@
    \else\global\@insht=\dimen@
    \fi}
\newinsert\bottomins
\skip\bottomins=\z@skip
\count\bottomins=1000
%%%
%%% Special pages to be inserted
%%%
\xdef\page@command{}
\newcount\s@count
\newcount\special@pages \special@pages=0
\def\specialpage#1{\global\advance\special@pages by1
    \global\s@count=\special@pages
    \global\advance\s@count by 100
    \global\setbox\s@count
    \vbox to\@txtht{\hsize=\@txtwd\parskip=0pt
    \par\noindent\noexpand#1\vfil}%
    \def\protect{\noexpand\protect\noexpand}%
    \xdef\page@command{\page@command
         \protect\global\advance\s@count by1
         \protect\begingroup
         \protect\setbox\z@\vbox{\protect\makeheadline
                                    \protect\box\s@count
            \protect\makefootline}%
         \protect{\shipout\box\z@}%
         \protect\endgroup\protect\advancepageno}%
    \let\protect=\relax
   }
%%%
%%%
%%% This little macro adjusts the top of figure boxes with the
%%% the top of the column. Previously they were adjusted with the
%%% the baseline of the first row in a column.
\def\@startins{\vskip \topskip\hrule height\z@
   \nobreak\vskip -\topskip\vskip3.7pt}
%%%
%%%  ==============  The main output routine ===============
%%%
%%%           The output routine was adapted to A&A
%%%
\let\retry=N
\output={\@makecolht \global\topskip=10pt \let\retry=N%
   \ifnum\count\topins>0 \ifdim\ht\topins>\@colht
       \global\count\topins=0 \global\let\retry=Y%
       \unvbox\@cclv \penalty\outputpenalty \fi\fi
   \if N\retry
    \if N\lr     % this is for single column output
       \@makecolumn
       \ifnum\the@end>0
          \setbox\z@=\vbox{\unvcopy\@outputbox}%
          \dimen@=\ht\z@ \advance\dimen@ by\ht\springer@macro
          \ifdim\dimen@<\@colht
             \setbox\@outputbox=\vbox to\@colht{\box\z@
             \unskip\vskip12pt plus0pt minus12pt
             \box\springer@macro\vfil}%
          \else \box\springer@macro \fi
          \global\the@end=0
       \fi
       \ifvoid\bothins\else\global\setbox\@bothcolumns\box\bothins\fi
       \@outputpage
       \ifvoid\rightins\else
       %  Hold \rightins back if there is already a \@savetopins
       \ifvoid\@savetopins\insert\bothins{\unvbox\rightins}\fi
       \fi
    \else
       \if L\lr    % this is the left of two columns
          \@makecolumn
          \global\setbox\@leftcolumn\box\@outputbox \global\let\lr=R%
          \ifx Y\firstpage %\ifnum\pageno=1
             \infuser{^^J[left\the\pageno]}%
          \else
             \infuser{[left\the\pageno]}\fi
          \ifvoid\bothins\else\global\setbox\@bothcolumns\box\bothins\fi
          \global\dimen\bothins=\z@
          \global\count\bothins=0
          \ifx Y\firstpage %\ifnum\pageno=1
             \global\topskip=\fullhead\fi
       \else    % the right column
          \@makecolumn
          \ifnum\the@end>0\ifnum\pageno>1\balance@right@left\fi\fi
          \@outputpage \global\let\lr=L%
          \global\dimen\bothins=\maxdimen
          \global\count\bothins=1000
          \ifvoid\rightins\else
          %  Hold \rightins back if there is already a \@savetopins
             \ifvoid\@savetopins \insert\bothins{\unvbox\rightins}\fi
          \fi
       \fi
    \fi
    \global\let\last@insert=N \put@default
    \ifnum\outputpenalty>-\@MM\else\dosupereject\fi
    \ifvoid\@savetopins\else
      \ifdim\ht\@savetopins>\@txtht
        \global\setbox\@tempboxa=\box\@savetopins
        \global\setbox\@savetopins=\vsplit\@tempboxa to\@txtht
        \global\setbox\@savetopins=\vbox{\unvbox\@savetopins}%
        \global\setbox\@savetopright=\box\@tempboxa \fi
    \fi
    \@makecolht
    \global\count\topins=1000
   \fi
   }
%
%%% ----------  Start one- or two-column output  ---------
%
\if N\lr
   \setuplr{O}{\fullhsize}{\hsize}% O = one column
\else
   \setuplr{T}{\fullhsize}{\hsize}% T = two columns
\fi
% This is aafigure.tex
%
% The macros cover the case of single column format as well
% as double column format. In single column format all 'double
% column' insertions are reduced to single column insertions.
% While real double column insertions will always appear at
% the top of page, the placement of single column insertions
% can be influenced by the definition of \put@default.
% Possible positions are at the current position (only if
% \insert@here expands to Y, there is no top and no bottom insertion
% so far, and there is enough space), at the bottom of the column
% (only if \insert@at@the@bottom expands to Y and there is enough
% space in this column) and at the top of the column but below
% any double column insertion.
% The default placement of single column figures is usually at the top.
% If the figure almost fills the page if inserted at the bottom,
% it may go there.
%
\catcode`@=11
\def\put@default{\global\let\insert@here=Y
   \global\let\insert@at@the@bottom=N}%
% Allow figures to be inserted a the current position (if possible):
\def\puthere{\global\let\insert@here=Y%
    \global\let\insert@at@the@bottom=N}
% All figures are inserted at the top:
\def\putattop{\global\let\insert@here=N%
    \global\let\insert@at@the@bottom=N}
% Figures are inserted at the bottom (if possible):
\def\putatbottom{\global\let\insert@here=N%
    \global\let\insert@at@the@bottom=X}
%--------------------------------------------------------------------
\put@default
\let\last@insert=N        % Always reset to 'N' when a column is finished
\def\end@skip{\smallskip} % This space is added except after bottom insertions
\newdimen\min@top
\newdimen\min@here
\newdimen\min@bot
\min@top=10cm
\min@here=4cm   % do not insert figures after a few lines of text only
\min@bot=\topskip % figures may be at the bottom but there is a \@startins
\def\figfuzz{\vskip 0pt plus 6pt minus 3pt}  % more flexible spacing
%--------------------------------------------------------------------
\def\check@here@and@bottom#1{\relax
   %%% Several conditions have to be true if a figure or table can be
   %%% inserted at the current position or at the bottom of the page.
   %%% These conditions should preserve the
   %%% order of single column figures and put floating figures
   %%% always to the top of a column. However, exceptions are
   %%% possible with a \puthere\begfig{...}\endfig appearing at
   %%% the current position and a later \begfig{...}\endfig
   %%% appearing at the top of the same column.
   %
   \ifvoid\topins\else       \global\let\insert@here=N\fi
   \if B\last@insert         \global\let\insert@here=N\fi
   \if T\last@insert         \global\let\insert@here=N\fi
   \ifdim #1<\min@bot        \global\let\insert@here=N\fi
   \ifdim\pagetotal>\@colht  \global\let\insert@here=N\fi
   \ifdim\pagetotal<\min@here\global\let\insert@here=N\fi
   %
   \if X\insert@at@the@bottom\global\let\insert@at@the@bottom=Y
     \else\if T\last@insert  \global\let\insert@at@the@bottom=N\fi
          \if H\last@insert  \global\let\insert@at@the@bottom=N\fi
          \ifvoid\topins\else\global\let\insert@at@the@bottom=N\fi\fi
   \ifdim #1<\min@bot        \global\let\insert@at@the@bottom=N\fi
   \ifdim\pagetotal>\@colht  \global\let\insert@at@the@bottom=N\fi
   \ifdim\pagetotal<\min@top \global\let\insert@at@the@bottom=N\fi
   %
   \ifvoid\bottomins\else    \global\let\insert@at@the@bottom=Y\fi
   \if Y\insert@at@the@bottom\global\let\insert@here=N\fi }
%
\def\single@column@insert#1{\relax
   \setbox\@tempboxa=\vbox{#1}%
   \dimen@=\@colht \advance\dimen@ by -\pagetotal
   \advance\dimen@ by-\ht\@tempboxa \advance\dimen0 by-\dp\@tempboxa
   \advance\dimen@ by-\ht\topins \advance\dimen0 by-\dp\topins
   \check@here@and@bottom{\dimen@}%
   \if Y\insert@here
      \par  % The insertion forces a new paragraph in this case.
      \midinsert\figfuzz\relax     %%%%%%%%%\bigskip
      \box\@tempboxa\end@skip\figfuzz\endinsert
      \global\let\last@insert=H
   \else \if Y\insert@at@the@bottom
      \begingroup\insert\bottomins\bgroup\if B\last@insert\end@skip\fi
      \floatingpenalty=20000\figfuzz\bigskip\box\@tempboxa\egroup\endgroup
      \global\let\last@insert=B
   \else
      \topinsert\box\@tempboxa\end@skip\figfuzz\endinsert
      \global\let\last@insert=T
   \fi\fi\put@default\ignorespaces}
%
% ---------------- The insertion macros for the user -------------------
%
\def\begfig#1cm#2\endfig{\par\the\everypar
\single@column@insert{\@startins\rahmen{#1}#2}\ignorespaces}
\def\begfigwid#1cm#2\endfig{\par\the\everypar
   \if N\lr  % Here the only difference to \begfig is the larger \hsize
      {\hsize=\fullhsize \begfig#1cm#2\endfig}%
   \else
      \setbox0=\vbox{\hsize=\fullhsize\bigskip#2\smallskip}%
      \dimen0=\ht0\advance\dimen0 by\dp0
      \advance\dimen0 by#1cm
      \advance\dimen0by7\normalbaselineskip\relax
      \ifdim\dimen0>\@txtht
         \infuser{^^JFigure plus legend too high, will try to put it on a
                  separate page. }%
         \begfigpage#1cm#2\endfig
      \else
         \bothtopinsert\line{\vbox{\hsize=\fullhsize
         \@startins\rahmen{#1}#2\smallskip}\hss}\figfuzz\endbothinsert
      \fi
   \fi}
%
\def\begfigside#1cm#2cm#3\endfig{\par\the\everypar
   \if N\lr  % Here the only difference to \begfig is the larger \hsize
      {\hsize=\fullhsize \begfig#1cm#3\endfig}%
   \else
      \dimen0=#2true cm\relax
      \ifdim\dimen0<\hsize
         \infuser{^^JYour figure fits in a single column; why don't
                  ^^Jyou use \string\begfig\space instead of
                  \string\begfigside? }%
      \fi
      \dimen0=\fullhsize
      \advance\dimen0 by-#2true cm
      \advance\dimen0 by-1true cc\relax
      \bgroup
         \ifdim\dimen0<8true cc\relax
            \infuser{^^JNo sufficient room for the legend;
                     using \string\begfigwid. }%
            \begfigwid #1cm#3\endfig
         \else
            \ifdim\dimen0<10true cc\relax
               \infuser{^^JRoom for legend to narrow;
                        legend will be set raggedright. }%
               \rightskip=0pt plus 2cm\relax
            \fi
            \setbox0=\vbox{\def\figure##1##2{\vbox{\hsize=\dimen0\relax
                           \@startins\noindent\petit{\bf
                           Fig.\ts##1\unskip.\ }\ignorespaces##2\par}}%
                           #3\unskip}%
            \ifdim#1true cm<\ht0\relax
               \infuser{^^JText of legend higher than figure; using
                        \string\begfig. }%
               \begfigwid #1cm#3\endfig
            \else
               \def\figure##1##2{\vbox{\hsize=\dimen0\relax
                                       \@startins\noindent\petit{\bf
                                       Fig.\ts##1\unskip.\
                                       }\ignorespaces##2\par}}%
               \bothtopinsert\line{\vbox{\hsize=#2true cm\relax
               \@startins\rahmen{#1}}\hss#3\unskip}\figfuzz\endbothinsert
            \fi
         \fi
      \egroup
   \fi\ignorespaces}
%
\def\begfigpage#1cm#2\endfig{\par\the\everypar\specialpage{\@startins
   \vskip3.7pt\rahmen{#1}#2}\ignorespaces}%
\def\begtab#1cm#2\endtab{\par\the\everypar
\single@column@insert{#2\rahmen{#1}}\ignorespaces}
\let\begtabempty=\begtab
\def\begtabfull#1\endtab{\par\the\everypar
\single@column@insert{#1}\ignorespaces}
\def\begtabemptywid#1cm#2\endtab{\par\the\everypar
   \if N\lr
      {\hsize=\fullhsize \begtabempty#1cm#2\endtab}%
   \else
      \bothtopinsert\line{\vbox{\hsize=\fullhsize
      #2\rahmen{#1}}\hss}\medskip\endbothinsert
   \fi\ignorespaces}
\def\begtabfullwid#1\endtab{\par\the\everypar
   \if N\lr
      {\hsize=\fullhsize \begtabfull#1\endtab}%
   \else
      \bothtopinsert\line{\vbox{\hsize=\fullhsize
      \noindent#1}\hss}\medskip\endbothinsert
   \fi\ignorespaces}
\def\begtabpage#1\endtab{\par\the\everypar\specialpage{#1}\ignorespaces}
\catcode`@=12
% This is startup.aa
% It defines additional macros and settings that are special for
% the AA package or that have to be changed in the macros
% already loaded. A control sequence \firsttodo is defined that is
% called first after TeX starts up and is executed by \everyjob for
% a format file use of the package.
%
\catcode`@=11 % use @ as a normal character
%
\def\SpringerMacroPackageNameA{AA}
%
\edef\firsttodo{%
\infuser{^^JThis is \SpringerMacroPackageNameA, the plain TeX macro
package from Springer-Verlag,^^Jfor the Astronomy and Astrophysics
Main Journal
\ifx\Initexing\undefined\else format file \fi
\fonttype\space version 3.0^^J}%
\catcode`@=\active   % This is reset by the \maketitle macro
}
%
\everyjob={\firsttodo}% set the startup variable of TeX
\firsttodo % (do things that are first to do :-)
\catcode`@=12 % at signs are no longer letters
%%% rest removed from cp-aa.ini
%% This is dodump.aa
%% it clears the first page possibly filled with junk material
%% adjusts the page counter and calls INITEX's \dump
%\pageno=0
%\let\header=N%
%\leavevmode
%\vfil
%You just produced a \TeX{} format file for the \fonttype\space
%version of the {\tt\SpringerMacroPackageNameA} macro package from
%Springer-Verlag.
%\bigskip
%\vfil
%\eject
%\let\Initexing=\undefined
%\dump
%.
