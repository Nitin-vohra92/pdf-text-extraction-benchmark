% file: mtexsis.tex -- TeXsis version 2.16
% $Revision: 16.2 $  :  $Date: 1994/11/12 22:32:02 $  :  $Author: myers $
%======================================================================*
%  mTeXsis - mini version of the TeX Macros for Physicists
%
%  This is mtexsis.tex, a "mini" version of the TeXsis macro package.
%  mtexsis.tex contains only the core macros of TeXsis, and there are
%  absolutely no comments (except these).  mtexsis.tex is convenient for
%  printing papers prepared with the TeXsis macros when you don't have
%  TeXsis installed on your system.  If you receive a manuscript file
%  prepared with TeXsis you can (usually) print it by inserting the line
%
%       \input mtexsis
%
%  at the begining of the manuscript file (if it's not already there)
%  and then running it through Plain TeX.
%
%  Authors who prepare papers with TeXsis for submission to physics
%  bulletin boards like hep-lat, hep-th or hep-ph may want to try
%  their manuscripts with mtexsis.tex and Plain TeX to be sure that
%  it works.  
%
%  If you want to know how a macro works you will have to look in the
%  full TeXsis source files.  Both the source and documentation for
%  TeXsis are available via anonymous ftp from lifshitz.ph.utexas.edu
%  in the /texsis subdirectory.  See the README file there for more
%  information.
%
%  Comments, suggestions, or bug reports should be directed to 
%  texsis@lifshitz.ph.utexas.edu
%
%  (C) copyright 1988, 1992, 1994 by Eric Myers and Frank Paige.
%  Permission is granted to use this software by any member the academic
%  community in conjunction with basic research.  This software may not
%  be used or distributed for profit without permission from the authors.
%======================================================================*
\def\TeXsis{\TeX sis}%                          % the TeXsis logo
\catcode`@=11                                   % @ is a letter in what follows
% file: TXSfonts.tex                                TeXsis version 2.16
%%> {TeXsis fonts:}
\catcode`@=11
\newskip\ttglue
%++ 9 pt fonts:
\def\ninefonts{%
   \global\font\ninerm=cmr9
   \global\font\ninei=cmmi9
   \global\font\ninesy=cmsy9
   \global\font\nineex=cmex10
   \global\font\ninebf=cmbx9
   \global\font\ninesl=cmsl9
   \global\font\ninett=cmtt9
   \global\font\nineit=cmti9
   \skewchar\ninei='177
   \skewchar\ninesy='60
   \hyphenchar\ninett=-1
   \moreninefonts
   \gdef\ninefonts{\relax}}
\def\moreninefonts{\relax}%
%++ 10 pt fonts:                               
\font\tenss=cmss10
\def\moretenfonts{\relax}%
%++ 11 pt fonts:
\def\elevenfonts{%
   \global\font\elevenrm=cmr10 scaled \magstephalf
   \global\font\eleveni=cmmi10 scaled \magstephalf
   \global\font\elevensy=cmsy10 scaled \magstephalf
   \global\font\elevenex=cmex10
   \global\font\elevenbf=cmbx10 scaled \magstephalf
   \global\font\elevensl=cmsl10 scaled \magstephalf
   \global\font\eleventt=cmtt10 scaled \magstephalf
   \global\font\elevenit=cmti10 scaled \magstephalf
   \global\font\elevenss=cmss10 scaled \magstephalf
   \skewchar\eleveni='177%
   \skewchar\elevensy='60%
   \hyphenchar\eleventt=-1%
   \moreelevenfonts
   \gdef\elevenfonts{\relax}}%
\def\moreelevenfonts{\relax}%
%++ 12 pt fonts:
\def\twelvefonts{%
   \global\font\twelverm=cmr10 scaled \magstep1%
   \global\font\twelvei=cmmi10 scaled \magstep1%
   \global\font\twelvesy=cmsy10 scaled \magstep1%
   \global\font\twelveex=cmex10 scaled \magstep1%
   \global\font\twelvebf=cmbx10 scaled \magstep1%
   \global\font\twelvesl=cmsl10 scaled \magstep1%
   \global\font\twelvett=cmtt10 scaled \magstep1%
   \global\font\twelveit=cmti10 scaled \magstep1%
   \global\font\twelvess=cmss10 scaled \magstep1%
   \skewchar\twelvei='177%
   \skewchar\twelvesy='60%
   \hyphenchar\twelvett=-1%
   \moretwelvefonts
   \gdef\twelvefonts{\relax}}
\def\moretwelvefonts{\relax}%
%++ 14 pt fonts:
\def\fourteenfonts{%
   \global\font\fourteenrm=cmr10 scaled \magstep2%
   \global\font\fourteeni=cmmi10 scaled \magstep2%
   \global\font\fourteensy=cmsy10 scaled \magstep2%
   \global\font\fourteenex=cmex10 scaled \magstep2%
   \global\font\fourteenbf=cmbx10 scaled \magstep2%
   \global\font\fourteensl=cmsl10 scaled \magstep2%
   \global\font\fourteenit=cmti10 scaled \magstep2%
   \global\font\fourteenss=cmss10 scaled \magstep2%
   \skewchar\fourteeni='177%
   \skewchar\fourteensy='60%
   \morefourteenfonts
   \gdef\fourteenfonts{\relax}}
\def\morefourteenfonts{\relax}%
%++ 16 pt fonts:
\def\sixteenfonts{%
   \global\font\sixteenrm=cmr10 scaled \magstep3%
   \global\font\sixteeni=cmmi10 scaled \magstep3%
   \global\font\sixteensy=cmsy10 scaled \magstep3%
   \global\font\sixteenex=cmex10 scaled \magstep3%
   \global\font\sixteenbf=cmbx10 scaled \magstep3%
   \global\font\sixteensl=cmsl10 scaled \magstep3%
   \global\font\sixteenit=cmti10 scaled \magstep3%
   \skewchar\sixteeni='177%
   \skewchar\sixteensy='60%
   \moresixteenfonts
   \gdef\sixteenfonts{\relax}}
\def\moresixteenfonts{\relax}%
%++ 20 pt fonts:
\def\twentyfonts{%
   \global\font\twentyrm=cmr10 scaled \magstep4%
   \global\font\twentyi=cmmi10 scaled \magstep4%
   \global\font\twentysy=cmsy10 scaled \magstep4%
   \global\font\twentyex=cmex10 scaled \magstep4%
   \global\font\twentybf=cmbx10 scaled \magstep4%
   \global\font\twentysl=cmsl10 scaled \magstep4%
   \global\font\twentyit=cmti10 scaled \magstep4%
   \skewchar\twentyi='177%
   \skewchar\twentysy='60%
   \moretwentyfonts
   \gdef\twentyfonts{\relax}}
\def\moretwentyfonts{\relax}%
%++ 24 pt fonts:
\def\twentyfourfonts{%
   \global\font\twentyfourrm=cmr10 scaled \magstep5%
   \global\font\twentyfouri=cmmi10 scaled \magstep5%
   \global\font\twentyfoursy=cmsy10 scaled \magstep5%
   \global\font\twentyfourex=cmex10 scaled \magstep5%
   \global\font\twentyfourbf=cmbx10 scaled \magstep5%
   \global\font\twentyfoursl=cmsl10 scaled \magstep5%
   \global\font\twentyfourit=cmti10 scaled \magstep5%
   \skewchar\twentyfouri='177%
   \skewchar\twentyfoursy='60%
   \moretwentyfourfonts
   \gdef\twentyfourfonts{\relax}}
\def\moretwentyfourfonts{\relax}%
\def\tenmibfonts{%
   \global\font\tenmib=cmmib10
   \global\font\tenbsy=cmbsy10
   \skewchar\tenmib='177%
   \skewchar\tenbsy='60%
   \gdef\tenmibfonts{\relax}}
\def\elevenmibfonts{%
   \global\font\elevenmib=cmmib10 scaled \magstephalf
   \global\font\elevenbsy=cmbsy10 scaled \magstephalf
   \skewchar\elevenmib='177%
   \skewchar\elevenbsy='60%
   \gdef\elevenmibfonts{\relax}}
\def\twelvemibfonts{%
   \global\font\twelvemib=cmmib10 scaled \magstep1%
   \global\font\twelvebsy=cmbsy10 scaled \magstep1%
   \skewchar\twelvemib='177%
   \skewchar\twelvebsy='60%
   \gdef\twelvemibfonts{\relax}}
\def\fourteenmibfonts{%
   \global\font\fourteenmib=cmmib10 scaled \magstep2%
   \global\font\fourteenbsy=cmbsy10 scaled \magstep2%
   \skewchar\fourteenmib='177%
   \skewchar\fourteenbsy='60%
   \gdef\fourteenmibfonts{\relax}}
\def\sixteenmibfonts{%
   \global\font\sixteenmib=cmmib10 scaled \magstep3%
   \global\font\sixteenbsy=cmbsy10 scaled \magstep3%
   \skewchar\sixteenmib='177%
   \skewchar\sixteenbsy='60%
   \gdef\sixteenmibfonts{\relax}}
\def\twentymibfonts{%
   \global\font\twentymib=cmmib10 scaled \magstep4%
   \global\font\twentybsy=cmbsy10 scaled \magstep4%
   \skewchar\twentymib='177%
   \skewchar\twentybsy='60%
   \gdef\twentymibfonts{\relax}}
\def\twentyfourmibfonts{%
   \global\font\twentyfourmib=cmmib10 scaled \magstep5%
   \global\font\twentyfourbsy=cmbsy10 scaled \magstep5%
   \skewchar\twentyfourmib='177%
   \skewchar\twentyfourbsy='60%
   \gdef\twentyfourmibfonts{\relax}}
%%> {\noexpand\mib,}
\def\mib{%
   \tenmibfonts
   \textfont0=\tenbf\scriptfont0=\sevenbf
   \scriptscriptfont0=\fivebf
   \textfont1=\tenmib\scriptfont1=\seveni
   \scriptscriptfont1=\fivei
   \textfont2=\tenbsy\scriptfont2=\sevensy
   \scriptscriptfont2=\fivesy}
%%> {\noexpand\scr,}
\newfam\scrfam
\def\scr{\scrfonts
   \global\textfont\scrfam=\tenscr
   \fam\scrfam\tenscr}
\def\scrfonts{%
   \global\font\twentyfourscr=rsfs10  scaled \magstep5
   \global\font\twentyscr=rsfs10  scaled \magstep4
   \global\font\sixteenscr=rsfs10  scaled \magstep3
   \global\font\fourteenscr=rsfs10  scaled \magstep2
   \global\font\twelvescr=rsfs10  scaled \magstep1
   \global\font\elevenscr=rsfs10  scaled \magstephalf
   \global\font\tenscr=rsfs10
   \global\font\sevenscr=rsfs7
   \global\font\fivescr=rsfs5
   \skewchar\tenscr='177 \skewchar\sevenscr='177 \skewchar\fivescr='177%
   \textfont\scrfam=\tenscr \scriptfont\scrfam=\sevenscr
   \scriptscriptfont\scrfam=\fivescr
   \gdef\scrfonts{\relax}}
%%> {9pt,}
\def\ninepoint{\ninefonts
   \def\rm{\fam0\ninerm}%
   \textfont0=\ninerm\scriptfont0=\sevenrm\scriptscriptfont0=\fiverm
   \textfont1=\ninei\scriptfont1=\seveni\scriptscriptfont1=\fivei
   \textfont2=\ninesy\scriptfont2=\sevensy\scriptscriptfont2=\fivesy
   \textfont3=\nineex\scriptfont3=\nineex\scriptscriptfont3=\nineex
   \textfont\itfam=\nineit\def\it{\fam\itfam\nineit}%
   \textfont\slfam=\ninesl\def\sl{\fam\slfam\ninesl}%
   \textfont\ttfam=\ninett\def\tt{\fam\ttfam\ninett}%
   \textfont\bffam=\ninebf
   \scriptfont\bffam=\sevenbf
   \scriptscriptfont\bffam=\fivebf\def\bf{\fam\bffam\ninebf}%
   \def\mib{\relax}%
   \def\scr{\relax}%
   \tt\ttglue=.5emplus.25emminus.15em
   \normalbaselineskip=11pt
   \setbox\strutbox=\hbox{\vrule height 8pt depth 3pt width 0pt}%
   \normalbaselines\rm\singlespaced}%
%%> {10pt,}
\def\tenpoint{%
   \def\rm{\fam0\tenrm}%
   \textfont0=\tenrm\scriptfont0=\sevenrm\scriptscriptfont0=\fiverm
   \textfont1=\teni\scriptfont1=\seveni\scriptscriptfont1=\fivei
   \textfont2=\tensy\scriptfont2=\sevensy\scriptscriptfont2=\fivesy
   \textfont3=\tenex\scriptfont3=\tenex\scriptscriptfont3=\tenex
   \textfont\itfam=\tenit\def\it{\fam\itfam\tenit}%
   \textfont\slfam=\tensl\def\sl{\fam\slfam\tensl}%
   \textfont\ttfam=\tentt\def\tt{\fam\ttfam\tentt}%
   \textfont\bffam=\tenbf
   \scriptfont\bffam=\sevenbf
   \scriptscriptfont\bffam=\fivebf\def\bf{\fam\bffam\tenbf}%
   \def\mib{%
      \tenmibfonts
      \textfont0=\tenbf\scriptfont0=\sevenbf
      \scriptscriptfont0=\fivebf
      \textfont1=\tenmib\scriptfont1=\seveni
      \scriptscriptfont1=\fivei
      \textfont2=\tenbsy\scriptfont2=\sevensy
      \scriptscriptfont2=\fivesy}%
   \def\scr{\scrfonts
      \global\textfont\scrfam=\tenscr\fam\scrfam\tenscr}%
   \tt\ttglue=.5emplus.25emminus.15em
   \normalbaselineskip=12pt
   \setbox\strutbox=\hbox{\vrule height 8.5pt depth 3.5pt width 0pt}%
   \normalbaselines\rm\singlespaced}%
%%> {11pt,}
\def\elevenpoint{\elevenfonts
   \def\rm{\fam0\elevenrm}%
   \textfont0=\elevenrm\scriptfont0=\sevenrm\scriptscriptfont0=\fiverm
   \textfont1=\eleveni\scriptfont1=\seveni\scriptscriptfont1=\fivei
   \textfont2=\elevensy\scriptfont2=\sevensy\scriptscriptfont2=\fivesy
   \textfont3=\elevenex\scriptfont3=\elevenex\scriptscriptfont3=\elevenex
   \textfont\itfam=\elevenit\def\it{\fam\itfam\elevenit}%
   \textfont\slfam=\elevensl\def\sl{\fam\slfam\elevensl}%
   \textfont\ttfam=\eleventt\def\tt{\fam\ttfam\eleventt}%
   \textfont\bffam=\elevenbf
   \scriptfont\bffam=\sevenbf
   \scriptscriptfont\bffam=\fivebf\def\bf{\fam\bffam\elevenbf}%
   \def\mib{%
      \elevenmibfonts
      \textfont0=\elevenbf\scriptfont0=\sevenbf
      \scriptscriptfont0=\fivebf
      \textfont1=\elevenmib\scriptfont1=\seveni
      \scriptscriptfont1=\fivei
      \textfont2=\elevenbsy\scriptfont2=\sevensy
      \scriptscriptfont2=\fivesy}%
   \def\scr{\scrfonts
      \global\textfont\scrfam=\elevenscr\fam\scrfam\elevenscr}%
   \tt\ttglue=.5emplus.25emminus.15em
   \normalbaselineskip=13pt
   \setbox\strutbox=\hbox{\vrule height 9pt depth 4pt width 0pt}%
   \normalbaselines\rm\singlespaced}%
%%> {12pt,}
\def\twelvepoint{\twelvefonts\ninefonts
   \def\rm{\fam0\twelverm}%
   \textfont0=\twelverm\scriptfont0=\ninerm\scriptscriptfont0=\sevenrm
   \textfont1=\twelvei\scriptfont1=\ninei\scriptscriptfont1=\seveni
   \textfont2=\twelvesy\scriptfont2=\ninesy\scriptscriptfont2=\sevensy
   \textfont3=\twelveex\scriptfont3=\twelveex\scriptscriptfont3=\twelveex
   \textfont\itfam=\twelveit\def\it{\fam\itfam\twelveit}%
   \textfont\slfam=\twelvesl\def\sl{\fam\slfam\twelvesl}%
   \textfont\ttfam=\twelvett\def\tt{\fam\ttfam\twelvett}%
   \textfont\bffam=\twelvebf
   \scriptfont\bffam=\ninebf
   \scriptscriptfont\bffam=\sevenbf\def\bf{\fam\bffam\twelvebf}%
   \def\mib{%
      \twelvemibfonts\tenmibfonts
      \textfont0=\twelvebf\scriptfont0=\ninebf
      \scriptscriptfont0=\sevenbf
      \textfont1=\twelvemib\scriptfont1=\ninei
      \scriptscriptfont1=\seveni
      \textfont2=\twelvebsy\scriptfont2=\ninesy
      \scriptscriptfont2=\sevensy}%
   \def\scr{\scrfonts
      \global\textfont\scrfam=\twelvescr\fam\scrfam\twelvescr}%
   \tt\ttglue=.5emplus.25emminus.15em
   \normalbaselineskip=14pt
   \setbox\strutbox=\hbox{\vrule height 10pt depth 4pt width 0pt}%
   \normalbaselines\rm\singlespaced}%
%%> {14pt,}
\def\fourteenpoint{\fourteenfonts\twelvefonts
   \def\rm{\fam0\fourteenrm}%
   \textfont0=\fourteenrm\scriptfont0=\twelverm\scriptscriptfont0=\tenrm
   \textfont1=\fourteeni\scriptfont1=\twelvei\scriptscriptfont1=\teni
   \textfont2=\fourteensy\scriptfont2=\twelvesy\scriptscriptfont2=\tensy
   \textfont3=\fourteenex\scriptfont3=\fourteenex
      \scriptscriptfont3=\fourteenex
   \textfont\itfam=\fourteenit\def\it{\fam\itfam\fourteenit}%
   \textfont\slfam=\fourteensl\def\sl{\fam\slfam\fourteensl}%
   \textfont\bffam=\fourteenbf
   \scriptfont\bffam=\twelvebf
   \scriptscriptfont\bffam=\tenbf\def\bf{\fam\bffam\fourteenbf}%
   \def\mib{%
      \fourteenmibfonts\twelvemibfonts\tenmibfonts
      \textfont0=\fourteenbf\scriptfont0=\twelvebf
      \scriptscriptfont0=\tenbf
      \textfont1=\fourteenmib\scriptfont1=\twelvemib
      \scriptscriptfont1=\tenmib
      \textfont2=\fourteenbsy\scriptfont2=\tenbsy
      \scriptscriptfont2=\tenbsy}%
   \def\scr{\scrfonts
      \global\textfont\scrfam=\fourteenscr\fam\scrfam\fourteenscr}%
   \normalbaselineskip=17pt
   \setbox\strutbox=\hbox{\vrule height 12pt depth 5pt width 0pt}%
   \normalbaselines\rm\singlespaced}%
%%> {16pt,}
\def\sixteenpoint{\sixteenfonts\fourteenfonts\twelvefonts
   \def\rm{\fam0\sixteenrm}%
   \textfont0=\sixteenrm\scriptfont0=\fourteenrm\scriptscriptfont0=\twelverm
   \textfont1=\sixteeni\scriptfont1=\fourteeni\scriptscriptfont1=\twelvei
   \textfont2=\sixteensy\scriptfont2=\fourteensy\scriptscriptfont2=\twelvesy
   \textfont3=\sixteenex\scriptfont3=\sixteenex\scriptscriptfont3=\sixteenex
   \textfont\itfam=\sixteenit\def\it{\fam\itfam\sixteenit}%
   \textfont\slfam=\sixteensl\def\sl{\fam\slfam\sixteensl}%
   \textfont\bffam=\sixteenbf
   \scriptfont\bffam=\fourteenbf
   \scriptscriptfont\bffam=\twelvebf\def\bf{\fam\bffam\sixteenbf}%
   \def\mib{%
      \sixteenmibfonts\fourteenmibfonts\twelvemibfonts
      \textfont0=\sixteenbf\scriptfont0=\fourteenbf
      \scriptscriptfont0=\twelvebf
      \textfont1=\sixteenmib\scriptfont1=\fourteenmib
      \scriptscriptfont1=\twelvemib
      \textfont2=\sixteenbsy\scriptfont2=\fourteenbsy
      \scriptscriptfont2=\twelvebsy}%
   \def\scr{\scrfonts
      \global\textfont\scrfam=\sixteenscr\fam\scrfam\sixteenscr}%
   \normalbaselineskip=20pt
   \setbox\strutbox=\hbox{\vrule height 14pt depth 6pt width 0pt}%
   \normalbaselines\rm\singlespaced}%
%%> {20pt,}
\def\twentypoint{\twentyfonts\sixteenfonts\fourteenfonts
   \def\rm{\fam0\twentyrm}%
   \textfont0=\twentyrm\scriptfont0=\sixteenrm\scriptscriptfont0=\fourteenrm
   \textfont1=\twentyi\scriptfont1=\sixteeni\scriptscriptfont1=\fourteeni
   \textfont2=\twentysy\scriptfont2=\sixteensy\scriptscriptfont2=\fourteensy
   \textfont3=\twentyex\scriptfont3=\twentyex\scriptscriptfont3=\twentyex
   \textfont\itfam=\twentyit\def\it{\fam\itfam\twentyit}%
   \textfont\slfam=\twentysl\def\sl{\fam\slfam\twentysl}%
   \textfont\bffam=\twentybf
   \scriptfont\bffam=\sixteenbf
   \scriptscriptfont\bffam=\fourteenbf\def\bf{\fam\bffam\twentybf}%
   \def\mib{%
      \twentymibfonts\sixteenmibfonts\fourteenmibfonts
      \textfont0=\twentybf\scriptfont0=\sixteenbf
      \scriptscriptfont0=\fourteenbf
      \textfont1=\twentymib\scriptfont1=\sixteenmib
      \scriptscriptfont1=\fourteenmib
      \textfont2=\twentybsy\scriptfont2=\sixteenbsy
      \scriptscriptfont2=\fourteenbsy}%
   \def\scr{\scrfonts
      \global\textfont\scrfam=\twentyscr\fam\scrfam\twentyscr}%
   \normalbaselineskip=24pt
   \setbox\strutbox=\hbox{\vrule height 17pt depth 7pt width 0pt}%
   \normalbaselines\rm\singlespaced}%
%%> {24pt.}
\def\twentyfourpoint{\twentyfourfonts\twentyfonts\sixteenfonts
   \def\rm{\fam0\twentyfourrm}%
   \textfont0=\twentyfourrm\scriptfont0=\twentyrm\scriptscriptfont0=\sixteenrm
   \textfont1=\twentyfouri\scriptfont1=\twentyi\scriptscriptfont1=\sixteeni
   \textfont2=\twentyfoursy\scriptfont2=\twentysy\scriptscriptfont2=\sixteensy
   \textfont3=\twentyfourex\scriptfont3=\twentyfourex
      \scriptscriptfont3=\twentyfourex
   \textfont\itfam=\twentyfourit\def\it{\fam\itfam\twentyfourit}%
   \textfont\slfam=\twentyfoursl\def\sl{\fam\slfam\twentyfoursl}%
   \textfont\bffam=\twentyfourbf
   \scriptfont\bffam=\twentybf
   \scriptscriptfont\bffam=\sixteenbf\def\bf{\fam\bffam\twentyfourbf}%
   \def\mib{%
      \twentyfourmibfonts\twentymibfonts\sixteenmibfonts
      \textfont0=\twentyfourbf\scriptfont0=\twentybf
      \scriptscriptfont0=\sixteenbf
      \textfont1=\twentyfourmib\scriptfont1=\twentymib
      \scriptscriptfont1=\sixteenmib
      \textfont2=\twentyfourbsy\scriptfont2=\twentybsy
      \scriptscriptfont2=\sixteenbsy}%
   \def\scr{\scrfonts
      \global\textfont\scrfam=\twentyfourscr\fam\scrfam\twentyfourscr}%
   \normalbaselineskip=28pt
   \setbox\strutbox=\hbox{\vrule height 19pt depth 9pt width 0pt}%
   \normalbaselines\rm\singlespaced}%
\def\Tbf{\fourteenpoint\bf}
\def\tbf{\twelvepoint\bf}
\def\printfont{\autoload\printfont{printfont.txs}\printfont}
% file: TXSmacs.tex  (TeXsis version 2.16)
%%> {TeXsis main macros.}
\catcode`@=11
\let\XA=\expandafter
\let\NX=\noexpand
\def\emsg#1{\relax
   \begingroup
     \def\@quote{"}%
     \def\TeX{TeX}\def\label##1{}\def\use{\string\use}%
     \def\ { }\def~{ }%
     \def\tt{\string\tt}\def\bf{}\def\Tbf{}\def\tbf{}%
     \def\break{}\def\n{\string\n}%
     \immediate\write16{#1}%
   \endgroup}
\newif\ifmarkerrors     \markerrorsfalse
\def\@errmark#1{\ifmarkerrors
   \vadjust{\vbox to 0pt{%
   \kern-\baselineskip
   \line{\hfil\rlap{{\tt\ <-#1}}}%
   \vss}}\fi}%
\def\draft{%
   \def\draftline{{\tentt Preliminary Draft \hfill
         -~\folio~- \hfill \runtime}}\footline={\draftline}%
   \eqnotracetrue
   \markerrorstrue
   \overfullrule=1em}%
\def\bye{%
    \endmode
    \par\vfill\supereject
    \checktags
    \end}
\def\endmode{\relax}%
\def\checktags{\relax}%
\def\setTableskip{\relax}%
\def\singlespaced{%
   \baselineskip=\normalbaselineskip
   \setRuledStrut
   \setTableskip}%
\def\singlespace{\singlespaced}%
\def\doublespaced{%
   \baselineskip=\normalbaselineskip
   \multiply\baselineskip by 150
   \divide\baselineskip by 100
   \setRuledStrut
   \setTableskip}%
\def\doublespace{\doublespaced}%
\def\TrueDoubleSpacing{%
   \baselineskip=\normalbaselineskip
   \multiply\baselineskip by 2
   \setRuledStrut
   \setTableskip}%
\def\truedoublespacing{\TrueDoubleSpacing}%
\def\triplespaced{%
   \baselineskip=\normalbaselineskip
   \multiply\baselineskip by 3
   \setRuledStrut
   \setTableskip}%
\def\widenspacing{%
   \multiply\baselineskip by 125
   \divide\baselineskip by 100
   \setRuledStrut
   \setTableskip}%
\def\whitespaced{\widenspacing}%
\def\whitespace{\widenspacing}%
\def\Footnote#1{%
   \let\@sf\empty
   \ifhmode\edef\@sf{\spacefactor\the\spacefactor}\/\fi
   ${}^{\scriptstyle\smash{#1}}$\@sf
   \Vfootnote{#1}}%
\def\Vfootnote#1{%
   \begingroup
     \def\@foot{\strut\egroup\endgroup}%
     \tenpoint
     \baselineskip=\normalbaselineskip
     \parskip=0pt
     \FootFont
     \vfootnote{${}^{\hbox{#1}}$}}%
\def\FootFont{\rm}%
\newcount\footnum \footnum=0
\let\footnotemark=\empty
\def\NFootnote{%
  \advance\footnum by 1
  \xdef\lab@l{\the\footnum}%
  \Footnote{\footnotemark\the\footnum}}
\def\setRuledStrut{\relax}              
\def\undertext#1{$\relax\underline{\hbox{#1}}$}
\def\n{\hfil\break}%
\def\nl{\hfil\break}%
\def\newpage{\vfill\eject}%
\def\Afour{\hsize=16.5cm \vsize=24.5cm\relax}%
\def\unobeylines{\catcode`\^^M=5}%
{\obeyspaces\gdef\unobeyspaces{\catcode`\ =10}}
\def\unraggedright{\rightskip=\z@\spaceskip=0pt\xspaceskip=0pt}
\catcode`\"=\active
\newcount\@quoteflag   \@quoteflag=\z@
\def"{\@quote}%
\def\@quote{%
   \ifnum\@quoteflag=\z@
     \@quoteflag=\@ne {``}%
   \else
     \@quoteflag=\z@ {''}%
   \fi}
\def\quoteon{\catcode`\"=\active}%
\def\quoteoff{\catcode`\"=12}%
\def\@checkquote#1{\ifnum\@quoteflag=\@ne\message{#1}\fi}
\quoteoff
\def\checkquote{{\quoteoff\@checkquote{> Unbalanced "}}}%
\def\obsolete#1#2{\def#1{\@obsolete#1#2}}
\def\@obsolete#1#2{%
   \emsg{>=========================================================}%
   \emsg{> \string#1 is now obsolete! It may soon disappear!}%
   \emsg{> Please use \string#2 instead.  But I'll try to do it anyway...}%
   \emsg{>=========================================================}%
   \let#1=#2\relax
   #2}%
\let\DUMP=\dump
\def\setcnt#1#2{%
  \edef\th@value{\the#1}%
  \aftergroup\global\aftergroup#1
  \aftergroup=\relax
  \XA\@ftergroup\th@value\endafter
  \global#1=#2\relax}%
\def\@ftergroup{\futurelet\next\@ftertoken} 
\long\def\@ftertoken#1{
   \ifx\next\endafter\relax
     \let\next=\relax
   \else\aftergroup#1\relax
     \let\next=\@ftergroup
   \fi\next}%
\def\ATlock{\catcode`@=12\relax}%
\def\ATunlock{\catcode`@=11\relax}%
\newhelp\AThelp{@: 
You've apparantly tried to use a macro which begins with ``@''.^^J
These macros are usually for internal TeXsis functions and should^^J
not be used casually.  If you really want to use the macro try first^^J
saying \string\ATunlock.  If you got this message by pure accident^^J
then something else is wrong.} 
\def\@{\begingroup
    \errhelp=\AThelp
    \newlinechar=10
    \errmessage{Are you tring to use an internal @-macro?}\relax
   \endgroup}
\def~{\ifmmode\phantom{0}\else\penalty10000\ \fi}%
\def\0{\phantom{0}}%
\def\,{\relax\ifmmode\mskip\the\thinmuskip\else\thinspace\fi}
\def\topspace{\hrule height \z@\vskip}
\long\def\comment#1/*#2*/{\relax}%
\long\def\Ignore#1\endIgnore{\relax}%
\def\endIgnore{\relax}%
{\catcode`\%=11 \gdef\@comment{% }}
\def\REV{\begingroup
   \def\endcomment{\endgroup}%
   \catcode`\|=12
   \catcode`(=12 \catcode`)=12
   \catcode`[=12 \catcode`]=12
   \comment}%
\def\begin#1{%
   \begingroup
     \let\end=\endbegin
     \expandafter\ifx\csname #1\endcsname\relax\relax
        \def\next{\beginerror{#1}}%
     \else
        \def\next{\csname #1\endcsname}%
     \fi\next}
\def\endbegin#1{%
   \endgroup
   \expandafter\ifx\csname end#1\endcsname\relax\relax
      \def\next{\begingroup\beginerror{end#1}}%
   \else
      \def\next{\csname end#1\endcsname}%
   \fi\next}
\newhelp\beginhelp{begin: 
    The \string\begin\space or \string\end\space marked above is for a
    non-existant^^J
    environment.  Check for spelling errors and such.}
\def\beginerror#1{%
   \endgroup
   \errhelp=\beginhelp
   \newlinechar=10
   \errmessage{Undefined environment for \string\begin\space or \string\end}}
\def\spine#1{\spineSetup
%   \headline={\LandscapeSpecial}%
   \setbox0=\hbox{\twentyfourpoint\bf\quad #1\quad}%
   \dimen0=\wd0
   \ifdim\wd0<15cm \dimen0=15cm\fi
   \vskip 1cm plus 2cm
   \tightboxit{\vbox to 1in{\vfill
     \hbox to \dimen0{\hfil\box0\hfil}%
     \vfill}}%
   \vskip 1cm plus 2cm}%
\def\spineSetup{%
   \LandscapeSpecial
   \vbox{\hsize=15cm
     \bigskip\bigskip
     \tenpoint\parskip=\smallskipamount
     \noindent
     How to install the spine label on a binder:
     \item{1.}
     Cut out the label around the outside of the 
     bounding ruled box.  Leave extra paper
     (about 1.5cm or so) on the RIGHT side.
     \item{2.}
     Obtain a long thin plastic ruler or similar object which can 
     slide down the pocket on the binder spine.
     \item{3.}
     Put the label along the ruler, face up, and fold the extra tab 
     of paper on the right of the label over the end of the ruler.
     \item{4.}
     Slide the ruler, with the label on it, down the spine of the 
     binder.  When the label is in place remove the ruler.  The label 
     will remain.
     \bigskip\bigskip
   \gdef\spineSetup{\relax}%
   }}
\def\\{\global\let\@stoken= }\\ 
\long\def\unexpandedwrite#1#2{%
   \def\@finwrite{\immediate\write#1}%
   \begingroup
    \aftergroup\@finwrite
    \aftergroup{\relax
    \@sanitize#2\endsanity
    \aftergroup}\relax
   \endgroup
 }
\long\def\writeNX#1#2{%
   \def\@finwrite{\write#1}%
   \begingroup
    \aftergroup\@finwrite
    \aftergroup{\relax
    \@sanitize#2\endsanity
    \aftergroup}\relax
   \endgroup}%
\def\@sanitize{\futurelet\next\@sanswitch} 
\def\@sanswitch{%
    \ifx\next\endsanity\relax
    \else\ifcat\noexpand\next\@stoken
        \aftergroup\space\let\next=\@eat
    \else\ifcat\noexpand\next\bgroup
        \aftergroup{\let\next=\@eat
    \else\ifcat\noexpand\next\egroup
        \aftergroup}\let\next=\@eat
     \else
        \let\next=\@copytoken
     \fi\fi\fi\fi 
     \next}%
\def\@eat{\afterassignment\@sanitize\let\next= } 
\long\def\@copytoken#1{%
    \ifcat\noexpand#1\relax
        \aftergroup\noexpand
    \else\ifcat\noexpand#1\noexpand~\relax
        \aftergroup\noexpand
    \fi\fi
    \aftergroup#1\relax
    \@sanitize}%
\def\endsanity\endsanity{}%
%%> {Checkpoint/Restart.}
\newwrite\checkpointout
\def\checkpoint#1{\emsg{\@comment\NX\checkpoint --> #1.chk}%
    \immediate\openout\checkpointout= #1.chk
    \@checkwrite{\pageno}%
    \@checkwrite{\chapternum}%
    \@checkwrite{\corollarynum}%
    \@checkwrite{\definitionnum}%
    \@checkwrite{\eqnum}%
    \@checkwrite{\fignum}%
    \@checkwrite{\lemmanum}
    \@checkwrite{\refnum}%
    \@checkwrite{\sectionnum}%
    \@checkwrite{\subsectionnum}%
    \@checkwrite{\tabnum}%
    \@checkwrite{\theoremnum}%
    \@checkwrite{\footnum}%
    \immediate\closeout\checkpointout}%
\def\@checkwrite#1{\edef\tnum{\the #1}%
     \immediate\write\checkpointout{\NX #1 = \tnum}}%
\def\restart#1{\relax
    \immediate\closeout\checkpointout
    \ATunlock
    \Input #1.chk \relax
    \@firstrefnum=\refnum
    \advance\@firstrefnum by \@ne
    \ATlock}%
\let\restore=\restart
\def\endstat{%
   \emsg{\@comment Last PAGE      number is \the\pageno.}%
   \emsg{\@comment Last CHAPTER   number is \the\chapternum.}%
   \emsg{\@comment Last EQUATION  number is \the\eqnum.}%
   \emsg{\@comment Last FIGURE    number is \the\fignum.}%
   \emsg{\@comment Last REFERENCE number is \the\refnum.}%
   \emsg{\@comment Last SECTION   number is \the\sectionnum.}%
   \emsg{\@comment Last TABLE     number is \the\tabnum.}%
   \tracingstats=1}%
{\catcode`\%=11 \gdef\@comment{% }}
\def\theBlank#1{\nobreak\hbox{\lower\jot\vbox{\hrule width #1\relax}}}
\def\Romannumeral#1{\uppercase\expandafter{\romannumeral #1}}
\def\monthname#1{\ifcase#1 \errmessage{0 is not a month}
    \or January\or February\or March\or April\or May\or June\or 
    July\or August\or September\or October\or November\or
    December\else \errmessage{#1 is not a month}\fi}
\def\leftpar#1{%
    \setbox\@capbox=\vbox{\normalbaselines
    \noindent #1\par
        \global\@caplines=\prevgraf}%
    \ifnum \@ne=\@caplines
        \leftline{#1}\else
        \hbox to\hsize{\hss\box\@capbox\hss}\fi}
\def\dospecials{\do\ \do\\\do\{\do\}\do\$\do\&\do\"\do\(\do\)\do\[\do\]%
  \do\#\do\^\do\^^K\do\_\do\^^A\do\%\do\~}
\def\tightbox#1{\vbox{\hrule\hbox{\vrule\vbox{#1}\vrule}\hrule}}
\let\tightboxit=\tightbox
\def\loosebox#1{%
    \vbox{\vskip\jot
        \hbox{\hskip\jot #1\hskip\jot}%
        \vskip\jot}}
\def\eqnbox#1{\lower\jot\tightbox{\loosebox{\quad $#1$ \quad}}}
\def\gloop#1\repeat{\gdef\body{#1}\iterate}
\newif\iflastarg\lastargfalse
\def\car#1,#2;{\gdef\@arg{#1}\gdef\@args{#2}}
\def\@apply{%
    \iflastarg
    \else
        \XA\car\@args;
        \islastarg
        \XA\@fcn\XA{\@arg}%
        \@apply
    \fi}
\def\apply#1#2{%
    \gdef\@args{#2,}\let\@fcn#1
    \islastarg
    \@apply
    }
\def\islastarg{\ifx \@args\empty\lastargtrue\else\lastargfalse\fi}%
\def\@seppuku{\errmessage{Interwoven alignment preambles are not allowed.}\end}
\def\jtem#1#2{\par\hangafter0\hangindent#1
              \noindent\llap{#2\enspace}\ignorespaces}
% file: TXSinit.tex  (TeXsis version 2.16)
%%> {Initialization.}
\catcode`@=11
\uchyph=\@ne
\brokenpenalty=\@M
\widowpenalty=\@M
\clubpenalty=\@M
\long\def\texsis{%
    \quoteon
    \Contentsfalse
    \autoparens
    \ATlock
    \resetcounters
    \pageno=1
    \colwidth=\hsize
    \headline={\HeadLine}\headlineoffset=0.5cm
    \footline={\FootLine}\footlineoffset=0.5cm
    \twelvepoint
    \doublespaced
    \newlinechar=`\^^M
}
\obsolete\inittexsis\texsis     \obsolete\texsisinit\texsis    
\obsolete\initexsis\texsis      \obsolete\initTeXsis\texsis    
\def\today{\number\day\ 
    \ifcase\month\or 
    January\or February\or March\or April\or May\or June\or
    July\or August\or September\or October\or November\or December\fi\
    \number\year}
\def\dated#1{\xdef\today{#1}}
\def\SetDate{%
  \xdef\adate{\monthname{\the\month}~\number\day, \number\year}%
  \xdef\edate{\number\day~\monthname{\the\month}~\number\year}%
  \count255=\time\divide\count255 by 60
  \edef\hour{\the\count255}%
  \multiply\count255 by -60 \advance\count255 by\time
  \edef\minutes{\ifnum 10>\count255 {0}\fi\the\count255}%
  \edef\runtime{\the\year/\the\month/\the\day\space\hour:\minutes}}
\def\gzero#1{\ifx#1\undefined\relax\else\global#1=\z@\fi}
\def\resetcounters{%
  \gzero\chapternum     \gzero\sectionnum
  \gzero\subsectionnum  \gzero\subsubsectionnum 
  \gzero\theoremnum     \gzero\lemmanum         
  \gzero\corollarynum   \gzero\definitionnum    
  \gzero\tabnum         \gzero\fignum        
  \gzero\@BadRefs       \gzero\@BadTags  
  \gzero\@envDepth      \gzero\enumDepth
  \gzero\enumcnt        \gzero\@quoteflag  
  \gzero\refnum         \gzero\eqnum         
  \global\@firstrefnum=1\global\@lastrefnum=1                   
}
\newread\patchfile
\def\LoadSiteFile{%
  \immediate\openin\patchfile=TXSsite.tex
  \ifeof\patchfile
     \emsg{> Trying to load the site info file TXSsite.tex...}%
  \fi
  \immediate\closein\patchfile
  \input TXSsite.tex \relax}%
\def\ReadPatches{%
    \immediate\openin\patchfile=\TXSpatches.tex
    \ifeof\patchfile
         \closein\patchfile
    \else\immediate\closein\patchfile
       \input\TXSpatches.tex \relax
    \fi
    \immediate\openin\patchfile=\TXSmods.tex \relax
    \ifeof\patchfile
       \closein\patchfile
    \else\immediate\closein\patchfile
       \input\TXSmods.tex \relax
    \fi}
\newinsert\botins 
\skip\botins=\bigskipamount
\count\botins=1000
\dimen\botins=\maxdimen
\newif\if@bot
\def\topinsert{\@midfalse\p@gefalse\@botfalse\@ins}
\def\midinsert{\@midtrue\p@gefalse\@botfalse\@ins}
\def\pageinsert{\@midfalse\p@getrue\@botfalse\@ins}
\def\bottominsert{\@midfalse\p@gefalse\@bottrue\@ins\bigskip}
\def\heavyinsert{\@midtrue\p@gefalse\@bottrue\@ins}
\def\endinsert{%
  \egroup
  \if@mid \dimen@\ht\z@ \advance\dimen@\dp\z@
    \advance\dimen@12\p@ \advance\dimen@\pagetotal
    \ifdim\dimen@>\pagegoal\@midfalse\p@gefalse\fi\fi
  \if@mid \bigskip\box\z@\bigbreak
  \else\if@bot\@insert\botins \else\@insert\topins \fi
  \fi
  \endgroup}
\def\@insert#1{%
  \insert#1{\penalty100
  \splittopskip\z@skip
  \splitmaxdepth\maxdimen \floatingpenalty\z@
  \ifp@ge \dimen@=\dp\z@
    \vbox to\vsize{\unvbox\z@\kern-\dimen@}%
  \else \box\z@ \nobreak
    \ifx #1\topins \ifp@ge\else\bigskip\fi\fi
  \fi}}
\def\pagecontents{%
  \ifvoid\topins\else\unvbox\topins
      \vskip\skip\topins\fi
  \dimen@=\dp\@cclv \unvbox\@cclv
  \ifvoid\footins\else
    \vskip\skip\footins
    \footnoterule
    \unvbox\footins\fi
  \ifvoid\botins\else\vskip\skip\botins
        \unvbox\botins\fi
  \ifr@ggedbottom \kern-\dimen@ \vfil \fi}
\newread\txsfile
\def\loadstyle#1#2{%
   \def#1{\@loaderr{#1}}%
   \ATunlock
   \immediate\openin\txsfile=#2
   \ifeof\txsfile
      \emsg{> Trying to load the style file #2...}%
   \fi
   \closein\txsfile
   \input #2 \relax
   \ATlock
   #1}%
\newhelp\@utohelp{%
loadstyle: The macro named above was supposed to be defined^^J
In the style file that was just read, but I couldn't find^^J
the definition in that file.  Maybe you can learn something^^J
from the comments in that style file, or find someone who knows^^J
something about it.}
\def\@loaderr#1{%
   \newlinechar=10
   \errhelp=\@utohelp
   \errmessage{No definition of \string#1 in the style file.}}
\def\autoload#1#2{%
   \def#1{\loadstyle#1{#2}}}
\autoload\PhysRev{PhysRev.txs}%
\autoload\PhysRevLett{PhysRev.txs}%
\autoload\PhysRevManuscript{PhysRev.txs}%
\autoload\nuclproc{nuclproc.txs}%
\autoload\NorthHolland{Elsevier.txs}%
\autoload\NorthHollandTwo{Elsevier.txs}%
\autoload\WorldScientific{WorldSci.txs}%
\autoload\IEEEproceedings{IEEE.txs}%
\autoload\IEEEreduced{IEEE.txs}%
\autoload\AIPproceedings{AIP.txs}%
\autoload\CVformat{CVformat.txs}%
\autoload\idx{index.tex}\autoload\index{index.tex}\autoload\theindex{index.tex}
\autoload\markindexfalse{index.tex}\autoload\markindextrue{index.tex}
\autoload\makeindexfalse{index.tex}\autoload\makeindextrue{index.tex}
% file: TXShead.tex                                 TeXsis version 2.16
%%> {Running Headlines.}
\newdimen\headlineoffset        \headlineoffset=0.0cm
\newdimen\footlineoffset        \footlineoffset=0.0cm
\newif\ifRunningHeads           \RunningHeadsfalse
\newif\ifbookpagenumbers        \bookpagenumbersfalse
\newif\ifrightn@m               \rightn@mtrue
\def\makeheadline{\vbox to 0pt{\vskip-22.5pt
   \vskip-\headlineoffset
   \line{\vbox to 8.5pt{}\the\headline}\vss}\nointerlineskip}
\def\makefootline{\baselineskip=24pt
   \vskip\footlineoffset
   \line{\the\footline}}
\def\HeadLine{%
   \edef\firstm{{\XA\iffalse\firstmark\fi}}%
   \edef\topm{{\XA\iffalse\topmark\fi}}%
   \ifRunningHeads
     \def\He@dText{{\HeadFont \HeadText}}%
   \else\def\He@dText{\relax}\fi
   \ifbookpagenumbers
      \ifodd\pageno\rightn@mtrue
      \else\rightn@mfalse\fi
   \else\rightn@mtrue\fi
   \ifx\topm\firstm
     \ifrightn@m
        {\hss\He@dText\hss\llap{\PageNumber}}%
     \else
        {\rlap{\PageNumber}\hss\He@dText\hss}%
      \fi
   \else \hfill \fi}%
\def\HeadText{\hfill}%
\def\He@dText{\relax}
\def\FootLine{%
   \edef\firstm{%
      {\expandafter\iffalse\firstmark\fi}}%
   \edef\topm{%
      {\expandafter\iffalse\topmark\fi}}%
   \ifx\topm\firstm \hss
    \else {\hss \FootText \hss} \fi}%
\def\FootText{\hfill}%
\def\PageNumber{{\tenrm\folio}}%
\def\HeadFont{\tenit}%
\def\nopagenumbers{\headline={\hfil}\footline={\hfil}}
\def\pagenumbers{\headline={\HeadLine}\footline={\FootLine}}
\def\bookpagenumbers{\bookpagenumberstrue}
\def\plainoutput{%
  \makeBindingMargin
  \shipout\vbox{\makeheadline\pagebody\makefootline}%
  \advancepageno
  \ifnum\outputpenalty>-\@MM \else\dosupereject\fi}
\newdimen\BindingMargin \BindingMargin=0pt
\def\makeBindingMargin{%
   \ifdim\BindingMargin>0pt
   \ifodd\pageno\hoffset=\BindingMargin\else
   \hoffset=-\BindingMargin\fi\fi}
% file: TXSeqns.tex  (TeXsis version 2.16)
%%> {Equation numbering.}
\newcount\eqnum         \eqnum=\z@
\def\@chaptID{}         \def\@sectID{}%
\newif\ifeqnotrace      \eqnotracefalse
\def\EQN{%
   \begingroup
   \quoteoff\offparens
   \@EQN}%
\def\@EQN#1$${%
   \endgroup
   \if ?#1? \EQNOparse *;;\endlist
   \else \EQNOparse#1;;\endlist\fi
   $$}%
\def\EQNOparse#1;#2;#3\endlist{%
  \if ?#3?\relax
    \global\advance\eqnum by\@ne
    \edef\tnum{\@chaptID\@sectID\the\eqnum}%
    \Eqtag{#1}{\tnum}%
    \@EQNOdisplay{#1}%
  \else\stripblanks #2\endlist
    \edef\p@rt{\tok}%
    \if a\p@rt\relax
      \global\advance\eqnum by\@ne\fi
    \edef\tnum{\@chaptID\@sectID\the\eqnum}%
    \Eqtag{#1}{\tnum}%
    \edef\tnum{\@chaptID\@sectID\the\eqnum\p@rt}%
    \Eqtag{#1;\p@rt}{\tnum}%
    \@EQNOdisplay{#1;#2}%
  \fi
  \global\let\?=\tnum
  \relax}%
\def\Eqtag#1#2{\tag{Eq.#1}{#2}} 
\def\@EQNOdisplay#1{%
   \@eqno
   \ifeqnotrace
     \rlap{\phantom{(\tnum)}%
        \quad{\tenpoint\tt["#1"]}}\fi
    (\tnum)}%
\let\@eqno=\eqno
\def\endlist{\endlist}%
\def\Eq#1{Eq.~($\use{Eq.#1}$)}%
\def\Eqs#1{Eqs.~($\use{Eq.#1}$)}%
\def\Ep#1{($\use{Eq.#1}$)}%
\def\EQNdisplaylines#1{%
   \@EQNcr
   \displ@y
   \halign{%
      \hbox to\displaywidth{%
      $\@lign\hfil\displaystyle##\hfil$}%
      &\llap{$\@lign\@@EQN{##}$}\crcr
   #1\crcr}%
   \@EQNuncr}%
\long\def\EQNalign#1{%
   \@EQNcr
   \displ@y
     \tabskip=\centering
   \halign to\displaywidth{%
   \hfil$\relax\displaystyle{##}$
     \tabskip=0pt
   &$\relax\displaystyle{{}##}$\hfil
     \tabskip=\centering
   &\llap{$\relax\@@EQN{##}$}%
     \tabskip=0pt\crcr
    #1\crcr}%
   }
\def\@@EQN#1{\if ?#1? \EQNOparse *;;\endlist
         \else \EQNOparse#1;;\endlist\fi}%
\def\@EQNcr{%
   \let\EQN=&
   \let\@eqno=\relax}%
\def\@EQNuncr{%
   \let\EQN=\@EQN
   \let\@eqno=\eqno}%
\def\EQNdoublealign#1{%
   \@EQNcr
   \displ@y
   \tabskip=\centering
   \halign to\displaywidth{%
      \hfil$\relax\displaystyle{##}$
      \tabskip=0pt
   &$\relax\displaystyle{{}##}$\hfil
      \tabskip=0pt
   &$\relax\displaystyle{{}##}$\hfil
      \tabskip=\centering
   &\llap{$\relax\@@EQN{##}$}%
      \tabskip=0pt\crcr
   #1\crcr}%
   \@EQNuncr}%
%\def\displaylines#1{%
%   \displ@y                            
%   \halign{\hbox to \displaywidth{$\hfil\displaystyle##\hfil$}%
%           &\llap{$##$}\crcr           
%    #1\crcr}}                          
\def\eqn#1$${\edef\tok\string#1
   \xdef#1{\NX\use{Eq.\tok}}%
   \EQNOparse \tok;;\endlist $$}%
\def\eqnmarker{\triangleright}%
\def\eqnmark{\quoteoff\offparens\@eqnmark}
\def\@eqnmark#1$${\@@eqnmark#1\eqno\eqno\endlist}
\def\@@eqnmark#1\eqno#2\eqno#3\endlist{\def\EQN{\relax}%
   \if ?#3? \@EQNmark#1\EQN\EQN\endlist
   \else\displaylines{\hbox to 0pt{$\eqnmarker$\hss}\hfill{#1}\hfill
                      \hbox to 0pt{\hss$#2$}}\fi$$}
\def\@EQNmark#1\EQN#2\EQN#3\endlist{%
   \if ?#3?\displaylines{\hbox to 0pt{$\eqnmarker$\hss}\hfill{#1}\hfill}%
   \else \let\@eqno=\relax
      \EQNdisplaylines{\hbox to 0pt{$\eqnmarker$\hss}\hfill{#1}\hfill
                \hbox to 0pt{\hss$\EQNOparse#2;;\endlist$}}\fi}
% file: TXSprns.tex                                 TeXsis version 2.16  
%%> {Auto-sizing of Parentheses.}
\catcode`@=11
\ifx\@left\undefined
 \let\@left=\left       \let\@right=\right
 \let\lparen=(          \let\rparen=)
 \let\lbrack=[          \let\rbrack=]
 \let\@vert=\vert
\fi
\begingroup
\catcode`\(=\active \catcode`\)=\active
\catcode`\[=\active \catcode`\]=\active
\gdef({\relax
   \ifmmode \push@delim{P}%
    \@left\lparen
   \else\lparen
   \fi}
\global\let\@lparen=(
\gdef){\relax
   \ifmmode\@right\rparen
     \pop@delim\@delim
     \if P\@delim \relax \else
       \if B\@delim\emsg{> Expecting \string] but got \string).}%
                   \@errmark{PAREN}%
       \else\emsg{> Unmatched \string).}\@errmark{PAREN}%
     \fi\fi
   \else\rparen
   \fi}
\gdef[{\relax
   \ifmmode \push@delim{B}%
     \@left\lbrack
   \else\lbrack
   \fi}
\global\let\@lbrack=[
\gdef]{\relax
   \ifmmode\@right\rbrack
     \pop@delim\@delim
     \if B\@delim \relax \else
       \if P\@delim\emsg{> Expecting \string) but got \string].}%
                   \@errmark{BRACK}%
       \else\emsg{> Unmatched \string].}\@errmark{BRACK}%
     \fi\fi
   \else\rbrack
   \fi}
\gdef\EZYleft{\futurelet\nexttok\@EZYleft}%
\gdef\@EZYleft#1{%
   \ifx\nexttok(  \let\nexttok=\lparen
   \else
   \ifx\nexttok[  \let\nexttok=\lbrack
   \fi\fi
   \@left\nexttok}%
\gdef\EZYright{\futurelet\nexttok\@EZYright}%
\gdef\@EZYright#1{%
   \ifx\nexttok)  \let\nexttok=\rparen
   \else
   \ifx\nexttok]  \let\nexttok=\rbrack
   \fi\fi
   \@right\nexttok}%
\endgroup
\toksdef\@CAR=0  \toksdef\@CDR=2
\def\push@delim#1{\@CAR={{#1}}%
     \@CDR=\XA{\@delimlist}%
    \edef\@delimlist{\the\@CAR\the\@CDR}}%
\def\pop@delim#1{\XA\pop@delimlist\@delimlist\endlist#1}%
\def\pop@delimlist#1#2\endlist#3{\def\@delimlist{#2}\def#3{#1}}    
\def\@delimlist{}%
\newif\ifEZparens   \EZparensfalse
\def\autoparens{\EZparenstrue
   \everydisplay={\@onParens}%
   }
\def\@onParens{%
   \ifEZparens
    \def\@delimlist{}%
    \let\left=\EZYleft
    \let\right=\EZYright
    \catcode`\(=\active \catcode`\)=\active
    \catcode`\[=\active \catcode`\]=\active
   \fi}
\def\offparens{%
   \EZparensfalse\@offParens
   \everymath={}\everydisplay={}}%
\def\@offParens{%
   \let\left=\@left
   \let\right=\@right
   \catcode`(=12 \catcode`)=12
   \catcode`[=12 \catcode`]=12
   }
\offparens
\def\onparens{%
   \EZparenstrue
   \everymath={\@onMathParens}%
   \everydisplay={\@onParens}%
   }
\def\easyparenson{\onparens}%
\def\@onMathParens#1{%
   \@SetRemainder#1\endlist
   \ifx#1\lparen\let\@remainder=\@lparen\fi
   \ifx#1\lbrack\let\@remainder=\@lbrack\fi
   \@onParens
   \@remainder}%
\def\@SetRemainder#1#2\endlist{%
   \ifx @#2@ \def\@remainder{#1}%
   \else  \def\@remainder{{#1#2}}%
   \fi}
\def\easyparensoff{\offparens}%
\def\pmatrix#1{\@left\lparen\matrix{#1}\@right\rparen}
\def\bordermatrix#1{\begingroup \m@th
  \setbox\z@\vbox{\def\cr{\crcr\noalign{\kern2\p@\global\let\cr\endline}}%
    \ialign{$##$\hfil\kern2\p@\kern\p@renwd&\thinspace\hfil$##$\hfil
      &&\quad\hfil$##$\hfil\crcr
      \omit\strut\hfil\crcr\noalign{\kern-\baselineskip}%
      #1\crcr\omit\strut\cr}}%
  \setbox\tw@\vbox{\unvcopy\z@\global\setbox\@ne\lastbox}%
  \setbox\tw@\hbox{\unhbox\@ne\unskip\global\setbox\@ne\lastbox}%
  \setbox\tw@\hbox{$\kern\wd\@ne\kern-\p@renwd\@left\lparen\kern-\wd\@ne
    \global\setbox\@ne\vbox{\box\@ne\kern2\p@}%
    \vcenter{\kern-\ht\@ne\unvbox\z@\kern-\baselineskip}\,\right\rparen$}%
  \;\vbox{\kern\ht\@ne\box\tw@}\endgroup}
\def\partitionmatrix#1{\,\vcenter{\offinterlineskip\m@th
   \def\tablerule{\noalign{\hrule}}
   \halign{\hfil\loosebox{$\mathstrut ##$}\hfil&&\quad\vrule##\quad&
      \hfil\loosebox{$##$}\hfil\crcr
   #1\crcr}}\,}
% file: TXSrefs.tex                                 TeXsis version 2.16
%%> {References and Citations.}
\catcode`@=11
\newcount\refnum        \refnum=\z@
\newcount\@firstrefnum  \@firstrefnum=1
\newcount\@lastrefnum   \@lastrefnum=1
\newcount\@BadRefs      \@BadRefs=0
\newif\ifrefswitch      \refswitchtrue
\newif\ifbreakrefs      \breakrefstrue
\newif\ifsuperrefs      \superrefstrue
\newif\ifmarkit         \markittrue
\newif\ifnullname       \nullnamefalse
\newif\iftagit
\newif\ifreffollows
\newif\ifrefpunct       \refpuncttrue
\def\refterminator{}%
\newdimen\refindent     \refindent=2em
\newdimen\refpar        \refpar=20pt
\newbox\tempbox
\newwrite\reflistout
\def\@refinit{\ifrefswitch
   \immediate\openout\reflistout=\jobname.ref
  \else
    \let\@refwrite=\@refwrong
    \let\@refNXwrite=\@refwrong
    \immediate\closeout\reflistout
  \fi
  \gdef\refinit{\relax}%
}%
\def\refinit{\@refinit}%
\def\@refwrite#1{\refinit
   \immediate\write\reflistout{#1}}%
\def\@refNXwrite#1{\refinit
   \unexpandedwrite\reflistout{#1}}%
\def\@refwrong#1{}%
\long\def\reference#1{%
  \markittrue
  \@tagref{#1}%
  \@GetRefText{#1}%
  }%
\long\def\addreference#1{%
  \markitfalse
  \@tagref{#1}%
  \@GetRefText{#1}%
  }%
\def\hiddenreference{\addreference}%
\def\@tagref#1{%
  \stripblanks #1\endlist
  \XA\ifstar\tok*\fi
  \ifnullname\relax\else
    \global\advance\refnum by \@ne
    \@lastrefnum=\refnum
    \edef\rnum{\the\refnum}%
    \tag{Ref.#1}{\rnum}%
    \ifnum\pageno>\z@
      \begingroup
      \quoteoff\offparens
      \immediate\write16{(\the\refnum)% write comment to .LIS and terminal
 First reference to "#1" on page \the\pageno.}%
      \endgroup
    \fi
  \fi
  }%
\def\ifstar#1#2\fi{\ifx*#1\relax\nullnametrue\else\nullnamefalse\fi}
\def\@GetRefText#1{%
  \ifnullname
    \p@nctwrite;\relax
    \begingroup\quoteoff
      \@refwrite{\@comment ... Reference text for%
      "#1" defined on page \number\pageno.}%
    \endgroup
  \else
    \ifnum\refnum>1
      \p@nctwrite.\fi
    \begingroup\quoteoff\@parensoff
      \@refwrite{\@comment }%
      \@refwrite{\@comment (\the\refnum) Reference text for%
      "#1" defined on page \number\pageno.}%
    \endgroup
    \@refwrite{\NX\@refitem{\the\refnum}}%
  \fi
  \begingroup
   \def\endreference{\NX\endreference}%
   \def\reference{\NX\reference}%
   \def\ref{\NX\ref}%
   \obeylines
   \@copyref
}%
{\obeylines\gdef\@copyref#1
   {\def\@arg{#1}\ifx\@arg\emtpy\relax\let\@nextline=\@copyref% get next line
    \else\ParseRefText#1\endreference\endreference\endlist\fi% parse line
    \@nextline}%
}%
\def\ParseRefText#1\endreference#2\endreference#3\endlist{%
   \def\@arg{#1}\ifx\@arg\empty\relax
   \else\begingroup
     \@refNXwrite{#1}%
     \endgroup
   \fi
   \def\@arg{#3}\ifx\@arg\empty
        \let\@nextline=\@copyref
   \else\let\@nextline=\@endcopyref
     \gdef\@remainder{\@endreference #2}%
   \fi}%
\def\@endcopyref{\endgroup
    \@remainder
    }%
{\catcode`\%=11 \gdef\@comment{% }}
\def\@parensoff{\catcode`\(=11 \catcode`\)=11}
\long\def\@endreference#1{%
  \reffollowsfalse
  \ifx#1\cite\reffollowstrue\fi
  \ifx#1\citerange\reffollowstrue\fi
  \ifx#1\refrange\reffollowstrue\fi
  \ifx#1\ref\reffollowstrue\fi
  \ifx#1\reference\reffollowstrue
  \else
   \ifnum\@firstrefnum>\@lastrefnum\relax
   \else\ifmarkit
    \ifnum\@firstrefnum=\@lastrefnum
      \xdef\@refmark{\the\@lastrefnum}%
    \else
      \xdef\@refmark{\the\@firstrefnum-\the\@lastrefnum}%
    \fi
    \global\@firstrefnum=\refnum
    \global\advance\@firstrefnum by \@ne
    \add@refmark
   \fi\fi
  \fi
  \flush@reflist{#1}%
  \def\@next{#1}\ifcat.\NX#1\def\@next{#1 }\fi
  \@next}%
\def\endreference{%
  \emsg{>  Whoops! \string\endreference was called without
  first calling \string\reference.}\@errmark{REF?}%
  \emsg{>  I'll just ignore it.}%
  }%
\def\@refspace{\ }
\def\citemark#1{%
   \relax\let\@sf\empty
   \ifhmode\edef\@sf{\spacefactor\the\spacefactor}\/\fi
   \ifsuperrefs
      $\relax{}^{\hbox{$\citestyle
           #1\refterminator$}}$\relax
   \else
      {}~[{#1}]\relax
   \fi
   \@sf}%
\def\citestyle{\scriptstyle}%
\def\referencelist{%
   \begingroup
   \pageno=0
   \def\citemark##1{\relax}%
   \def\@refspace{\relax}}%
\def\endreferencelist{%
   \endgroup}%
\long\def\cite#1#2{%
  \markittrue
  \reffollowsfalse
  \ifx#2\cite\reffollowstrue\fi
  \ifx#2\citerange\reffollowstrue\fi
  \ifx#2\refrange\reffollowstrue\fi
  \ifx#2\ref\reffollowstrue\fi
  \ifx#2\reference\reffollowstrue\fi
  \make@refmark{#1}%
  \add@refmark
  \flush@reflist{#2}%
  #2}%
\let\ref=\cite
\def\@refmarklist{}%
\gdef\refbef@re{F}%
\def\make@refmark#1{%
  \testtag{Ref.#1}\ifundefined
    \emsg{> UNDEFINED REFERENCE #1 ON PAGE \number\pageno.}%
    \global\advance\@BadRefs by 1
    \xdef\@refmark{{\tenbf #1}}%
    \@errmark{REF?}%
  \else
    \xdef\@refmark{\csname\tok\endcsname}%
  \fi}%
\def\add@refmark{%
  \if\refbef@re T\relax
    \xdef\@refmarklist{\@refmarklist,\@refmark}%
  \else
    \xdef\@refmarklist{\@refmark}%
    \gdef\refbef@re{T}%
  \fi}
\long\def\flush@reflist#1{%
  \ifmarkit
  \ifreffollows\else
    \citemark{\@refmarklist}%
    \gdef\refbef@re{F}%
    \gdef\@refmarklist{}%
    \ifx#1\par\else\space@head{#1}\fi
  \fi\fi}%
{\quoteon
\gdef\space@head#1{\def\next{\space}%
    \ifcat.\NX#1\relax\def\next{\relax}\fi
    \ifx)#1\def\next{\relax}\fi
    \ifx]#1\def\next{\relax}\fi
    \ifx"#1\def\next{\relax}\fi
   \next}%
}
\def\Ref#1{Ref.~\use{Ref.#1}}
\long\def\refrange#1#2#3{%
  \reffollowsfalse
  \ifx#3\cite\reffollowstrue\fi
  \ifx#3\ref\reffollowstrue\fi
  \ifx#3\reference\reffollowstrue\fi
  \ifx#3\refrange\reffollowstrue\fi
  \make@refmark{#2}%
  \xdef\@refmarktwo{\@refmark}%
  \make@refmark{#1}%
  \xdef\@refmark{\@refmark\hbox{--}\@refmarktwo}%
  \add@refmark
  \flush@reflist{#3}%
  #3}%
\let\citerange=\refrange
\def\vol#1{\undertext{#1}}
\def\booktitle#1{{\sl #1}}
\def\etal{{\it et al.}}
\def\ie{{\it i.e.}}
\def\cf{{\it cf.}}
\def\ibid{{\it ibid.}}
\def\NP{Nucl.\ Phys.}
\def\PL{Phys.\ Lett.}
\def\PR{Phys.\ Rev.}
\def\PRL{Phys.\ Rev.\ Lett.}
\def\ao{Appl.\  Opt.\ }
\def\ap{Appl.\  Phys.\ }
\def\apl{Appl.\ Phys.\ Lett.\ }
\def\apj{Astrophys.\ J.\ }
\def\bell{Bell Syst.\ Tech.\ J.\ }
\def\jqe{IEEE J.\ Quantum Electron.\ }
\def\assp{IEEE Trans.\ Acoust.\ Speech Signal Process.\ }
\def\aprop{IEEE Trans.\ Antennas Propag.\ }
\def\mtt{IEEE Trans.\ Microwave Theory Tech.\ }
\def\iovs{Invest.\ Ophthalmol.\ Vis.\ Sci.\ }
\def\jcp{J.\ Chem.\ Phys.\ }
\def\jmo{J.\ Mod.\ Opt.\ }
\def\josa{J.\ Opt.\ Soc.\ Am.\ }
\def\josaa{J.\ Opt.\ Soc.\ Am.\ A }
\def\josab{J.\ Opt.\ Soc.\ Am.\ B }
\def\jpp{J.\ Phys.\ (Paris) }
\def\nat{Nature (London) }
\def\oc{Opt.\ Commun.\ }
\def\ol{Opt.\ Lett.\ }
\def\pl{Phys.\ Lett.\ }
\def\pra{Phys.\ Rev.\ A }
\def\prb{Phys.\ Rev.\ B }
\def\prc{Phys.\ Rev.\ C }
\def\prd{Phys.\ Rev.\ D }
\def\pre{Phys.\ Rev.\ E }
\def\prl{Phys.\ Rev.\ Lett.\ }
\def\rmp{Rev.\ Mod.\ Phys.\ }
\def\pspie{Proc.\ Soc.\ Photo-Opt.\ Instrum.\ Eng.\ }
\def\sjqe{Sov.\ J.\ Quantum Electron.\ }
\def\vr{Vision Res.\ }
\def\ListReferences{\emsg{Reference List}%
  \ifnum\refnum>\z@ \p@nctwrite{.}%
    \@refwrite{\@comment>>> EOF \jobname.ref <<<}% and EOF marker comment
    \immediate\closeout\reflistout
  \fi
  \ifnum\@BadRefs>\z@
    \emsg{>}\emsg{> There were \the\@BadRefs\ undefined references.}%
    \emsg{> See the file \jobname.log for the citations, or try running}%
    \emsg{> TeXsis again to resolve forward references.}\emsg{>}%
  \fi
  \begingroup
    \offparens
    \immediate\openin\reflistout=\jobname.ref
    \ifeof\reflistout
       \closein\reflistout
       \emsg{> \string\ListReferences: no references.}%
    \else
       \catcode`@=11
       \unobeylines
       \setbox\tempbox\hbox{\the\refnum.\quad}%
       \refindent=\wd\tempbox
       \leftskip=\refindent
       \parindent=\z@
       \def\reference{\@noendref}%
       \refFormat
       \Input\jobname.ref  \relax
       \vskip 0pt
    \fi
  \endgroup
  \emsg{}%
  \refReset
  }%
\def\References{\ListReferences}%
\def\refFormat{\relax}%
\def\@noendref#1{%
   \emsg{>  Whoops! \string\reference{#1} was given before the}%
   \emsg{>  \string\endreference for the previous \string\reference.}%
   \emsg{>  I'll just ignore it and run the two together.}%
   }%
\def\@refitem#1#2{\message{#1.}%
   \refskip\noindent\hskip-\refindent
   \hbox to \refindent {\hss #1.\quad}%
   #2}
\def\refskip{\smallskip}%
\def\@refpunct#1{\unskip#1}%
\def\p@nctwrite#1{%
   \ifrefpunct
      \@refwrite{\NX\@refpunct#1\NX\@refbreak}%
   \else
      \@refwrite{\NX\@refbreak}%
   \fi}
\def\@refbreak{\ifbreakrefs\par\fi}
\def\refReset{%
   \global\refnum=\z@
   \global\@firstrefnum=1
   \global\@lastrefnum=1
   \global\@BadRefs=0
   \gdef\refinit{\@refinit}%
   }
\newif\ifEurostyle     \Eurostylefalse
{\catcode`\.=\active
   \gdef.{\hbox{\p@riod\null}}}%
\def\p@riod{.}%
\def\journal{%
  \bgroup
   \catcode`\.=\active
   \offparens
   \j@urnal}%
\offparens
 \def\j@urnal#1;#2,#3(#4){%
   \ifEurostyle
      {#1} {\vol{#2}} (\@fullyear{#4}) #3\relax
   \else
      {#1} {\vol{#2}}, #3 (\@fullyear{#4})\relax
   \fi
  \egroup}%
\def\@fullyear#1{%
  \begingroup
   \count255=\year
      \divide \count255 by 100 \multiply \count255 by 100
   \count254=\year
      \advance \count254 by -\count255 \advance \count254 by 1
   \count253=#1\relax
   \ifnum\count253<100
     \ifnum \count253>\count254
       \advance \count253 by -100\fi
      \advance \count253 by \count255
   \fi
   \number\count253
  \endgroup
}
% file: TXSsects.tex  (TeXsis version 2.16)
%%> {Section and Chapter divisions.}
\catcode`@=11
\newcount\chapternum            \chapternum=\z@
\newcount\sectionnum            \sectionnum=\z@
\newcount\subsectionnum         \subsectionnum=\z@
\newcount\subsubsectionnum      \subsubsectionnum=\z@
\newif\ifshowsectID             \showsectIDtrue
\def\@sectID{}%
\newif\ifshowchaptID            \showchaptIDtrue
\def\@chaptID{}%
\newskip\sectionskip            \sectionskip=1cm plus4pt minus4pt
\newskip\subsectionskip         \subsectionskip=0.5cm plus2pt minus2pt
\newdimen\sectionminspace       \sectionminspace = 0.25\vsize
\def\chapter#1{%
  \vfill\supereject
  \global\advance\chapternum by \@ne
  \global\sectionnum=\z@
  \global\def\@sectID{}%
  \edef\lab@l{\ChapterStyle{\the\chapternum}}%
  \ifshowchaptID
    \global\edef\@chaptID{\lab@l.}%
    \r@set
  \else\edef\@chaptID{}\fi
  \everychapter
  \ifx\Tbf\undefined\def\Tbf{\bf}\fi
  \ifshowchaptID
    \leftline{\Tbf{Chapter\ \@chaptID}}%
    \nobreak\smallskip\fi
  \begingroup
    \raggedright\pretolerance=2000\hyphenpenalty=2000
    \parindent=\z@ {\Tbf{#1}}\relax
    \bigskip
  \endgroup
  \nobreak\bigskip
  \begingroup
    \def\label##1{}%
    \xdef\ChapterTitle{#1}%
    \def\n{}\def\nl{}\def\mib{}%
    \setHeadline{#1}%
    \emsg{Chapter \@chaptID\space #1}%
    \def\@quote{\string\@quote\relax}%
    \addTOC{0}{\TOCcID{\lab@l.}#1}{\folio}%
  \endgroup
  \@Mark{#1}%
  \s@ction
  \afterchapter}%
\def\everychapter{\relax}%
\def\afterchapter{\relax}%
\def\ChapterStyle#1{#1}%
\def\setChapterID#1{\edef\@chaptID{#1.}}%
\def\r@set{%
  \global\subsectionnum=\z@
  \global\subsubsectionnum=\z@
  \ifx\eqnum\undefined\relax
    \else\global\eqnum=\z@\fi
  \ifx\theoremnum\undefined\relax
  \else
    \global\theoremnum=\z@
    \global\lemmanum=\z@
    \global\corollarynum=\z@
    \global\definitionnum=\z@
    \global\fignum=\z@
    \ifRomanTables\relax
    \else\global\tabnum=\z@\fi
  \fi}
\long\def\s@ction{%
  \checkquote
  \checkenv
  \nobreak\noindent}
\def\@Mark#1{%
   \begingroup
     \def\label##1{}%
     \def\goodbreak{}%
     \def\mib{}\def\n{}%
     \mark{#1\NX\else\lab@l}%
   \endgroup}%
\def\@noMark#1{\relax}%
\def\setHeadline#1{\@setHeadline#1\n\endlist}%
\def\@setHeadline#1\n#2\endlist{%
   \def\@arg{#2}\ifx\@arg\empty
      \global\edef\HeadText{#1}%
   \else
      \global\edef\HeadText{#1\dots}%
   \fi
}
\def\section#1{%
   \vskip\sectionskip
   \goodbreak\pagecheck\sectionminspace
   \global\advance\sectionnum by \@ne
   \edef\lab@l{\@chaptID\SectionStyle{\the\sectionnum}}%
   \ifshowsectID
     \global\edef\@sectID{\SectionStyle{\the\sectionnum}.}%
     \global\edef\@fullID{\lab@l.\space\space}%
     \r@set
   \else\gdef\@fullID{}\fi
   \everysection
   \ifx\tbf\undefined\def\tbf{\bf}\fi
   \vbox{%
     {\raggedright\pretolerance=2000\hyphenpenalty=2000
     \setbox0=\hbox{\noindent\tbf\@fullID}%
     \hangindent=\wd0 \hangafter=1
     \noindent\unhbox0{\tbf{#1}}\relax
     \medskip}}%
   \nobreak
   \begingroup
     \def\label##1{}%
     \global\edef\SectionTitle{#1}%
     \def\n{}\def\nl{}\def\mib{}%
     \ifnum\chapternum=0\setHeadline{#1}\fi
     \emsg{Section \@fullID #1}%
     \def\@quote{\string\@quote\relax}%
     \addTOC{1}{\TOCsID{\lab@l.}#1}{\folio}%
   \endgroup
   \s@ction
   \aftersection}%
\def\everysection{\relax}%
\def\aftersection{\relax}%
\def\setSectionID#1{\edef\@sectID{#1.}}%
\def\SectionStyle#1{#1}%
\def\subsection#1{%
   \vskip\subsectionskip
   \goodbreak\pagecheck\sectionminspace
   \global\advance\subsectionnum by \@ne
   \subsubsectionnum=\z@
   \edef\lab@l{\@chaptID\@sectID\SubsectionStyle{\the\subsectionnum}}%
   \ifshowsectID
     \global\edef\@fullID{\lab@l.\space}%
   \else\gdef\@fullID{}\fi
   \everysubsection
   \vbox{%
     {\raggedright\pretolerance=2000\hyphenpenalty=2000
     \setbox0=\hbox{\noindent\bf\@fullID}%
     \hangindent=\wd0 \hangafter=1
     \noindent\unhbox0{\bf{#1}}\relax
     \medskip}}%
   \nobreak
   \begingroup
     \def\label##1{}%
     \global\edef\SubsectionTitle{#1}%
     \def\n{}\def\nl{}\def\mib{}%
     \emsg{\@fullID #1}%
     \def\@quote{\string\@quote\relax}%
     \addTOC{2}{\TOCsID{\lab@l.}#1}{\folio}%
   \endgroup
   \s@ction
   \aftersubsection}%
\def\everysubsection{\relax}%
\def\aftersubsection{\relax}%
\def\SubsectionStyle#1{#1}%
\def\subsubsection#1{%
   \vskip\subsectionskip
   \goodbreak\pagecheck\sectionminspace
   \global\advance\subsubsectionnum by \@ne
   \edef\lab@l{\@chaptID\@sectID\SubsectionStyle{\the\subsectionnum}.%
           \SubsubsectionStyle{\the\subsubsectionnum}}%
   \ifshowsectID
     \global\edef\@fullID{\lab@l.\space\space}%
   \else\gdef\@fullID{}\fi
   \everysubsubsection
   \vbox{%
     {\raggedright\bf
     \setbox0=\hbox{\noindent\@fullID}%
     \hangindent=\wd0 \hangafter=1
     \noindent\@fullID\relax
     #1\relax}}%
   \nobreak\medskip
   \begingroup
     \def\label##1{}%
     \global\edef\SubsectionTitle{#1}%
     \def\n{}\def\nl{}\def\mib{}%
     \emsg{\@fullID #1}%
     \def\@quote{\string\@quote\relax}%
     \addTOC{3}{\TOCsID{\lab@l.}#1}{\folio}%
   \endgroup
   \s@ction
   \aftersubsection}%
\def\everysubsubsection{\relax}%
\def\aftersubsubsection{\relax}%
\def\SubsubsectionStyle#1{#1}%
\def\Appendix#1#2{%
  \vfill\supereject
  \global\sectionnum=\z@
  \def\@arg{#1}%
  \gdef\@sectID{}%
  \edef\lab@l{#1}%
  \ifshowchaptID
    \ifx\@arg\empty\else
       \global\edef\@chaptID{\lab@l.}\fi
    \r@set
  \else\def\@chaptID{}\fi
  \everychapter
  \ifx\Tbf\undefined\def\Tbf{\bf}\fi
  \leftline{\Tbf{Appendix\ \@chaptID}}%
  \begingroup
    \nobreak\smallskip
    \parindent=\z@\raggedright
    {\Tbf{#2}}\relax
    \nobreak\bigskip
  \endgroup
  \nobreak\bigskip
  \begingroup
    \def\label##1{}%
    \global\edef\ChapterTitle{#2}%
    \def\n{}\def\nl{}\def\mib{}%
    \setHeadline{#2}%
    \emsg{Appendix \@chaptID\space #2}%
    \def\@quote{\string\@quote\relax}%
    \addTOC{0}{\TOCcID{\lab@l.}#2}{\folio}%
  \endgroup
  \@Mark{#2}%
  \s@ction
  \afterchapter}%
\def\appendix#1#2{%
   \vskip\sectionskip
   \goodbreak\pagecheck\sectionminspace
           \global\advance\sectionnum by \@ne
   \def\@arg{#1}%
   \gdef\@sectID{}%
   \edef\lab@l{#1}%
   \ifshowsectID
     \r@set
     \ifx\@arg\empty\else
       \global\edef\@sectID{\lab@l.}%
       \global\edef\@fullID{\lab@l.\space\space}\fi
   \else\gdef\@fullID{}\fi
   \everysection
   \ifx\tbf\undefined\def\tbf{\bf}\fi
   \vbox{%
     {\raggedright\tbf
     \setbox0=\hbox{\tbf\@fullID}%
     \hangindent=\wd0 \hangafter=1
     \noindent\@fullID
     {#2}}}\relax
   \nobreak\medskip
   \begingroup
     \def\label##1{}%
     \global\edef\SectionTitle{#2}%
     \def\n{}\def\nl{}\def\mib{}%
     \ifnum\chapternum=0\setHeadline{#2}\fi
     \emsg{appendix \@fullID #2}%
     \def\@quote{\string\@quote\relax}%
     \addTOC{1}{\TOCsID{\lab@l.}#2}{\folio}%
   \endgroup
   \s@ction
   \aftersection}%
\def\pagecheck#1{%
   \dimen@=\pagegoal
   \advance\dimen@ by -\pagetotal
   \ifdim\dimen@>0pt
   \ifdim\dimen@< #1\relax
      \vfil\break \fi\fi
   }
\def\nosechead#1{%
   \vskip\subsectionskip
   \goodbreak\pagecheck\sectionminspace
   \checkquote\checkenv
   \vbox{%
     {\raggedright\bf\noindent
     {#1}%
     \nobreak\medskip}}%
   }
\def\checkenv{%
   \ifx\@envdepth\undefined\relax
   \else\ifnum\@envdepth=\z@
      \else\emsg{> Unclosed environment \@envname in the last section!}\fi 
   \fi}%
% file: TXStags.tex  (TeXsis version 2.16)
%%> {Labels and tags.}
\newread\auxfilein
\newwrite\auxfileout
\newif\ifauxswitch      \auxswitchtrue
\let\XA=\expandafter    \let\NX=\noexpand
\catcode`"=12
\catcode`@=11
\newcount\@BadTags   \@BadTags= 0
\def\auxinit{%
  \ifauxswitch
    \immediate\openout\auxfileout=\jobname.aux
    \immediate\write\auxfileout{\string\let\string\XA=\string\expandafter}%
  \else
    \gdef\auxout##1##2{}%
  \fi
  \gdef\auxinit{\relax}}%
\def\auxout#1#2{\auxinit
   \immediate\write\auxfileout{%
   \string\XA\string\gdef\string\csname\space
     #1\string\endcsname{#2}}%
   }       
\def\ReadAUX{%
   \openin\auxfilein=\jobname.aux
   \ifeof\auxfilein\closein\auxfilein
   \else\closein\auxfilein
     \begingroup
      \unSpecial
      \input\jobname.aux \relax
     \endgroup
   \fi}%
\def\tag{%
   \begingroup\unSpecial
    \@tag}%
\def\@tag#1#2{%
   \endgroup
   \edef\@@temp{#2}%
   \stripblanks @#1@\endlist
   \XA\let\csname\tok\endcsname=\z@%   
   \XA\xdef\csname\tok\endcsname{\@@temp}%
   \auxout{\tok}{\@@temp}%
   }
\def\label{%
   \begingroup\unSpecial
    \@label}
\def\@label#1{%
   \endgroup
   \tag{#1}{\lab@l}}%
\def\lab@l{\relax}%
\def\use{%
   \begingroup
    \unSpecial
    \@use}%
\def\@use#1{%
   \endgroup
   \stripblanks @#1@\endlist
   \XA\ifx\csname\tok\endcsname\relax\relax
     \emsg{> UNDEFINED TAG #1 ON PAGE \folio.}%
     \global\advance\@BadTags by 1
     \@errmark{UNDEF}%
     \edef\tok{{\bf\tok}}%
   \else
     \edef\tok{\csname\tok\endcsname}%
   \fi
   \tok}%
\def\unSpecial{%
   \catcode`@=11\catcode`"=12
   \catcode`(=12\catcode`)=12
   \catcode`[=12\catcode`]=12
   \catcode`\&=12\catcode`\#=12
   }
\def\stripblanks{%
   \let\tok=\empty\@stripblanks}
\def\@stripblanks#1{\def\next{#1}\@striplist}
\def\@striplist{%
   \ifx\next\stripblanks\message{>\NX\@striplist: Oops!}\next=\endlist\fi
   \ifx\next\endlist\let\next=\relax
   \else\@stripspace\let\next=\@stripblanks\fi
   \next}
\def\@stripspace{\XA\if\space\next\else\edef\tok{\tok\next}\fi}
\def\endlist{\endlist}%
\def\tok{}
\newif\ifundefined      \undefinedfalse
\def\testtag#1{\stripblanks @#1@\endlist   
   \XA\ifx\csname\tok\endcsname\relax\undefinedtrue
   \else\undefinedfalse\fi}
\def\checktags{%
  \ifnum\@BadTags>\z@
    \emsg{>}\emsg{> There were \the\@BadTags\ undefined tags.}%
    \emsg{> See the file \jobname.log for the citations, or try running}%
    \emsg{> TeXsis again to resolve forward references.}\emsg{>}%
  \fi}
\def\LabelParse#1;#2;#3\endlist{%
  \if ?#3?\relax
    \global\advance\@count by\@ne
    \xdef\@ID{\@chaptID\@sectID\the\@count}%
    \tag{\@prefix#1}{\@ID}%
  \else
    \stripblanks #2\endlist
    \edef\p@rt{\tok}%
    \if a\p@rt\relax
      \global\advance\@count by\@ne\fi
    \xdef\@ID{\@chaptID\@sectID\the\@count}%
    \tag{\@prefix#1}{\@ID}%
    \xdef\@ID{\@chaptID\@sectID\the\@count\p@rt}%
    \tag{\@prefix#1;\p@rt}{\@ID}%
  \fi
}%
\def\@ID{}%
% file: TXStitle.tex                                TeXsis version 2.16  
%%> {Title Page macros.}
\catcode`@=11
\def\pubcode#1{\gdef\@DOCcode{#1}}
\def\PUBcode#1{\gdef\@DOCcode{#1}}%
\def\DOCcode#1{\PUBcode{#1}}%
\def\BNLcode#1{\PUBcode{#1}\banner}%
\def\@DOCcode{\TeXsis~\fmtversion}%
\def\pubdate#1{\gdef\@PUBdate{#1}}
\def\PUBdate#1{\gdef\@PUBdate{#1}}%
\def\@PUBdate{\monthname{\month},~\number\year}%
\def\ORGANIZATION{BROOKHAVEN\ NATIONAL\ LABORATORY}%
\def\banner{%
   \line{\hfil
      \vbox to 0pt{\vss \hbox{\twelvess \ORGANIZATION}}%
      \hfil}%
   \vskip 12pt
   \hrule height 0.6pt \vskip 1pt \hrule height 0.6pt
   \vskip 4pt \relax
   \line{\twelvepoint\rm\@PUBdate \hfil \@DOCcode}%
   \vskip 3pt
   \hrule height 0.6pt \vskip 1pt \hrule height 0.6pt
   \vskip 0pt plus 1fil
   \vskip 1.0cm minus 1.0cm
   \relax}
\def\titlepage{%
   \bgroup
   \pageno=1
   \hbox{\space}%
   \let\title=\Title
   \let\endmode=\relax
   }
\def\endtitlepage{%
   \endmode
   \vfil\eject
   \egroup}%
\def\Title{%
   \endmode
   \vskip 0pt
   \mark{Title Page\NX\else Title Page}% mark page so no \HeadLine
   \bgroup
   \let\endmode=\endTitle
   \center\Tbf}%
\let\title=\Title
\def\endTitle{%
   \endcenter
   \bigskip
   \gdef\title{%
      \emsg{> Please use \NX\booktitle instead of \NX\title.}%
      \@errmark{OLD!}%
      \booktitle}%
   \egroup}%
\def\endtitle{\endTitle}%
\def\Tbf{\sixteenpoint\bf}%
\def\author{%
  \endmode
  \bgroup
   \let\endmode=\endauthor
   \singlespaced\parskip=0pt
   \obeylines\def\\{\par}%
   \@getauthor}%
{\obeylines\gdef\@getauthor#1
  #2
  {#1\bigskip\def\n{\egroup\centerline\bgroup\bf}%
   \centerline{\bf #2}%
   \medskip\center}%
}
\def\endauthor{\endcenter\egroup\bigskip}
\def\authors{%
   \endmode
   \bigskip
   \bgroup
    \let\endmode=\endauthors
    \let\@uthorskip=\medskip
    \raggedcenter\singlespaced}%
\def\endauthors{%
   \endraggedcenter
   \egroup
   \bigskip}%
\def\note#1#2{%
  ${}^{\hbox{#1}}\ $
  \space@head#2
  #2}%
\def\institution#1#2{%
   \@uthorskip\let\@uthorskip=\relax
   \raggedcenter
      ${}^{\rm #1}$\space #2%                  
   \endraggedcenter
   }
\let\@uthorskip=\medskip
\long\def\titlenote#1#2{%
   \footnote{}{%
   \llap{\hbox to \parindent{\hfil
   ${}^{\rm #1}$\space}}#2}}%
\def\and{\centerline{and}\medskip}
\def\abstract{%
   \endmode
   \bigskip\bigskip
    \centerline{ABSTRACT}%
    \medskip
    \bgroup
    \let\endmode=\endabstract
    \narrower\narrower
    \singlespaced}%
\def\endabstract{\smallskip\egroup}
\def\pacs#1{\medskip\centerline{PACS numbers: #1}\smallskip}
\def\submit#1{\bigskip\centerline{Submitted to {\sl #1}}}
\def\submitted#1{\submit{#1}}%
\def\toappear#1{\bigskip\raggedcenter
     To appear in {\sl #1}
     \endraggedcenter}
\def\disclaimer#1{\footnote{}\bgroup\tenrm\singlespaced
   This manuscript has been authored under contract number #1
   \@disclaimer\par}
\def\disclaimers#1{\footnote{}\bgroup\tenrm\singlespaced
   This manuscript has been authored under contract numbers #1
   \@disclaimer\par}
\def\@disclaimer{%
with the U.S. Department of Energy.  Accordingly, the U.S.
Government retains a non-exclusive, royalty-free license to publish
or reproduce the published form of this contribution,
or allow others to do so, for U.S. Government purposes.
\egroup}
% file: TXSenvmt.tex  (TeXsis version 2.16)
%%> {Environments.}
\catcode`@=11
\chardef\other=12
\def\center{%
   \flushenv
   \advance\leftskip \z@ plus 1fil
   \advance\rightskip \z@ plus 1fil
   \obeylines\@eatpar}%
\def\flushright{%
    \flushenv
    \advance\leftskip \z@ plus 1fil
    \obeylines\@eatpar}%
\def\flushleft{%
   \flushenv
   \advance\rightskip \z@ plus 1fil
   \obeylines\@eatpar}%
\def\flushenv{%
    \vskip \z@
    \bgroup
     \def\flushhmode{F}%
     \parindent=\z@  \parfillskip=\z@}%
\def\endcenter{\endflushenv}
\def\endflushleft{\endflushenv}
\def\endflushright{\endflushenv}
\def\@eatpar{\futurelet\next\@testpar}
\def\@testpar{\ifx\next\par\let\@next=\@@eatpar\else\let\@next=\relax\fi\@next}
\long\def\@@eatpar#1{\relax}
\def\raggedcenter{%
    \flushenv
    \advance\leftskip\z@ plus4em
    \advance\rightskip\z@ plus 4em
    \spaceskip=.3333em \xspaceskip=.5em
    \pretolerance=9999 \tolerance=9999
    \hyphenpenalty=9999 \exhyphenpenalty=9999
    \@eatpar}%
\def\endraggedcenter{\endflushenv}%
\def\hcenter{\hflushenv
   \advance\leftskip \z@ plus 1fil
   \advance\rightskip \z@ plus 1fil
   \obeylines\@eatpar}%
\def\hflushright{\hflushenv
    \advance\leftskip \z@ plus 1fil
    \obeylines\@eatpar}%
\def\hflushleft{\hflushenv
    \advance\rightskip \z@ plus 1fil
    \obeylines\@eatpar}%
\def\hflushenv{%
   \def\par{\endgraf\indent}%
   \hbox to \z@ \bgroup\hss\vtop
   \flushenv\def\flushhmode{T}}%
\def\endflushenv{%
   \ifhmode\endgraf\fi
   \if T\flushhmode \egroup\hss\fi
   \egroup}%
\def\endhcenter{\endflushenv}
\def\endhflushleft{\endflushenv}
\def\endhflushright{\endflushenv}
\newskip\EnvTopskip     \EnvTopskip=\medskipamount
\newskip\EnvBottomskip  \EnvBottomskip=\medskipamount
\newskip\EnvLeftskip    \EnvLeftskip=2\parindent
\newskip\EnvRightskip   \EnvRightskip=\parindent
\newskip\EnvDelt@skip   \EnvDelt@skip=0pt
\newcount\@envDepth     \@envDepth=\z@
\def\beginEnv#1{%
   \begingroup
   \def\@envname{#1}%
   \ifvmode\def\@isVmode{T}%
   \else\def\@isVmode{F}\vskip 0pt\fi
   \ifnum\@envDepth=\@ne\parindent=\z@\fi
   \global\advance\@envDepth by \@ne
   \EnvDelt@skip=\baselineskip
   \advance\EnvDelt@skip by-\normalbaselineskip
   \@setenvmargins\EnvLeftskip\EnvRightskip
   \setenvskip{\EnvTopskip}%
   \vskip\skip@\penalty-500
   }
\def\endEnv#1{%
   \ifnum\@envDepth<1
      \emsg{> Tried to close ``#1'' environment, but no environment open!}%
      \begingroup
   \else
      \def\test{#1}%
      \ifx\test\@envname\else
         \emsg{> Miss-matched environments!}%
         \emsg{> Should be closing ``\@envname'' instead of ``\test''}%
      \fi
   \fi
   \vskip 0pt
   \setenvskip\EnvBottomskip
   \vskip\skip@\penalty-500
   \xdef\@envtemp{\@isVmode}%
   \endgroup
   \global\advance\@envDepth by -\@ne
   \if F\@envtemp\vskip-\parskip\noindent\fi
   }
\def\setenvskip#1{\skip@=#1 \divide\skip@ by \@envDepth}
\def\@setenvmargins#1#2{%
   \advance \leftskip  by #1    \advance \displaywidth by -#1
   \advance \rightskip by #2    \advance \displaywidth by -#2
   \advance \displayindent by #1}%
\def\itemize{\beginEnv{itemize}%
   \let\itm=\itemizeitem
   \if F\@isVmode\vskip-\parskip\fi
   }
\def\itemizeitem{%
   \par\noindent
   \hbox to 0pt{\hss\itemmark\space}}%
\def\enditemize{\endEnv{itemize}}%
\def\itemmark{$\bullet$}%
\newcount\enumDepth     \enumDepth=\z@
\newcount\enumcnt
\def\enumerate{\beginEnv{enumerate}%
   \global\advance\enumDepth by \@ne
   \setenumlead
   \enumcnt=\z@
   \let\itm=\enumerateitem
   \if F\@isVmode\vskip-\parskip\fi
   }
\def\enumerateitem{%
    \par\noindent                 
    \advance\enumcnt by \@ne
    \edef\lab@l{\enumlead \enumcur}%
    \hbox to \z@{\hss \lab@l \enummark
       \hskip .5em\relax}%
    \ignorespaces}%
\def\endenumerate{%
   \global\advance\enumDepth by -\@ne
   \endEnv{enumerate}}%
\def\enumPoints{%
   \def\setenumlead{\ifnum\enumDepth>1
          \edef\enumlead{\enumlead\enumcur.}%
      \else\def\enumlead{}\fi}%
   \def\enumcur{\number\enumcnt}%
   }
\def\enumpoints{\enumPoints}%
\def\enumOutline{%
   \def\setenumlead{\def\enumlead{}}%
   \def\enumcur{\ifcase\enumDepth
     \or\uppercase{\XA\romannumeral\number\enumcnt}%
     \or\LetterN{\the\enumcnt}%
     \or\XA\romannumeral\number\enumcnt
     \or\letterN{\the\enumcnt}%
     \or{\the\enumcnt}%
     \else $\bullet$\space\fi}%
   }
\def\enumoutline{\enumOutline}%
\def\enumNumOutline{%
   \def\setenumlead{\def\enumlead{}}%
   \def\enumcur{\ifcase\enumDepth
      \or{\XA\number\enumcnt}%
      \or\letterN{\the\enumcnt}%
      \or{\XA\romannumeral\number\enumcnt}%
      \else $\bullet$\space\fi}%
   }
\def\enumnumoutline{\enumNumOutline}%
\def\LetterN#1{\count@=#1 \advance\count@ 64 \XA\char\count@}
\def\letterN#1{\count@=#1 \advance\count@ 96 \XA\char\count@}
\def\enummark{.}%
\def\enumlead{}%
\enumpoints
\newbox\@desbox
\newbox\@desline
\newdimen\@glodeswd
\newcount\@deslines
\newif\ifsingleline \singlelinefalse
\def\description#1{\beginEnv{description}%
   \setbox\@desbox=\hbox{#1}%
   \@glodeswd=\wd\@desbox
   \@setenvmargins{\@glodeswd}{0pt}%
   \let\itm=\descriptionitem
   \if F\@isVmode\vskip-\parskip\fi
  }%
\def\descriptionitem#1{%
   \goodbreak\noindent
   \setbox\@desline=\vtop\bgroup
      \hfuzz=100cm\hsize=\@glodeswd
      \rightskip=\z@ \leftskip=\z@
      \raggedright
      \noindent{#1}\par
      \global\@deslines=\prevgraf
      \egroup
   \ifsingleline
     \ifnum\@deslines>1
        \@deslineitm{#1}%
     \else
        \setbox\@desline=\hbox{#1}%
        \ifdim \wd\@desline>\wd\@desbox
            \@deslineitm{#1}%
        \else\@desitm\fi
     \fi
   \else
     \@desitm
   \fi
   \ignorespaces}
\def\@desitm{%
   \noindent
   \hbox to \z@{\hskip-\@glodeswd
     \hbox to \@glodeswd{\vtop to \z@{\box\@desline\vss}%
     \hss}\hss}}%
\def\@deslineitm#1{%
   \hbox{\hskip-\@glodeswd {#1}\hss}%
   \vskip-\parskip\nobreak\noindent
   }
\def\enddescription{\ifhmode\par\fi
   \@setenvmargins{-\wd\@desbox}{0pt}%
   \endEnv{description}}
\def\example{\beginEnv{example}%
   \parskip=\z@ \parindent=\z@
   \baselineskip=\normalbaselineskip
   }%
\def\endexample{\endEnv{example}%
   \noindent}%
\let\blockquote=\example
\let\endblockquote=\endexample
\def\Listing{%
   \beginEnv{Listing}%
   \vskip\EnvDelt@skip
   \baselineskip=\normalbaselineskip
   \parskip=\z@ \parindent=\z@
   \def\\##1{\char92##1}%
   \catcode`\{=\other \catcode`\}=\other
   \catcode`\(=\other \catcode`\)=\other
   \catcode`\"=\other \catcode`\|=\other
   \catcode`\%=\other \catcode`\&=\other        
   \catcode`\-=\other \catcode`\==\other
   \catcode`\$=\other \catcode`\#=\other
   \catcode`\_=\other \catcode`\^=\other
   \catcode`\~=\other
   \obeylines
   \tt\Listingtabs
   \everyListing}%
\def\endListing{\endEnv{Listing}}%
\def\everyListing{\relax}
\def\ListCodeFile#1{%
   \Listing
   \hsize=2\hsize\raggedright
   \catcode`\\=\other
   \input #1\relax
   \endListing}
{\catcode`\^^I=\active\catcode`\ =\active
\gdef\Listingtabs{\catcode`\^^I=\active\let^^I\@listingtab
\catcode`\ =\active\let \@listingspace}%
}%
\def\@listingspace{\hskip 0.5em\relax}%
\def\@listingtab{\hskip 4em\relax}%
\def\TeXexample{\beginEnv{TeXexample}%
   \vskip\EnvDelt@skip
   \parskip=\z@ \parindent=\z@
   \baselineskip=\normalbaselineskip
   \def\par{\leavevmode\endgraf}%
   \obeylines
   \catcode`|=\z@
   \ttverbatim
   \@eatpar}%
\def\endTeXexample{%
   \vskip 0pt
   \endgroup
   \endEnv{TeXexample}}%
\def\ttverbatim{\begingroup
   \catcode`\(=\other \catcode`\)=\other
   \catcode`\"=\other \catcode`\[=\other
   \catcode`\]=\other
   \let\do=\uncatcode \dospecials
   \obeyspaces\obeylines
   \def\n{\vskip\baselineskip}%
   \tt}%
\def\uncatcode#1{\catcode`#1=\other}%
{\obeyspaces\gdef {\ }}%
\def\TeXquoteon{\catcode`\|=\active}%
\let\TeXquoteson=\TeXquoteon
\def\TeXquoteoff{\catcode`\|=\other}%
\let\TeXquotesoff=\TeXquoteoff
{\TeXquoteon\obeylines% active "|" calls \ttverbatim
   \gdef|{\ifmmode\vert\else% | is \vert in math mode, but
     \ttverbatim\spaceskip=\ttglue% to use \tt type
     \let^^M=\ \relax% and to ignore ^^M
     \let|=\endgroup\fi}%
}     
\def\ttvert{\hbox{\tt\char`\|}}
\outer\def\begintt{$$\let\par=\endgraf \ttverbatim \parskip=0pt
   \catcode`\|=0 \rightskip=-5pc \ttfinish}
{\catcode`\|=0 |catcode`|\=\other% | is temporary escape character
   |obeylines% end of line is active
   |gdef|ttfinish#1^^M#2\endtt{#1|vbox{#2}|endgroup$$}%
}
\def\beginlines{\par\begingroup\nobreak\medskip\parindent=0pt
   \hrule\kern1pt\nobreak \obeylines \everypar{\strut}}
\def\endlines{\kern1pt\hrule\endgroup\medbreak\noindent}
\def\beginproclaim#1#2#3#4#5{\medbreak\vskip-\parskip
   \global\XA\advance\csname #2\endcsname by \@ne
   \edef\lab@l{\@chaptID\@sectID
      \number\csname #2\endcsname}%
   \tag{#4#5}{\lab@l}%
   \noindent{\bf #1 \lab@l.\space}%
   \begingroup #3}%
\def\endproclaim{%
   \par\endgroup\ifdim\lastskip<\medskipamount
   \removelastskip\penalty55\medskip\fi}%
\newcount\theoremnum           \theoremnum=\z@
\def\theorem#1{\beginproclaim{Theorem}{theoremnum}{\sl}{Thm.}{#1}}
\let\endtheorem=\endproclaim
\def\Theorem#1{Theorem~\use{Thm.#1}}
\newcount\lemmanum             \lemmanum=\z@
\def\lemma#1{\beginproclaim{Lemma}{lemmanum}{\sl}{Lem.}{#1}}
\let\endlemma=\endproclaim
\def\Lemma#1{Lemma~\use{Lem.#1}}
\newcount\corollarynum         \corollarynum=\z@
\def\corollary#1{\beginproclaim{Corollary}{corollarynum}{\sl}{Cor.}{#1}}
\let\endcorollary=\endproclaim
\def\Corollary#1{Corollary~\use{Cor.#1}}
\newcount\definitionnum        \definitionnum=\z@
\def\definition#1{\beginproclaim{Definition}{definitionnum}{\rm}{Def.}{#1}}
\let\enddefinition=\endproclaim
\def\Definition#1{Definition~\use{Def.#1}}
\def\proof{\medbreak\vskip-\parskip\noindent{\it Proof. }}
\def\blackslug{%
   \setbox0\hbox{(}%
   \vrule width.5em height\ht0 depth\dp0}%
\def\QED{\blackslug}%
\def\endproof{\quad\blackslug\par\medskip}
% file: TXSfmts.tex                                 TeXsis version 2.16
%%> {Layout macros.}
\catcode`@=11
\def\paper{%
   \auxswitchtrue
   \refswitchtrue
   \texsis
   \def\titlepage{%
      \bgroup
      \let\title=\Title
      \let\endmode=\relax
      \pageno=1}%
   \def\endtitlepage{%
      \endmode
      \goodbreak\bigskip
      \egroup}%
   \autoparens
   \quoteon
   }
\def\Tbf{\fourteenpoint\bf}%
\def\tbf{\twelvepoint\bf}%
\def\preprint{%
   \auxswitchtrue
   \refswitchtrue
   \texsis
   \def\titlepage{%
      \bgroup
      \pageno=1
      \let\title=\Title
      \let\endmode=\relax
      \banner}%
   \def\endtitlepage{%
      \endmode
      \vfil\eject
      \egroup}%
   \autoparens
   \quoteon
   }
\def\Manuscript{%
   \preprint
   \showsectIDfalse
   \showchaptIDfalse
   \def\SectionStyle##1{\uppercase
         \expandafter{\romannumeral ##1}}%
   \RomanTablestrue
   \TablesLast
   \FiguresLast
   \TrueDoubleSpacing
   }
\autoload\PhysRevManuscript{PhysRev.txs}%
\def\book{%
   \ContentsSwitchtrue
   \refswitchtrue
   \auxswitchtrue
   \texsis
   \RunningHeadstrue
   \bookpagenumbers
   \def\titlepage{%
      \bgroup
      \pageno=-1
      \let\title=\Title
      \let\endmode=\relax
      \def\FootText{\relax}}%
   \def\endtitlepage{%
      \endmode
      \vfil\eject
      \egroup
      \pageno=1}%
   \def\abstract{%
      \endmode
      \bigskip\bigskip\medskip
      \bgroup\singlespaced
         \let\endmode=\endabstract
         \narrower\narrower}%
   \def\endabstract{%
      \medskip\egroup\bigskip}%
   \def\FootText{--\ \tenrm\folio\ --}%
   \def\Tbf{\sixteenpoint\bf}%
   \def\tbf{\fourteenpoint\bf}%
   \twelvepoint
   \doublespaced
   \autoparens
   \quoteon
   }%
\autoload\thesis{thesis.txs}
\autoload\UTthesis{thesis.txs}
\autoload\YaleThesis{thesis.txs}
\def\Letter{%
   \ContentsSwitchfalse
   \refswitchfalse
   \auxswitchfalse
   \texsis
   \singlespaced
   \LetterFormat}%
\def\letter{\Letter}%
\def\Memo{%
   \ContentsSwitchfalse
   \refswitchfalse
   \auxswitchfalse
   \texsis
   \singlespaced
   \MemoFormat}%
\def\memo{\Memo}%
\def\Referee{%
   \ContentsSwitchfalse
   \auxswitchfalse
   \refswitchfalse
   \texsis
   \RefReptFormat}%
\def\referee{\Referee}%
\def\Landscape{%
   \texsis
   \hsize=9in
   \vsize=6.5in
   \voffset=.5in
   \nopagenumbers
   \LandscapeSpecial
}
\def\landscape{\Landscape}%
\def\LandscapeSpecial{\special{papersize=11in,8.5in}}
\def\slides{%
   \quoteon
   \autoparens
   \ATlock
   \pageno=1
   \twentyfourpoint
   \doublespaced
   \raggedright\tolerance=2000
   \hyphenpenalty=500
   \raggedbottom
   \nopagenumbers
   \hoffset=-.25in \hsize=7.0in
   \voffset=-.25in \vsize=9.0in
   \parindent=30pt
   \def\bl{\vskip\normalbaselineskip}%
   \def\np{\vfill\eject}%
   \def\nospace{\nulldelimiterspace=0pt
      \mathsurround=0pt}%
   \def\big##1{{\hbox{$\left##1
      \vbox to2ex{}\right.\nospace$}}}%
   \def\Big##1{{\hbox{$\left##1
      \vbox to2.5ex{}\right.\nospace$}}}%
   \def\bigg##1{{\hbox{$\left##1
       \vbox to3ex{}\right.\nospace$}}}%
   \def\Bigg##1{{\hbox{$\left##1
      \vbox to4ex{}\right.\nospace$}}}%
  }
\autoload\twinout{twin.txs}%
\def\twinprint{%
   \preprint
   \let\t@tl@=\title
   \def\title{\vskip-1.5in\t@tl@}%
   \let\endt@tlep@ge=\endtitlepage
   \def\endtitlepage{\endt@tlep@ge
       \twinformat}%
}
\def\twinformat{%
   \tenpoint\doublespaced
   \def\Tbf{\twelvebf}\def\tbf{\tenbf}%
   \headlineoffset=0pt
   \twinout}%
% file: TXSfigs.tex (TeXsis version 2.16)
%%> {Figures and Tables.}
\catcode`\@=11
\let\NX=\noexpand\let\XA=\expandafter
\offparens
\newcount\tabnum        \tabnum=\z@
\newcount\fignum        \fignum=\z@
\newif\ifRomanTables    \RomanTablesfalse
\long\def\figure{\@figure\topinsert}
\long\def\topfigure{\@figure\topinsert}%
\long\def\midfigure{\@figure\midinsert}
\long\def\fullfigure{\@figure\pageinsert}
\long\def\bottomfigure{\@figure\bottominsert}
\long\def\heavyfigure{\@figure\heavyinsert}
\long\def\widefigure{\@figure\widetopinsert}
\long\def\widetopfigure{\@figure\widetopinsert}
\long\def\widefullfigure{\@figure\widepageinsert}
\obsolete\topFigure\figure   \obsolete\midFigure\midfigure 
\obsolete\fullFigure\fullfigure \obsolete\TOPFIGURE\figure              
\obsolete\MIDFIGURE\midfigure   \obsolete\FULLFIGURE\fullfigure            
\obsolete\endFigure\endfigure   \obsolete\ENDFIGURE\endfigure
\def\@figure#1#2{%
  \ifvmode\else\vskip 0pt\fi
  \begingroup
   \let\@count=\fignum
   \def\@prefix{Fg.}%
   \let\@FigInsert=#1\relax
   \if ?#2?\relax \def\@ID{}%
   \else\LabelParse #2;;\endlist\fi
   \def\CaptionName{Figure}%
   \ifFigsLast
     \emsg{\CaptionName\space\@ID. {#2} [storing in \jobname.fg]}%
     \@fgwrite{\@comment> \CaptionName\space\@ID.\space{#2}}%
     \@fgwrite{\NX\@FigureItem{\CaptionName}{\@ID}{\NX#1}}%
     \newlinechar=`\^^M
     \obeylines
     \let\@next=\@copyfig
   \else
     \@FigInsert
     \setbox\@capbox\vbox to 0pt{}%
     \def\@whereCap{N}%
     \emsg{\CaptionName\ \@ID.\ {#2}}%
     \let\endfigure=\@endfigure
     \let\@next=\@findcap
   \fi
   \@next}
\long\def\@findcap#1{%
   \ifx #1\Caption \def\@whereCap{T}\fi
   \ifx #1\caption \def\@whereCap{T}\fi
   #1}%
\def\@whereCap{N}%
\def\@endfigure{\relax
   \if B\@whereCap\relax
     \vskip\baselineskip\medskip
     \centerline{\box\@capbox}%
   \fi
   \endinsert \endgroup}%
\def\endfigure{\emsg{> \NX\endfigure before \NX\figure!}}
\def\figuresize#1{\vglue #1}%
\newbox\@capbox
\newcount\@caplines
\def\CaptionName{}%
\def\@ID{}%
\def\caption#1{%
   \def\lab@l{\@ID}%
   \global\setbox\@capbox=\vbox\bgroup
     \def\@inCaption{T}%
     \captionspacing
     \dimen@=20\parindent
     \ifdim\colwidth>\dimen@\narrower\narrower\fi
     \noindent{\bf \CaptionName~\@ID:\space}%
     #1\relax
     \vskip0pt
     \global\@caplines=\prevgraf
   \egroup
   \ifnum\@ne=\@caplines
    \global\setbox\@capbox=\vbox\bgroup
       \noindent
       \hfil{\bf \CaptionName~\@ID:\space}%
       #1\hfil\egroup
   \fi
   \def\@inCaption{F}%
   \if N\@whereCap\def\@whereCap{B}\fi
   \if T\@whereCap
     \centerline{\box\@capbox}%
     \vskip\baselineskip\medskip
   \fi}
\def\captionspacing{\normalbaselines}%
\def\@inCaption{F}%
\long\def\Caption#1\endCaption{\caption{#1}}
\def\endCaption{\emsg{> \NX\endCaption called before \NX\Caption.}}
\def\endcaption{\emsg{> try using \NX\caption{ text... }}}
\long\def\table{\@table\topinsert}
\long\def\toptable{\@table\topinsert}%
\long\def\midtable{\@table\midinsert}
\long\def\fulltable{\@table\pageinsert}
\long\def\bottomtable{\@table\bottominsert}
\long\def\heavytable{\@table\heavyinsert}
\long\def\widetable{\@table\widetopinsert}
\long\def\widetoptable{\@table\widetopinsert}
\long\def\widefulltable{\@table\widepageinsert}
\let\topTable=\toptable
\let\midTable=\midtable
\let\fullTable=\fulltable
\let\TOPTABLE=\toptable
\let\MIDTABLE=\midtable
\let\FULLTABLE=\fulltable
\def\@table#1#2{%
  \vskip 0pt
  \begingroup
   \def\CaptionName{Table}%
   \def\@prefix{Tb.}%
   \let\@count=\tabnum
   \if ?#2?\relax \def\@ID{}%
   \else
     \ifRomanTables
      \global\advance\@count by\@ne
      \edef\@ID{\uppercase\expandafter
         {\romannumeral\the\@count}}%
      \tag{\@prefix#2}{\@ID}%
     \else
       \LabelParse #2;;\endlist\fi
   \fi
   \ifTabsLast
    \emsg{\CaptionName\space\@ID. {#2} [storing in \jobname.tb]}%
    \@tbwrite{\@comment> \CaptionName\space\@ID.\space{#2}}%
    \@tbwrite{\NX\@FigureItem{\CaptionName}{\@ID}{\NX#1}}%
    \newlinechar=`\^^M
    \obeylines
    \let\@next=\@copytab
   \else
    #1\relax
    \setbox\@capbox\vbox to 0pt{}%
    \def\@whereCap{N}%
    \emsg{\CaptionName\ \@ID.\ {#2}}%
    \let\endtable=\@endfigure
    \let\endTable=\@endfigure
    \let\ENDTABLE=\@endfigure
    \let\@next=\@findcap
   \fi
   \@next}                                      
\def\endtable{\emsg{> \NX\endtable before \NX\table!}}
\def\endTable{\endtable}%
\def\ENDTABLE{\endtable}%
\autoload\Tablebody{Tablebod.txs}
\autoload\Tablebodyleft{Tablebod.txs}          
\autoload\tablebody{Tablebod.txs}
\def\Fig#1{Fig.~\use{Fg.#1}}
\def\Figs#1{Figs.~\use{Fg.#1}}
\def\Fg#1{\use{Fg.#1}}
\def\Tab#1{Table~\use{Tb.#1}}
\def\Tbl#1{Table~\use{Tb.#1}}
\def\FIG{\@obsolete\FIG\Fig\Fig}%
\def\TBL{\@obsolete\TBL\Tbl\Tbl}%
\newif\ifFigsLast       \FigsLastfalse
\def\FiguresLast{\FigsLasttrue}
\def\FiguresNow{\FigsLastfalse}
\newwrite\fgout
\def\@fginit{%
   \immediate\openout\fgout=\jobname.fg \relax
   \gdef\@fginit{\relax}%
   \immediate\write\fgout{\@comment FIGURES for job \jobname}%
   \immediate\write\fgout{\@comment ====================================}%
}
\def\@fgwrite#1{%
  \@fginit
  \immediate\write\fgout{#1}%
  }
\long\def\@fgNXwrite#1{%
  \@fginit
  \unexpandedwrite\fgout{#1}%
  }
{\obeylines\long\gdef\@copyfig#1
  {\@ParseFigText#1\endfigure\endfigure\endlist% Parse line
    \@nextline}%
}%
\long\def\@ParseFigText#1\endfigure#2\endfigure#3\endlist{%
   \ifx?#1?\relax
   \else\@fgNXwrite{#1}\fi
   \ifx?#3?\let\@nextline=\@copyfig
   \else
     \gdef\@remainder{#2}%
     \let\@nextline=\@endcopyfig
   \fi
   }
\def\@endcopyfig{\@fgNXwrite{\@endfigure}%
   \endgroup
   \@remainder
   }
\def\PrintFigures{%
  \ifFigsLast\@PrintFigures\fi}
\def\@PrintFigures{%
   \@fgwrite{\@comment>>> EOF \jobname.fg <<<}%
   \immediate\closeout\fgout
   \begingroup
      \FigsLastfalse
      \vbox to 0pt{\hbox to 0pt{\ \hss}\vss}%
      \ATunlock
      \offparens
      \catcode`@=11
      \unobeylines
      \emsg{[Getting figures from file \jobname.fg]}%
      \Input\jobname.fg \relax
   \endgroup
}
\def\@FigureItem#1#2#3{%
   \begingroup
    #3\relax
    \def\@ID{#2}%
    \def\CaptionName{#1}%
    \setbox\@capbox\vbox to 0pt{}%
    \def\@whereCap{N}%
    \@findcap
    }
\newif\ifTabsLast       \TabsLastfalse
\def\TablesLast{\TabsLasttrue}
\def\TablesNow{\TabsLastfalse}
\newwrite\tbout
\def\@TBinit{%
   \immediate\openout\tbout=\jobname.tb \relax
   \gdef\@TBinit{\relax}%
   \immediate\write\tbout{\@comment TABLES FOR \jobname.tex}%
   \immediate\write\tbout{\@comment ====================================}%
}
\def\@tbwrite#1{%
  \@TBinit
  \immediate\write\tbout{#1}%
  }
\long\def\@tbNXwrite#1{%
  \@TBinit
  \unexpandedwrite\tbout{#1}%
  }
{\obeylines\long\gdef\@copytab#1
   {\@ParseTabText#1\endtable\endtable\endlist% PARSE line
    \@nextline}%
}%
\long\def\@ParseTabText#1\endtable#2\endtable#3\endlist{%
   \ifx?#1?\relax
   \else\@tbNXwrite{#1}\fi
   \ifx?#3?\let\@nextline=\@copytab
   \else
     \gdef\@remainder{#2}%
     \let\@nextline=\@endcopytab
   \fi
   }
\def\@endcopytab{\@tbNXwrite{\@endfigure}%
   \endgroup
   \@remainder
   }
\def\PrintTables{%
  \ifTabsLast\@PrintTables\fi}
\def\@PrintTables{%
   \@tbwrite{\@comment>>> EOF \jobname.tb <<<}%
   \immediate\closeout\tbout
   \TabsLastfalse
   \begingroup
     \catcode`@=11
     \offparens
     \unobeylines
     \emsg{[Getting tables from file.]}%
     \Input\jobname.tb \relax
   \endgroup
}
\newwrite\figlist
\newwrite\tablelist
\def\FigListInit{\relax
   \immediate\openout\figlist=\jobname.fgl\relax
   \immediate\write\figlist{\@comment List of Figures for job `\jobname'
                                -- Created at \runtime}%
   \immediate\write\figlist{\@comment ============================}%
   \gdef\FigListInit{\relax}%
 }
\def\TabListInit{\relax
   \immediate\openout\tablelist=\jobname.tbl\relax
   \immediate\write\tablelist{\@comment List of Tables for job `\jobname'
                                -- Created: \runtime}%
   \immediate\write\tablelist{\@comment ===========================}%
   \gdef\TabListInit{\relax}%
 }
\def\FigListWrite#1#2{\relax
  \ifx#1\figlist\relax   \FigListInit\fi
  \ifx#1\tablelist\relax \TabListInit\fi
  \edef\@line@{{#2}}%
  \write#1\@line@}%
\def\FigListWriteNX#1#2{\relax
  \ifx#1\figlist\relax\FigListInit\fi
  \ifx#1\tablelist\relax\TabListInit\fi
  \writeNX#1{#2}%
  }
\def\infiglist{%
   \begingroup\obeylines
   \newlinechar=`\^^M
     \@infiglist\figlist}%
\def\intablelist{%
   \begingroup\obeylines
   \newlinechar=`\^^M
     \@infiglist\tablelist}                               
\def\@infiglist#1#2{%
   \FigListWrite#1{\@comment -- \CaptionName\space\@ID:}%
   \FigListWrite#1{\string\FIGLitem{\CaptionName} {\@ID.\space}}%
   \FigListWriteNX#1{{#2}}%
   \FigListWrite#1{{\folio}}%
   \endgroup}%
\def\FIGLitem#1#2#3#4{%
   \medskip
   \begingroup
     \raggedright\tolerance=1700
     \advance\rightskip by \TOCmargin
     \parfillskip=-\TOCmargin
     \hangindent=1.41\parindent\hangafter=1
     \noindent \ifshowsectID #1\ \fi #2
        #3 \hskip 0pt plus 10pt
     \leaddots
     \hbox to 2em{\hss #4}%
     \vskip 0pt
   \endgroup
   }
\def\ListFigures{%
   \emsg{\@comment List of Figures:}%
   \immediate\closeout\figlist
   \openin\figlist=\jobname.fgl
   \ifeof\figlist\closein\figlist
      \emsg{> \NX\ListFigures: no list of figures.}%
   \else\closein\figlist
     \begingroup
      \ATunlock\quoteoff\offparens
      \Input\jobname.fgl \relax
     \endgroup
   \fi}%
\def\ListTables{%
   \emsg{\@comment List of Tables}%
   \immediate\closeout\tablelist
   \openin\tablelist=\jobname.tbl
   \ifeof\tablelist\closein\tablelist
      \emsg{> \NX\ListTables: no list of tables.}%
   \else\closein\tablelist
     \begingroup
      \ATunlock\quoteoff\offparens
      \Input\jobname.tbl \relax
     \endgroup
   \fi}%
\def\ListFigureCaptions{%
   \showsectIDtrue
   \emsg{\@comment List of Figure Captions:}%
   \immediate\closeout\figlist
   \openin\figlist=\jobname.fgl
   \ifeof\figlist\closein\figlist
      \emsg{> \NX\ListFigureCaptions: no list of figures.}%
   \else\closein\figlist
     \begingroup
      \ATunlock\quoteoff\offparens
      \let\FIGLitem=\CAPLitem
      \Input\jobname.fgl \relax
     \endgroup
   \fi}%
\def\ListTableCaptions{%
   \showsectIDtrue
   \emsg{\@comment List of Table Captions:}%
   \immediate\closeout\figlist
   \openin\figlist=\jobname.tbl
   \ifeof\figlist\closein\figlist
      \emsg{> \NX\ListFigureCaptions: no list of figures.}%
   \else\closein\figlist
     \begingroup
      \ATunlock\quoteoff\offparens
      \let\FIGLitem=\CAPLitem
      \Input\jobname.tbl \relax
     \endgroup
   \fi}%
\def\CAPLitem#1#2#3#4{%
   \medskip
   \begingroup
     \raggedright\tolerance=1700
     \advance\rightskip by \TOCmargin
     \parfillskip=-\TOCmargin
     \hangindent=1.41\parindent\hangafter=1
     \noindent \ifshowsectID #1\ \fi #2
        #3 \hskip 0pt plus 10pt
     \vskip 0pt
   \endgroup}%
\autoload\epsffile{epsf.tex}
\autoload\epsfbox{epsf.tex}
\autoload\epsfxsize{epsf.tex}
\autoload\epsfysize{epsf.tex}   
\autoload\epsfverbosetrue{epsf.tex}
\autoload\epsfverbosefalse{epsf.tex}
% file: TXSruled.tex -- TeXsis 2.16
\catcode`@=11
\catcode`\|=12
\catcode`\&=4
\newcount\ncols         \ncols=\z@
\newcount\nrows         \nrows=\z@
\newcount\curcol        \curcol=\z@
\let\currow=\nrows
\newdimen\thinsize      \thinsize=0.6pt
\newdimen\thicksize     \thicksize=1.5pt
\newdimen\tablewidth    \tablewidth=-\maxdimen
\newdimen\parasize      \parasize=4in
\newif\iftableinfo      \tableinfotrue
\newif\ifcentertables   \centertablestrue
\def\centeredtables{\centertablestrue}%
\def\noncenteredtables{\centertablesfalse}%
\def\nocenteredtables{\centertablesfalse}%
\let\plaincr=\cr
\let\plainspan=\span
\let\plaintab=&
\def\ampersand{\char`\&}%
\let\lparen=(
\let\NX=\noexpand
\def\ruledtable{\relax
    \@BeginRuledTable
    \@RuledTable}%
\def\@BeginRuledTable{%
   \ncols=0\nrows=0
   \begingroup
    \offinterlineskip
    \def~{\phantom{0}}%
    \def\span{\plainspan\omit\relax\colcount\plainspan}%
    \let\cr=\crrule
    \let\CR=\crthick
    \let\nr=\crnorule
    \let\|=\Vb
    \def\hfill{\hskip0pt plus1fill\hbox{}}%
    \ifx\tablestrut\undefined\relax
    \else\let\tstrut=\tablestrut\fi
    \catcode`\|=13 \catcode`\&=13\relax
    \TableActive
    \curcol=1
    \ifdim\tablewidth>-\maxdimen\relax
      \edef\@Halign{\NX\halign to \NX\tablewidth\NX\bgroup\TablePreamble}%
      \tabskip=0pt plus 1fil
    \else
      \edef\@Halign{\NX\halign\NX\bgroup\TablePreamble}%
      \tabskip=0pt
    \fi
    \ifcentertables
       \ifhmode\vskip 0pt\fi
       \line\bgroup\hss
    \else\hbox\bgroup
    \fi}%
\long\def\@RuledTable#1\endruledtable{%
   \vrule width\thicksize
     \vbox{\@Halign
       \thickrule
       #1\killspace
       \tstrut
       \linecount
       \plaincr\thickrule
     \egroup}%
   \vrule width\thicksize
   \ifcentertables\hss\fi\egroup
  \endgroup
  \global\tablewidth=-\maxdimen
  \iftableinfo
      \immediate\write16{[Nrows=\the\nrows, Ncols=\the\ncols]}%
   \fi}%
\def\TablePreamble{%
   \TableItem{####}%
   \plaintab\plaintab
   \TableItem{####}%
   \plaincr}%
\def\@TableItem#1{%
   \hfil\tablespace
   #1\killspace
   \tablespace\hfil
    }%
\def\@tableright#1{%
   \hfil\tablespace\relax
   #1\killspace
   \tablespace\relax}%
\def\@tableleft#1{%
   \tablespace\relax
   #1\killspace
   \tablespace\hfil}%
\let\TableItem=\@TableItem
\def\RightJustifyTables{\let\TableItem=\@tableright}%
\def\LeftJustifyTables{\let\TableItem=\@tableleft}%
\def\NoJustifyTables{\let\TableItem=\@TableItem}%
\def\LooseTables{\let\tablespace=\quad}%
\def\TightTables{\let\tablespace=\space}%
\LooseTables
\def\TrailingSpaces{\let\killspace=\relax}%
\def\NoTrailingSpaces{\let\killspace=\unskip}%
\TrailingSpaces
\def\setRuledStrut{%
   \dimen@=\baselineskip
   \advance\dimen@ by-\normalbaselineskip
   \ifdim\dimen@<.5ex \dimen@=.5ex\fi
   \setbox0=\hbox{\lparen}%
   \dimen1=\dimen@ \advance\dimen1 by \ht0% space above line
   \dimen2=\dimen@ \advance\dimen2 by \dp0% space below line
   \def\tstrut{\vrule height\dimen1 depth\dimen2 width\z@}%
   }%
\def\tstrut{\vrule height 3.1ex depth 1.2ex width 0pt}%
\def\bigitem#1{%
   \dimen@=\baselineskip
   \advance\dimen@ by-\normalbaselineskip
   \ifdim\dimen@<.5ex \dimen@=.5ex\fi
   \setbox0=\hbox{#1}%
   \dimen1=\dimen@ \advance\dimen1 by \ht0
   \dimen2=\dimen@ \advance\dimen2 by \dp0
   \vrule height\dimen1 depth\dimen2 width\z@
   \copy0}%
\def\vctr#1{\hfil\vbox to 0pt{\vss\hbox{#1}\vss}\hfil}%
\def\nextcolumn#1{%
   \plaintab\omit#1\relax\colcount
   \plaintab}%
\def\tab{%
   \nextcolumn{\relax}}%
\let\novb=\tab
\def\vb{%
   \nextcolumn{\vrule width\thinsize}}%
\def\Vb{%
   \nextcolumn{\vrule width\thicksize}}%
\def\dbl{%
   \nextcolumn{\vrule width\thinsize
   \hskip\thinsize\vrule width\thinsize}}%
{\catcode`\|=13 \let|0
 \catcode`\&=13 \let&0
 \gdef\TableActive{\let|=\vb \let&=\tab}%
}%
\def\crrule{\killspace
   \tstrut
   \linecount
   \plaincr\tablerule
  }%
\def\crthick{\killspace
   \tstrut
   \linecount
   \plaincr\thickrule
  }%
\def\crnorule{\killspace
   \tstrut
   \linecount
   \plaincr
   }%
\def\crpart{\killspace
   \linecount
   \plaincr}%
\def\tablerule{\noalign{\hrule height\thinsize depth 0pt}}%
\def\thickrule{\noalign{\hrule height\thicksize depth 0pt}}%
\def\cskip{\omit\relax}%
\def\crule{\omit\leaders\hrule height\thinsize depth0pt\hfill}%
\def\Crule{\omit\leaders\hrule height\thicksize depth0pt\hfill}%
\def\linecount{%
   \global\advance\nrows by1
   \ifnum\ncols>0
      \ifnum\curcol=\ncols\relax\else
      \immediate\write16
      {\NX\ruledtable warning: Ncols=\the\curcol\space for Nrow=\the\nrows}%
      \fi\fi
   \global\ncols=\curcol
   \global\curcol=1}%
\def\colcount{\relax
   \global\advance\curcol by 1\relax}%
\long\def\para#1{%
   \vtop{\hsize=\parasize
   \normalbaselines
   \noindent #1\relax
   \vrule width 0pt depth 1.1ex}%
}%
\def\begintable{\relax
    \@BeginRuledTable
    \@begintable}%
\long\def\@begintable#1\endtable{%
   \@RuledTable#1\endruledtable}%
% file: TXSsymb.tex (TeXsis version 2.16)
%%> {Extended math/physics symbols,}
\def\E#1{\hbox{$\times 10^{#1}$}}
\def\square{\hbox{{$\sqcup$}\llap{$\sqcap$}}}%
\def\grad{\nabla}%
\def\del{\partial}%
\def\frac#1#2{{#1\over#2}}
\def\smallfrac#1#2{{\scriptstyle {#1 \over #2}}}
\def\half{\ifinner {\scriptstyle {1 \over 2}}%
          \else {\textstyle {1 \over 2}}\fi}
\def\bra#1{\langle#1\vert}%
\def\ket#1{\vert#1\/\rangle}%
\def\vev#1{\langle{#1}\rangle}%
\def\simge{%
    \mathrel{\rlap{\raise 0.511ex 
        \hbox{$>$}}{\lower 0.511ex \hbox{$\sim$}}}}
\def\simle{%
    \mathrel{\rlap{\raise 0.511ex 
        \hbox{$<$}}{\lower 0.511ex \hbox{$\sim$}}}}
\def\gtsim{\simge}%
\def\ltsim{\simle}%
\def\therefore{%
   \setbox0=\hbox{$.\kern.2em.$}\dimen0=\wd0
   \mathrel{\rlap{\raise.25ex\hbox to\dimen0{\hfil$\cdotp$\hfil}}%
   \copy0}}
\def\|{\ifmmode\Vert\else \char`\|\fi}          
\def\sterling{{\hbox{\it\char'44}}}     
\def\degrees{\hbox{$^\circ$}}%
\def\degree{\degrees}%
\def\real{\mathop{\rm Re}\nolimits}%
\def\imag{\mathop{\rm Im}\nolimits}%
\def\tr{\mathop{\rm tr}\nolimits}%
\def\Tr{\mathop{\rm Tr}\nolimits}%
\def\Det{\mathop{\rm Det}\nolimits}%
\def\mod{\mathop{\rm mod}\nolimits}%
\def\wrt{\mathop{\rm wrt}\nolimits}%
\def\TeV{{\rm TeV}}%
\def\GeV{{\rm GeV}}%
\def\MeV{{\rm MeV}}%
\def\KeV{{\rm KeV}}%
\def\eV{{\rm eV}}%
\def\Ry{{\rm Ry}}%
\def\mb{{\rm mb}}%
\def\mub{\hbox{\rm $\mu$b}}%
\def\nb{{\rm nb}}%
\def\pb{{\rm pb}}%
\def\cmsec{{\rm cm^{-2}s^{-1}}}%
\def\units#1{\hbox{\rm #1}} 
\let\unit=\units
\def\dimensions#1#2{\hbox{$[\hbox{\rm #1}]^{#2}$}}
\def\parenbar#1{{\null\!
   \mathop{\smash#1}\limits
   ^{\hbox{\fiverm(--)}}%
   \!\null}}%
\def\nunubar{\parenbar{\nu}}
\def\ppbar{\parenbar{p}}
\def\buildchar#1#2#3{{\null\!
   \mathop{\vphantom{#1}\smash#1}\limits
   ^{#2}_{#3}%
   \!\null}}%
\def\overcirc#1{\buildchar{#1}{\circ}{}}
\def\slashchar#1{\setbox0=\hbox{$#1$}%
   \dimen0=\wd0
   \setbox1=\hbox{/} \dimen1=\wd1
   \ifdim\dimen0>\dimen1
      \rlap{\hbox to \dimen0{\hfil/\hfil}}%
      #1
   \else
      \rlap{\hbox to \dimen1{\hfil$#1$\hfil}}%
      /
   \fi}%
\def\subrightarrow#1{%
  \setbox0=\hbox{%
    $\displaystyle\mathop{}%
    \limits_{#1}$}%
  \dimen0=\wd0
  \advance \dimen0 by .5em
  \mathrel{%
    \mathop{\hbox to \dimen0{\rightarrowfill}}%
       \limits_{#1}}}%
\newdimen\vbigd@men
\def\vbigl{\mathopen\vbig}
\def\vbigm{\mathrel\vbig}
\def\vbigr{\mathclose\vbig}
\def\vbig#1#2{{\vbigd@men=#2\divide\vbigd@men by 2
   \hbox{$\left#1\vbox to \vbigd@men{}\right.\n@space$}}}
\def\Leftcases#1{\smash{\vbigl\{{#1}}}
\def\Rightcases#1{\smash{\vbigr\}{#1}}}
%% $Revision: 16.2 $  :  $Date: 1994/11/12 22:32:02 $  :  $Author: myers $
%======================================================================*
% First, null definitions for ``missing'' things from other parts of TeXsis
%
% Make double column stuff do nothing nicely:
%
\def\doublecolumns{\relax}
\def\enddoublecolumns{\relax}
\def\leftcolrule{\relax}
\def\rightcolrule{\relax}
\def\longequation{\relax}
\def\endlongequation{\relax}
\def\newcolumn{\relax}
\def\widetopinsert{\topinsert}
\def\widepageinsert{\pageinsert}
\def\forceleft{\relax}\def\forceright{\relax}   
\def\SetDoubleColumns#1{\relax}

% ...and other stuff that's not in mTeXsis:

\def\addTOC#1#2#3{\relax}\def\Contents{\relax}  % disable table of contents
\newif\ifContents                               % disable table of contents
\def\ContentsSwitchtrue{\Contentstrue}\def\ContentsSwitchfalse{\Contentsfalse}

\def\obsolete#1#2{\let#1=#2\relax #2}		% \obsolete is silent here

\let\Input=\input                               % TeXsis synonym
\newdimen\colwidth      \colwidth=\hsize        % default column width
\def\ORGANIZATION{}%                            % default is empty

% You shouldn't use style files that might not be around when using
% mTeXsis.  But we will try to be friendly about it...

\newhelp\@utohelp{%
loadstyle: The definition of the macro named above is actually contained^^J%
in a style file, and so it cannot be used with mTeXsis.  If you really^^J%
need to load the definition from that file, you should do so explicitly^^J%
at the begining of your manuscript file, with %
'\string\input\space filename.txs',^^J%
and that still may not give you what you want.}

\def\loadstyle#1#2{% disable \loadstyle in mTeXsis
   \newlinechar=10                              % ^^J is line break
   \errhelp=\@utohelp                           % longer help message
   \emsg{> Whoops! Trying to load \string#1\space from style file #2.}%
   \errmessage{You cannot use macro definitions from style files in mTeXsis}}

 
% -- These make mtexsis more friendly than TeX or TeXsis when printing
%    e-prints, but you won't find your errors as easily.

\hbadness=10000         % don't complain about overfull hboxes!
\overfullrule=0pt       % and don't mark them in the output!
\vbadness=10000         % and don't complain about vboxes either.

% -- Run time setup:
\ATunlock
\SetDate				% sets \adate and \edate
\ReadAUX				% reads .aux file, if it exists
\def\fmtname{TeXsis}\def\fmtversion{2.16}%
\def\revdate{12 November 1994}%
\def\imsg#1{\emsg{\@comment #1}}%
\imsg{================================================== \@comment}
\imsg{This is m\TeXsis, containing the core macros from \TeXsis, the \TeX}
\imsg{macro package for physicists.  Complete source code and documentation}
\imsg{for \TeXsis\space can be obtained via anonymous ftp from %
lifshitz.ph.utexas.edu.}
\imsg{See the README file there for more information.}
\imsg{================================================== \@comment}
\emsg{m\fmtname\space version \fmtversion\space (\revdate)  loaded.}%
\ATlock					% lock internal @ macros
\texsis					% assumed, since they asked for it
%%>>> EOF mtexsis.tex <<<
