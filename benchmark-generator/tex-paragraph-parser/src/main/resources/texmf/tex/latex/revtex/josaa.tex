%%% ======================================================================
%%%  @LaTeX-file{
%%%     filename        = "josaa.tex",
%%%     version         = "3.0",
%%%     date            = "October 20, 1992",
%%%     ISO-date        = "1992.10.20",
%%%     time            = "15:41:54.18 EST",
%%%     author          = "Optical Society of America",
%%%     contact         = "Frank E. Harris",
%%%     address         = "Optical Society of America
%%%                        2010 Massachusetts Ave., N.W.
%%%                        Washington, D.C.  20036-1023",
%%%     email           = "fharris@pinet.aip.org (Internet)",
%%%     telephone       = "(202) 416-1903",
%%%     FAX             = "(202) 416-6120",
%%%     supported       = "yes",
%%%     archived        = "pinet.aip.org/pub/revtex,
%%%                        Niord.SHSU.edu:[FILESERV.REVTEX]",
%%%     keywords        = "REVTeX, version 3.0, sample, Optical
%%%                        Society of America",
%%%     codetable       = "ISO/ASCII",
%%%     checksum        = "13245 559 3639 28302",
%%%     docstring       = "This is a sample JOSA A paper under REVTeX
%%%                        3.0 (release of November 10, 1992).
%%%
%%%                        The checksum field above contains a CRC-16
%%%                        checksum as the first value, followed by the
%%%                        equivalent of the standard UNIX wc (word
%%%                        count) utility output of lines, words, and
%%%                        characters.  This is produced by Robert
%%%                        Solovay's checksum utility."
%%% }
%%% ======================================================================
%%%%%%%%%%%%%%%%%%% file josaa.tex %%%%%%%%%%%%%%%%%%%%
%                                                     %
%   Copyright (c) Optical Society of America, 1992.   %
%                                                     %
%%%%%%%%%%%%%%%%%% October 20, 1992 %%%%%%%%%%%%%%%%%%%
%
\documentstyle[osa,manuscript]{revtex}  % DON'T CHANGE %
\newcommand{\MF}{{\large{\manual META}\-{\manual FONT}}}
\newcommand{\manual}{rm}        % Substitute rm (Roman) font.
\newcommand\bs{\char '134 }     % add backslash char to \tt font %
%
\begin{document}                % INITIALIZE - DONT CHANGE % %  %

\title{Strehl ratios with various types of anisoplanatism}

\author{Richard J. Sasiela}

\address{Lincoln Laboratory, Massachusetts Institute of Technology,
Lexington,  Massachusetts 02173-9108} %

\maketitle
\begin{abstract}
There are many ways in which the paths of two waves through
turbulence  can become separated, thereby leading to anisoplanatic
effects.  Among  these are a parallel path separation, an angular
separation, one caused  by a time delay, and one that is due to
differential refraction at two  wavelengths.  All these effects can
be treated in the same manner.   Gegenbauer polynomials are used to
obtain an approximation for the  Strehl ratio for these
anisoplanatic effects, yielding a greater range  of applicability
than the Mar\'{e}chal approximation.
\end{abstract}

\section{ INTRODUCTION}
Adaptive-optics systems are  used to correct images of objects.
These systems work by measuring the  phase distortion on a
downpropagating wave called a beacon and applying  the negative of that
phase to a deformable mirror.  If this is done  well, then the
image of the beacon is close to diffraction limited; and  if a
laser beam is projected along the corrected path, it will have
propagation characteristics approaching those of a wave propagating
in  vacuum.  It is not possible to make a perfect correction; one
of the  major error sources is due to the fact  that the rays of
the object to  be imaged or the laser beam to be propagated are
along a path displaced  from that of the beacon.  A measurement of
this degradation is the  Strehl ratio, which is the ratio of the
intensity of the actual beam on  axis to that of a
diffraction-limited beam.

\begin{center}
{\small  \copyright\ Optical Society of America, 1992.}
\end{center}

This displacement can  have several causes.  The receiving and the
transmitting apertures may  be displaced from each other owing to
misalignment or vignetting of the  beams.  The paths can be
separated in angle, for instance, when the  object to be imaged is
different from the beacon.  The correction is  applied with a time
delay after the measurements.  In this time the  turbulence is
displaced by winds and slewing of the telescope.  The  paths may be
separated because the beacon and the imaging wavelengths  differ,
in which case refraction operates differently on the two waves.
All the  effects are typically present simultaneously.

These  anisoplanatisms have been treated separately in the
past\cite{1,2,3,4,5,6,7}; however,  they are all manifestations of
the same effect. \ldots   A better analytic approximation that
applies in the  range of operation of a typical adaptive-optics
system is developed  here.  This is applied to obtain expressions
for the various types of  anisoplanatism discussed above.


In Section \ref{SR} the general formula  for the Strehl ratio with
any type of anisoplanatism is derived.   Gegenbauer polynomials
provide a convenient way to keep track of the  series terms and to
cancel terms that lead to numerical difficulties if  the integral
is evaluated numerically.  In Sections \ref{da}--\ref{ca}  the
general  formula is applied to obtain the Strehl ratio for various
types of  anisoplanatism.  The cases considered are parallel path
displacements,  angular offsets, time-delay-induced offsets, and
offsets that are due  to refractive effects that vary with
wavelength.  The Strehl ratio in  the presence of several effects
is examined in Section \ref{cd}.  It is shown  that, depending on
the direction of the relative displacements, one can  get a
cancellation of the displacements so that the Strehl ratio is high
or  an enhancement so that the Strehl ratio is less than the
product of the  Strehl ratios of the individual terms.

\section{ STREHL RATIO WITH ANISOPLANATISM}
\label{SR}
For a perfect correction the  paths of the beacon signal and the
imaging or projected laser should be  the same.  In general, this
is not possible to achieve, and there is a  degradation in
performance caused by time delays, displacement of the  two paths
by translation and angle, and differences in wavelength of the
beacon and the measurement or projecting systems.  The effects of
displacement, angular mispointing, time delay, and atmospheric
dispersion can each be treated as an anisoplanatic effect.  In
fact, if  all the effects are present simultaneously, they can be
combined to get  a total offset of the measurement from the imaging
paths.  In this  section the effect of a general displacement on
the Strehl ratio is  determined.

The Strehl ratio (SR) for a circular aperture \cite{7} from  the
Huygens--Fresnel approximation is  \begin{eqnarray}{\rm  SR}
={1 \over {2\pi }}\int {{\rm d}\bbox  \alpha }\,K(\alpha )\,\exp
\,\left[ {-{{{\cal D}\left( {\bbox \alpha } \right)}  \over 2}}
\right].\end{eqnarray}  The integral is over a circular aperture of
unit radius,  ${\cal D}( {\bbox \alpha } )$  is the structure
function, and  $K(\alpha )$  is a  factor times the optical
transfer function given by  \begin{eqnarray}K(\alpha )={{16} \over
\pi }\left[ {\cos ^{- 1}(\alpha )-\alpha \left( {1-\alpha ^2}
\right)^{1/ 2}} \right]\,U(1- \alpha ),\end{eqnarray}  where
$U\left(  x \right)$  is the unit step function defined as
\begin{eqnarray} U( x )&=&1\,\,\,\,\,\,\,\,{\rm for}\,\,\,\,x\ge
0\,,  \nonumber \\   U( x )&=&0\,\,\,\,\,\,\,\,{\rm
for}\,\,\,\,x<0\,\,.    \end{eqnarray}

To find  the Strehl ratio, one must first determine the structure
function.  It  was found by Fried\cite{4}  for angular
anisoplanatism.  If the source  is collimated and a general
displacement is introduced, his expression  for a wave propagating
from ground to space becomes
\begin{eqnarray}
{\cal D}({\alpha
\kern 1ptD} )&=& 2(2.91)\,{k_0}^2\int\limits_{\,\,\; 0}^{\,\,\,\,\,\;
\infty}   {\rm d}z\,{C_n}^2(z)\left[ {( {\alpha \kern 1ptD}  )^{5/
3}+d^{5/ 3}(z)}\right.  \nonumber\\
&&\left.
{-{\slantfrac{1}{2}}\,\left| {{\bbox \alpha} \kern 1ptD+{\bbox
d}(z)\,} \right|^{5/ 3} -{\textstyle \slantfrac{1}{2}}\left|
{\,{\bbox \alpha}  \kern 1ptD-{\bbox d}(z)\,} \right|^{5 / 3}}
\right],
\end{eqnarray}
where  ${C_n}^2(z)$  is the turbulence
strength as a function of altitude;  $k_0=2\kern 1pt\pi / \lambda
,$  where $\lambda $ is the wavelength  of operation; $D$ is the
aperture diameter; and  ${\bbox d}(z)$   is the vector displacement
of the two paths.

The sums of the terms in  brackets almost cancel, thus causing
difficulties if one tries to  evaluate this integral numerically.
The terms in the absolute-value  sign are equal to
\begin{eqnarray}\left| {\,{\bbox \alpha}  \kern 1ptD\pm {\bbox
d}(z)\,} \right|^{5/ 3}=\left[ {\left( {\alpha \kern  1ptD}
\right)^2\pm 2\alpha \kern 1ptD\,d(z)\cos \left( \varphi
\right)+d^2(z)} \right]^{5/ 6},\end{eqnarray}  where  is the angle
between  ${\bbox \alpha} $  and  ${\bbox d}( z )$ .    This
expression can be simplified and  the numerical difficulties can be
eliminated by using Gegenbauer  polynomials.\cite{8}  Their
generating function is  \begin{eqnarray}\left( {1-2ax+a^2}
\right)^{-\lambda }=\sum\limits_{p=0}^\infty  {{C_p}^\lambda
(x)\,a^p}. \end{eqnarray}   These functions are sometimes referred
to as ultraspherical functions because they are a generalization of
the Legendre polynomials  $P_n(t)$ , whose generating function is
\begin{eqnarray}\left( {1- 2ax+a^2} \right)^{-1/
2}=\sum\limits_{p=0}^\infty  {P_p(x)\,a^p}.\end{eqnarray}      The
Gegenbauer polynomials with the cosine of a variable as the
argument are given in Eq. (8.934  \#2) of Ref. \onlinecite{8}  and
can be rewritten as  \begin{eqnarray}{C_p}^\lambda \left[ {\cos
\left( \varphi   \right)} \right]=\sum\limits_{m=0}^p
{}{{\Gamma\,\left[ {\lambda +m}  \right]\,\Gamma\,\left[ {\lambda
+p-m} \right]\cos \left[ {(p-2m)\varphi }  \right]} \over
{m!\,(p-m)!\,\left( {\Gamma\,\left[ \lambda  \right]}
\right)^2}},\end{eqnarray}     where - $\Gamma\left[ x \right]$  is
the gamma function.  A particular Gegenbauer  polynomial that is
required is  \begin{eqnarray}{C_2}^{-5/ 6}\left[ {\cos (\varphi )}
\right]={\textstyle{\slantfrac{5}{6}}}\left[ {1- {\textstyle{
\slantfrac{1}{3}}}\cos ^2\left( \varphi  \right)} \right].
\end{eqnarray}   For  $\alpha \kern 1ptD>d(z)$ , the terms in the
structure function can  be expanded in Gegenbauer polynomials.  The
zeroth- and all odd-order  terms cancel.  When the summation index
is changed by the substitution  $p\to 2\kern 1ptp$  the result is
\begin{eqnarray} {\cal D}(\alpha \kern
1ptD)=2(2.91)\,{k_0}^2\int\limits_{\,\,\, 0}^{\,\,\,\,\,\,\infty}{\rm
d}z\,{C_n}^2(z) \left\{ {d^{5/  3}(z)- (\alpha \kern 1ptD)^{5/
3}\sum\limits_{p=1}^\infty  {{C_{2p}}^{- 5/ 6}\,\left[ {\cos \left(
\varphi  \right)} \right]}\,\left[ {{{d(z)}  \over {\alpha \kern
1ptD}}} \right]^{2p}} \right\}.\end{eqnarray} It is this  canceling
of the first two terms of the power series that would cause
numerical difficulties.  Define a distance moment as
\begin{eqnarray}d_m\equiv  2.91\,{k_0}^2\int\limits_{\,\,\,
0}^{\,\,\,\,\,\,\infty}{\rm d}z\,{C_n}^2(z)\,d^m(z) \end{eqnarray}
and a phase variance as  \begin{eqnarray}{\sigma _\varphi}^2=d_{5/
3}.\end{eqnarray}    Unlike the calculation for Strehl ratio for
uncorrected  turbulence and for corrected turbulence with tilt
jitter, an exact  analytical solution cannot be found for
anisoplanatism.  Fortunately,  for adaptive-optics systems, the
Strehl ratio should be fairly high by  design, which requires the
structure function to be small.  This  assumption allows one to
retain only the first term of the Gegenbauer  expansion to give
\begin{eqnarray}{\cal D}(\alpha \kern  1ptD)=2{\sigma
_\varphi}^2-2x,\end{eqnarray} where
\begin{eqnarray}x=d_{2}\left[ {1-
{\textstyle{\slantfrac{1}{3}}}\cos ^2\left( \varphi  \right)}
\right]{\slantfrac{5}{6}}(\alpha \kern 1ptD)^{-1/ 3}.\end{eqnarray}
 We justify this single-term approximation below by showing that it
produces a result close to the exact result. \\      \ldots \\ The
Strehl ratio with the six term approximation is
\begin{eqnarray}{\rm   SR} \approx  {{\exp \left( {-\sigma
_\varphi} ^2 \right)} \over {2\pi }}\int {\rm d{\bbox  \alpha}
\,K(\alpha )\,}\kern-.5em\left( {1+x+{{x^2} \over 2}+{{x^3} \over 6}+{{x^4}
\over {24}}+{{x^5} \over {120}}} \right).\end{eqnarray}  If just
the  first term in the last parenthetical expression  is retained,
the result is equivalent to  the extended Mar\'{e}chal
approximation.  It is shown below that the six-term  approximation
is best for aperture sizes normally encountered.   The angle
integral for the $n$th term, after use of the binomial theorem,  is
proportional to  \begin{eqnarray}\Phi (n)={1 \over {2\pi
}}\int\limits_{\,\,\, 0}^{\,\,\,\,\,\,\,\, 2\pi } {\rm d}\varphi \,\left[
{1-\slantfrac{1}{3}} \cos ^2\left( \varphi  \right) \right]^n={1
\over {2\pi  }}\sum\limits_{m=0}^n {\left( \begin{array}{c} n \\
n-m\end{array}  \right)}\,3^{-m}\int\limits_{\,\,\, 0}^{\,\,\,\,\,\, 2\pi
} {\rm d\varphi }\, \cos ^{2m}\left( \varphi
\right),\end{eqnarray}  where       \begin{eqnarray}\left(
\begin{array}{c} n \\ n-m \end{array} \right)={{n!} \over {\left(
{n-m} \right)!\,\,m!}}.\end{eqnarray}  Equation (4.641 \# 4) in
Gradshteyn  and Ryzhik\cite{8} is
\begin{eqnarray}\int\limits_{\,\,\, 0}^{\,\,\,\,\,\, \pi /  2}{\rm
d\varphi \,}\cos ^{2m}\left( \varphi  \right)={{\pi (2m-1)!!} \over
{2(2m)!!}},\end{eqnarray}   where
\begin{eqnarray}(2m-1)!!&=&(2m-1)(2m-3)\ldots (3)(1), \\
(2m)!!&=&(2m)(2m-2)\ldots (4)(2).\end{eqnarray}  With these
relations, the angle integral is equal to
\begin{eqnarray}\Phi (n)=1-\sum\limits_{m=1}^n {\left(
\begin{array}{c}n \\  n-m \end{array} \right)}\,3^{-m}{{(2m-1)!!}
\over {(2m)!!}}.\end{eqnarray}  The values of interest to us are
$\Phi (0) = 1$, $\Phi (1) = 0.8333$, $\Phi (2) = 0.7083$, $ \Phi
(3) = 0.6134$, $\Phi (4) = 0.5404$, and  $\Phi (5) = 0.4836$.   The
aperture integration for the $n$th term is proportional to
\begin{eqnarray}Y(n)=\int\limits_{\,\,\, 0}^{\,\,\,\,\,\, 1} {\rm d\alpha
\,}\alpha ^{1-n/ 3}K(\alpha ).\end{eqnarray}  This is a
generalization of  the integral evaluated by Tatarski in Sec.\ 55,
Eq.\ (22) of Ref.  \onlinecite{9}.  Its value is
\begin{eqnarray}Y\left( n  \right)={8 \over {(2-n/ 3)\,\sqrt \pi
}}\,\Gamma\,\left[ \begin{array}{c}  -n/ 6+{3 \over 2} \\  { -n/
6+3} \end{array} \right]\,\,\quad\,\,\,\,\,\,\,\, {\rm for}\,\,\,
n<6.\end{eqnarray}  The values of interest to us are  $Y\left( 0
\right)=1$,  $Y\left( 1 \right)=1.402$,  $Y\left( 2 \right)=2.087$,
$Y\left( 3 \right)=3.396$,  $Y\left( 4 \right)=6.419$, and $Y\left(
5 \right)=16.94$.   With these values for  the integral, the Strehl
ratio approximation is  \begin{eqnarray}{\rm SR} \approx
({1+0.9736\,E+0.5133\,E^2+0.2009\,E^3+0.0697\,E^4+0.02744\,E^5}
)\exp ({-\sigma _\varphi }^2),\end{eqnarray}   where
\begin{eqnarray}E={{d_{\,2}} \over  {D^{1/ 3}}}.
\end{eqnarray}
\ldots \\
There is an error made in using this approximation for the central
part  of the aperture that increases with each term in the
approximation.  One  has to determine whether this error is less
than or greater than the  increased accuracy achieved over the
remainder of the aperture by using  additional series terms.  To
resolve these uncertainties, I compared the  Strehl ratio, using
various numbers of terms, with exact calculations.

I calculated the Strehl ratio numerically for the case in which
the displacement does not vary with propagation distance.  In
Fig.~\ref{f1}   are plotted the exact Strehl ratio versus
displacement for the  Hufnagel--Valley 21 (HV-21) model of
turbulence\cite{10,11,12} and  the Strehl ratio from relation (24)
for  $D/ r_o  = 1$, with only the unity term in parenthesis
(extended  Marechal approximation) and with different numbers of
terms in the  parenthesis. \\
\ldots         \\

\section{ DISPLACEMENT ANISOPLANATISM}
\label{da}
In the simplest case of displacement  anisoplanatism, which was
treated in Section \ref{SR}, the displacement is  constant along
the propagation direction.  The terms to use to find the  Strehl
ratio are  \begin{eqnarray}  d(z)&=&d  ,  \\
d_{\,2}&=&2.91\,k_0^2\,\mu _0\,d^2  ,     \\ E&=&6.88\,\left( {{d
\over D}} \right)^2\left(  {{D \over {r_o}}} \right)^{5/3}  ,
 \\ \sigma _\varphi ^2&=&2.91\,k_0^2\,\mu _0\,d^{5/3}=6.88\, \left(
{{d \over {r_o}}} \right)^{5/3}  .  \end{eqnarray} The Strehl
ratios are plotted in Figs.~\ref{f5}  and ~\ref{f10}.

\section{ ANGULAR ANISOPLANATISM}
\label{aa}
When the propagation beam is offset by a  constant angle from the
direction along which turbulence is measured,  the effect is called
angular anisoplanatism.\cite{4}  It arises naturally  when one is
tracking a satellite target and directing a laser beam at  it.
Because of the finite speed of light, the laser beam has to lead
the tracking direction, resulting in an angular difference between
the  direction along which the target is tracked and the one along
which the  laser beam is directed.  This error can be eliminated if
the target has  a reflector for the beacon that extends a suitable
distance in the  point-ahead direction.  For the case of an angular
error  \begin{eqnarray} d(z)&=&\theta \,z  ,         \\
d_{\,2}&=&2.91\,k_0^2\,\mu  _2\,\theta ^2  ,               \\
E&=&6.88\,{{\mu _2} \over {\mu _0}}\left(  {{\theta  \over D}}
\right)^2\left( {{D \over {r_o}}} \right)^{5/3}   ,
        \\ \sigma  _\varphi ^2&=&2.91\,k_0^2\,\theta ^{5/
3}\int\limits_{\,\,\, 0}^{\,\,\,\,\,\, L}  {{\rm d}z\,{C_n}^2(z)}\kern
1ptz^{5/ 3}=\left( {\theta / \theta _o}  \right)^{5/ 3}  ,
\end{eqnarray} where the isoplanatic angle is defined by
\begin{eqnarray} \theta _o^{}=\left( {2.91\,k_0^2\,\mu _{5/ 3}}
\right)^{-3/ 5}  .   \end{eqnarray} ...
\section{ TIME DELAY}
\label{td}
If there is a time delay  between when turbulence is measured and
when a correction is applied to the deformable mirror, there is  a
degradation in performance.\cite{7}  This effect is not often
thought of  as an anisoplanatic effect; however, it can be treated
as such.  ...  \begin{eqnarray}   d(z)&=&v(z)\tau   ,  \\
d_2&=&2.91\,k_0^2\int\limits_{\,\,\, 0}^{\,\,\,\,\,\, L} {\rm
d}z\,{C_n}^2(z)\,v^2(z)\,\tau ^2=\left(  {\tau / \tau _2} \right)^2
,   \\    E&=&{{\tau ^2} \over {\tau _2^2D^{1/ 3}}}  ,  \\  \sigma
_\varphi ^2&=&2.91\,k_0^2\int\limits_{\,\,\, 0}^{\,\,\,\,\,\, L}  {\rm
d}z\,{C_n}^2(z)\,v^{5/ 3}(z)\,\tau ^{5/ 3}=\left( {\tau / \tau _{5/
3}} \right)^{5/ 3}  ,   \end{eqnarray} where the temporal moment is
defined as  \begin{eqnarray}  1/ \tau _m^{5/
3}=2.91\,k_0^2\int\limits_{\,\,\, 0}^{\,\,\,\,\,\, L}  {\rm
d}z\,{C_n}^2(z)\,v^m(z)  .   \end{eqnarray}
\ldots

\section{ CHROMATIC ANISOPLANATISM}
\label{ca}
If the beacon beam that senses the  turbulence has a wavelength
different from that of the laser beam that  is sent out, then the
two beams will follow different paths through the  atmosphere
because of the dispersive properties of the atmosphere.  The
analysis given here parallels that given by Belsher and
Fried.\cite{1}

\ldots
The change of refractive index with wavelength has been  given by
Allen\cite{16} as   \begin{eqnarray} \Delta \kern 1ptn_0=\left(
{\lambda _1^2-\lambda  _2^2} \right)\left[ {{{29\,498.1} \over
{\left( {146\lambda _2^2-1}  \right)\left( {146\lambda _1^2-1}
\right)}}+{{255.4} \over {\left(  {41\lambda _2^2-1} \right)\left(
{41\lambda _1^2-1} \right)}}}  \right]10^{-6}  .
\end{eqnarray} The atmospheric density versus altitude is  given by
Cole.\cite{17}  The ratio of the ... .  Thus the beam  displacement
along the path is  \begin{eqnarray} {\rm  \pmb{d}}_c(z)=-{{{
\rm \bbox{\xi}} \,\sin  \left( \xi  \right)\,\Delta \kern 1ptn_0} \over
{\xi \,\cos ^2\left( \xi   \right)}}\,\left[ {\int\limits_{\,\,\,
0}^{\,\,\,\,\,\, z} {\rm d}z' \alpha \left( {z'} \right)-{z  \over
L}\int\limits_{\,\,\, 0}^{\,\,\,\,\,\, L} {\rm d}z'\alpha \left( {z'}
\right)} \right]  .   \end{eqnarray} Define the integral of the air
density as  \begin{eqnarray} I\left( z \right)=\int\limits_{\,\,\,
0}^{\,\,\,\,\,\, z}{\rm d}z'\alpha  \left( {z'} \right)  .
\end{eqnarray}  Evaluating the integral and     \ldots

The moments of this displacement are \begin{eqnarray} d_m=\left[
{{{\sin \left( \xi  \right)\Delta \kern 1ptn_0} \over {\cos
^2\left( \xi  \right)}}} \right]^mT_m  ,        \end{eqnarray}
where
\begin{eqnarray} T_m=2.91\,k_0^2\sec \left( \xi
\right)\int\limits_{\,\,\, 0}^{\,\,\,\,\,\, H}  {\rm
d}h\,{C_n}^2(h)\,\left[ {I(h)-{{h\sec \left( \xi  \right)} \over
L}I(L)}  \right]^m  .
\end{eqnarray} $H$ is the altitude of the
target.  The last term in brackets goes to zero  as the range
becomes infinite.  \ldots For the infinite range, this reduces to
     \begin{eqnarray} T_m=2.91\,k_0^2\sec \left( \xi
\right)\int\limits_{\,\,\, 0}^{\,\,\,\,\,\, H} {\rm
d}h\,{C_n}^2(h)\,I^m(h)  .
\end{eqnarray}
\ldots

\section{ COMBINED DISPLACEMENT}
\label{cd}
If there are several anisoplanatic  effects present, with each not
decreasing the Strehl ratio much, it is a  common practice to
multiply the Strehl ratios for the individual effects  to get a
combined Strehl ratio.  The validity of this assumption is now
examined.  The total displacement that is due to a translation, an
angular offset, a time delay, and a chromatic offset is
\begin{eqnarray}
{\rm \pmb{d}}_t(z)={\rm  \pmb{d}}+{ \rm \bbox{
\theta}} \kern 1ptz+{\rm \pmb{v}}(z)\tau +{\rm  \pmb{d}}_c(z)  ,
\end{eqnarray} where chromatic displacement is given in Eq. (50).
The two terms  necessary for calculating the Strehl ratio are
\begin{eqnarray} E&=&{{d_{\,2}}  \over {D^{1/ 3}}}  ,          \\
\sigma _\varphi ^2&=&d_{\,5/  3}  ,   \end{eqnarray} where
\begin{eqnarray} d_m=2.91\,k_0^2\int\limits_{\,\,\, 0}^{\,\,\,\,\,\,
\infty}   {\rm d}z\,{C_n}^2(z)\,\left| {d_t(z)} \right|^m  .
\end{eqnarray}
\ldots

\ldots
Tyler {\it et al.}\cite{18} took advantage of the vector nature  of
the displacement almost to eliminate the effect of chromatic
anisoplanatism on an adaptive-optics system by choosing an optimal
offset angle of a beacon from the propagation direction.

\section{ SUMMARY}
\label{Su}
An approximate expression for the Strehl ratio that is  easily
evaluated for any turbulence distribution was derived.  It  applies
for various anisoplanatic effects.  This expression was shown to
give much better agreement with the exact answer than the extended
Marechal approximation.  The zenith dependence is included in the
formula.  This approximation was applied to parallel path
displacements,  angular offsets, time-delay induced offsets, and
offsets owing to  refractive effects that vary with wavelength.
Examples for each type of  anisoplanatism at various zenith angles
were evaluated.

The  Strehl ratio in the presence of several effects was examined.
It was  shown that, depending on the direction of the relative
displacements,  one can get a cancellation or an enhancement of the
effect of the  displacements.  Therefore it is possible for there
to be little  reduction in the Strehl ratio if there is little net
path displacement.   If the displacements are in the same
direction, the Strehl ratio is less  than the product of the Strehl
ratios of the individual terms.

\acknowledgments This research was  sponsored by the Strategic
Defense Initiative Organization through the  U.S. Department of the
Air Force.

\begin{references}
\bibitem{1}  J. Belsher and D. Fried, ``Chromatic refraction
induced pseudo  anisoplanatism,'' tOSC Rep. TR-433 (Optical
Sciences Co.,  Placentia, Calif., 1981).
\bibitem{2}  B. L. Ellerbroek and P. H. Roberts,  ``Turbulence
induced angular separation errors; expected values for the  SOR-2
experiment,'' tOSC Rep.  TR-613 (Optical Sciences Co.,  Placentia,
Calif., 1984).
\bibitem{3}  D. L. Fried, ``Differential angle of  arrival: theory,
evaluation, and measurement feasibility,'' Radio  Sci. {\bf 10,}
71-76 (1975).
\bibitem{4}  D. Fried, ``Anisoplanatism in adaptive  optics,''
\josa {\bf 72,} 52-61 (1982).  \bibitem{5}  D. Korff, G. Druden,
and  R. P. Leavitt, ``Isoplanicity: the translation invariance of
the  atmospheric Green's function,'' \josa {\bf 65,} 1321-1330
(1975).
\bibitem{6}   J. H. Shapiro, ``Point-ahead limitation on
reciprocity tracking,''  \josa {\bf 65,} 65-68 (1975).
\bibitem{7}  G. A. Tyler, ``Turbulence-induced  adaptive-optics
performance degradation: evaluation in the time domain,''  \josaa
{\bf  1,} 251-262 (1984).
\bibitem{8}  I. S. Gradshteyn and I. M.  Ryzhik, {\it Table of
Integrals, Series, and Products}  (Academic, New York, 1980).
\bibitem{9}  V. I. Tatarski, {\it The Effects Of The Turbulent
Atmosphere  On Wave Propagation} (U. S. Department of Commerce,
Washington, D.C., 1971).
\bibitem{10}  R. E.  Hufnagel, {\it Optical Propagation through
Turbulence} (Optical Society of America, Washington, D. C., 1974).
\bibitem{11}   J. L. Bufton, P. O. Minott, M. W. Fitzmaurice, and
P. J. Titterton,  ``Measurements of turbulence profiles in the
troposphere,''  \josa  {\bf 62,}  1068-1070 (1972).
\bibitem{12}  G. C. Valley, ``Isoplanatic degradation of tilt
correction and short-term imaging system,'' \ao {\bf 19,} 574-577
(1980).
\bibitem{13}  M. G. Miller and P. L. Zieske, ``Turbulence
environmental  characterization,'' RADC-TR-79-131 (Rome Air
Development Center,  Griffiss Air Force Base, N.Y., 1979).
\bibitem{14}  D. P. Greenwood, ``Bandwidth specifications for
adaptive optics  systems,'' \josa {\bf 67,} 390-393 (1977).
\bibitem{15}  D. L. Fried, ``Time-delay-induced mean-square error
in adaptive optics,'' \josaa  {\bf 7,} 1224-1225 (1990).
\bibitem{16}  C. W. Allen, {\it Astrophysical Quantities}
(Athlone, London, 1963).
\bibitem{17}  A. E. Cole, A. Court, and A. J. Kantor,  {\it
Handbook  of Geophysics and Space Environments,} S.\ L.\ Valley,
ed.  (McGraw-Hill, New York, 1965).
\bibitem{18}  G. Tyler, J.  Belsher and D. Fried, ``Amelioration of
chromatic refraction induced  pseudoanisoplanatism," tOSC Rep.
TR-465 (Optical Sciences  Co., Placentia, Calif., 1982).
\end{references}

\begin{figure}
\caption{ Comparison of the Mar\'{e}chal and the two- to six-term
approximations  with the exact value of the Strell ratio, for an
anisoplanatic displacement, for $D/r_0$  equal to 1.}\label{f1}
\end{figure}

\begin{figure}
\caption{ Comparison of the Mar\'{e}chal and the two- to six-term
approximations  with the exact value of the Strell ratio, for an
anisoplanatic displacement, for $D/r_0$  equal to 5. } \label{f5}
\end{figure}
\begin{figure}
\caption{ Comparison of the Mar\'{e}chal and the two- to six-term
approximations  with the exact value of the Strell ratio, for an
anisoplanatic displacement, for $D/r_0$  equal to 10. } \label{f10}
\end{figure}
\begin{figure}
\caption{Strehl ratio for angular anisoplanatic error at zenith,
for  various turbulence models, versus separation angle for a 0.6-m
system.   Upper-altitude turbulence has a strong effect on the
Strehl ratio.}
\label{faaz}
\end{figure}
\begin{figure}
\caption{ Strehl ratio for angular anisoplanatism at $30^{\circ}$
for a 0.6-m system.}
\label{faa30}
\end{figure}
\begin{figure}
\caption{ Strehl ratio versus time delay at zenith for a 0.6-m
system.}
\label{ftdz}
\end{figure}
\begin{figure}
\caption{ Strehl ratio versus time delay for a 0.6-m system at
$30^{\circ}$ zenith angle.     Strehl ratio  at $30^{\circ}$ for a
0.6-m system. }
\label{ftd30}
\end{figure}
\begin{figure}
\caption{ Difference ($\times 10^6$) in refractive index between
$0.5 \, \mu \rm m$ and other wavelengths.}\label{fri}
\end{figure}

\begin{table}
\caption{Values of $T_2$ and  $T_{5/3}$  to Solve for the Chromatic
Displacement for Various  Turbulence Models for a Wavelength of 0.5
$\mu \rm m$}
\begin{tabular}{lcc}
Model&$T_2$\tablenote{The units of $T_2$ are $m^{1/3}$.}&
$T_{5/3}$\tablenote{$T_{5/3}$  is dimensionless.} \\ \tableline
SLC-Day&$2.71 \, \times \, 10^{-6}$&$2.00 \, \times \, 10^{-7}$\\
HV-21&$6.16 \, \times \, 10^{-6}$&$3.60 \, \times \, 10^{-7}$\\
HV-54&$3.40 \, \times \, 10^{-5}$&$1.87 \, \times \, 10^{-6}$\\
HV-72&$5.95 \, \times \, 10^{-5}$&$3.25 \, \times \, 10^{-6}$\\
\end{tabular}
\end{table}

\end{document}

%%% file josaa.tex %%%
