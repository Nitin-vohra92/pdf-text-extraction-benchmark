% SAMPLE.TEX -- AGUTeX style file tutorial paper.

   % This document is intended to demonstrate how to use commands 
   % described in the aguguide.tex user manual.  Most instructions 
   % and explanations are commented out; thus they will not appear 
   % if authors wish to LaTeX this file and view it as a sample 
   % document.  

   % \documentstyle[12pt,agums]{article}
   % \documentstyle[agupp]{article}
\documentstyle[jgrga]{article}

   % The first item in a LaTeX file must be a \documentstyle command to
   % declare the overall style of the paper.  Three of the \documentstyle 
   % lines that are relevant for the AGUTeX style files are shown; two 
   % are commented out so that the file can be processed.

   % LaTeX's automatic section numbering is turned off by default.
   % If you feel that numbering your sections is imperative you should
   % manually place numbers in the section headers in the following
   % format:  \section{1. Introduction}

% \tighten

   % The \tighten command is used to turn off double spacing in the agums 
   % substyle.  This command should appear only in manuscripts intended for 
   % personal use or for distribution to colleagues.  Manuscripts submitted 
   % for editorial review should not contain this command, or reviewers and 
   % copy editors will have no space to include their comments.

% Preamble Information

\lefthead{KURTZ ET AL.}
\righthead{TUTORIAL FOR AGU JOURNALS}
\received{September 1, 1992}
\revised{February 12, 1993}
\accepted{February 13, 1993}
\journalid{JGRA}{November 1993}
\articleid{1}{13}
\paperid{92JB04601}
\ccc{0148-0227/93/92JB-04601\$05.00}
   % \cpright{PD}{1993}
   % \cpright{Crown}{1993}
   % (Crown copyrights have no "\ccc{}" information.)
\cpright{AGU}{1993}

   % The running head information, slug-line data, and author addresses 
   % are placed in the preamble (before the \begin{document} command).  
   % The style files cause the actual text to be typeset after the 
   % reference section.  If there is no reference section, you must 
   % use the \forcesluginfo command referred to in the aguguide.tex 
   % file.  Editors will supply received, revised, and accepted dates 
   % to authors preparing camera ready copy.  

   % Author address material should be in alphabetical order:

\authoraddr{J. M. Albert and C. D. Biemesderfer, Ferberts Associates, 
    1895 Mt. Lemmon Hwy., Oracle, AZ 85623.}

\authoraddr{R. J. Hall, Space Telescope Science Institute, 3700 San 
    Martin Drive, Baltimore, MD 21218.}

\authoraddr{E. R. King, S. Kurtz, and A. E. Simpson, Astronomy Department, 
    University of California, Berkeley, CA 94720.}

\slugcomment{To appear in the {\it Journal of Geophysical Research}, 1994.}
   % The \slugcomments appear only in manuscript and preprint styles.

   % This ends the "preamble" section.  The body of the paper is started
   % with the \begin{document} command, which is followed by the front 
   % matter (title, author, address data, and abstract).

% Front Matter

\begin{document}

\title{Tutorial for authors preparing camera ready copy for submission 
    to AGU journals}

\author{S. Kurtz, Evan R. King,\altaffilmark{1} and A. E. 
    Simpson\altaffilmark{2}}
\affil{Astronomy Department, University of California, Berkeley}

\author{R. J. Hall}
\affil{Space Telescope Science Institute, Baltimore, Maryland}

\author{C. D. Biemesderfer and J. M. Albert}
\affil{Ferberts Associates, Oracle, Arizona}

\altaffiltext{1}{Also at Physics Department, University of 
    California, Berkeley.}

\altaffiltext{2}{Now at National Center for Atmospheric Research, 
    Boulder, Colorado.} 

   % Notice that some of the authors have alternate affiliations, 
   % which are identified by the \altaffilmark command after their
   % names.  The alternate affiliation information is specified in 
   % \altaffiltext commands.  There is a separate \altaffiltext for
   % each alternate affiliation indicated above.  The actual typeset 
   % information appears in footnotes after the reference section.  

   % Important:  If there are more than three author and affiliation 
   % groupings, all of the affiliate information should be footnoted 
   % using the \altaffilmark{} command as in the following example:
   %
   % \author{S. Kurtz,\altaffilmark{1} Evan R. King,\altaffilmark{1,2} 
   %   A. E. Simpson,\altaffilmark{1,3} R. J. Hall,\altaffilmark{4} 
   %   C. D. Biemesderfer,\altaffilmark{5} Jason M. Albert,\altaffilmark{5} 
   %   G. G. Kaye,\altaffilmark{6} and Antony Ignatov\altaffilmark{7}}
   % 
   % \altaffiltext{1}{Astronomy Department, University of California, 
   %     Berkeley.}
   % \altaffiltext{2}{Also at Physics Department, University of California, 
   %     Berkeley.}
   % \altaffiltext{3}{Now at National Center for Atmospheric Research, 
   %     Boulder, Colorado.} 
   % \altaffiltext{4}{Space Telescope Science Institute, Baltimore, Maryland}
   % \altaffiltext{5}{Ferberts Associates, Oracle, Arizona.}
   % \altaffiltext{6}{Space Telescope Science Institute, Baltimore, Maryland.}
   % \altaffiltext{7}{Department of Astronomy, California Institute of 
   %     Technology, Pasadena.}

\begin{abstract}
This tutorial includes codes and explanations which will not print in 
an ordinary \LaTeX\ document.  In order to use this tutorial you should 
view or print it using an editor or word processing program.  Also 
included are samples of marked-up equations, tables, and figure captions, 
with some hints on how to use them.
\end{abstract}

% Main Body Text 

   % In the first two sections, note the use of the LaTeX \markcite 
   % command to identify citations.

\section{Introduction}

A focal problem today in the dynamics of globular clusters 
is core collapse.  It has been predicted by theory for decades 
[\markcite{{\it King,} 1966}; \markcite{{\it Lynden-Bell and Wood,} 
1968}; \markcite{{\it Spitzer,} 1985}], but observation has 
been less alert to the phenomenon.  For many years the central 
brightness peak in M15 [\markcite{{\it King,} 1975}; \markcite{{\it 
Newell and O'Neil,} 1978}] seemed a unique anomaly.  Then 
\markcite{{\it Auri\'{e}re} [1982]} suggested a central peak 
in NGC 6397, and a limited photographic survey of \markcite{{\it 
Kurtz and King} [1984]} found three more cases (hereinafter 
referred to as paper 1) including NGC 6624, whose sharp center 
had often been noted [e.g., \markcite{{\it Canizares et al.,} 1978]}.

\section{This is a Particularly Long Section Header Which Will 
Demonstrate How Headers Wrap Flush Left}

All our observations were short direct exposures with CCDs.  
At Central Observatory we used a TI 500$\times$500 chip and a 
GEC 575$\times$385 on the 1-m Nickel reflector.  The only 
filter available at Central was red.  At CTIO we used a GEC 
575$\times$385 with $B, V,$ and $R$ filters and an RCA 
512$\times$320 with $U, B, V, R,$ and $I$ filters on the 
1.5-m reflector.  In the CTIO observations we tried to 
concentrate on the shortest practicable wavelengths; but 
faintness, reddening, and poor short-wavelength sensitivity 
often kept us from observing in $U$ or even in $B$.  

The CCD images are unfortunately not always suitable for 
very poor clusters or for clusters with large cores.  Since 
the latter are easily studied by other means, we augmented 
our own CCD profiles by collecting from the literature a 
number of star count profiles [\markcite{{\it King et al.,\,} 
1968}; \markcite{{\it Peterson,} 1976}; \markcite{{\it Harrold 
and Bernstein,} 1984}; \markcite{{\it Ortolani et al.,\,} 1985}],
as well as photoelectric profiles [\markcite{{\it King,} 1966}] 
and electronographic profiles [\markcite{{\it Kromlin et al.,\,} 
1984}].  In a few cases we judged normality by eye estimates on 
one of the Sky Surveys. 

   % In the next section we see the use of the \subsection command 
   % to set off independent subsections.  Note that if subsections 
   % or subsubsections are included, AGU style requires authors to 
   % use at least two per section (this is standard etiquette for 
   % outlines).
   % 
   % We show the use of displayed math environments (as described in
   % the AGU guide), and there are several mathematical typesetting
   % examples.  

\section{This is a Level One Head}

It has been realized that helicity amplitudes provide a convenient 
means for Feynman diagram evaluations.  These amplitude level 
techniques are particularly convenient for calculations involving 
many Feynman diagrams, where the usual trace techniques for the 
amplitude squared become unwieldy.  Our calculations use helicity 
techniques [\markcite{{\it Hagiwara and Zeppenfeld,} 1986}]; we 
briefly summarize below.

\subsection{Formalism}
A three-level amplitude in $ e^+e^-$ collisions can be expressed 
in terms of fermion strings of the form 
\begin{equation}
   \bar v(p_2,\sigma_2)P_{-\tau}\not\!a_1\not\!a_2
   \cdots\not\!a_nu(p_1,\sigma_1)\;,
\end{equation}
where $p$ and $\sigma$ label the initial $e^{\pm}$ four momenta and 
helicities $(\sigma = \pm 1)$, $\not\!a_i=a^\mu_i\gamma_\nu$, and 
$P_\tau=\frac{1}{2}(1+\tau\gamma_5)$ is a chirality projection 
operator $(\tau = \pm1)$.  The $a^\mu_i$ may be formed from particle
four momenta, gauge boson polarization vectors, or fermion strings with 
an uncontracted Lorentz index associated with final state fermions.  

\subsection{This is a Level Two Head}
In the chiral representation the $\gamma$ matrices are expressed in 
terms of $2\times 2$ Pauli matrices $\sigma$ and the unit matrix 1 as
\begin{mathletters}
\begin{eqnarray}
   \gamma^\mu & = & \left(\begin{array}{cc} 0 & \sigma^\mu_+ \\
   \sigma^\mu_- & 0 \end{array} \right), \;\gamma^5= \left(
   \begin{array}{cc}
   -1 & 0\\
   \; \; 0 & 1  
   \end{array} \right), \\ & & \nonumber \\
   \sigma^\mu_{\pm} & = & ({\bf 1} ,\pm \sigma)\;, 
\end{eqnarray}
giving 
\begin{eqnarray}
   \not\!a= \left(\begin{array}{cc}0 & (\not\!a)_+\\(\not\!a)_- 
   & 0\end{array}\right), \;(\not\!a)_\pm=a_\mu\sigma^\mu_\pm\;,
\end{eqnarray}
\end{mathletters}
These amplitude level techniques are particularly convenient for 
calculations involving many Feynman diagrams, where the usual 
trace techniques for the amplitude squared become unwieldy.

   % Note that the "mathletters" command automatically letters 
   % the above equations.

\subsubsection{Level three heads capitalize only the first word 
are indented one em space.}
The spinors are expressed in terms of two-component Weyl spinors as
\begin{equation}
   u=\left(\begin{array}{c}(u)_-\\(u)_+\end{array}\right),\;v={\bf 
   (}(v)^\dagger_+{\bf ,} \; (v)^\dagger_-{\bf )}\;.
\end{equation}
All four cameras had scales of the order of 0.4 arc sec/pixel, and our 
field sizes were around 3 arc min.  The Weyl spinors are given in terms 
of helicity eigen\-states $\chi_\lambda(p)$ with $\lambda=\pm1$ by
\begin{eqnarray}u(p, \lambda)_\pm & = & (E\pm\lambda|{\bf 
   p}|)^{1/2}\chi_\lambda(p)\;, \nonumber \\ & & \\v(p,\lambda)_\pm 
   & = & \pm\lambda(E\mp\lambda|{\bf p}|)^{1/2}\chi_{-\lambda}(p).
\nonumber\end{eqnarray}

\subsubsection{Level three heads are run into the text and end with a period.}
These spinors are expressed in terms of two-component Weyl spinors as
\begin{equation}
   u=\left(\begin{array}{c}(u)_-\\(u)_+\end{array}\right),\;v={\bf 
   (}(v)^\dagger_+{\bf ,} \; (v)^\dagger_-{\bf )}\;.
\end{equation}
The Weyl spinors are given in terms of helicity eigen\-states 
   $\chi_\lambda(p)$ with $\lambda=\pm1$ 
   by\begin{eqnarray}u(p, \lambda)_\pm & = & (E\pm\lambda|{\bf 
   p}|)^{1/2}\chi_\lambda(p)\;, \nonumber \\ & & \\v(p,\lambda)_\pm 
   & = & \pm\lambda(E\mp\lambda|{\bf p}|)^{1/2}\chi_{-\lambda}(p).
\nonumber\end{eqnarray}
The CCD images are unfortunately not always suitable for very poor 
clusters or for clusters with large cores.

   % In the next sections we have a reference to one of the tables 
   % (which occurs later in the document).  There is also some 
   % additional math-related markup.  In the second paragraph, 
   % note the use of in-text math ($stuff$) including a couple of 
   % the miscellaneous symbol commands defined in the AGUTeX style
   % file package.

\section{Floating Material}
Consider a task that computes profile parameters for a modified 
Lorentzian of the following form:
\begin{equation}
   I = \frac{1}{1 + d_{1}^{P (1 + d_{2} )}}, 
\end{equation}
where
\begin{mathletters}
\begin{eqnarray}
   d_{1} = \frac{3}{4} \sqrt{ \left( \begin{array}{c} 
   \frac{x_{1}}{R_{\rm maj}} 
   \end{array} \right) ^{2} + \left( 
   \begin{array}{c} \frac{y_{1}}{R_{\rm min}} 
   \end{array} \right) ^{2} } \\
   d_{2} = \case{3}{4} \sqrt{ \left( 
   \begin{array}{c} \frac{x_{1}}{P R_{\rm maj}}
   \end{array} \right) ^{2} + \left( 
   \begin{array}{c} \case{y_{1}}{P R_{\rm min}} 
   \end{array} \right) ^{2} }
\end{eqnarray}
\end{mathletters}
\begin{mathletters}
\begin{eqnarray}
   x_{1} & = & (x - x_{0}) \cos \Theta + (y - y_{0}) \sin \Theta \\ 
   y_{1} & = & -(x - x_{0}) \sin \Theta + (y - y_{0}) \cos \Theta. 
\end{eqnarray}
\end{mathletters}
In these expressions, $x_{0}$,$y_{0}$ is the star center, and $\Theta$ 
is the angle with the $x$ axis.  Results of this task are shown in 
Table~\ref{tbl-1}.  It is not clear how these sorts of analyses may 
affect determination of $M_{\sun}$ and $M_{\earth}$, but the assumption 
is that the alternate results should be less than 90\deg\ out of phase 
with previous values.  We have no observations of \ion{Ca}{2}.  Roughly 
four fifths of electronically submitted abstracts are error-free.

\appendix
\section{Appendix A: Your Title}

   % If you have several appendix sections, the appendix header should 
   % match the above example.  If there is only one appendix, the lettering 
   % should be left out of the header: \section{Appendix: Your Title}.

   % Important:  The \appendix command causes all ensuing equation and 
   % table headers to use a lettered caption.  This is to distinguish 
   % appendix tables and equations from the tables and equations that 
   % will appear in the main body of the manuscript.  Since AGU submission
   % style asks for all tables to be printed at the end of a manuscript, 
   % manuscripts with both appendix sections and regular tables must use 
   % the \tablenum{} command to number the regular tables.  This will not 
   % pose a problem for manuscripts with either no tables or no appendix 
   % section.  This manuscript includes examples of tables with \tablenum{} 
   % commands.

Consider a task that computes profile parameters for a modified 
Lorentzian of the form
\begin{equation}
   I = \frac{1}{1 + d_{1}^{P (1 + d_{2} )}}, 
\end{equation}
where
\begin{mathletters}
\begin{equation}
   d_{1} = \frac{3}{4} \sqrt{ \left( \begin{array}{c} \frac{x_{1}}{R_{maj}} 
   \end{array} \right) ^{2} + \left( \begin{array}{c} \frac{y_{1}}{R_{min}} 
   \end{array} \right) ^{2} } 
\end{equation}
\begin{equation}
   d_{2} = \case{3}{4} \sqrt{ \left( \begin{array}{c} \frac{x_{1}}{P R_{maj}}
   \end{array} \right) ^{2} + \left( \begin{array}{c} \case{y_{1}}{P R_{min}} 
   \end{array} \right) ^{2} }  
\end{equation}
which leaves us with the conclusion that
\begin{equation}
   x_{1} = (x - x_{0}) \cos \Theta + (y - y_{0}) \sin \Theta 
\end{equation}
\begin{equation}
   y_{1} = -(x - x_{0}) \sin \Theta + (y - y_{0}) \cos \Theta. 
\end{equation}
\end{mathletters}
\section{Appendix B: One Last Equation}
For completeness, here is one last equation.
\setcounter{equation}{0}  \begin{equation}
e = mc^2  
\end{equation}
Notice how the table and equation numbering is reset when a new appendix
section is added.  The appendix table at the end of this manuscript will
also include the letter ``B'' in the table caption.

   % This is the last section of the paper, so we include an 
   % \acknowledgments section.  Acknowledgments should be placed 
   % directly in front of the references section.  Please place a 
   % \newpage command after the acknowledgments section; this will 
   % force the references section to begin on a fresh page.  

\acknowledgments
We are grateful to V. Barger, T. Han, and R. J. N. Phillips for
doing the math in the formalism section.

\newpage

% Reference List

   % In this document we use the \markcite command to call out citations, 
   % so we must enclose the reference list in a "references" environment.  

\begin{references}
\reference
Auri\`ere, M., Title of article, \apj {\it , 109,} 301-305, 1982.

\reference
Canizares, C. R., J. E. Grindlay, W. A. Hiltner, W. Liller, and 
    J. E. McClintock, Title of article, \apj {\it , 224,} 39, 1978.

\reference
Kurtz, S., and E. R. King, Title of article, \apj {\it , 277,} 49, 1984.

\reference
Hagiwara, K., and D. Zeppenfeld, Title of article, \rs {\it , 274,} 1, 1986.

\reference
Harrold, G. E., and W. Bernstein, Title of article, \rg {\it , 49,} 181, 1984.

\reference
King, I. R., Title of article, \apj {\it , 71,} 276, 1966.

\reference
King, I. R., Title of article, in {\it Dynamics of Stellar Systems,} edited 
    by A. Hayli, pp. 99-103, D. Reidel, Norwell, Mass., 1975.

\reference
King, I. R., E. Hedemann, S. M. Hodge, and R. E. White, Title of article, 
    \apj {\it , 73,} 456, 1968.

\reference
Kromlin, M. R., L. W. Mueller, and H. J. Waterman, Title of article, \jatp 
    {\it , 96,} 198, 1984.

\reference
Lynden-Bell, D., and R. Wood, Title of article, {\it Mon. Not. R. Astron. 
    Soc., 138,} 495, 1968.

\reference
Newell, E. B., and E. J. O'Neil, Title of article, {\it Astrophys. J.,} Supp., 
    {\it 37,} 27, 1978.

\reference
Ortolani, S., L. Rosino, and A. Sandage, Title of article, \prc {\it , 90,} 
    473, 1985.

\reference
Peterson, C. J. Title of article, \jgr {\it , 81,} 617, 1976.

\reference
Spitzer, L. J. Goodman, and P. Hut, Title of article, in {\it Dynamics of 
    Star Clusters,} pp. 109-125, D. Reidel, Norwell, Mass., 1985.
\end{references}

   % An alternate way of handling the references would be to use the 
   % "\cite" and "\bibitem" commands to call out citations and use LaTeX's
   % "thebibliography" environment for the reference list.  Please note 
   % that the \begin{thebibliography}{} command is followed by a null 
   % argument (see example below).  If you forget this, mayhem ensues, 
   % and LaTeX will say "Perhaps a missing item?" when you run it.
   %
   % Each reference has a \bibitem command to define the citation format
   % and the symbolic tag, as well as a \reference command which sets up
   % formatting parameters for the reference list itself.  References in 
   % the text would use markup such as [\cite{can78}; \cite{kur84}; 
   % \cite{hag86}] or [\cite{aur82}].  A drawback to using this system 
   % is that the references will be typeset exactly as they appear in 
   % the \bibitem command.  Thus the \cite-\bibitem system would be unable 
   % to produce format such as {\it Kurtz and King} [1984].  If your 
   % manuscript ever uses this format you should probably stick with 
   % the \markcite commands.
   %
   % The following example is commented out so the file will LaTeX properly.
   %
   % \begin{thebibliography}{}
   % \bibitem[{\it Auri\`ere,} 1982]{aur82} \reference Auri\`ere, M., 
   %     Title of article, \apj {\it , 109,} 301-305, 1982.
   % 
   % \bibitem[{\it Canizares et al.,} 1978]{can78} \reference Canizares, 
   %     C. R., J. E. Grindlay, W. A. Hiltner, W. Liller, and J. E. 
   %     McClintock, Title of article, \apj {\it , 224,} 39, 1978.
   % 
   % \bibitem[ {\it Kurtz and King,} 1984]{kur84} \reference Kurtz, S., 
   %     and E. R. King, Title of article, \apj {\it , 277,} 49, 1984.
   % 
   % \bibitem[{\it Hagiwara and Zeppenfeld,} 1986]{hag86} \reference 
   %     Hagiwara, K., and D. Zeppenfeld, Title of article, \rs {\it , 274,} 
   %     1, 1986.
   % \end{thebibliography}

\newpage

% Figure Captions

   % Next, we have figure captions.  It is permissible to have several
   % figure captions on the same page.  The \caption command in the 
   % figure environment works like the one in the "table" environment 
   % (it's the same one, actually), except that this one produces 
   % identification text that reads "Figure N."

\begin{figure}
\caption{We use the \LaTeX\ {\tt figure} environment syntax to set this
figure caption.  It can be a paragraph containing several sentences.}
\end{figure}

\begin{figure}
\figurewidth{35pc}
\caption{We use the \LaTeX\ {\tt figure} environment syntax to set another
figure caption.  This caption contains a ``figurewidth'' command.  AGU style
accepts several different figure caption widths.}
\end{figure}

\begin{plate}
\platewidth{35pc}
\caption{We use the AGU\TeX\ {\tt plate} environment syntax to set a
plate caption.  This caption contains a ``platewidth'' command.  AGU style
accepts several different plate caption widths.}
\end{plate}

% Tables

   % Tables should be submitted one or two to a page following the main body 
   % of the text.  Use a \clearpage command to force page breaks as needed. 
   % There should be a \clearpage after the last table so that it is forced 
   % onto its own page as well.
   %
   % We've used the planotable environment to set the next three tables.  
   % Please note that we use no vertical rules.  Requisite horizontal 
   % rules are automatically created by the \startdata command.  If you 
   % must include other rules, use the \tableline command (not an \hline 
   % command).  Note that some of the column headings require special 
   % annotation, i.e., table footnotes.  These are marked and tagged 
   % with \tablenotemark.  The \tablenotemark commands may be placed on 
   % individual data entries as well, but try to keep them to a minimum.
   % Short table notes may be placed in a \tablenotetext{\null}{TEXT} with 
   % a \null argument as the TAG.  Longer notes may be placed in a 
   % \tablecomment{TEXT} command.  Lists of table references should use 
   % \tablecomment{References:  Names of your references.}  There is 
   % an example of this in Table 3, below.
   % 
   % The \colhead{} command will automatically create centered column
   % headers, in accordance with AGU style.  This command would be used
   % in place of the usual \multicolumn{} command.  If you prefer to use
   % multicolumn{}, or if you have a more complicated table, AGUTeX will  
   % still accept that coding.  It is not recommended that both the 
   % \colhead{} and \multicolumn{} commands be used in the same table.
   % 
   % The caption contains only the caption text.  The "Table N." 
   % identification is generated by the \tablecaption{} or \caption{} 
   % commands.  The \tablecaption{} command is used with planotables 
   % and the \caption{} command is used with tabular matter inside a 
   % table environment.  The table or figure number is generated by the 
   % \caption command, not by the \begin{} command, so if you wish to 
   % \label tables and figures, the \label command should appear after 
   % the \caption has been specified.
   %
   % It is possible to use the "tabular" environment to create tables.
   % The tabular environment should be embedded within a "center" 
   % environment, which, in turn, should be inside a table environment.  
   % The table environment is needed for autonumbering and caption 
   % generation, which is why it is not enough to have a centered 
   % tabular environment.  

\clearpage

\begin{planotable}{crrrrrrrrrrr}
\tablewidth{41pc}
\tablecaption{Table with Examples of Footnotes}
\tablenum{1}
   % Note the \tablenum{} command above.  Since this manuscript
   % includes an appendix, a \tablenum command is needed or the 
   % table caption will appear "Table B1. Relevant Information"
\tablehead{
\colhead{Star}             &      \colhead{Height}            &
\colhead{$d_{x}$}          &      \colhead{$d_{y}$}           &
\colhead{$n$}              &      \colhead{$\chi^2$}          &
\colhead{$R_{\rm maj}$}    &      \colhead{$R_{\rm min}$}     &
\colhead{$P$\tablenotemark{\it a}}                            & 
\colhead{$P R_{\rm maj}$}  &      \colhead{$P R_{\rm min}$}   &
\colhead{$\Theta$\tablenotemark{\it b}}}

   % Text for table footnotes should directly precede the "startdata" 
   % command.  If they are placed between the table data and the
   % \end{planotable} command then the last cell of information will
   % not indent properly.  Note that it is acceptable to use \ref 
   % commands in \tablenotetext commands.

\tablenotetext{\it a}{Sample footnote for Table~\ref{tbl-1}.}
\tablenotetext{\it b}{Another sample footnote for Table~\ref{tbl-1}.}
\startdata
\label{tbl-1}
1 &33472.5 &-0.1 &0.4  &53 &27.4 &2.065  &1.940 &3.900 &68.3 &116.2 &-27.639\nl
2 &27802.4 &-0.3 &-0.2 &60 &3.7  &1.628  &1.510 &2.156 &6.8  &7.5 &-26.764\nl
3 &29210.6 &0.9  &0.3  &60 &3.4  &1.622  &1.551 &2.159 &6.7  &7.3 &-40.272\nl
4 &32733.8 &-1.2 &-0.5 &41 &54.8 &2.282  &2.156 &4.313 &117.4 &78.2 &-35.847\nl
5 & 9607.4 &-0.4 &-0.4 &60 &1.4  &1.669  &1.574 &2.343 &8.0  &8.9 &-33.417\nl
6 &31638.6 &1.6  &0.1  &39 &315.2 & 3.433 &3.075 &7.488 &92.1 &25.3 &-12.052
\end{planotable}

\begin{planotable}{crrrrrrrrrrr}
\tablewidth{41pc}
\tablecaption{Table with Example of a Table Note}
\tablenum{2}
\tablehead{
\colhead{Star}                     &      \colhead{Height}          &
\colhead{$d_{x}$}                  &      \colhead{$d_{y}$}         &
\colhead{$n$}                      &      \colhead{$\chi^2$}        &
\colhead{$R_{\rm maj}$}            &      \colhead{$R_{\rm min}$}   &
\colhead{$P$\tablenotemark{\it t}} & 
\colhead{$P R_{\rm maj}$}          &      \colhead{$P R_{\rm min}$} &
\colhead{$\Theta$\tablenotemark{\it u}}}
\tablenotetext{\null}{Occasionally, authors may wish to append a 
                      short paragraph of explanatory notes which 
                      pertain to an entire table but are different
                      than the caption.  Such notes may be placed in 
                      a ``tablenotetext'' environment with a ``null''
                      argument as demonstrated in this example.}
\startdata
1 &33472.5 &-0.1 &0.4  &53 &27.4 &2.065  &1.940 &3.900 &68.3 &116.2 &-27.639\nl
2 &27802.4 &-0.3 &-0.2 &60 &3.7  &1.628  &1.510 &2.156 &6.8  &7.5 &-26.764\nl
3 &29210.6 &0.9  &0.3  &60 &3.4  &1.622  &1.551 &2.159 &6.7  &7.3 &-40.272\nl
4 &32733.8 &-1.2 &-0.5 &41 &54.8 &2.282  &2.156 &4.313 &117.4 &78.2 &-35.847\nl
5 & 9607.4 &-0.4 &-0.4 &60 &1.4  &1.669  &1.574 &2.343 &8.0  &8.9 &-33.417\nl
6 &31638.6 &1.6  &0.1  &39 &315.2 & 3.433 &3.075 &7.488 &92.1 &25.3 &-12.052
\end{planotable}

\begin{planotable}{lrrrrcrrrrr}
\tablewidth{30pc}
\tablecaption{Literature Data for Program Stars}
\tablenum{3}
\tablehead{
\colhead{Star}           & \colhead{V}      &
\colhead{b$-$y}          & \colhead{m$_1$}  &
\colhead{c$_1$}          & \colhead{ref}    &
\colhead{T$_{\rm eff}$}  & \colhead{log g}  &
\colhead{v$_{\rm turb}$} & \colhead{[Fe/H]} &
\colhead{Ref}}
\tablecomments{References:  
1, {\it Barbuy, et al.} [1985]; 
2, {\it Bond} [1980]; 
3, {\it Carbon et al.} [1987];
4, {\it Hobbs and Duncan} [1987]; 
5, {\it Gilroy et al.} [1988]; 
6, {\it Gratton and Ortolani} [1986]; 
7, {\it Gratton and Sneden} [1987, 1988, 1991];
8, {\it Kraft et al.} [1982]; 
9, {\it Laird} [1990]; 
10, {\it Leep and Wallerstein} [1981]; 
11, {\it Luck and Bond} [1981, 1985]; 
12, {\it Magain} [1987, 1989]; 
13, {\it Peterson} [1981]; 
14, {\it Peterson et al.} [1990]; 
15, {\it Royce et al} [1988];
16, {\it Schuster and Nissen} [1988 a,b]; 
17, {\it Spite et al.} [1984]; 
18, {\it Spite and Spite} [1986]; 
19, {\it Hobbs and Thorburn} [1991].  Occasionally, authors may wish 
    to append longer paragraphs of explanatory notes which pertain to 
    an entire table but are different from the caption.  These notes 
    may be placed in a ``tablecomments'' environment as demonstrated 
    in this example.  Only one paragraph of material is permitted at 
    the end of a table, (excluding called-out references), so if both
    references and notes exist, they should be run in together.}
\tablenotetext{\it a}{This is another example of a footnote.}
\tablenotetext{\it b}{Star LP 608--62 is also known as BD+1\deg 2341p.}
\startdata
HD 97 & 9.7& 0.51& 0.15& 0.35& 2 & \nodata & \nodata & \nodata & $-1.50$ 
& 2 \nl 
& & & & & & 5015 & \nodata & \nodata & $-1.50$ & 10 \nl
HD 2665 & 7.7& 0.54& 0.09& 0.34& 2 & \nodata & \nodata & \nodata & $-2.30$ 
& 2 \nl 
& & & & & & 5000 & 2.50 & 2.4 & $-1.99$ & 5 \nl
& & & & & & 5120 & 3.00 & 2.0 & $-1.69$ & 7 \nl
& & & & & & 4980 & \nodata & \nodata & $-2.05$ & 10 \nl
HD 4306 & 9.0& 0.52& 0.05& 0.35& 20, 2& \nodata & \nodata & \nodata & 
$-2.70$ & 2 \nl
& & & & & & 5000 & 1.75 & 2.0 & $-2.70$ & 13 \nl
& & & & & & 5000 & 1.50 & 1.8 & $-2.65$ & 14 \nl
& & & & & & 4950 & 2.10 & 2.0 & $-2.92$ & 8 \nl
& & & & & & 5000 & 2.25 & 2.0 & $-2.83$ & 18 \nl
& & & & & & \nodata & \nodata & \nodata & $-2.80$ & 21 \nl
& & & & & & 4930 & \nodata & \nodata & $-2.45$ & 10 \nl
HD 84937 & 8.3& 0.30& 0.06& 0.35& 20,11& 6200 & \nodata & \nodata & 
$-2.10$ & 4 \nl
& & & & & & 6250 & \nodata & \nodata & $-2.18$ & 17 \nl
& & & & & & 6216 & \nodata & \nodata & $-2.42$ & 11 \nl
& & & & & & 6240 & \nodata & \nodata & $-2.13$ & 3 \nl
& & & & & & \nodata & \nodata & \nodata & $-2.14$ & 21 \nl
& & & & & & 6200 & 3.60 & 1.5 & $-2.43$ & 16 \nl
& & & & & & 6250 & 4.00 & \nodata & $-2.10$ & 22 \nl
\cutinhead{This Is an Example of a Center Head} \nl
HD 87140 & 9.0& 0.48& 0.12& 0.28& 20 & 5000 & 4.50 & 1.0 & $-1.41$ & 7 \nl
& & & & & & \nodata & \nodata & \nodata & $-1.56$ & 21 \nl
HD 88609 & 8.6& 0.68& 0.09& 0.54& 2 & \nodata & \nodata & \nodata & 
$-2.50$ & 2 \nl
& & & & & & 4500 & 1.10 & 2.8 & $-2.77$ & 5 \nl
& & & & & & 4500 & 0.80 & 3.2 & $-2.65$ & 14 \nl
& & & & & & 4600 & \nodata & \nodata & $-2.75$ & 10 \nl
HD 94028 & 8.2& 0.34& 0.08& 0.25& 20 & 5795 & 4.00 & \nodata & $-1.70$ & 
22 \nl
& & & & & & 5860 & \nodata & \nodata & $-1.70$ & 4 \nl
& & & & & & 5910 & 3.80 & \nodata & $-1.76$ & 15 \nl
& & & & & & 5800 & \nodata & \nodata & $-1.67$ & 17 \nl
& & & & & & 5902 & \nodata & \nodata & $-1.50$ & 11 \nl
& & & & & & 5900 & \nodata & \nodata & $-1.57$ & 3 \nl
& & & & & & \nodata & \nodata & \nodata & $-1.32$ & 21 \nl
HD 97916 & 9.2& 0.29& 0.10& 0.41& 20 & 6125 & 4.00 & \nodata & 
$-1.10$ & 22 \nl
& & & & & & 6160 & \nodata & \nodata & $-1.39$ & 3 \nl
& & & & & & 6240 & 3.70 & \nodata & $-1.28$ & 15 \nl
& & & & & & 5950 & \nodata & \nodata & $-1.50$ & 17 \nl
& & & & & & 6204 & \nodata & \nodata & $-1.36$ & 11 \nl
\cutinhead{At Least Two Center Heads Are Required if Any Are Used} \nl
+26\deg2606& 9.7&0.34&0.05&0.28&20,11& 5980 & \nodata & \nodata & 
$<-2.20$ & 19 \nl
& & & & & & 5950 & \nodata & \nodata & $-2.89$ & 24 \nl
+26\deg3578& 9.4&0.31&0.05&0.37&20,11& 5830 & \nodata & \nodata & 
$-2.60$ & 4 \nl
& & & & & & 5800 & \nodata & \nodata & $-2.62$ & 17 \nl
& & & & & & 6177 & \nodata & \nodata & $-2.51$ & 11 \nl
& & & & & & 6000 & 3.25 & \nodata & $-2.20$ & 22 \nl
& & & & & & 6140 & 3.50 & \nodata & $-2.57$ & 15 \nl
+30\deg2611& 9.2&0.82&0.33&0.55& 2 & \nodata & \nodata & \nodata & 
$-1.70$ & 2 \nl
& & & & & & 4400 & 1.80 & \nodata & $-1.70$ & 12 \nl
& & & & & & 4400 & 0.90 & 1.7 & $-1.20$ & 14 \nl
& & & & & & 4260 & \nodata & \nodata & $-1.55$ & 10 \nl
+37\deg1458& 8.9&0.44&0.07&0.22&20,11& 5296 & \nodata & \nodata & 
$-2.39$ & 11 \nl
& & & & & & 5420 & \nodata & \nodata & $-2.43$ & 3 \nl
+58\deg1218&10.0&0.51&0.03&0.36& 2 & \nodata & \nodata & \nodata & 
$-2.80$ & 2 \nl
& & & & & & 5000 & 1.10 & 2.2 & $-2.71$ & 14 \nl
& & & & & & 5000 & 2.20 & 1.8 & $-2.46$ & 5 \nl
& & & & & & 4980 & \nodata & \nodata & $-2.55$ & 10 \nl
+72\deg0094&10.2&0.31&0.09&0.26&12 & 6160 & \nodata & \nodata & 
$-1.80$ & 19 \nl
G5--36\tablenotemark{\it a} & 10.8& 0.40& 0.07& 0.28& 20 & \nodata & 
\nodata & \nodata & $-1.19$ & 21 \nl
G18--54 & 10.7& 0.37& 0.08& 0.28& 20 & \nodata & \nodata & \nodata & 
$-1.34$ & 21 \nl
G20--08 & 9.9& 0.36& 0.05& 0.25& 20,11& 5849 & \nodata & \nodata & 
$-2.59$ & 11 \nl
& & & & & & \nodata & \nodata & \nodata & $-2.03$ & 21 \nl
G20--15 & 10.6& 0.45& 0.03& 0.27& 20,11& 5657 & \nodata & \nodata & 
$-2.00$ & 11 \nl
& & & & & & 6020 & \nodata & \nodata & $-1.56$ & 3 \nl
& & & & & & \nodata & \nodata & \nodata & $-1.58$ & 21 \nl
G205--42 & 10.0& 0.43& 0.10& 0.23& 11 & 5301 & \nodata & \nodata & 
$-2.18$ & 11 \nl
G217--08 & 10.5& 0.35& 0.06& 0.28& 11 & 6066 & \nodata & \nodata & 
$-2.29$ & \nodata\nl
G243--63 & 7.7& 0.48& 0.13& 0.30& 11 & \nodata & \nodata & \nodata & 
$-1.66$ & 21 \nl
LP 608--62\tablenotemark{\it b} & 10.5& 0.30& 0.07& 0.35& 11 & 6250 & 
\nodata & \nodata & $-2.70$ & 4
\end{planotable}

\begin{planotable}{llrr}
\tablewidth{20pc}
\tablecaption{Example of an Appendix Table}
\tablenum{A1}
\tablehead{\multicolumn{2}{c}{}&
\multicolumn{2}{c}{\it dl/dt, \rm mm/yr}\\[.3ex]
\cline{3-4}\\[-1.6ex]
\multicolumn{1}{c}{From} &
\multicolumn{1}{c}{To} &
\multicolumn{1}{c}{Observed} &
\multicolumn{1}{c}{Model}}
\tablenotetext{\null}{Note that this table uses the ``tablenum'' command
since it needs a letter ``A'' in the table caption.}
\startdata
Alamillo & Palvadero & $0.6 + 0.8$ & $1.4$\nl
Campana & Canas &$0.4 + 1.1$ & $-0.7$\nl
& Chupadera & $ -0.5 + 1.0$ & $0.2$
\end{planotable}

\begin{planotable}{llrr}
\tablewidth{20pc}
\tablecaption{Example of a Second Appendix Table}
\tablehead{\multicolumn{2}{c}{}&
\multicolumn{2}{c}{ \it dl/dt, \rm mm/yr}\\[.3ex]
\cline{3-4}\\[-1.6ex]
\multicolumn{1}{c}{From} &
\multicolumn{1}{c}{To} &
\multicolumn{1}{c}{Observed \tablenotemark{\it a}} &
\multicolumn{1}{c}{Model}}
\tablenotetext{\null}{If there is no ``tablenum'' command, table captions 
will automatically include the lettering of the last appendix.}
\tablenotetext{\it a}{The quoted uncertainty is one standard deviation.}
\startdata
Alamillo & Palvadero & $0.6 + 0.8$ & $1.4$\nl
Campana & Canas &$0.4 + 1.1$ & $-0.7$\nl
& Chupadera & $ -0.5 + 1.0$ & $0.2$
\end{planotable}

\begin{table}
\caption{Example of a Table in a Tabular \protect\\ Environment} 
\tablenum{A1}
\tablenotetext{\null}{This table is an example of \LaTeX's \verb"table" 
and \\ \verb"tabular" environments!}
\vspace{5pt}
\begin{tabular}{llrr} 
\tableline 
& & & \\[-5pt]
\multicolumn{2}{c}{} & \multicolumn{2}{c}{ \it dl/dt, \rm mm/yr}\\[4pt]
\cline{3-4}\\[-7pt]
\multicolumn{1}{c}{From} & \multicolumn{1}{c}{To} & \multicolumn{1}{c}{Observed} & \multicolumn{1}{c}{Model} \\[4pt]
\tableline
& & & \\[-6pt]
Alamillo & Palvadero & $0.6 + 0.8$ & $1.4$\\
Campana & Canas &$0.4 + 1.1$ & $-0.7$\\
& Chupadera & $ -0.5 + 1.0$ & $0.2$\\[4pt]
\tableline
& & & \\[-8pt]
\end{tabular}
\end{table}

   % This is the last table for this paper so we should follow 
   % it with a \clearpage command.  We use \clearpage rather 
   % than \newpage in order to force all the floating tables 
   % out of their buffers.

\clearpage

   % LaTeX's technique of segregating major semantic components 
   % of a document within marked up "environments" is a very 
   % good one, but you as an author have to come up with a way 
   % of making sure each \begin{tagname} has a corresponding 
   % \end{tagname}.  If you miss one, LaTeX will probably 
   % complain a great deal during the composition of the document.  
   % Occasionally you can get away with it right up to the 
   % \end{document}, in which case you will see an error message 
   % saying "\begin{tagname} ended by \end{document}" for instance, 
   % "\begin{planotable} ended by \end{document}."

\end{document}
