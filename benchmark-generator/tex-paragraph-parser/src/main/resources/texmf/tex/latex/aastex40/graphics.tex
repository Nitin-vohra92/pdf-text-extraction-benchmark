% GRAPHICS.TEX -- EPS graphics inclusion proposal for WGAS.

% The style file name was modified so it was compatible with v4.0. jb 4/20/95

\documentstyle[12pt,aaspp4,flushrt]{article}

\def\emphasize#1{{\sl #1\/}}

\begin{document}
\title{Graphics inclusions in manuscripts}
\author{C. D. Biemesderfer}
\affil{NOAO, Tucson, AZ}

There has been considerable development of text processing and
typesetting tools over the past decade, and more and more
scientists are using these complex programs to produce the
material that they publish.
Over the past several years, technology for creating
finished-quality charts and graphics has started to come of age:
versatile plotting and drawing programs are now widely available,
high-resolution bit-mapped video displays are becoming commonplace,
and laser printer hardware and software has matured.

\begin{figure}[h]
\plotone{sgi9289.eps}
\caption{IRAF plot} \label{onebarrel}
\end{figure}

It is natural for astronomers who are publishing scientific
results to desire a mechanism for merging their graphical and
textual data in a way that ensures a certain integrity in the
page layout.
Fortunately, the adoption of PostScript\footnote{PostScript
is a registered trademark of Adobe Systems Incorporated.}
as a de facto \emphasize{page description language} standard
by a substantial fraction of the relevant sectors of the
industry makes it simpler for programmers to set up the machinery.

\section{Why PostScript is good}

PostScript has a number of merits that make it a good choice
as the language to use for graphics exchange.
Most important is that it is a \emphasize{device-independent}
means of describing printing on a page, and has been widely
implemented across a variety of printers and display devices.
It is fairly well-documented and tested, so different
implementations work reliably and behave predictably.

Furthermore, PostScript is composed of text using the ASCII
character set.  This makes is very easy to transport files
on networks, and they are susceptible to manipulation by
standard text processing tools, e.g., editors, pagers, etc.

\subsection{Encapsulated PostScript}

It is desirable for graphics inclusions to conform to certain
codes of behavior, so that the graphics can be manipulated
readily and reliably.  PostScript can be generated at any of
several levels of so-called \emphasize{structuring conventions},
which are a more or less inevitable consequence of PostScript's
heritage as a programming language.

When a PostScript program is to be interpreted as a simple
page description, it is convenient if it obeys some rules of
form.  For the purposes of graphics inclusion, the most
important property is that the PostScript be \emphasize{encapsulated}.
What this means essentially is that the including application
can determine the size and location on the page of the graphic
\emphasize{without} having to interpret any PostScript code.

The boundaries of such a ``capsule'' graphic are defined by
a \verb"BoundingBox" comment that specifies the $x$ and $y$
coordinates of two opposing corners of the graphic.
There are other ``do's and don'ts'' in the quest to produce
encapsulated PostScript; however, it is generally sufficient
for the including application to specify that imported
PostScript graphics conform to the encapsulation standard.

\section{Overview of the DVIPS commands}

This is intended as a cursory look at the commands that can
be used to work with graphics inclusions.  It is by no means
necessary for authors to know \emphasize{any} of this; markup
language writers can easily shield the end-user from all of
the details of the PostScript interface.

A graphic that is delivered to an application in a capsule
defined by a bounding box can be subjected to a limited number
of operations: translation, truncation, scaling, and rotation.
Translation, truncation (called \emphasize{clipping} in PostScript
context), and scaling can be performed on each coordinate
independently, hence we can identify seven primitive functions:
\begin{center}
\begin{tabular}{ll@{\hspace{2em}}ll}
\tt hoffset & horizontal offset & \tt voffset & vertical offset \\
\tt hsize & horizontal clip size & \tt vsize & vertical clip size \\
\tt hscale & horizontal scale factor & \tt vscale & vertical scale factor \\
\tt angle & rotation angle \\
\end{tabular}
\end{center}
For most graphics inclusions, scaling is the most important
function.  It is rare to clip or rotate imported graphics,
and moving the coordinate system origin from the current
point is often troublesome in a text formatter.
Furthermore, there is usually no reason to alter the aspect
ratio of the graphic, so including encapsulated PostScript is
often as simple as specifying a scale factor or a dimension
and reading the file.

\begin{figure}[h]
\plottwo{sgi9279.eps}{sgi9259.eps}
\caption{Dual IRAF plots} \label{twobarrel}
\end{figure}

\section{Implications for the AAS\TeX\ package}

We are concerned with the import of two-dimensional graphics and
grey-scale images into scientific manuscripts and other technical
documentation.  At this point in time, it should be adequate to
require that all graphics for import be in the form of encapsulated
PostScript, and to declare that figures will be imported in their
entirety and with the same aspect ratio as in the original.
It is then trivial to build some simple macros based on the \verb"epsf"
substyle that is supplied with the DVIPS program by its author
Tom Rokicki.

The \verb"epsf" macros we need to worry about are \verb"\epsfxsize"
and \verb"\epsfbox", which perform the two functions we determined
in the preceding section to be critical.  For purposes of having
prototype macros, I wrote two simple macros:

\hangindent=3pc \hangafter=1
\noindent
\verb"\plotone{FILE}" reads the PostScript in FILE and adjusts
the scale such that the $x$-coordinate width of the graphic matches
the \verb"\textwidth" of the manuscript.
Figure~\ref{onebarrel} on page~\pageref{onebarrel} is an example.

\hangindent=3pc \hangafter=1
\noindent
\verb"\plottwo{FILE1}{FILE2}" reads the PostScript from two
files and scales each to fit across half the \verb"\textwidth"
(actually, slightly less than half so the graphics don't collide).
Figure~\ref{twobarrel} on page~\pageref{twobarrel} shows a pair
of dueling inclusions.

The arguments to the \verb"\plotone" and \verb"\plottwo" commands
should not include additional scaling and rotation information.
These commands are suggested as part of the AAS\TeX\ markup
conventions, but it is entirely reasonable for authors to use the
markup syntax of DVIPS for handling encapsulated PostScript graphics.
If direct access to the graphic is desired, it is necessary to use
the interface syntax defined by Rokicki.

\begin{figure}[h]
\vbox to2.6in{\rule{0pt}{2.6in}}
\special{psfile=sgi9259.eps angle=180 hoffset=424 voffset=232 vscale=60 hscale=60}
\caption{Inverted IRAF plot}
\end{figure}

One has to be careful, though, when using this approach, since the
automatic link between the formatter (\LaTeX) and the PostScript
is abandoned.  One can get great effects, or totally funky ones \ldots

\begin{figure}[h]
\vbox to3in{\rule{0pt}{3in}}
\special{psfile=sgi9259.eps voffset=-218 hoffset=60 vscale=75 hscale=55 angle=30}
\caption{Mangled IRAF plot}
\end{figure}

\end{document}
