% File jydoc.tex, documentation for (and written in) jytex.

\input jytex

%********** First, the \typesize command **********

\typesize=12pt

%********** Setting up the document **********

\hsize=6in
\vsize=8.9in

\leftmargin=1.25in
%\oddleftmargin=.5in
%\evenleftmargin=0pt

\abovedisplayskip=\smallskipamount
\belowdisplayskip=\smallskipamount
\abovedisplayshortskip=\smallskipamount
\belowdisplayshortskip=\smallskipamount

\chapternumstyle{Roman}
\sectionnumstyle{arabic}
\subsectionnumstyle{blank}

\baselinestretch=1300

%********** The \draft command, if appropriate **********

%\draft

%********** Now some special definitions for this document **********

\def\~{\discretionary{}{}{}}            % Special miscellaneous

\catcode`\@=11                          % Don't worry about this
\def\err@stylenotavailable#1{{\normalfonts$\bullet$}}
\catcode`\@=12

\def\tts{\bgroup\tt                     % This is for doing typewriter
     \catcode`\\=12 \catcode`\{=12 \catcode`\}=12 \catcode`\$=12 \catcode`\#=12
     \catcode`\&=12 \catcode`\^=12 \catcode`\|=0  \catcode`\(=1  \catcode`\)=2
     \gettts}
\def\gettts#1/{#1\/\egroup}
\hyphenchar\xiipttt=-1                  % No hyphenation of typewriter
\def\({\char"28 }
\def\){\char"29 }

\long\def\docmarginnote#1{\leavevmode   % A special \marginnote
     \edef\@@marginsf{\spacefactor=\the\spacefactor\relax}%
     \strut\vadjust{%
          \hbox to0pt{\hskip\hsize\hskip.1in
               \vbox to0pt{\vskip-\dp\strutbox
                    \subscriptsize\baselinestretch=1000
                    \hsize=1in \tolerance=5000 \hbadness=5000
                    \leftskip=0pt \rightskip=0pt \everypar={}%
                    \raggedright\parskip=0pt \parindent=0pt
                    \vskip-\ht\strutbox
                    \noindent\strut#1\par
                    \vss}%
               \hss}}%
     \@@marginsf}


%********** Here we go!  Title page **********

\pagenumstyle{blank}

\null\vskip0pt plus.6fil

\centertext{\Bigfonts\jyTeX{} version \fmtversion}

\newpage

\pagenumstyle{arabic}
\pagenum=0


%********** First chapter **********

{\bigfonts\bfs\chapter{BASIC COMMANDS}}

This chapter will provide what you need to know to create a document with
\jyTeX.  Generally speaking, the commands described below can be divided into
two groups: those that control the ``global'' parameters of the document (like
the base type size, page dimensions, {\it etc.}), and those that affect the
``local'' appearance of the text (size or style changing, centering, {\it
etc.}).  There are some restrictions on the use of the global commands.  First,
they generally should be invoked only once (if at all), at the beginning of
your \TeX{} file.  (This is not a hard-and-fast rule; most can be used more
than once, but it should be rarely necessary and is not recommended unless you
have a good understanding of what they do.)  Second, the order in which these
commands appear is mildly important, so read carefully.  You can also consult
the \TeX{} file that produced this document for an example of their use.


{\bigfonts\bfs\section{Using Fonts}}

This section describes how to access the veritable plethora of fonts provided
by \jyTeX.  Many examples of the different sizes and styles are given here; the
symbol~``$\bullet$'' will appear in inappropriate places below if any are
unavailable on your machine. (Normally, warnings about unavailable fonts are
written to the terminal as the job is run.)

{\sl\subsection{The\/ {\tt\string\typesize} command
     and size-changing}}

The first line of your \jyTeX{} file should contain the \tts\typesize/ command,
which establishes what the normal size of fonts in the document will be.  (This
manual is set with \tts\typesize=12pt/.)  If you're the kind that doesn't like
to be tied down to a particular font size, have no fear---a larger font, for
example, is only a \tts\bigfonts/ {\bigfonts command away.  As should be
obvious by now, \tts\bigfonts/ changes the normal size fonts, {\it in all
styles,} into their larger size cousins.}

{\bigsize

Actually, there's a little more to it than that.  There are two commands in
\jyTeX{} that will produce these large fonts: \tts\bigfonts/ and \tts\bigsize/.
\tts\bigfonts/ does exactly what is described above, {\it i.e.,} replaces the
normal fonts with larger ones.  \tts\bigsize/ also does this, but in addition
scales a number of internal parameters (such as the \tts\baselineskip/) so that
typesetting with the new size looks more like a scaled up version of the normal
one.  (This paragraph was typeset after first giving a \tts\bigsize/ command.
Notice how the line spacing is larger than in the previous paragraph.)

}

So which command do you use if you want larger fonts?  The simple answer is to
use \tts\bigfonts/ to enlarge single words or one-line bits of text, and use
\tts\bigsize/ for multi-line text or paragraphing.  One consideration is
that \tts\bigsize/ is a fairly slow and difficult command for \jyTeX{} to
execute compared to \tts\bigfonts/, so if typesetting speed is an issue it
might be worthwhile to eliminate \tts\bigsize/ commands in favor of
\tts\bigfonts/ whereever possible.

By this time you've probably guessed that \tts\bigfonts/ and \tts\bigsize/
aren't the only size-changing commands available to you.  In fact, there's 
a whole range of larger sizes:

\smallskip

\halign{\hskip\parindent\strut#\hfil&\quad\hfil#\hfil&\qquad#\hfil\cr
     \underline{Command\vphantom p}%
          &\underline{Scale factor\vphantom p}%
          &\underline{Example}\cr
     \noalign{\smallskip}
     \tts\bigfonts/&\phantom(1.2\phantom{)$^1$}&\bigsize\strut Big.\cr
     \tts\Bigfonts/&$(1.2)^2$&\Bigsize\strut Real big.\cr
     \tts\biggfonts/&$(1.2)^3$&\biggsize\strut SUPER big.\cr
     \tts\Biggfonts/&$(1.2)^4$&\Biggsize\strut YOW!\cr
     \noalign{\goodbreak\medskip\noindent
          Of course, there are plenty of smaller sizes available also:
          \medskip}
     \underline{Command\vphantom p}%
          &\underline{Scale factor\vphantom p}%
          &\underline{Example}\cr
     \noalign{\smallskip}
     \tts\smallfonts/&1/1.1\phantom{$()^1$}%
          &\smallfonts``We begin bombing in five minutes.''\cr
     \tts\footnotefonts/&1/1.2\phantom{$()^1$}%
          &\footnotefonts``We could do it, but it would be wrong.''\cr
     \tts\subscriptfonts/&$1/(1.2)^2$&\subscriptfonts``I am not a crook.''\cr
     \tts\subsubscriptfonts/
          &$1/(1.2)^4$&\subsubscriptfonts``Mistakes were made.''\cr}

\smallskip

\noindent (The names of the last three entries describe their use in \jyTeX.)
Each of these has a corresponding \tts\...size/ partner.  In addition, there
are also the commands \tts\tinyfonts/ and \tts\tinysize/ (the smallest type
available, usually~\tts5pt/), \tts\HUGEFONTS/ and \tts\HUGESIZE/ (the largest
type available, usually~\tts25pt/), and \tts\normalfonts/ and \tts\normalsize/
(which return \jyTeX{} to the base size). None of these commands work in math
mode (they are ignored).

As explained above, the \tts\...size/ version of each command differs from its
\tts\...fonts/ counterpart in that it causes internal parameters to be scaled.
The most important in practice besides the \tts\baselineskip/ are the sizes of
\tts\bigskip/, \tts\medskip/, \tts\smallskip/, which are always normalized
after a \tts\...size/ command to correspond to a one-line, half-line, and
quarter-line skip (with some stretch and shrink), respectively.

\tts\typesize/ is a ``global'' command and has some limitations. It may
only be invoked once and, as stated above, should be the first command in
your file. Because only certain sizes are generally available, the only
base sizes supported are \tts10pt/, \tts12pt/, and \tts14pt/. This means
that if you want to typeset a document in, say, \tts11pt/, you can only get
it by \tts\typesize=12pt \smallsize/.

{\sl\subsection{Style-changing commands}}

\begin{block}

\catcode`\@=11                          % Don't worry about this
\def\err@bfstobf{{\normalsize$\bullet$}}
\def\err@sltoit{{\normalsize$\bullet$}}
\catcode`\@=12

You can also change the type style, rather than the size, in the usual way:
for example, \tts\it/ switches to {\it italic\/} type. There's also
\tts\rm/ (roman, the default), \tts\bf/ {\bf (bold)}, \tts\bfs/ (a slimmer
{\bfs bold} than \tts\bf/, giving a sort of nineteenth-century feel),
\tts\sc/ {\sc (Small Caps)}, \tts\sl/ {\sl (slanted)}, \tts\ss/ {\ss (sans
serif)}, and \tts\tt/ {\tt (typewriter)}. (A size-changing command always
leaves you with roman type, so if you want large italic type you should
specify \tts\bigfonts\it/, {\it not\/} \tts\it\bigfonts/.) In math mode
things are a bit different: you have the special math styles \tts\mit/
($math$ $italic$, the default), \tts\bmit/ (bold math, for bold
$\bmit\Gamma\mskip-3mu \rho\epsilon\epsilon\kappa$ and other symbols), and
\tts\cal/ (for $\cal CAL$ligraphic characters), as well as the usual
\tts\rm/, \tts\bf/, \tts\bfs/, and \tts\it/, but you can't say \tts\sc/,
\tts\sl/, \tts\ss/, or \tts\tt/ (except inside an \tts\hbox/).

Style-changing commands should be used with caution in math mode.  In
particular, they should be kept confined within braces (the penalty for letting
them run free could be some unexplainable output).  For example, you can't make
an equation in boldface by typing \tts$$\bmit/\dots\tts$$/.  The reason for
this is that different bold symbols come from different fonts, and some symbols
have no bold analog at all.  \tts${\bmit\Phi}$/, for example, gives you
$\bmit\Phi$, the bold version of the $\mit\Phi$ in the math italic font.  But
if you want $\bf\Phi$ (the upright version) you have to say \tts${\bf\Phi}$/,
because the unbold $\Phi$ comes from the roman font.  The basic rule is this:
\tts\bf/ and \tts\bfs/ give you a bold roman characters and \tts\bmit/ gives
you bold math italic.  If the symbol you want to make bold isn't in either of
those fonts you're out of luck.  (To see which characters reside in what fonts,
see Appendix F of the \TeX book.)

You may have noticed that \jyTeX's \tts\ss/ command for invoking the sans serif
font supercedes the \TeX{} command that produces ``\sharps,'' the German sharp
``s.'' In \jyTeX{} you get this character with \tts\sharps/.

To close this section, you should make sure you understand that a
{\subscriptfonts sentence} {\bf like} {\bigfonts\it THIS!\/} is produced by

\nobreak\smallskip

\dots\tts a {\subscriptfonts sentence} {\bf like} {\bigfonts\it THIS!}/\dots.

\end{block}


{\bigsize\bfs\section{Setting Up a Document}}

After giving the \tts\typesize/ command it's time to specify the general format
of the document.  \jyTeX{} ``wakes up'' with a format already in place, so you
may find that you don't need to use many of the commands described below.
Generally speaking, these are all global commands and should be invoked only
once at the beginning of the file (except where noted), but if you insist on
using them again, at least do it after an explicit page break (see
\tts\newpage/, below).

{\sl\subsection{Normal vs.\ two-page output}}

The \tts\outputstyle/ command determines the general layout of the page. There
are two styles available: they are selected by the commands
\tts\outputstyle{normal}/ and \tts\outputstyle{twoup}/.  The default is \tts
normal/.

When the \tts\outputstyle/ is \tts normal/, the text forms a box on the page
whose dimensions are the values of the \TeX{} registers \tts\vsize/ and
\tts\hsize/. The distance of this box from the top and left edges of the paper
is determined by the \jyTeX{} registers \tts\topmargin/ and \tts\leftmargin/.
These are initialized by the commands

\nobreak\smallskip

\tts\vsize=8.9in \hsize=6.5in/ \par
\tts\topmargin=1in \leftmargin=1in/ \par

\smallskip

\noindent Above and below the page are single lines of text called the {\it
headline} and {\it footline}, the contents of which are determined by the
token variables \tts\head/ and \tts\foot/. These have the default values

\nobreak\smallskip

\tts\head={\hfil}/ \par
\tts\foot={\hfil\normalfonts\numstyle\pagenum\hfil}/ \par

\smallskip

\noindent As you can see, the \tts\head/ is normally empty.  The definition of
\tts\foot/ requires some explanation (see below), but it should be reasonably
obvious that it contains a centered page number in the base size.

If you select \tts\outputstyle{twoup}/, two pages of the document will appear
side by side, printed ``sideways'' on each page of output (you should check
that your printer has a mode that allows you to print this way).  In this case,
\tts\vsize/ and \tts\hsize/ are the dimensions of the document page, a new
register \tts\fullhsize/ holds the width of the two pages combined, and
\tts\topmargin/ and \tts\leftmargin/ establish the positions of the top and
left margins of the output page.  They are initialized to

\nobreak\smallskip

\tts\vsize=6.9in \hsize=4.75in \fullhsize=10in/ \par
\tts\topmargin=.75in \leftmargin=.5in/ \par

\smallskip

\noindent Each page of the document has an empty headline and a footline that
contains the page number.

If you find that the default values of these page parameters are not to your
liking you can simply change them.  Such changes should (obviously) appear {\it
after\/} the \tts\outputstyle/ command, if there is one.

{\sl\subsection{Two-sided output}}

The previous section described \tts\topmargin/ as a register whose value
determines the distance of the text from the top of the page.  This was a lie.
There are actually two registers that handle this task: \tts\oddtopmargin/
controls the margin on odd-numbered pages, and \tts\eventopmargin/ controls it
on even-numbered pages.  The \tts\topmargin/ command is just a convenient
shorthand for setting both of these registers to the same value.  By assigning
different values to these parameters you can make odd and even pages different.

The above comments also apply to the \tts\leftmargin/ command.  There are
really two registers that determine where the left margin of the page is:
\tts\oddleftmargin/ and \tts\evenleftmargin/.  It is useful to set these to
different values when you want to produce two-sided output (``two-sided''
meaning that the pages are meant to be placed back-to-back).  This way you can
arrange it that the text ends up back-to-back even when the left and right
margins are not the same.  Again, \tts\leftmargin/ sets both registers to the
same value.

You may have guessed by now that the \tts\head/ and \tts\foot/ commands conceal
the true registers \tts\oddhead/, \tts\evenhead/, \tts\oddfoot/, and
\tts\evenfoot/.  It is very common to set odd and even differently for
two-sided output.  A book, for example, might have a headline that contains the
chapter name:

\nobreak\smallskip

\tts\oddhead={\hfil Chapter 1} \evenhead={Chapter 1\hfil}/ \par

\smallskip

\noindent These assignments cause the odd-numbered pages to show the chapter
name on the right and even numbered pages to show it on the left.

{\sl\subsection{The\/ {\tt\string\baselinestretch} command}}

The distance between lines is initialized to an appropriate value when \jyTeX{}
starts up, but most technical papers look better if the lines are moved apart
somewhat to accommodate math symbols in the text. \jyTeX{} provides the command
\tts\baselinestretch/ to accomplish this.  A typical example might be:

\nobreak\smallskip

\tts\baselinestretch=1200/

\smallskip

\noindent The number 1200 is 1000 times the ``stretch'' applied to the default
line separation.  The command above would expand the baseline distance by a
factor of 1.2.  It will also stretch \tts\bigskip/,
\tts\medskip/, and \tts\smallskip/ so that they still skip a full line, half
line, and quarter line, respectively.

If you make the \tts\baselinestretch/ very large (a double-spaced document
would have \tts\baselinestretch|~=|~2000/, for example), you may find that the
positions of the headline and footline don't look right.  This is because the
line separation is now bigger than the distance of these lines from the text.
This can be fixed by adjusting the values of \tts\headskip/ and \tts\footskip/.
These are normally both set to twice the normal line separation (\tts24pt/ when
\tts\typesize=10pt/).

The \tts\baselinestretch/ can be changed anywhere in the document (but only
between paragraphs, please).

{\sl\subsection{Sectioning commands}}

\jyTeX{} provides the \tts\chapter/, \tts\section/, and \tts\subsection/
commands to facilitate the division of a document into (surprise) chapters,
sections, and subsections.  These all work much the same way.

Let's start with an example of \tts\chapter/.  To start a new chapter, all you
have to do is say

\nobreak\smallskip

\tts\chapter{Martian Holiday}/

\smallskip

\noindent (you would probably want to use a different chapter name, but you get
the idea).  That's all there is to it.  \jyTeX{} takes care of things like
starting a new page, putting space above and below, and printing the title
left-justified.

If you choose, \jyTeX{} will also keep track of the chapter number and prepend
it to the title (more on this below).  In fact, assuming that you are set up
for chapter numbers, and that ``Martian Holiday'' is the third chapter of your
tome, the command above would yield

\nobreak\smallskip

\noindent 3. Martian Holiday

\smallskip

\noindent It would appear at the top of a fresh page, and subsequent text would
start after a space of about one and a half lines.

OK, so it looks rather unimpressive in 12-point roman type.  If you want it
instead to come out big and bold, just type

\nobreak\smallskip

\tts{\bigfonts\bf\chapter{Martian Holiday}}/

\smallskip

\noindent and that's what you'll get.  Notice that you don't write the
font-changing commands inside the argument to \tts\chapter/---that would indeed
make the title ``Martian Holiday'' come out big and bold, but would not change
the font of the chapter number.

If \jyTeX{} encounters a \tts\section/ or \tts\subsection/ command the behavior
is similar.  For example,

\nobreak\smallskip

\tts\section{Martian Customs}/

\smallskip

\noindent first causes a page break if the current page is nearly full (unlike
\tts\chapter/, which always starts a new page).  If there is no page break, a
little space is skipped, then the section title is printed (preceded by a
composite of the chapter and section numbers, if you so choose), then a little
more space is skipped before any following text.  The same goes for
\tts\subsection/. 

Of course, you may not like the way \jyTeX{} handles a lot of these details, so
here's the full story about what goes on behind the scenes (skip this if you
don't care).  When you give a command like \tts\section{Martian Customs}/ all
hell breaks loose. First, the macro \tts\sectionbreak/ decides whether or not
to break the page, based on how full it is.  Next, there is a \tts\vskip/ by
the amount \tts\abovesectionskip/.  Now the section title is actually printed,
but how it is output depends on the current \tts\sectionstyle/; \jyTeX{} is
initialized with \tts\sectionstyle{left}/, meaning that the section title will
appear flush left, but you can change this behavior by preceding the
\tts\section/ command by \tts\sectionstyle{center}/ or
\tts\sectionstyle{right}/, with obvious consequences. Finally, there is another
\tts\vskip/ by the amount \tts\belowsectionskip/.

If this sounds complicated, just remember that all {\it you} have to do is
give the \tts\section{Martian Customs}/ command, and the rest (page
breaking, putting space around the title, justifying) is taken care of for
you. The point of mentioning these other commands in the ``\tts\section/
family'' is that you can change them if you don't like what they do. For
example, if you want more space after a section name, just increase the
value of \tts\belowsectionskip/ (as usual, all of these parameters can be
found in the \tts Initialization/ section at the end of \tts jymacros.tex/).

{\sl\subsection{OK, there's more: sectioning (and other) counters}}

The rosy picture painted above is a slight oversimplification, because it
doesn't really explain what happens if you're numbering your chapters,
sections, and subsections. It is basic \jyTeX{} philosophy that you should
never have to number anything ``by hand,'' so internal counters for chapters,
sections, and subsections are provided to take care of this chore for you.
Before getting into how these counters are used in the sectioning commands
described above, let's first discuss how \jyTeX{} manipulates numbers.

Numbers in \jyTeX{} may be output in any one of a variety of styles. The
number ``16,'' for example, can be easily and automatically converted into the
roman numeral ``\Roman{16}'' simply by typing \tts\Roman{16}/. The full
list of number styles includes:

\smallskip

\halign{\hskip\parindent#\hfil&\qquad(#)\hfil\cr
     \tts\arabic/&the usual, {\it e.g.,} ``\arabic{16}''\cr
     \tts\roman/&lower case roman numerals, ``\roman{16}''\cr
     \tts\Roman/&upper case roman numerals, ``\Roman{16}''\cr
     \tts\alphabetic/&lower case letters, ``\alphabetic{16}''\cr
     \tts\Alphabetic/&upper case letters, ``\Alphabetic{16}''\cr
     \tts\symbols/&footnote symbols, see below\cr}

\smallskip

\noindent These styles may be applied to \jyTeX's counters as well.  For
example, the page number is stored in a register called \tts\pagenum/.  If 
you'd like to print the page number in roman numerals, and 
\tts\pagenum/ has the value 3, the command \tts\roman\pagenum/
produces ``\thinspace\roman3.''

This is all well and good, but the macros that output the page number and
other counters are already written, so how do you change the output style
of these counters without delving into the code?  No problem: if you want
the page numbers to appear as roman numerals, for example, you just put at
the beginning of your document

\nobreak\smallskip

\tts\pagenumstyle{roman}/

\smallskip

\noindent This will change the style of \tts\pagenum/ from its default,
which is \tts arabic/, to \tts roman/. Note that {\it all\/} \jyTeX{}
counters have an associated \tts style/ command; a quick glance at this
manual reveals that it was typeset with \tts\chapternumstyle{Roman}/ and
\tts\sectionnumstyle{arabic}/.

The \tts\...style/ commands also accept the argument \tts blank/, which 
suppresses the output of the relevant counter. The \tts\subsectionnum/ 
counter in this manual has been \tts blank/ed out, which is why there are 
no numbers preceding the slanted subsection titles.  The default for 
all of the sectioning counters, in fact, is \tts blank/.

You now know enough to understand the rest of the story on the sectioning
commands. When you give a \tts\chapter/ command everything proceeds as
described above, except that when it's time to print the chapter title, if
the \tts\chapternumstyle/ is not \tts blank/, \jyTeX{} increments
\tts\chapternum/, sets all ``lesser'' counters ({\it i.e.,}
\tts\sectionnum/, \tts\subsectionnum/, and a couple to be named later) to
zero, prints the \tts\chapternum/ as specified by the macro
\tts\chapternumformat/ (another user-modifiable macro which just prints the
counter in the current style followed by a period), then prints the chapter
title. If the \tts\chapternumstyle/ is \tts blank/, \tts\chapternum/ is not
incremented, no counters are set to zero, and the title appears without a
chapter number.

Similarly, if \tts\sectionnumstyle/ is not \tts blank/, the \tts\section/
command increments \tts\sectionnum/, sets lesser counters to zero, prints
the \tts\sectionnum/ as specified by the macro \tts\sectionnumformat/
(which prints the chapter number if it isn't blank, followed by the section
number), then prints the section title. No number is printed if
\tts\sectionnumstyle/ is \tts blank/. You can figure out how the 
subsection counter works.

To end this discussion let me mention a few tricks, which are especially
useful if you are thinking of playing around with any of the macros. Any
counter can be set to a desired value via a simple \TeX{} assignment
command, {\it e.g.,} \tts\chapternum=6/. The sectioning counters also
permit the constructions \tts\newchapternum=6/ (which sets the specified
counter) and \tts\newchapternum=\next/ (which increments the counter); both
of these also reset lesser counters to zero. Finally, to print any counter
in its current style you can simply precede it with \tts\numstyle/, as in
\tts\numstyle\pagenum/. Doing things this way makes the printing of the
counter sensitive to the value of counter style (go back and look at the
definition of \tts\foot/ given above for an example.)


{\bigfonts\bfs\section{Arranging Text}}

Now it's time for the macros that help you with actual text.  Some of these
make use of a block structure set up in \jyTeX, so this is described first.

{\sl\subsection{Block structure}}

It is sometimes desirable to set off part of a file for visual or logical
reasons.  This can be done in an uninteresting and boring way by using comment
lines, or in a clever and exciting way using \tts\begin{(|it
name|/)}/\~\dots\~\tts\end{(|it name|/)}/ blocks.  To typeset the title page of
a paper, for example, you could type

\nobreak\smallskip

\tts\begin{titlepage}/ \par
{\it (Stuff)} \par
\tts\end{titlepage}/ \par

\smallskip

\noindent The name given to the block can be anything you want.  Blocks may be
nested but they cannot overlap; in other words,
\tts\begin{a}/\~\dots\~\tts\begin{b}/\~\dots\~%
\tts\end{b}/\~\dots\~\tts\end{a}/ is allowed, but
\tts\begin{a}/\~\dots\~\tts\begin{b}/\~\dots\~%
\tts\end{a}/\~\dots\~\tts\end{b}/ is not.

Blocking things in this way has some advantages. The \tts\begin/ statement
opens a group that is closed by \tts\end/, so parameters can be changed inside
the block without affecting the goings-on outside.  For example, if you were to
suppress paragraph indentation in the case above by giving a
\tts\parindent=0pt/ command after \tts\begin{titlepage}/, the normal
indentation would be automatically restored after \tts\end{titlepage}/.

You're probably thinking that this isn't too exciting (braces accomplish the
same thing), but there is another feature: when processing \tts\begin{(|it
name|/)}/, \jyTeX{} executes the command \tts\begin(|it name|/)/ (if it exists)
immediately after opening the group. If you understand \TeX{} macro definitions
this allows you to create blocks with special conditions set.  Going back to
the \tts titlepage/ example, suppose you want to set \tts\pagenumstyle{blank}/
and \tts\parindent=0pt/ after \tts\begin{titlepage}/. You can do this by
defining

\nobreak\smallskip

\tts\def\begintitlepage{\pagenumstyle{blank}\parindent=0pt}/

\smallskip

\noindent This is automatically invoked when the \tts titlepage/ block opens.
Similarly, if there is an \tts\endtitlepage/ command defined, it is executed
when \tts\end{titlepage}/ is processed, just before closing the group.  For
this example an appropriate definition might be

\nobreak\smallskip

\tts\def\endtitlepage{\newpage}/

\smallskip

\noindent (\tts\newpage/ does the obvious thing, and is defined below.) In this
way you can define blocks tailored to your own needs and keep them tucked away
in a personal definitions file.

The astute reader will recognize that the \tts\end{(|it name|/)}/ command
supercedes the plain \TeX{} \tts\end/ command.  Not to worry. \jyTeX{} has it's
own command that does the same thing and more, described below.  A more serious
problem is that \TeX{} has two commands that start with the word \tts begin/:
\tts\beginsection/ and \tts\begingroup/.  If you name a block either \tts
section/ or \tts group/, \jyTeX{} will attempt to run these, probably with
unintended results.

{\sl\subsection{Line and page breaking: {\tt\string\\}
     and\/ {\tt\string\newpage}}}

\jyTeX{} provides the \tts\\/ command to force a line break\\
(like that one, filled at the right with blank space), and the \tts\newpage/
command to force a page break (not demonstrated, but filled at the bottom with
blank space). No, \tts\\/ does not work in math mode.

{\sl\subsection{Left-justifying, right-justifying, and centering}}

The commands \tts\lefttext/, \tts\righttext/, and\/ \tts\centertext/ are
generalizations of \TeX's \tts\leftline/, \tts\rightline/, and \tts\centerline/
that allow several lines of text to be positioned either

\lefttext{on the left,}
\righttext{on the right,}
\centertext{or in the center.}

\noindent They take a single argument, which may be several lines of text
separated by \tts\\/.  For example, \tts\righttext{There once was a
man\\ from Nantucket,}/ produces

\nobreak\smallskip

\righttext{There once was a man\\ from Nantucket,}

\smallskip

\noindent (There should be no \tts\\/ after the last line.)  These commands are
most useful for short applications such as address labels or section headings.

If you need to justify several paragraphs of text or if you need greater
control over the output (you may need to add vertical spacing between certain
lines, for example), the commands above may fail.  The blocks
\tts\begin{left}/\~\dots\~\tts\end{left}/,
\tts\begin{right}/\~\dots\~\tts\end{right}/, and
\tts\begin{center}/\~\dots\~\tts\end{center}/ will handle these more difficult
cases.  Generally you should find that \tts\lefttext/, \tts\righttext/, and\/
\tts\centertext/ do everything you need, but if some weird problem arises these
other commands are available.  The format is what you expect:

\nobreak\smallskip

\tts\begin{center}/ \par
\tts He said with a grin,\\/ \par
\tts as he wiped off his chin,/ \par
\tts\end{center}/ \par

\nobreak\smallskip

\noindent produces

\smallskip

\begin{center}
He said with a grin,\\
as he wiped off his chin,
\end{center}

{\sl\subsection{The\/
     {\tt\string\begin{narrow}}\~\dots\~{\tt\string\end{narrow}} block}}

\TeX{} already has a \tts\narrower/ command that moves the left and right
margins in by the amount of the usual paragraph indentation. The \jyTeX{}
version is similar, but is constructed with a \tts\begin/\~\dots\~\tts\end/
block and can take an optional argument \tts[/in square braces\tts]/ specifying
the amount the margins are to be moved.

\begin{narrow}

For example, this paragraph was produced by typing \tts\begin{narrow} For
example,/\~\dots\~\tts\end{narrow}/. Notice that the margins have moved in by
an amount equal to an indentation.

\end{narrow}

\begin{narrow}[.80in]

This paragraph also uses \tts\begin{narrow}/, but with a specified shift of .80
inch. The actual sequence of commands is \tts\begin{narrow}[.80in] |hskip0pt
plus6pt This paragraph/\~\dots\~\tts\end{narrow}/. The amount of the shift can
be specified in any units that \TeX{} understands and should appear in square
braces immediately following \tts\begin{narrow}/.

\end{narrow}

{\sl\subsection{Skipping text}}

It may occasionally happen that you'd like to \jyTeX{} only part of a file (say
you're working on one section and, to save time, you want to prevent \TeX{}
from processing the other sections).  This is the {\it raison d'\^etre\/} of
the \tts\begin{ignore}/\~\dots\~\tts\end{ignore}/ block.  Absolutely everything
sitting between these two commands is skipped over by \jyTeX{} like it wasn't
even there.


{\bigfonts\bfs\section{Footnotes}}

{\sl\subsection{The\/ {\tt\string\footnote} command}}

The \tts\footnote/ command in \jyTeX{} is similar to the one in \TeX, so
you should go read about that one first. The main improvement over \TeX{}
is the addition of a counter to number the footnotes automatically. This
has the nice side-effect that it makes the form of the command a little
simpler. Just say

\nobreak\smallskip

\tts\footnote{(|it text of footnote|/)}/

\smallskip

\noindent wherever you want the footnote mark to be\footnote{See how easy?}
and a footnote appears at the bottom of the page.

The counter \tts\footnotenum/ is used to generate marks sequentially. Just
as the \tts\pagenumstyle/ command determines how the counter \tts\pagenum/
is output, so the \tts\footnotenumstyle/ command determines the appearance
of \tts\footnotenum/. Probably the most useful style (the default, in fact)
is \tts\footnotenumstyle{symbols}/, which marks the first footnote with a
\symbols1, followed by \symbols2, \symbols3, \symbols4, \symbols5,
\symbols6, \symbols7, \symbols8, \symbols9, and \symbols{10}.

There is a variation of \jyTeX's \tts\footnote/ you can use if you really
want to specify the footnote mark yourself (like the original \TeX{}
command).  If you put

\nobreak\smallskip

\tts\footnote[(|it mark|/)]{(|it text of footnote|/)}/

\smallskip

\noindent\jyTeX{} takes whatever appears between the square braces and uses it
as the footnote mark,\footnote[Hello]{\dots even something as ridiculous as the
word ``Hello.''} even something as ridiculous as the word ``Hello.'' Use this
form if you want to imitate \TeX's \tts\footnote/ command.

If for some reason \TeX{} won't let you put a footnote somewhere (like
inside another footnote, for example), it is possible to split up the
\tts\footnote/ command into two pieces: \tts\footnotemark/ (or
\tts\footnotemark[(|it mark|/)]/), which positions the mark, and
\tts\footnotetext{(|it text of footnote|/)}/, which puts the footnote at
the bottom of the page. The idea is that you put the \tts\footnotemark/
command where the \tts\footnote/ command can't go, and put the
\tts\footnotetext/ command as soon afterward as possible.

One last little goodie: the spacing between footnotes at the bottom of the
page can be adjusted by varying the value of \tts\footnoteskip/. It is
initialized to \tts0pt/.


{\bigfonts\bfs\section{Labels}}

\jyTeX{} has a set of macros that allow a character string to be assigned a
{\it label,} which may then be invoked elsewhere as a way to print out the
string.  If the character string is changed by some modification to the
document, you need only change the definition of the label and all references
to the label will substitute the new string.  This scheme is the basis of the
equation and reference numbering routines described below.

{\sl\subsection{Creating and using labels}}

Creating a label is easy: just say \tts\label{(|it name)}{(|it character
string)}/.  To then print out the character string, say \tts\putlab{(|it
name)}/.  The label name can include numbers and most symbols, but should {\it
not\/} contain spaces or commas.  As a simple example, suppose you are
constantly referring to Figure~3 of the document you are \jyTeX ing, but you're
afraid that before you're through it might become Figure~4 because of some
modification, and it would be a pain to have to go through the document
changing all references to it.  Simply make the definition
\tts\label{lab1}{Figure~3}/\label{lab1}{Figure 3}, and refer to the figure by
saying, ``\dots\tts as we see from \putlab{lab1}/,'' which gives ``\dots as we
see from \putlab{lab1}.'' If Figure~3 becomes Figure~4, you need only change
the label definition (\tts\label{lab1}{Figure~4}/) instead of the references.
(By the way, \jyTeX{} will issue the warning

\smallskip

\tts--> Label (|it name|/) redefined/

\smallskip

\noindent if a label is defined more than once, since this could generate
problems if done unintentionally.)

The sections below describe applications of this facility.


{\bigfonts\bfs\section{Automatic Equation Numbering}}

\jyTeX{} has a set of commands built out of the labelling macros that will
automatically number equations in a document.  They allow you to name and refer
to an equation via a label, which is translated into a number (automatically
generated by \jyTeX) wherever it appears.  You need never type an equation
number again.

{\sl\subsection{The\/ {\tt\string\eqnlabel}
     and {\tt\string\puteqn} commands}}

To have \jyTeX{} generate an equation number automatically, all you have to do
is type \tts\eqnlabel{(|it name|/)}/ where you would normally put the number.
The {\it name\/} is the label to which you will refer instead of the number.
For example, to get
$$\rm strings = everything,\eqno(1)$$
type

\nobreak\smallskip

\tts$$\rm strings = everything,\eqno\eqnlabel{string}$$/

\smallskip

\noindent The equation number advances each time \tts\eqnlabel/ is invoked, so
if the above equation is followed by

\nobreak\smallskip

\tts$$\rm physics = chemistry,\eqno\eqnlabel{phys}$$/

\smallskip

\noindent it appears as
$$\rm physics = chemistry,\eqno(2)$$
These equations can now be referred to by invoking their {\it labels\/} instead
of their {\it numbers}. This is done with a customized version of the
\tts\putlab/ command called \tts\puteqn/. So to get, ``Equations (1) and (2)
are silly,'' type

\nobreak\smallskip

\tts Equations |(\puteqn{string}|) and |(\puteqn{phys}|) are silly./

\smallskip

\noindent Notice that \tts\puteqn/ does not put parentheses around the number
it returns. This is so you can make constructions like ``\dots equations
(1,2)'' by typing ``\dots\tts equations |(\puteqn{string},\puteqn{phys}|)/.''

{\sl\subsection{Generating the equation number}}

In general, equation numbers are generated from five counters: the three
sectioning counters (\tts\chapternum/, \tts\sectionnum/, and
\tts\subsectionnum/), a counter for the equation itself (\tts\equationnum/),
and one more for further numbering just in case these aren't enough
(\tts\subequationnum/). As usual, these two new counters have their associated
\tts style/ commands. The equation numbers for \tts string/ and \tts phys/
above were generated under the assumption that all of these styles were
\tts blank/ except \tts\equationnumstyle/.  This is the default.  If
none of the counter styles were \tts blank/, a typical equation might come out
like this:
$$F=ma\eqno\hbox{(II.1.B.6c)}$$
Notice that all counters are separated by dots except the last two,
\tts\equationnum/ and \tts\subequationnum/.

As the compound form of the label suggests, the equation and subequation
counters are reset whenever the sectioning counters are advanced.  This means,
for example, that if you have ``turned on'' the section counter ({\it i.e.,}
made it non-\tts blank/), your equations will automatically be numbered by
section.

The \tts\eqnlabel/ command takes care of incrementing the equation number.
As long as \tts\subequationnumstyle/ is \tts blank/, the \tts\equationnum/
counter is advanced each time \tts\eqnlabel/ is used, otherwise
\tts\subequationnum/ is advanced. This makes it easy to produce lettered
equations, like

\nobreak\smallskip

\subequationnumstyle{alphabetic}
$$\eqalignno{I &= he,&\rm(3a)\cr
     you &= he,&\rm(3b)\cr
     you &= me,&\rm(3c)\cr
     we  &= all\;together&\rm(3d)\cr}$$
\subequationnumstyle{blank}

\smallskip

\noindent just by saying

\nobreak\smallskip

\tts\subequationnumstyle{alphabetic}/\par
\tts$$\eqalignno{I &= he, & \eqnlabel{walrus-a} \cr/\par
\hskip\parindent\tts you &= he, & \eqnlabel{walrus-b} \cr/\par
\hskip\parindent\tts you &= me, & \eqnlabel{walrus-c} \cr/\par
\hskip\parindent
     \tts we  &= all\;together & \eqnlabel{walrus-d} \cr}$$/\par
\tts\subequationnumstyle{blank}/\par

\smallskip

\noindent All you have to do is ``turn on'' the \tts\subequationnum/ counter at
the start of the set of equations (by setting
\tts\subequationnumstyle{alphabetic}/, for example), and turn it off at the end
(by setting it \tts blank/).  This last command gets \tts\equationnum/ counting
again.

There is a slightly better way to do the same thing, sort of a shorthand except
that it doesn't save much typing. The first \tts\subequationnumstyle/ command
above can be replaced by \tts\begin{eqnseries}/ and the second by
\tts\end{eqnseries}/. Inside this block the style of \tts\subequationnum/ is
set to \tts\eqnseriesstyle/, which is initially \tts alphabetic/ but can be
changed in the usual way ({\it e.g.,} \tts\eqnseriesstyle{roman}/). If that
were all there wouldn't be much point, but the \tts eqnseries/ block has one
other feature: you may specify an optional label name in square braces
immediately following the \tts\begin{eqnseries}/ command, to which will be
assigned the equation number as it would appear without the attached
\tts\subequationnum/. An example will make this clear. Consider the same
equation as above, now using the \tts eqnseries/ block:

\nobreak\smallskip

\tts\begin{eqnseries}[walrus]/\par
\tts$$\eqalignno{I &= he, & \eqnlabel{walrus-a} \cr/\par
\hskip\parindent\tts you &= he, & \eqnlabel{walrus-b} \cr/\par
\hskip\parindent\tts you &= me, & \eqnlabel{walrus-c} \cr/\par
\hskip\parindent
     \tts we  &= all\;together & \eqnlabel{walrus-d} \cr}$$/\par
\tts\end{eqnseries}/\par

\smallskip

\noindent As in the example without \tts eqnseries/, the equation labels \tts
walrus-a/, \tts walrus-b/, \tts walrus-c/, and \tts walrus-d/ are assigned
numbers like ``3a'', ``3b'', ``3c'', and ``3d.''  But here a label \tts walrus/
is also generated, and assigned the number ``3.''  This can be used in
references like ``\dots\tts equations |(\puteqn{walrus}b-d|)/, which produces
``\dots equations (3b-d).''

If you don't like the way the compound equation number looks, you can change it
to suit your tastes by modifying \tts\puteqnformat/.  This is defined in a
fairly straightforward manner in the \tts Initialization/ section of \tts
jymacros.tex/.

{\sl\subsection{{\tt\string\equation}, {\tt\string\Equation},
     and\/ {\tt\string\putequation}}}

These commands are for those of you that like to put all your equations in a
separate file, then call them into text when appropriate (by far the best
method). The command

\nobreak\smallskip

\tts\equation{energy}{E=mc^2}/

\smallskip

\noindent does two things: first, it assigns the name \tts energy/ (the first
argument) to the equation \tts E=mc^2/ (the second argument), and second, it
appends an equation number generated by \tts\eqnlabel/ (it actually uses the
\tts\eqno/ construction from \TeX). The {\it equation\/} can be inserted into
the document by using the \tts\putequation/ command; typing

\nobreak\smallskip

\tts$$\putequation{energy}$$/

\smallskip

\noindent produces
$$E = mc^2\eqno(4)$$
The {\it label\/} can be referred to by typing \tts\puteqn{energy}/, as in
``\dots\tts the famous equation \puteqn{energy},/'' which gives ``\dots the
famous equation (4).''  Don't be confused by the fact that the {\it
equation\/} and the {\it label\/} have the same name (\jyTeX{} will figure out
what you mean from context).

In the event that you don't want an equation to have an equation number, or you
want to specify it yourself, or you're using an equation format that doesn't
permit the use of \tts\eqno/ (like \tts\eqalignno/), you can't use
\tts\equation/.  For these cases \jyTeX{} supplies the \tts\Equation/ command
instead.  \tts\Equation/ behaves like \tts\equation/ but doesn't put in an
\tts\eqnlabel/; you must put one in yourself, if you want one.
\tts\putequation/ works the same for equations defined by \tts\equation/ or
\tts\Equation/.

As mentioned above, these commands are useful if you like to put your equations
in a separate file.  There is a disadvantage to this: equations input in this
way must be stored before use, and if you have too many you could run out of
memory.  There are advantages, however.  One is that you can run spelling
checkers on the text part of your job without getting bogged down by ``words''
in the equations.  Another is that you can redefine the \tts\equation/ and
\tts\Equation/ commands to do operations on just the equations.  For example,
suppose all your equations are in a file called \tts eqns.tex/ and you want to
print them out.  Just create another file that contains the following commands:

\smallskip

\tts\typesize=12pt/ \par
\tts\draft/ \par
\tts\def\equation#1#2{$$#2\eqno\eqnlabel{#1}$$}/ \par
\tts\def\Equation#1#2{$$#2$$}/ \par
\tts\input eqns/ \par
\tts\bye/ \par

\smallskip

\noindent (The \tts\draft/ command will be explained later, but what is does
here is cause the equation labels to appear in the right margin.  This makes
a handy reference.)

{\sl\subsection{Figures and tables}}

There is currently no provision in \jyTeX{} for creating or positioning figures
and tables, but there are counters in place for you to use if you like.  For
figures there are registers \tts\figurenum/ and \tts\subfigurenum/ with
corresponding \tts style/ commands, \tts\figlabel/ and \tts\putfig/, a \tts
figseries/ block, and a \tts\figlabelformat/ macro.  All of these work in the
same way as their equation counterparts.  Similarly, for tables you have
\tts\tablenum/ and \tts\subtablenum/ and their \tts style/ commands,
\tts\tbllabel/ and \tts\puttbl/, a \tts tblseries/ block, and
\tts\tbllabelformat/.  Like \tts\equationnum/ and \tts\subequationnum/, the
figure and table counters are reset when the sectioning counters are
incremented.


{\bigfonts\bfs\section{Automatic Reference Numbering}}

Below are a set of commands to handle the numbering of references, which are
assumed to appear in a list at the end of the document.  Like the equation
numbering routine, you assign a label to each reference instead of a number and
\jyTeX{} takes care of the rest.  Two referencing schemes are supported:  in
one, references are numbered in the order they are cited; in the other, the
numbering is determined by the order of the final list.  In either case, only
the references that are actually cited are printed.

{\sl\subsection{The\/ {\tt\string\referencestyle} command}}

\tts\referencestyle/ (not to be confused with \tts\referencenumstyle/!) selects
one of the two reference numbering schemes: \tts sequential/, which numbers the
references in the order they are {\it cited\/} and sees to it that they are
output in that order, and \tts preordered/, which numbers the references in the
order they are {\it listed\/} at the end of the document.
\tts\referencestyle{sequential}/ is the default.

A little reflection on the dynamics of \tts preordered/ referencing reveals
that it requires two passes through the document to implement.  You should make
sure you understand the pitfalls of forward-referenced labels, explained in the
Advanced Topics chapter, before using this style.

{\sl\subsection{{\tt\string\reference} and the\/
     {\tt\string\begin{putreferences}}\dots{\tt\string\end{putreferences}}
     block}}

The command \tts\reference/ is used to define a reference and assign to it a
label. A typical example might be:

\smallskip

\tts\reference{ref1}{J. Bloggs, SAT Rev.| B21:74 1986}/

\smallskip

\noindent The first argument (\tts ref1/) is the reference label, the second
argument is the reference itself.  If you want to list more than one paper in a
single reference, separate the entries by \tts\\/:

\nobreak\smallskip

\tts\reference{ref2}{M. Golden, J. Blog.| Phys.| 106:44 1985\\|\
     |null|kern3|parindent
     O. Cheyette, {\it Getting paid to draw pretty pictures|\
     |null|kern3|parindent
     on my terminal,} to appear in Comm.| Time Wast.}/

\smallskip

All \tts\reference/ commands should be listed inside a
\tts\begin{putreferences}/\~\dots\~\tts\end{putreferences}/ block at the place
where you want the references to print.  For the references above, the end of
your document might therefore look like this:

\nobreak\smallskip

\tts{\bf\section{References}}/ \par\nobreak
\tts\begin{putreferences}/ \par
\tts\reference{ref1}{J. Bloggs, SAT Rev.| B21:74 1986}/ \par
\tts\reference{ref2}{M. Golden, J. Blog.| Phys.| 106:44 1985\\|\
     |null|kern3|parindent
     O. Cheyette, {\it Getting paid to draw pretty pictures|\
     |null|kern3|parindent
     on my terminal,} to appear in Comm.| Time Wast.}/ \par
\tts\end{putreferences}/ \par

\smallskip

\noindent How these references are printed depends on the \tts\referencestyle/.
If you're using the \tts sequential/ style, the references will print in the
order they are cited. If you're using the \tts preordered/ style then the order
of citing doesn't matter, and the above commands will produce

\begin{block}

\smallskip

\lefttext{\bf References}

\putreferenceformat
\setbox0=\hbox{1.\ \ }\leftskip=\wd0

\leavevmode\llap{1.\ \ }J. Bloggs, SAT Rev.\ B21:74 1986

\leavevmode\llap{2.\ \ }M. Golden, J. Blog.\ Phys.\ 106:44 1985\\
     O. Cheyette, {\it Getting paid to draw pretty pictures on my terminal,}
     to appear in Comm.\ Time Wast.

\smallskip

\end{block}

\noindent There are a couple of things to notice about how the references are
formatted.  The number prefacing each reference is the \tts\referencenum/,
another counter just like all the others.  It has its associated \tts style/
command, here set to \tts arabic/ (the default).  Its appearance in square
braces is a consequence of the macro \tts\referencenumformat/, which can be
found in the \tts Initialization/ of \tts jymacros.tex/ and is easily changed
if it's not to your liking.  The ``paragraphing'' of the reference itself (how
\tts\\/ is handled, how lines are indented) is controlled by
\tts\putreferenceformat/, also in the \tts Initialization/ section.  Space
between references can be controlled with \TeX's \tts\parskip/.

One final word concerning the output of references: {\it only references that
are cited somewhere in the document are actually printed.} (This means that in
the example above it has been assumed that both references were cited.)  Thus,
if your document only needs a handful of references out of a larger database,
you can list the entire database in the \tts putreferences/ block (or better
yet, call it in with an \tts\input/ command).  The extraneous \tts\reference/'s
will not appear in the output.

{\sl\subsection{Citing a reference with\/ {\tt\string\putref}}}

Citing a reference should be like referring to an equation---you should specify
the reference {\it label\/} which is then turned into a {\it number} by \jyTeX.
This is the job tackled by \tts\putref/, the referencing scheme's variant of
\tts\putlab/.

In the simplest case, \tts\putref/ allows you to refer to the work of Dr.\
Bloggs above by saying ``\dots\tts thus the result of reference [\putref{ref1}]
must be wrong/,'' and have it come out ``\dots thus the result of reference~[1]
must be wrong.''  But it is frequently desirable to cite several sources at
once, so \tts\putref/ takes multiple arguments.  For example, if the references
\tts yin/ and \tts yang/ are assigned the numbers 6 and~9, the command
\tts\putref{yin,yang}/ produces ``6,9''. Furthermore, if \tts\putref/ is given
three or more arguments, it looks for sequences and outputs them appropriately.
Specifically, if \tts larry/, \tts moe/, and \tts curly/ correspond to
references 1, 2, and~3, the command \tts\putref{larry,moe,curly}/ yields
``1--3'', {\it not\/} ``1,2,3''.  As you would expect, \tts\putref/ puts out
numbers in the form demanded by \tts\referencenumstyle/ (here \tts
arabic/)---it does not look at \tts\referencenumformat/, which only affects the
appearance of the reference number in the final reference list.

If you're the kind that likes to make citations raised and in smaller type like
this\markup{[1]} (that's an example, not a footnote), you can use the
\tts\markup/ command in conjunction with a \tts\putref/.  \tts\markup/ takes
whatever follows and puts it up, so that last example could have been done
``\dots\tts like this\markup{[\putref{ref1}]}/.''


{\bigfonts\bfs\section{Remaining Bits}}

{\sl\subsection{Underlining and overlining}}

This is a fast one.  In \TeX{} the commands \tts\underline/ and \tts\overline/
work only in math mode.  In \jyTeX{} they work in normal text as well.

{\sl\subsection{The\/ {\tt\string\draft} command}}

This is a very good command to put in your \TeX{} file if you're not printing
out the final copy of the document.  In \tts\draft/ mode, the word ``{\sl
DRAFT\/}'' appears in the headline along with the date and time, overfull boxes
are marked with a black bar, and the names assigned to the equations and
references (if you're using the automatic numbering routines) are written in
the margin next to the things they label.  This makes it easy to look up the
names when you want to refer to these items.

\tts\draft/ mode also shifts the text over to accommodate notes in the right
margin.\docmarginnote{Are\\ you\\ following\\ this?}  You just have to say
\tts\marginnote{(|it stuff|/)}/ somewhere in the text of a paragraph (it's not
allowed between paragraphs) and the {\it stuff\/} will migrate out to the
margin and appear in very small print. \jyTeX{} is not very sophisticated in
how it handles this, so if your notes are very long or very frequent you may
get weird output.  A \tts\marginnote/ command is ignored unless you're in
\tts\draft/ mode.  (The one above was doctored to make it come out anyway.)

\tts\draft/'s monkeying with the margins makes for a couple of restrictions.
First, you can't use it with \tts\outputstyle{twoup}/, because the margins
aren't big enough.  Second, since \tts\draft/ needs to have the last word on
margin settings, it should be one of the last commands to appear before the
start of the text.

{\let\bye=\relax
\sl\subsection{Signing off: saying good\/{\tt\string\bye}}}

The correct way to end a \jyTeX{} file is with the command \tts\bye/.  This is
a slight change from \TeX, which allows both \tts\end/ and \tts\bye/. As
mentioned above, \tts\end/ has been superceded.  (If you {\it really\/} need
the \TeX{} version of \tts\end/, you can get it with \tts\TeXend/.)


%********** Next chapter **********

{\bigfonts\bfs\chapter{ADVANCED TOPICS}}

{\bigfonts\bfs\section{Label Tricks}}

{\sl\subsection{Forward-referencing}}

The label routines are so sophisticated that they actually allow you to refer
to a label before it's been defined.  The basic idea is this: during the first
pass through a document the values of all such forward-referenced labels are
stored in a file, which is then automatically read in during subsequent passes
to supply the label values.  The user doesn't have to do anything differently,
but some care must be taken as there are all kinds of warnings and errors that
can be generated.

Here are the details.  When \jyTeX{} sees a \tts\putlab/ for a label that has
yet to be defined, it immediately prints the warning

\smallskip

\tts--> Label (|it name|/) referenced before its definition/

\smallskip

\noindent If it hasn't occurred already, \jyTeX{} also opens a file in which to
save the as-yet-undefined label; the name of this file is the same as the file
being \jyTeX ed, but with the extension \tts.lab/. When this happens, you will
see the message

\smallskip

\tts--> Creating file (|it file).lab/

\smallskip

\noindent The values of the needed labels are then written to this file as they
are defined.  If any are missing ({\it i.e.,} if you forget to define a label
you use), you get the warning

\smallskip

\tts--> Label (|it name|/) referenced but never defined/

\smallskip

\noindent Once you succeed in running without generating this error, you will
have a \tts.lab/ file that contains all the values of all the
forward-referenced labels.

Now comes the fun part.  When you run the file again, the first thing \jyTeX{}
does is to go get the \tts.lab/ file:

\smallskip

\tts--> Getting labels from file (|it file).lab/

\smallskip

\noindent You should not see any warning messages about forward-referencing,
because the label values are supplied from the \tts.lab/ file.  However, if
you've changed any labels between the first run and the second, you may get the
warning

\smallskip

\tts--> Definition of label (|it name|/) doesn't match value in (|it file).lab/

\smallskip

\noindent This is telling you that the label value pulled from the \tts.lab/
file and the value assigned by the subsequent \tts\label/ command don't match.
The solution is to \jyTeX{} the file yet again, until you get no more warnings.
You can then be sure that all forward-referenced labels are supplied with the
correct character strings.

Any label that is likely to be forward-referenced should be fairly short. This
is because the label will have to be written to a file, and regardless of its
length \TeX{} will output it as a single line. On some systems a line of more
than a couple of hundred characters creates problems.

{\sl\subsection{\TeX{} commands in labels}}

You may wonder whether you can put \TeX{} commands in a label instead of a
character string.  The answer is yes (in fact the reference numbering routine
does this), but it's tricky.  The \tts\label/ macro creates an {\it expanded\/}
definition, so any commands appearing in the label string will be expanded
until \TeX{} can't expand anymore. If you want to play around with this, check
out the section of the \TeX book on expanded definitions (Chapter~20).


\begin{ignore}


{\bigfonts\bfs\section{More About Referencing}}

{\sl\subsection{The {\tt\string\begin{prereferences}}\dots
     {\tt\string\end{prereferences}} block}}

Any \tts preordered/ reference scheme that only prints cited references
necessarily requires two passes through the document.  However, if you don't
care whether printed references are actually cited, or if you're willing to
check this for yourself, \jyTeX{} provides a way to do \tts preordered/
referencing in one pass.  What you do is put the \tts\reference/ commands at
the {\it beginning\/} of the text (before any citations) in a
\tts\begin{preferences}/\~\dots\~\tts\end{prereferences}/ block, instead of at
the end in the \tts{putreferences}/ block.  A typical application might
therefore look like this:

\smallskip

\tts\typesize=12pt/ \par
\tts\referencestyle{preordered}/ \par
\tts\begin{prereferences}/ \par
\tts\reference{ref1}{J. Bloggs, SAT Rev.| B21:74 1986}/ \par
\tts\reference{ref2}{M. Golden, J. Blog.| Phys.| 106:44 1985\\|\
     |null|kern3|parindent
     O. Cheyette, {\it Getting paid to draw pretty pictures|\
     |null|kern3|parindent
     on my terminal,} to appear in Comm.| Time Wast.}/ \par
\tts\end{prereferences}/ \par
\smallskip
{\it (Stuff: text, reference citations, etc.)}
\smallskip
\tts\centertext{\bfs References}/ \par
\tts\begin{putreferences}/ \par
\tts\end{putreferences}/ \par
\tts\bye/

\smallskip

\noindent In this example, all the references are assigned numbers by the
\tts{prereferences}/ block {\it before\/} any citations have been made.
Therefore citations do not generate forward-referenced labels and the
referencing can be done in one pass.  Entering the \tts{putreferences}/ block
causes all prereferences to print, regardless of whether they were cited.
Notice that the \tts{putreferences}/ block is still required to print the
references even though none of them are listed there.

If you keep your references in a separate file, there is a way to get a fast
listing by using the \tts{prereferences}/ block.  The following \TeX{} file
will do the trick, assuming that the references are in \tts refs.tex/:

\smallskip

\tts\typesize=12pt/ \par
\tts\draft/ \par
\tts\begin{prereferences}/ \par
\tts\input refs/ \par
\tts\end{prereferences}/ \par
\tts\begin{putreferences}/ \par
\tts\end{putreferences}/ \par
\tts\bye/

\smallskip

\noindent The \tts\draft/ command puts the reference labels in the right
margin, so you might refer to this list when making citations.  Notice that a
\tts\referencestyle/ is not specified;  as long as all your references appear
in the \tts{prereferences}/ block there is no difference between the \tts
sequential/ and \tts preordered/ styles.

So far you've seen the \tts\reference/ command used in the \tts{prereferences}/
block and the \tts{putreferences}/ block.  There's one more possibility.  A
\tts\reference/ command can appear in normal text, where it will close the
current paragraph and print the reference, formatted in the usual way but
without a reference number.  This is for those instances when you don't want a
numbered list ({\it e.g.,} a list of ``suggested readings'').

\end{ignore}

{\bigfonts\bfs\section{Job Streams}}

The best way to typeset a long document (at least in the draft stage) is to
break it up into sections that can be \TeX ed separately.  A problem with this
approach, however, is that a label reference in one section to material in an
earlier section will be unrecognized.  For example, suppose you're typesetting
a book chapter-by-chapter, and you're using the automatic equation numbering
routine described above.  If in chapter two you try to refer to an equation in
chapter one you will get a complaint from \jyTeX{} that the equation label is
not defined. The problem is simply that the labels defined in the first chapter
are not available to the second unless the two chapters are typeset together.
This problem can be overcome by {\it job streaming.}  The trick is to have
chapter one produce a file containing label information that chapter two can
input for reference.

{\sl\subsection{The\/ {\tt\string\streamto} and\/
     {\tt\string\streamfrom} commands}}

If you give the command \tts\streamto/ at the beginning of a job, it causes all
labels (including all automatically generated equation numbers and references)
and all \jyTeX{} counter values to be dumped into a file.  The name of the file
is either generated from the name of the job ({\it jobname\/}\tts.str/ for the
straight \tts\streamto/ command), or can be specified by giving an optional
argument in square braces (\tts\streamto[(|it filename|/)]/ creates a file {\it
filename\/}\tts.str/). \jyTeX{} always supplies the extension \tts.str/.

This file can be retrieved by another job by giving a \tts\streamfrom/ command.
The format is the same as \tts\streamto/: \tts\streamfrom/ looks for a file
{\it jobname\/}\tts.str/, \tts\streamfrom[(|it filename|/)]/ looks for {\it
filename\/}\tts.str/.  In this way all of the labels and counter values are
made available to the new job.

For example, suppose chapter one of your book is in the file \tts chap1.tex/
and chapter two is in \tts chap2.tex/, and you want to make the labels and
counter values of chapter one available to chapter two. Simply put the command
\tts\streamto[chap2]/ at the beginning of \tts chap1.tex/ and \tts\streamfrom/
at the beginning of \tts chap2.tex/. Now when you \jyTeX{} \tts chap1.tex/ the
stream file \tts chap2.str/ is created, and when you \jyTeX{} \tts chap2.tex/
this file is automatically input by the \tts\streamfrom/ command. As long as
the stream file is around, chapter two will be able to refer to all the labels
in chapter one without generating errors. Of course, it is up to you to see
that the stream file has up-to-date information; in other words, if you make
changes to chapter one, you should re-\jyTeX{} it to create a new stream file
for chapter two.

If there is a \tts chap3.tex/ and you want to make available to it the labels
of chapter two, you can put a \tts\streamto[chap3]/ command in \tts chap2.tex/.
If this command is given {\it before\/} the \tts\streamfrom/ then the contents
of \tts chap2.str/ will be copied to \tts chap3.str/, so that chapter three
will also be able to refer to the labels of chapter one. Using this technique
you can arrange it so that later chapters always have all of the labels of
previous chapters available to them.

\goodbreak

{\sl\subsection{Writing to a stream file}}

To send information of your own out to the stream file just write to
\tts\streamout/ using \TeX's \tts\immediate\write/ command. If you introduce a
special counter, for example, the command

\nobreak\smallskip

\tts\immediate\write\streamout{\mycounter=\the\mycounter}/

\smallskip

\noindent will write it's current value to the stream file.  Obviously, you
should make sure you understand \TeX's \tts\write/ command before attempting
this.  The conditional \tts\ifstreaming/ is available to test whether or not
the stream file has been opened.


{\bigfonts\bfs\section{Rolling Your Own}}

{\sl\subsection{Getting extra fonts}}

There may come a time when the fonts provided by \jyTeX{} just aren't enough,
and you want to use the \TeX{} \tts\font/ command to go and get some of your
own.  Herein lies a problem.  The amount of space available for extra fonts is
limited because \jyTeX{} fills up virtually all of its memory with the various
sizes and styles described earlier.  Nevertheless, there is room for a few
additional fonts.

If you need more than a few new fonts you have a couple of alternatives.  One
is to not specify a \tts\typesize/.  This may seem pretty radical, but in fact
\jyTeX{} has been designed to behave very much like plain \TeX{} if the
\tts\typesize/ command is not given: the base size is \tts10pt/, fonts are
available in \tts7pt/ and \tts5pt/ for subscripts and sub-subscripts, and some
size-changing commands are disabled.  There is also a little more room for
extra fonts, maybe ten or twenty depending on their size.

If you want to input even more fonts you have to get basic and use plain \TeX.
This can be tricky and tedious.  You may find the file \tts jyfonts.tex/
helpful.

{\sl\subsection{Dates and times}}

The current date and time is made available to \TeX{} each time a job is run.
\TeX{} gives you access to this information through the following commands:

\nobreak\smallskip

\halign{\hskip\parindent\strut#\hfil&\qquad(#)\hfil\cr
     \tts\year/&current year\cr
     \tts\month/&number of the current month\cr
     \tts\day/&current day of the month\cr
     \tts\time/&current time in minutes since midnight\cr
     \noalign{\goodbreak\smallskip
          \noindent\jyTeX{} also provides\smallskip}%
     \tts\shortyear/&the last two digits of the current year\cr
     \tts\standardhour/&current hour in 12-hour format\cr
     \tts\militaryhour/&current hour in 24-hour format\cr
     \tts\amorpm/&either ``am'' or ``pm''\cr}

\smallskip

\noindent All of these are token registers, so their values are retrieved using
the command \tts\the/, {\it e.g.,} \tts\the\year/, \tts\the\amorpm/.
Additionally, there are the commands \tts\monthword/ and \tts\monthabbr/, which
take a single numerical argument and return the name of the corresponding month
(full name and three letter abbreviation, respectively).

{}From these pieces you should be able to format dates and times in the usual
ways.  \jyTeX{} provides a few of the more common formats: \tts\today/ (the
full month name, day, and year), \tts\standardtime/ (current hour and minute in
12-hour format), and \tts\militarytime/ (current hour and minute in 24-hour
format).  Retrieving these values does not require using \tts\the/.


{\sl\subsection{A couple of hacks: {\tt\string\everyoutput} and\/
     {\tt\string\everybye}}}

\TeX{} has registers that allow you to insert your own tokens or commands
whenever you enter certain modes.  For example, you can use the register
\tts\everypar/ to put something at the beginning of every paragraph. Continuing
in this tradition, \jyTeX{} adds the registers \tts\everyoutput/ (for tokens to
insert whenever \jyTeX{} starts to output a page) and \tts\everybye/ (for
tokens to insert whenever \jyTeX{} ends a job).

The job of \tts\everybye/ is to finish off the final page, and as a
consequence it is not initially empty, in contrast to other \tts\every.../
commands. Its default value is

\nobreak\smallskip

\tts\everybye={\par\vfil}/

\smallskip

\noindent which closes the final paragraph and adds some infinite glue to
fill the page. This is a very useful register if you want to modify the
closing routine, say to accommodate a special format.

\bye
% ---------------------- End of jydoc.tex ----------------

