%% BEGIN: pst-coil.tex
%% Generated on <1993/3/12> from `pst-coil.doc'.
%% For use with the PostScript header file `pst-coil.pro'.
%%
\def\fileversion{0.93a}
\def\filedate{93/03/12}
%%
%% For stroking and filling characters with PSTricks' line and fill styles.
%%
%% COPYRIGHT 1993, by Timothy Van Zandt, tvz@Princeton.EDU
%% See pstricks.doc or pstricks.tex for copying restrictions.
%%
%% See the PSTricks read-me file and the User's Guide for documentation.
\message{ v\fileversion, \filedate}
\csname PSTcoilsLoaded\endcsname
\let\PSTcoilsLoaded\endinput
\ifx\PSTricksLoaded\endinput\else
\def\next{\input pstricks.tex}\expandafter\next
\fi
\edef\TheAtCode{\the\catcode`\@}
\catcode`\@=11
\pstheader{pst-coil.pro}
\edef\pst@theheaders{\pst@theheaders,pst-coil.pro}
\def\pst@coildict{tx@CoilDict begin }
\def\psset@coilwidth#1{\pst@getlength{#1}\psk@coilwidth}
\psset@coilwidth{1cm}
\def\psset@coilheight#1{\pst@checknum{#1}\pscoilheight}
\psset@coilheight{1}
\def\psset@coilarmA#1{\pst@getlength{#1}\psk@coilarmA}
\def\psset@coilarmB#1{\pst@getlength{#1}\psk@coilarmB}
\def\psset@coilarm#1{%
\pst@getlength{#1}\psk@coilarmA
\let\psk@coilarmB\psk@coilarmA}
\psset@coilarm{.5cm}
\def\psset@coilaspect#1{\pst@getangle{#1}\psk@coilaspect}
\psset@coilaspect{45}
\def\psset@coilinc#1{\pst@getangle{#1}\psk@coilinc}
\psset@coilinc{10}
\def\pscoil{\def\pst@par{}\pst@object{pscoil}}
\def\pscoil@i{\pst@getarrows\pscoil@ii}
\def\pscoil@ii(#1){%
\@ifnextchar(%
{\pscoil@iii{1}(#1)}%
{\pscoil@iii{\z@}(0,0)(#1)}}
\def\pscoil@iii#1(#2)(#3){%
\begin@OpenObj
\pst@getcoor{#2}\pst@tempa
\pst@getcoor{#3}\pst@tempb
\pst@optcp{#1}\pst@tempa
\addto@pscode{%
\pst@tempa \pst@tempb
\psk@coilwidth \pscoilheight
\psk@coilarmA \psk@coilarmB
\psk@coilaspect \psk@coilinc
\pst@coildict \tx@Coil end}%
\showpointsfalse
\end@OpenObj}
\def\tx@CoilLoop{CoilLoop }
\def\tx@Coil{Coil }
\def\psCoil{\def\pst@par{}\pst@object{psCoil}}
\def\psCoil@i#1#2{%
\begin@AltOpenObj
\showpointsfalse
\pst@getangle{#1}\pst@tempa
\pst@getangle{#2}\pst@tempb
\addto@pscode{%
\pst@tempa
\pst@tempb
\psk@coilwidth
\pscoilheight
\psk@coilaspect
\psk@coilinc
\pst@coildict \tx@AltCoil end
\@nameuse{psls@\pslinestyle}}%
\end@OpenObj}
\def\tx@AltCoil{AltCoil }
\def\pszigzag{\def\pst@par{}\pst@object{pszigzag}}
\def\pszigzag@i{\pst@getarrows\pszigzag@ii}
\def\pszigzag@ii(#1){%
\@ifnextchar({\pszigzag@iii{1}(#1)}{\pszigzag@iii{\z@}(0,0)(#1)}}
\def\pszigzag@iii#1(#2)(#3){%
\begin@OpenObj
\pst@getcoor{#2}\pst@tempa
\pst@getcoor{#3}\pst@tempb
\pst@optcp{#1}\pst@tempa
\addto@pscode{%
\pst@tempa
\pst@tempb
\pscoilheight
\psk@coilwidth
\psk@coilarmA
\psk@coilarmB
\pst@coildict \tx@ZigZag end
\psline@iii
\tx@Line}%
\end@OpenObj}
\def\tx@ZigZag{ZigZag }
\def\nccoil{\def\pst@par{}\pst@object{nccoil}}
\def\nccoil@i{\check@arrow{\nccoil@ii}}
\def\nccoil@ii#1#2{\nc@object{#1}{#2}{.5}{%
\tx@NCCoor
tx@Dict begin
\psk@coilwidth \pscoilheight
\psk@coilarmA \psk@coilarmB
\psk@coilaspect \psk@coilinc
\pst@coildict \tx@Coil end
end}}
\def\pccoil{\def\pst@par{}\pst@object{pccoil}}
\def\pccoil@i{\pc@object\nccoil@ii}
\def\nczigzag{\def\pst@par{}\pst@object{nczigzag}}
\def\nczigzag@i{\check@arrow{\nczigzag@ii}}
\def\nczigzag@ii#1#2{\nc@object{#1}{#2}{.5}{%
\tx@NCCoor
tx@Dict begin
\pscoilheight
\psk@coilwidth
\psk@coilarmA
\psk@coilarmB
\pst@coildict \tx@ZigZag end
\psline@iii
\tx@Line
end}}
\def\pczigzag{\def\pst@par{}\pst@object{pczigzag}}
\def\pczigzag@i{\pc@object\nczigzag@ii}
\catcode`\@=\TheAtCode\relax
\endinput
%%
%% END: pst-coil.tex
