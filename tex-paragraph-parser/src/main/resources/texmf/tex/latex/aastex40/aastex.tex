% MANUAL.TEX -- User input guide for AAS\TeX\ markup package.
% Revised 4/23/94 - mostly reordering with some editing - J. Barnes
% Final editing started 8/8/94 - J. Barnes

\documentstyle[aas2pp4]{article}

\textwidth 7.3in
\hoffset=-0.4in  % This centers the extra wide page
\voffset=-0.4in  % This is because the printer I use prints too low...
\def\baselinestretch{0.96}

\hyphenation{com-pu-scripts}

\begin{document}
\twocolumn

\title{The AAS\TeX\ Macros for\\
       Manuscript Preparation}

\author{Chris Biemesderfer}
\author{Revised by Jeannette Barnes, May 1995}
\author{\it American Astronomical Society}
\authoraddr{NOAO, P.O. Box 26732, Tucson, AZ 85726 \\}

\vspace{.2in}

\section{Introduction}

The AAS has developed an author markup package to assist authors in
preparing manuscripts that
are intended for submission to the AAS-affiliated journals, or to other
journals that wish to use it, such as the PASP.
The most important aspect of the AAS\TeX\
package is that it defines the set of commands (called {\sl markup\/})
that can be used to identify the structural elements of papers.
When articles are marked up using this set of standard commands,
the papers may also be submitted electronically to the editorial
offices, with the ultimate goal being electronic generation of the journals
themselves.

This guide contains basic instructions for creating
manuscripts using the AAS\TeX\ markup package, which functions
as substyles to the standard \LaTeX\ {\tt article} style.
Authors are expected to be familiar with the editorial
requirements of the journals so that they can make
appropriate submissions, as well as to have at least
a rudimentary knowledge of \LaTeX\ (for instance, knowing
how to set up equations using \LaTeX\ commands).
A number of useful publications about \LaTeX\ (and \TeX) are listed in the
reference section of this guide.

{\bf Authors who wish to submit papers electronically to the ApJ, the ApJ
Letters, or the AJ {\sl must} use the AAS\TeX\ markup package as 
described in this guide.}  The package is an {\bf option} for PASP submissions
at this time.

%\section{Introduction}

\section{The AAS\TeX\ Package and Electronic Submissions}

Information about acquiring the AAS\TeX\ package will be mailed to you
electronically if you send an empty e-mail message (some local mailers require
a short subject line) to {\tt aastex-instruct@aas.org}. 

For purposes of producing paper output, the AAS\TeX\ package contains
a number of \LaTeX\ style files that produce variously formatted pages.
There is a ``manuscript'' style, two ``preprint'' styles, and
two ``deluxetable'' styles.
The manuscript style ({\tt aasms4.sty}) is used for papers that are submitted to
the AAS journals and the PASP for review.
The preprint styles render articles in a compact form that may
be suitable for distribution among colleagues.
The deluxetable styles are used by authors who wish to submit
``camera-ready'' tables to the journals for publication.  See Sections~\ref{styles} 
and \ref{docs} for further details about the contents of the AAS\TeX\ package.

Electronic submission instructions for the various journals are also
available electronically.  Send an empty e-mail message to
the appropriate address below.

\begin{quote}
\begin{tabular}{l@{\quad}p{2in}}
{\tt ApJ} & {\tt apj-instruct@aas.org}\\
{\tt ApJ Letters} & {\tt apjlett-instruct@aas.org}\\
{\tt AJ} & {\tt aj-instruct@aas.org}\\
{\tt PASP} & {\tt pasp-instruct@aas.org}\\
\end{tabular}
\end{quote}

The AAS provides user support for the AAS\TeX\ package and for other
aspects of electronic submission to the journals.  This support is 
available electronically by sending e-mail to {\tt aastex-help@aas.org}.

\section{Command Descriptions for Journal Submissions}
%\setcounter{secnumdepth}{1}

This section describes the commands in the AAS\TeX\
package that an author might enter in a manuscript that is being prepared for 
{\bf electronic} submission to one of the journals.
The commands will be described in roughly the same order as they
would appear in a manuscript.
The reader will also find it helpful to examine the
sample files (\verb"sample1.tex" and \verb"sample2.tex") that are distributed
with the package.
Authors are reminded to check the electronic submission procedures and the 
instructions to authors for the appropriate journals.

\subsection{Preamble}

In \LaTeX\ manuscripts, the preamble is that portion of the file before the
\verb"\begin{document}" command. 

\subsubsection{Getting Started}

The first piece of markup in the manuscript must declare the
overall style of the document.  Any commands that appear before this markup
will be ignored.
\begin{quote}
\verb"\documentstyle[12pt,aasms4]{article}"
\end{quote}
This specifies the main style to be
the {\tt article} style using twelve point fonts,
with modifications and additions for the {\tt aasms4} substyle.
The {\tt aasms4} substyle will issue a warning message
if the font size is smaller than twelve points, and the
size will be set to twelve points; the file will still be processed.

The {\tt aasms4} style must be used for the main body of the manuscript. 
The paper copy produced by this style file will be double-spaced.
Any tables included in the main body of the manuscript will also be 
double-spaced. 

Other styles are available, which will be discussed 
later in this guide (see Section \ref{styles}).

\subsubsection{Defining New Commands}

The AAS\TeX\ package supports author-defined commands using \verb"\newcommand" 
and \verb"\def". These definitions should be
placed in the document preamble only.  The remainder of this section refers
to manuscripts submitted electronically to the ApJ or the ApJ Letters; other
users of the package may skip to the next section.
 
Manuscripts submitted electronically to the ApJ or the
ApJ Letters are translated to a language called SGML. 
In general, author-defined commands that are
abbreviations are acceptable and can be translated;
commands that attempt to define new symbols (e.g.,
\verb"\new"- \linebreak \verb"command{\lte}{\lower 0.5ex\hbox{${}\buildrel<\" \linebreak \verb"over\sim{}$}}") 
will not translate correctly. In creating new commands authors are 
encouraged to use \verb"\newcommand" rather than the \verb"\def" command. 
 
Authors who need additional symbols not found in \LaTeX\
or AAS\TeX\ may use the AMS symbol definitions and fonts
that are discussed in Appendix \ref{ams}.  If it is not
feasible to use the AMS fonts, authors may create their
own symbols via \verb"\newcommand" provided that the
symbol is given exactly the same name as in the AMS
fonts.  This works because author-defined commands with
names that match AMS symbol names will be translated as
if they were references to the AMS symbols, regardless
of the author's own definition.  
 
Author-defined commands that use any of the commands
listed below are unlikely to translate correctly:
\verb"\hskip", \verb"\vskip", \verb"\raise",
\verb"\raisebox", \verb"\lower",
\verb"\rlap", \verb"\kern", \verb"\lineskip",
\verb"\baselineskip",
\verb"\char", \verb"\mathchar", \verb"\mathcode",
\verb"\buildref", and
\verb"\mathrel".
 
Authors submitting manuscripts electronically to the
ApJ and ApJ Letters will have their manuscripts screened
by the Production Office at the University of Chicago
Press, and a screening report will be returned to the
authors along with the referees' comments.  Any
definitions or commands that would cause problems in
translation to SGML will be identified in the screening
report and must be fixed by the author if the electronic
file is to be processed successfully.

\subsubsection{Editorial Information}

A number of markup commands are available for the 
editorial offices to record the publication history and slug-line data for each
manuscript.  These commands should be commented out by the author if they 
appear in a template copy.  At this point authors may wish to skip to the 
next section.

\begin{quote}
\verb"\received{RECEIPT DATE}"\\
\verb"\revised{REVISION DATE}"\\
\verb"\accepted{ACCEPT DATE}"\\
\verb"\cpright{type}{year}"\\

%For preprints and manuscripts in draft/referee format, etc.,
%the slug-line information is irrelevant and generally in those styles
%of that nature, the data are never formatted or printed.
%Receipt and acceptance dates (or blank lines representing them) are
%printed on {\tt aasms4} articles, that is, on true manuscripts.
%The {\tt aaspp4} and {\tt aas2pp4} styles do not print the rules or the dates.
%Authors do not know what these dates are, however,
%so there is no reason for the author to include
%\verb"\received" and \verb"\accepted" commands in manuscripts.
%Editorial staff will insert the correct information as appropriate.

%\begin{quote}
\verb"\journalid{VOL}{JOURNAL DATE}"\\
\verb"\articleid{START PAGE}{END PAGE}"\\
\verb"\paperid{MANUSCRIPT ID}"
%\end{quote}

%The \verb"\journalid" and \verb"articleid" commands are used to identify
%the volume and page numbers of a scheduled article.
%The manuscript identification number used to track the manuscript is
%specified in the \verb"\paperid" command.

%\begin{quote}
\verb"\cpright{TYPE}{YEAR}"\\
\verb"\ccc{CODE}"
\end{quote}

Copyright information is specified through the commands \verb"\cpright"
and \verb"\ccc".  The ``type'' of copyright and the corresponding year
are given in \verb"\cpright"; valid copyright types are as follows.
\begin{quote}
\begin{tabular}{l@{\quad}p{2in}}
\tt AAS & Copyright has been assigned to the AAS.\\
\tt ASP & Copyright has been assigned to the ASP.\\
\tt PD & The article is in the public domain.\\
\tt none & No copyright is claimed for the article.\\
\end{tabular}
\end{quote}
The copyright type is case sensitive, so the type string must be
entered exactly as given above.
The Copyright Clearing Center code may be given in the \verb"\ccc"
command; the code is taken as regular text, so any special characters,
notably `\$', must be properly specified.

\subsubsection{Short Comment on Title Page}

Authors who wish to include a short remark on the title page,
such as the name and date of the journal in which an article
has been scheduled, may do so with the following command.
\begin{quote}
\verb"\slugcomment{TEXT}"
\end{quote}
In the {\tt aasms4} style, such comments appear on the title page after the
title and authors;
in the {\tt aaspp4} style, they are placed at the upper right corner
of the title page.

\subsubsection{Running Heads}

Authors are invited to supply running head information.
There are generally two different kinds of data in running heads;
the left head contains an author list (last
names, possibly truncated as ``et al.''), while the right head
is an abbreviated form of the paper title.  This running head information will
not be appear on the printed page.
\begin{quote}
\verb"\lefthead{TEXT}"\\
\verb"\righthead{TEXT}"
\end{quote}
Editors and publishers impose varying requirements
on the brevity of these data.  A good rule of thumb is to limit the list
of authors to three, otherwise use et al., and limit the short form of the
title to 44 characters.  The editors may choose to modify the author-supplied
running heads.


\subsection{Starting the Main Body}

None of the markup that appears in the preamble actually typesets
anything; the preamble is only a control section.
The author must include a
\begin{quote}
\verb"\begin{document}"
\end{quote}
command to identify the beginning of the main textual
portion of the manuscript.

\subsection{Title and Author Information}

Title and author identification are by way of the markup commands
\verb"\title" and \verb"\author".
The author's principal affiliation is specified with
a separate command \verb"\affil".
Each \verb"\author" command
should be followed by a corresponding \verb"\affil".
\begin{quote}
\verb"\title{LUCID TEXT}"\\[.5ex]
\verb"\author{NAME(S)}"\\
\verb"\affil{AFFILIATION}"\\
\verb"\authoraddr{ADDRESS}"\\[.5ex]
\verb"\and"
\end{quote}

Line breaks are permitted in the title if the author wishes
to specify them with the \verb"\\" command.  Long titles will
be broken automatically, so the \verb"\\" markup is not required.
If the title is explicitly broken over several lines, the
preferred style for titles in AAS and ASP journals is the so-called
``inverted pyramid'' style.  In this style, the longest line
is the first (or top) line, and each succeeding line is shorter.
The text of the title should be entered in mixed case;
it will be converted to uppercase by the publisher.
Footnotes are permissible in titles; be careful to ensure that
alternate affiliations (see below) are properly numbered if a
footnote to the title is specified.

Authors' names are given in \verb"\author" commands,
and should be entered in mixed case.
Names that appear together in the author list for authors who
have the same primary affiliation should be specified in a single
\verb"\author" command.
Each author group (\verb"\author" command)
should be followed by an \verb"\affil" command, giving the principal
affiliation of that author.  Physical and postal address information
for the institution specified may be included with its name.
The address can be broken over several lines by using the
\verb"\\" command to indicate the line breaks.
Usually, however, postal information will fit on one line.
When there is more than one \verb"\author" command, the last
one should be preceded by an \verb"\and" command.

When there is a lengthy author list, all authors' names may be
specified in a single \verb"\author" command, with affiliations
specified using the \verb"\altaffilmark" mechanism described below.
In this case, no \verb"\affil" commands would be used, and the
affiliations would all be listed in a footnote block at the bottom
of the title page.  The style file performs this formatting.

Postal addresses for individual authors may be specified in
\verb"\authoraddr" commands.  This command does not produce
any formatted text in most AAS\TeX\ styles and can be used to specify the
corresponding address of the first author for purposes of
editorial communication.

Authors often have affiliations in addition to their principal employer,
and these are specified with the \verb"\altaffilmark"
and \verb"\altaffiltext" commands.
These behave like the \verb"\footnotemark"
and \verb"\footnotetext" commands of \LaTeX, except there are no optional
arguments in the AAS\TeX\ commands.
\verb"\altaffilmark" is appended to author's names in the \verb"\author"
lists and generates superscript identification numbers.
The text for the individual alternate affiliations is generated by the
\verb"\altaffiltext" command.
\begin{quote}
\verb"\altaffilmark{KEY NUMBER(S)}"\\
\verb"\altaffiltext{NUMERICAL KEY}{TEXT}"
\end{quote}
It is up to the author to make sure that \verb"\altaffilmark" numbers
attached to authors' names correspond to the correct alternate affiliation,
i.e., that each {\small KEY NUMBER} matches the {\small NUMERICAL KEY} for
the corresponding {\small TEXT}.

\subsection{Abstract}

The paper abstract should be enclosed in an {\tt abstract} environment.
\begin{quote}
\verb"\begin{abstract}"\\
{\it abstract text\/}\\
\verb"\end{abstract}"
\end{quote}

\subsection{Keywords}

Keywords, subject headings, etc., are accommodated
as a single piece of text.
\begin{quote}
\verb"\keywords{TEXT}"
\end{quote}
If authors supply keywords they must be delimited by the appropriate punctuation
required by the journal.  Keywords should be specified in alphabetical order.
The \verb"\keywords" command
supplies the proper leading text (``Keywords:'', ``Subject headings:'',
etc.), according to the style.

The list of keywords used by the ApJ and the ApJ Letters are in the file 
{\tt keywords.apj} in the directory {\tt pubs/aastex-misc} on the anonymous
ftp node {\tt aas.org}.

\subsection{Comments to Editors}

Authors may make notes or comments to the editors with the
\verb"\notetoeditor{TEXT}" command.  This command will behave like a 
footnote and appear on the bottom of the page, e.g., $^{E1}$: {\small NOTE TO
THE EDITOR: \verb"<TEXT>"}.  Output to the printed page is produced with the 
\verb"aasms4.sty" file only.
\begin{quote}
\verb"\notetoeditor{TEXT}"
\end{quote}


\subsection{Sections}

The {\tt article} style for AAS\TeX\ manuscripts supports four levels of
sectioning.
\begin{quote}
\verb"\section{HEADING}"\\[.5ex]
\verb"\subsection{HEADING}"\\[.5ex]
\verb"\subsubsection{HEADING}"\\[.5ex]
\verb"\paragraph{HEADING}"
\end{quote}
Section headings should be given in mixed case.
Note that these commands delimit sections by marking the
{\sl beginning\/} of each section;
there are not separate commands to identify the ends.

\subsection{Figure and Table Placement}   \label{place}

Generally figures and tables are not ``placed'' in the text of the document
where an author would like them to appear physically, but rather follow the
main body of the text.
However, authors may indicate to the editors the preferred placement of these 
items in the text by use of the \verb"\place*" commands.
\begin{quote}
\verb"\placetable{KEY}"\\
\verb"\placefigure{KEY}"
\end{quote}
The \verb"\place*{KEY}" commands are similar to the \verb"\ref" command
in La\TeX\
and require corresponding \verb"\label" commands to link them to the
proper elements for assisting the editors.   The \verb"\label{KEY}" command
may be included in a \verb"\figcaption" command used to produce figure
captions on the figure caption page or in a
\verb"\caption" command used in the figure environment (see 
Sections \ref{legends} and \ref{figs}); in a \verb"\tablecaption" 
command in the deluxetable 
environment (see Section \ref{dte}), or in a 
\verb"\caption" command in a La\TeX\ table environment.

On the printed page the \verb"\place*" commands will print a short message to 
the editor about figure and table placements when used with the 
{\tt aasms4.sty} file.  Nothing will be 
printed with the other style files.


\subsection{Acknowledgments}

In addition, AAS\TeX\ manuscript styles support an
\verb"\acknowledgments" section.
\begin{quote}
\verb"\acknowledgments"
\end{quote}
In the AAS\TeX\ substyles, acknowledgments are set off from the
concluding main body text simply by vertical space
(no heading or type size change).

\subsection{Appendices}

When one or more appendices are needed in a paper, the end of the
main body text must be marked.
\begin{quote}
\verb"\appendix"
\end{quote}
Note that the \verb"\appendix" command has no arguments;
sections in the appendix must be headed with \verb"\section"
commands containing the section headings, as described earlier.
The \verb"\appendix" command takes care of a number of internal
housekeeping concerns, such as identifying sections with letters
instead of numerals, resetting the equation counter, etc.

\subsection{Citations}   \label{cites}

Two options are available for marking citations within a manuscript.
The \verb"\cite" command is used in conjunction with {\tt thebibliography}.

\begin{quote}
\verb"\cite{KEY}"
\end{quote}

\LaTeX's {\tt thebibliography} environment allows authors to identify
references symbolically using unique keys (the author makes these up).
If this mechanism is used, the {\small KEY} given in the \verb"\cite"
command must correspond to a {\small KEY} given in a \verb"\bibitem"
command in the {\tt thebibliography} environment.
The {\tt LABEL} in the \verb"\bibitem" command is used as the citation in
the text. See Section \ref{bib}.

The conventional method used by authors to manage the citations and
reference list in a paper has often been a manual one.  These authors can use 
the {\tt references} environment and \verb"\markcite" command.

\begin{quote}
\verb"\markcite{KEY}"
\end{quote}

The {\small KEY} supplied in the \verb"\markcite" command must match the 
key used by the \verb"\reference" command for the corresponding reference 
in the {\tt references} environment.  The \verb"\markcite" command is
only used to point to the proper reference in the reference list---the actual
citation must be supplied by the author as part of the running text.
See Section \ref{refenv}.

The use of the {\tt references} environment and the \verb"\markcite" command
has changed with the v4.0 release of the AAS\TeX\ package and is not 
backward or forward compatible with v3.0.  


\subsection{Equations}

Displayed equations can be typeset in many ways using the standard
displayed math environments of \LaTeX;
these three are probably of greatest use.
\begin{quote}
\verb"\begin{displaymath}"\\
\verb"\end{displaymath}"\\[.5ex]
\verb"\begin{equation}"\\
\verb"\end{equation}"\\[.5ex]
\verb"\begin{eqnarray}"\\
\verb"\end{eqnarray}"
\end{quote}
The {\tt displaymath} environment will break out a single,
unnumbered formula.  The same formula will appear the same if it
is set in an {\tt equation} environment, except it will be
autonumbered by \LaTeX.
In order to set several formul\ae\ in which vertical alignment
is required or to display a long equation across multiple lines, the 
{\tt eqnarray} environment should be used.

Authors occasionally wish to group related equations together and
identify them with letters appended to the same equation number,
as opposed to having each with a separate numeral.
When this is desired, such related equations should still be set
in {\tt equation} or {\tt eqnarray} environments (whichever is
appropriate may be used), and this grouping should be placed within
a {\tt mathletters} environment.
\begin{quote}
\verb"\begin{mathletters}"\\
{\it {\tt equation} or {\tt eqnarray} environment(s)\/}\\
\verb"\end{mathletters}"
\end{quote}

It is possible to override \LaTeX's automatic numbering within
{\tt equation} or {\tt eqnarray} environments.
\begin{quote}
\verb"\eqnum{TEXT}"
\end{quote}
When \verb"\eqnum" is specified inside an {\tt equation} environment,
or on a particular equation within an {\tt eqnarray}, the text supplied
as an argument to \verb"\eqnum" is used as the equation identifier.
\LaTeX's equation counter is {\sl not\/} incremented when \verb"\eqnum"
is used.
\verb"\eqnum" must be used {\sl inside\/} the environment.

When unnumbered equations are desired, authors can use either the
{\tt displaymath} environment (for single displayed equations) or place
a \verb"\nonumber" command before the equation delimiter (\verb"\\")
in a particular equation in an {\tt eqnarray}.
\LaTeX's equation counter is {\sl not\/} incremented when
\verb"\nonumber" is used.

If, as a consequence of the use of \verb"\eqnum" or \verb"\nonumber",
\LaTeX's equation counter gets out of the author's intended sequence,
the counter may be reset to a particular value.
\begin{quote}
\verb"\setcounter{equation}{NUMBER}"
\end{quote}
The equation counter should be set to the number corresponding to the
last equation that was formatted, so it is most appropriate for this
command to appear immediately after an {\tt equation} or {\tt eqnarray}
environment ends.
\verb"\setcounter{equation}" must be used {\sl outside\/} the math
environments.

\subsection{Tables}  \label{tables}

There is support in the AAS\TeX\ package for tables via two mechanisms:
\LaTeX's standard {\tt table} and {\tt tabular} environments,
and a {\tt deluxetable} environment that facilitates the formatting
of tabular (particularly lengthy tabular) material.  Tables may be
marked up using either mechanism although the {\tt deluxetable} environment 
is preferred.  Authors should {\it not} use the \LaTeX\
{\tt tabbing} environment or the \verb"\hline" command for electronic
submissions.

\LaTeX\ permits the preparation of fairly complex tables with
arbitrary spacing, straddle heads and rules, and the like.
Authors who need to specify complicated column headings and
so forth are advised to consult the \LaTeX\ manual (Lamport 1985)
for details.
Most of the capabilities are applicable to AAS\TeX's {\tt deluxetable}
environment as well as \LaTeX's {\tt tabular}.

Authors should be aware that all tables produced using the {\tt aasms4}
style will be double-spaced.  If single-spaced, or ``camera-ready'' tables
are required, then an alternate style file must be used for the tables only
(see Section \ref{crtabs}).

\subsubsection{The {\tt deluxetable} Environment}  \label{dte}

This section describes the use of the {\tt deluxetable} environment.
There are several desiderata that are somewhat above and beyond LaTeX's
{\tt tabular} environment that facilitate the formatting of such tables.
Among these are breaking long tables across pages, using footnotes
in a table, specifying comments and references for tables, etc.

The {\tt deluxetable} environment is delimited by \LaTeX's familiar
\verb"\begin" and \verb"\end" constructs.
\begin{quote}
\verb"\begin{deluxetable}{COLS}"\\
\verb"\end{deluxetable}"
\end{quote}
{\small COLS} specifies the justification for each column.
One of the letters `l', `c', or `r' is given for each column,
indicating left, center, or right justification.  Authors are
referred to the \LaTeX\ manual (Lamport 1985) for further information.

There are several items in a {\tt deluxetable} environment that
must be given before the data for the table.

\begin{quote}
\verb"\small"\\
\verb"\footnotesize"\\
\verb"\scriptsize" \\
\verb"\tablewidth{DIMEN}"\\
\verb"\tablenum{TEXT}"\\
\verb"\tablecaption{TEXT \label{KEY}}"
\end{quote}

If a table is too wide for the printed page, it is permissible to change the 
font size within the {\tt deluxetable} environment using the \verb"\small" 
(11pt), the \verb"\footnotesize" (10pt) or the \verb"\scriptsize" (8pt) command.

The width of a deluxetable is defined by \verb"\tablewidth";
the default width, if this command is omitted, is the width of the body text.
The table can be set to its natural width by specifying
a {\small DIMEN} of 0pt.  Long tables may have a natural width that is 
different for each page.  The natural width for each page will be printed 
to the log file during processing;
authors may then define a fixed table width based 
on this information giving the 
tables a more uniform appearance across the pages.

It is possible to override \LaTeX's automatic numbering within the
{\tt deluxetable} environment.
When \verb"\tablenum" is specified inside a {\tt deluxetable} environment,
the text supplied as an argument to \verb"\tablenum" is used as the
table identifier.
\LaTeX's equation counter is {\sl not\/} incremented when \verb"\tablenum"
is used.
\verb"\tablenum" must be used {\sl inside\/} the {\tt deluxetable} environment
and before the \verb"\tablecaption" command.

The caption (actually, the title) of the table is specified
in \verb"\tablecaption".
The intent is for the text of \verb"\tablecaption" to be brief;
explanatory notes may be specified in the end notes to the table
(\verb"\tablecomments", see below).  If the caption does not appear 
centered on the table after processing, then specify the width of 
the table explicitly in the \verb"\tablewidth" command and rerun 
\LaTeX\ on the file.

The {\small KEY} in the {\tt label} command should be the same {\small KEY}
used in
the \verb"\placetable{KEY}" command used to call out the preferred location
of the table in the text.  The {\tt label} command, if used, must be included 
in the \verb"\tablecaption" command.

\begin{quote}
\verb"\tablehead{TEXT}"\\
\verb"\colhead{HEADING}"
\end{quote}

Column headings are specified within \verb"\tablehead".
Within \verb"\tablehead", each column heading can be given
in a \verb"\colhead", which will ensure that the heading is centered
on the natural width of the column; this is the typical disposition
of column headings, and the use of \verb"\colhead" is encouraged.
There should be a heading for each column, so there should be as
many \verb"\colhead" commands in the \verb"\tablehead" as there
are data columns.
If more complicated column headings are required, 
any valid {\tt tabular} commands that constitute a proper
head line for the table may be used.
Consult the \LaTeX\ manual (Lamport 1985) for details about using
the {\tt tabular} environment to prepare tables.

The \verb"\tablecolumns{NUM}" command is necessary if the author has
multiline column headings produced by \verb"\tablehead" or other \LaTeX
commands and is using either the \verb"\cutinhead" or \verb"\sidehead"
markup (see below).  The {\tt NUM} is 
set to the true number of columns in the 
table, and the command must come before the \verb"\startdata" command.

\begin{quote}
\verb"\tablecolumns{NUM}"
\end{quote}
See the {\tt complext.tex} file in the AAS\TeX\ package for an example
of a more complex table format using the \verb"\tablecolumn" command.

Column alignment within the data columns can be adjusted with the \TeX\
\verb"\phantom{STRING}" command, where {\tt STRING} can be any character
representation, e.g., \verb"\phantom{$\arcmin$}".  A blank character of
width {\tt STRING} is then inserted in the table.  Three
characters have been predefined for this purpose as well as a shorter version
of the generic \verb"phantom" command:
\begin{quote}
\begin{tabular}{l@{\quad}p{2in}}
{\verb"\phn"} & {phantom numeral 0-9}\\
{\verb"\phd"} & {phantom decimal point}\\
{\verb"\phs"} & {phantom plus or minus sign}\\
{\verb"\phm{STRING}"} & {generic phantom command}\\
\end{tabular}
\end{quote}

It is possible that a complicated table heading will overflow
the vertical space allotted for the table heading.
The fraction of the page allocated
for the table heading may be changed with \verb"\tableheadfrac".
The {\small NUM} argument to \verb"\tableheadfrac" should be the
decimal fraction of the page used for heading information.
The default value is 0.1, meaning that 10\% of the page height
is reserved for the table heading.  It should rarely be necessary
to change this value.
\begin{quote}
\verb"\tableheadfrac{NUM}"
\end{quote}

After the table title and column headings are specified, data lines
can be entered.  The data lines are within two ``data'' 
commands.
Data elements within a row of the table are separated with \& (ampersand)
characters.  The end of each row is indicated with \verb"\nl".
\begin{quote}
\verb"\startdata"\\
{\it data lines\/}\\
\verb"\enddata"
\end{quote}

\begin{quote}
\verb"\nl"\\
\verb"\tablevspace{DIMEN}"
\end{quote}

Extra vertical space can be inserted between rows with a
\verb"\tablevspace" command; the argument is a dimension
and may be specified in any units that are legitimate in \LaTeX.

\begin{quote}
\verb"\tablebreak"\\
\verb"\nodata"
\end{quote}

If a page break needs to be forced in a deluxetable,
\verb"\tablebreak" should be used instead of \verb"\nl".
This is sometimes necessary when several rows of data are
associated with a single object or item; such logical groupings
should not be broken across pages, and \verb"\tablebreak" can
be used to ensure that the page breaks are reasonable in these cases.

The journals often require that elements for which there are no data
be explicitly marked.  This is to differentiate such elements from
blank elements, which are frequently interpreted as implicitly
repeating the entry in the corresponding element in the row preceding.
Data elements for which there are no data should contain
a \verb"\nodata" command; an appropriate symbol will be placed in
that data element.

Within the deluxetable body, two kinds of ``specialty'' heads are
recognized.  A cut-in head is a piece of text that is centered on the 
table width;
it is spaced above and below from the data rows that precede and
follow it, and there may be rules associated, depending on the
journal or manuscript style.  All of these formatting particulars
are managed by the style files.  The author need only specify the
text to be centered with a \verb"\cutinhead" command.
Similarly, a side head is a piece of text that is left-justified.

\begin{quote}
\verb"\cutinhead{TEXT}"\\
\verb"\sidehead{TEXT}"
\end{quote}

Table footnotes (more properly, table {\sl endnotes\/})
may be used in the {\tt deluxetable} environment;
their use is described in detail in the section on
{\it Table Footnotes\/}, below.
These notes must follow the \verb"\enddata" command.
The markup commands for such endnotes are as follows.
\begin{quote}
\verb"\enddata"\\
\verb"\tablenotemark{KEY LETTER(S)}"\\
\verb"\tablenotetext{ALPHA KEY}{TEXT}"\\
\verb"\tablecomment{TEXT}"\\
\verb"\tablerefs{TEXT}"
\end{quote}


\subsubsection{The {\tt table} Environment}

Tables may be marked and composed using the standard \LaTeX\ tools for tables,
although the use of the {\tt deluxetable} environment is encouraged.
Tables should appear in {\tt table} environments.
\begin{quote}
\verb"\begin{table}"\\
\verb"\end{table}"
\end{quote}
The {\tt table} environment encloses not only the tabular
material but also any title (caption) or footnote information
associated with the table.

Titles or captions for tables are indicated with
\begin{quote}
\verb"\caption{TEXT  \label{KEY}}"
\end{quote}
Tables will be identified with arabic numerals, e.g., ``Table 2'';
the identifying text, including the number, is generated automatically
by \verb"\caption" within the \verb"table" environment.
The {\small KEY} in the {\tt label} command should correspond to the 
{\small KEY}
used in the \verb"\placetable{KEY}" command used to identify the placement 
of the table in the
text.

Tabular information is typeset within the
{\tt tabular} environment:
\begin{quote}
\verb"\begin{tabular}{COLS}"\\
\verb"\end{tabular}"
\end{quote}
where {\small COLS} specifies the justification for each column.
One of the letters `l', `c', or `r' is given for each column,
indicating left, center, or right justification.
Consult the \LaTeX\ manual (Lamport 1985) for details about using
the {\tt tabular} environment to prepare tables.
A {\tt tabular} table must appear within a {\tt table} environment.
And there should be only one {\tt tabular} table per {\tt table} environment.
If the journal requests manuscripts with only one table per page,
the author may need to insert a \verb"\clearpage" command after
especially short tables.

There is a \verb"\tableline" command for use in {\tt tabular}
environments.
\begin{quote}
\verb"\tableline"
\end{quote}
This command produces the horizontal rule(s) between the column headings
and the body of the table.
Authors are discouraged from using any \verb"\hlines" themselves;
vertical rules are typically forbidden by editorial preference.

It is possible to override \LaTeX's automatic numbering within the
{\tt table} environment.
\begin{quote}
\verb"\tablenum{TEXT}"
\end{quote}
When \verb"\tablenum" is specified inside a {\tt table} environment,
the text supplied as an argument to \verb"\tablenum" is used as the
table identifier.
\LaTeX's equation counter is {\sl not\/} incremented when \verb"\tablenum"
is used.
\verb"\tablenum" must be used {\sl inside\/} the {\tt table} environment.

\subsubsection{Table Footnotes}

AAS\TeX\ supports footnotes (endnotes) that are associated with tables;
this support applies to both the {\tt deluxetable} environment
and the standard \LaTeX\ {\tt table} environment.
Footnotes for tables are usually identified by lowercase letters
rather than numbers.
Marking and assigning associated text is achieved with
the \verb"\tablenotemark" and \verb"\tablenotetext"
commands, in which the note identifier is required
(cf.\ \verb"\altaffilmark" and \verb"\altaffiltext").
The \verb"\tablenotetext" {\sl must} be specified before
the enclosing \verb"\end{table}", since the text of
notes to tables are displayed by that command.
\begin{quote}
\verb"\tablenotemark{KEY LETTER(S)}"\\
\verb"\tablenotetext{ALPHA KEY}{TEXT}"
\end{quote}
Note that the {\small KEY LETTER} should be the same as the
{\small ALPHA KEY} for the corresponding {\small TEXT}.
It is the responsibility of the author to make the correspondence correct.

AAS\TeX\ also supports special kinds of table endnotes.
Sometimes authors tabulate materials which have corresponding references,
and it may be desirable to associate these references with the table
rather than (or in addition to) the formal reference list.
Occasionally, authors wish to append a short paragraph of explanatory
notes that pertain to the entire table, but which are different than
the caption.
\begin{quote}
\verb"\tablerefs{REFERENCE LIST}"\\
\verb"\tablecomments{TEXT}"
\end{quote}

The table endnotes are coupled to the table in which they occur, rather
than being associated with a particular page, and they are printed with
the table (relatively close to the caption) instead of appearing at the
extreme bottom of the page.  This is done to ensure that the notes wind
up on the same page as the table, since tables are floats and can migrate
from one page to another.

\subsection{Bibliography}

Observe that the {\it bibliographic data\/} supplied by the author must
conform to the standards of the journal.  We have elected not to burden
authors with tedious markup commands
to delimit the bibliographic fields because many of the journals we have
targeted in this project have agreed to reduce typographic overhead (bolding,
italicizing, etc.) in reference lists (Abt 1990).
It is the responsibility of the author to arrange these fields in the proper
order with the correct punctuation; the information will be typeset as is,
i.e., in roman with no size or style changes.

An identifier denoting that the article  was
prepared with the AAS\TeX\ package appears as a footnote
on the last page of references.

Authors has two options for marking citations in the text and providing
the associated reference lists. 

\subsubsection{The {\tt thebibliography} Environment}   \label{bib}

Authors are encouraged to use the semantics of \LaTeX's 
{\tt thebibliography} environment, marking citations with \verb"\cite" 
and associating references with them via \verb"\bibitem".
The \verb"\cite"-\verb"\bibitem"
mechanism associates citations and references symbolically
while maintaining proper citation syntax within the paper.
The author must create the citation label for each reference
in proper journal format in the \verb"\bibitem" command.
\begin{quote}
\verb"\begin{thebibliography}{DUM}"\\
\verb"\bibitem[LABEL]{KEY}" {\it bibliographic data\/}\\
\verb"   ."\\
\verb"   ."\\
\verb"\end{thebibliography}"
\end{quote}
where {\small LABEL} must adhere to journal standards, e.g., ``Abt 1986''.
The value of {\small LABEL} is then inserted into the body of the text
at the place of the \verb"\cite" command during the processing.
The argument {\small DUM} to {\tt thebibliography} environment is not
used by the AAS\TeX\ package.
Note that it is not possible to use \verb"\bibitem"s
within the {\tt references} environment,
nor will \verb"\cite" commands work properly in the main body
if \verb"\bibitem"s are not properly specified.
This technique can be a bit tricky, and there are limitations on the
way that the citation {\small LABEL} is formatted.

Citation management can be complex, and systems have been
developed to assist authors in preparing bibliographies.
The program that manages references within the \TeX\ family
is called BIB\TeX, and it is designed to work in conjunction
with the citation and reference list capabilities of \LaTeX.
Authors are advised to consult the \LaTeX\ manual (Lamport 1985).
The AAS\TeX\ package has no additional support for BIB\TeX\/.

\subsubsection{The {\tt references} Environment}\label{refenv}

Many authors are comfortable with the standard
process of entering citations in the proper format directly
in the body of an article and then organizing the reference
list themselves.
Such authors would use \verb"\markcite{KEY}" in the text before the citation,
e.g., \verb"...as discussed by" \verb"Marlow \markcite{a3} (1993)",
and then employ the {\tt references} environment for the
reference list, where each reference is preceded by a \verb"\reference{KEY}" 
command, e.g., \verb"\reference{a3} Marlow, R. 1993, \nat," \verb"445, 12". 
The \verb"\markcite{KEY}" command indicates in the text where a certain 
reference in the reference list is used; the author must still supply the 
correct citation.
The {\tt references} environment simply sets off
the list of references and adjusts spacing parameters.
\begin{quote}
\verb"\begin{references}"\\
\verb"\reference{KEY}" {\it bibliographic data\/}\\
\verb"   ."\\
\verb"   ."\\
\verb"\end{references}"
\end{quote}

The implementation of the \verb"\markcite-\reference" markup has changed
with the v4.0 release of the AAS\TeX\ package.  Its use is not backward
or forward compatible with v3.0. 

\subsubsection{Abbreviations for Journals}

There are markup commands for many of the oft-referenced journals so that
authors may use the markup names as a shorthand rather than having to look
up a particular journal's specific abbreviation.
In principle, all the journals should be using the
same abbreviations as well, but it is fair to anticipate some changes in the
specific abbreviations before a system is finally settled on.  As long as
these commands are kept up to date, authors need not be concerned about
such editorial preferences and changes.

\begin{center}
\begin{tabular}{ll}
\verb"\araa" & Annual Review of Astronomy\\
  & \hspace*{1em} and Astrophysics\\
\verb"\ao" & Applied Optics\\
\verb"\aj" & Astronomical Journal\\
\verb"\azh" & Astronomicheskii Zhurnal\\
\verb"\aap" & Astronomy and Astrophysics\\
\verb"\aapr" & Astronomy and Astrophysics Reviews\\
\verb"\apj" & Astrophysical Journal\\
\verb"\apjl" & \rule[.5ex]{2em}{.4pt}, Letters to the Editor\\
\verb"\apjs" & \rule[.5ex]{2em}{.4pt}, Supplement Series\\
\verb"\aplett" & Astrophysics Letters\\
\verb"\apspr" & Astrophysics Space Physics Research\\
\verb"\apss" & Astrophysics and Space Science\\
\verb"\aaps" & \rule[.5ex]{2em}{.4pt}, Supplement Series\\
\verb"\baas" & Bulletin of the AAS\\
\verb"\bain" & Bulletin Astronomical Inst. Netherlands\\
\verb"\gca" & Geochimica Cosmochimica Acta\\
\verb"\grl" & Geophysics Research Letters\\
\verb"\iaucirc" & IAU Circular\\
\verb"\jcp" & Journal of Chemical Physics\\
\verb"\jgr" & Journal of Geophysics Research\\
\verb"\jrasc" & Journal of the RAS of Canada\\
\verb"\memras" & Memoirs of the RAS\\
\verb"\mnras" & Monthly Notices of the RAS\\
\verb"\nat" & Nature\\
\verb"\nphysa" & Nuclear Physics A\\
\verb"\physscr" & Physica Scripta\\
\verb"\pra" & Physical Review A: General Physics\\
\verb"\prb" & Physical Review B: Solid State\\
\verb"\prc" & Physical Review C:\\
\verb"\prd" & Physical Review D:\\
\verb"\pre" & Physical Review E:\\
\verb"\prl" & Physical Review Letters\\
\verb"\physrep" & Physics Reports\\
\verb"\planss" & Planetary Space Science\\
\verb"\procspie" & Proceedings of the SPIE\\
\verb"\pasj" & Publications of the ASJ\\
\verb"\pasp" & Publications of the ASP\\
\verb"\qjras" & Quarterly Journal of the RAS\\
\verb"\skytel" & Sky and Telescope\\
\verb"\solphys" & Solar Physics\\
\end{tabular}
\end{center}

\subsection{Figure Captions}      \label{legends}

Figure captions or legends are included in the manuscript on a 
separate page using the \verb"\figcaption" command for each figure in the 
paper.

\begin{quote}
\verb"\newpage"

\verb"\figcaption[FILENAME]{TEXT \label{KEY}}"
\end{quote}

The optional argument, {\small FILENAME}, can be used to identify
the PostScript file for the corresponding figure, e.g., graph.eps; 
{\small TEXT} refers to the caption for that figure. 

The {\small KEY} in the {\tt label} command refers to the {\small KEY} in the
\verb"\place*{KEY}" command used to identify the placement of the figure in the 
manuscript (see Section \ref{place}).

When the \verb"\figcaption" command is used, the figure 
identification,
e.g., ``Figure 1,'' is generated automatically by the command itself.

Some journals may accept figures placed in the {\tt figure} environment
(see Section \ref{figs}) if the manuscript is being submitted electronically. 
Please check the appropriate electronic submission instructions
for your target journal.

\subsection{Miscellaneous}

When discussing atomic species, ionization levels can be indicated
with the following command.
\begin{quote}
\verb"\ion{ELEMENT}{LEVEL}"
\end{quote}
The ionization state is specified as the second argument,
and should be given as a numeral.
For example, \ion{Ca}{3} is specified by typing \verb"\ion{Ca}{3}".

AAS\TeX\ contains two commands that permit authors to specify alternate
forms for fractions.  Authors submitting manuscripts electronically
will generally find it unnecessary to use any markup other than the
standard \LaTeX\ \verb"\frac".
\begin{quote}
\verb"\case{NUM}{DENOM}"\\
\verb"\slantfrac{NUM}{DENOM}"
\end{quote}
\LaTeX\ will set fractions in displayed math as built-up fractions;
it is sometimes desirable to use case fractions in displayed equations.
In such instances, one should use \verb"\case" rather than \verb"\frac".
Occasionally, authors wish to typeset fractions with a solidus but in
which the type size is reduced and the numerals are oriented diagonally.
In this case, \verb"\slantfrac" should be used instead of \verb"\frac".
Note that this is different from a shilled fraction, which authors can
produce without any special markup.
\begin{displaymath}
\renewcommand{\arraystretch}{1.4}
\begin{array}{llc}
\mbox{Built-up} & \verb"\frac{1}{2}" & \displaystyle\frac{1}{2} \\[.5ex]
\mbox{Case} & \verb"\case{1}{2}" & \case{1}{2} \\
\mbox{Slant} & \verb"\slantfrac{1}{2}" & \slantfrac{1}{2} \\
\mbox{Shilled} & \verb"1/2" & 1/2 \\
\end{array}
\end{displaymath}

The AAS\TeX\ package also contains a collection of assorted macros
for special symbols (or abbreviations) that authors tend to work
out for themselves anyway.
Some of the definitions come from the {\it A\&A\/} package (Springer 1989);
some are contributions from individuals.  We have
tried to select a tractable number that were useful and also somewhat
difficult to get right because fussy kerning or some such is required.

\begin{center}
\begin{tabular}{ll@{\hspace*{3em}}ll}
\verb"\arcdeg" & \arcdeg & 
\verb"\sq" & \sq \\
\verb"\sun" & \sun &
\verb"\earth" & \earth \\
\verb"\arcmin" & \arcmin & 
\verb"\arcsec" & \arcsec \\
\verb"\fd" & \fd & 
\verb"\fh" & \fh \\
\verb"\fm" & \fm & 
\verb"\fs" & \fs \\
\verb"\fdg" & \fdg & 
\verb"\farcm" & \farcm \\
\verb"\farcs" & \farcs & 
\verb"\fp" & \fp \\
\verb"\micron" & \micron & \\[.8ex]
\verb"\onehalf" & \onehalf &
\verb"\ubvr" & \ubvr \\
\verb"\onethird" & \onethird &
\verb"\ub" & \ub \\
\verb"\twothirds" & \twothirds &
\verb"\bv" & \bv \\
\verb"\onequarter" & \onequarter &
\verb"\vr" & \vr \\
\verb"\threequarters" & \threequarters &
\verb"\ur" & \ur \\[.8ex]
\verb"$\lesssim$" & $\lesssim$ &
\verb"$\gtrsim$" & $\gtrsim$ \\ 
\end{tabular}
\end{center}
Most of these commands can be used in running text as well as when
setting mathematical expressions.
\verb"\lesssim" and \verb"\gtrsim" can only be used in math mode,
which is sensible since they are relations.
It is possible to use \verb"\earth" and \verb"\sun" as subscripts,
e.g., \verb"$1.4 M_{\sun}$" yields $1.4 M_{\sun}$.

\subsection{Concluding the File}

The last command in the electronic manuscript file should be the
\begin{quote}
\verb"\end{document}"
\end{quote}
command, which appears after all the material in the paper.
This command directs the formatter to perform assorted termination
activities and finish processing.

\section{Camera-ready Tables}  \label{crtabs}

Long tables that are prepared as ``camera-ready'' and submitted on paper will 
require the use of a {\tt deluxetable} substyle file.

There are special \LaTeX\ substyle options within the AAS\TeX\ package
which authors can use to produce deluxetable output suitable for
photographic reproduction in specific AAS and ASP journals; these are the
\verb"aj_pt4" and \verb"apjpt4" substyles.
\begin{quote}
\verb"\documentstyle[aj_pt4]{article}"\\[.5ex]
\verb"\documentstyle[apjpt4]{article}"
\end{quote}
When tabular material is prepared with one of these style files,
the pages produced will be suitable for submission to the
journal as camera-ready ``art.''
These special styles are intended for producing deluxetables {\sl only\/};
manuscripts should {\sl not\/} be formatted with the \verb"aj_pt4" or
the \verb"apjpt4" substyle.
See the example files {\tt samp2tbl.tex}, {\tt samptbls.tex}, and
{\tt complext.tex} in the AAS\TeX\ package.

In general, the same markup commands apply to these tables as were described 
earlier for the {\tt deluxetable} environment (see Section \ref{dte}).
The commands \verb"\small", \verb"\footnotesize", and \verb"\scriptsize" are 
not permitted, however.

If the table is too wide for the page, the 
\verb"\ptlandscape" command may be used to format the table in landscape
mode; this command must be issued {\bf before} the \verb"\begin{deluxetable}"
command.  The author must remember to print the table using the landscape
option in the {\it dvi} print program, e.g., {\tt dvips -t landscape}.

These special substyle options for tables, {\tt aj\_pt4.sty} and 
{\tt apjpt4.sty}, may also
be used to produce tables that are not meant to be ``camera-ready'' but are
too wide to fit on a page even after using the techniques described in Section 
\ref{dte}.  The author could use the \verb"\ptlandscape" option but
double-space the table as required for normal manuscript submission.
\begin{quote}
\verb"\ptlandscape" \\
\verb"\begin{deluxetable}" \\
\verb"\doublespace"
\end{quote}

Authors may reference these external tables from within their main manuscript
file with the \verb"\ref{KEY}" command by using the \verb"\dummytable" command 
within the table environment as part of their main manuscript file.

\begin{quote}
\verb"\begin{table}" \\
\verb"\dummytable\label{KEY}" \\
\verb"\end{table}"
\end{quote}

\section{Figures}  \label{figs}

At this time, the most widely used means of including figures or other types of
nontextual data in electronic manuscripts is to generate such
files as PostScript.\footnote{PostScript
is a registered trademark of Adobe Systems Incorporated.}

For manuscripts that are submitted electronically the figures or PostScript
files are generally sent as individual files to the editors and are not 
included as part of the main body of text.
The legends or captions for these files are included in the main body
of the text, however, on a separate page with specific formats (see
Section \ref{legends}).
Authors are referred to the submission instructions for the appropriate
journals if PostScript figures are to be submitted along with an electronically 
submitted manuscript.  

If the author wishes the PostScript figures to be included in the
pages being produced, it is necessary to use 
the {\it dvips\/} program.
This program is available via anonymous ftp
from {\tt labrea.stanford.edu}.
It is also necessary that the graphics files being included conform
to the Encapsulated PostScript standard (Adobe 1990).
Encapsulated PostScript will be referred to as `EPS' in what follows.

There are several commands associated with including EPS files 
in AAS\TeX\ manuscripts,
and these should be placed within a {\tt figure} environment.
\begin{quote}
\verb"\begin{figure}"\\
\verb"\figurenum{TEXT}"\\
\verb"\epsscale{NUM}"\\
\verb"\plotone{EPSFILE}"\\
\verb"\plottwo{EPSFILE}{EPSFILE}"\\
\verb"\plotfiddle{EPSFILE}{VSIZE}"\\
\hspace*{2em}\verb"{ROT}{HSF}{VSF}{HTRANS}{VTRANS}"\\
\verb"\caption{TEXT}"\\
\verb"\end{figure}"
\end{quote}
When \verb"\figurenum" is specified inside a {\tt figure} environment,
the text supplied as an argument to \verb"\figurenum" is used as the
figure identifier.
\LaTeX's figure counter is {\sl not\/} incremented when \verb"\figurenum"
is used.
\verb"\figurenum" must be used {\sl inside\/} the {\tt figure} environment.

\verb"\plotone" inserts the graphic in the named {\small EPSFILE},
scaled (in both dimensions) so that the horizontal
dimension fits in the body text width;
the vertical dimension is scaled to maintain the aspect ratio.
\verb"\plottwo" inserts two plots next to each other.
Scale factors are determined automatically from information in the
EPS file.

The automatic scaling may be overridden with the command 
\verb"\epsscale{NUM}", where {\small NUM}
is in decimal units, i.e., 0.80.

For the adventuresome, there is also \verb"\plotfiddle", which can be
used to override the automatic figure placement, scaling, and rotation.
When this method is used, the author must specify the {\sl vertical\/}
space allotment for the graphic (see {\tt VSIZE} below).
The scaling and placement of the figure are controlled by these parameters:
\begin{quote}
\begin{tabular}{l@{\quad}p{2in}}
\tt VSIZE & vertical white space to allow for plot (\LaTeX\ dimension)\\
\tt ROT & rotation angle (degrees)\\
\tt HSF & horizontal scale (percentage)\\
\tt VSF & vertical scale (percentage)\\
\tt HTRANS & horizontal translation (PS points)\\
\tt VTRANS & vertical translation (PS points)\\
\end{tabular}
\end{quote}
PostScript points are 1/72 inches, so an {\small HTRANS} of 72 moves
the graphic 1 inch to the right.  The PostScript reference point is
the lower left corner of the page, so a {\small VTRANS} of 72 moves
the graphic {\sl up} 1 inch.

There is an upper limit of seven \verb"\begin...\end" figure environments
per page.

Footnotes are {\sl not} supported for figures.

\section{Alternate Style Options}   \label{styles}

It is not the objective of the AAS\TeX\ project to develop \LaTeX\ styles
that produce pages that mimic the appearance of specific journals.
Since this is the case, we have chosen
to provide several format options within the preprint styles
themselves so that output format can be varied.
The {\tt aaspp4} substyle offers a single-spaced alternative for manuscripts,
for instance. 
The {\tt aaspp4} substyle is similar to the manuscript style,
but it is single-spaced, and it is possible to use a smaller typeface.
For institutional purposes, it may be preferable for preprints
to be set in two columns, have running heads, etc.
The {\tt aas2pp4} substyle may be used instead to produce
two-column pages.
This style can be used as is, but it can also serve as a point of
departure for \LaTeX\ style writers at institutions that want
preprints of this general nature.  These two style options are discussed
further below.

The preparation of ``plano tables'' (camera-ready, or {\sl planographic\/})
is possible by using one of the style files \verb"aj_pt4" or \verb"apjpt4".
These styles produce tables in formats suitable for the \aj\ and the
\apj, respectively; the \verb"aj_pt4" style could be used to create
camera-ready tables for the \pasp\ as well.  These styles are intended
primarily for the production of tables that the author wishes to
submit camera-ready; they are not to be used for producing manuscript pages.

The {\tt eqsecnum} style file can be used to modify the way equations are
numbered.  Normally, equations are just numbered sequentially through the 
entire paper.  If {\tt eqsecnum} appears in the {\tt documentstyle} command, 
equation 
numbers will be sequential through each section and will be formatted 
``(sec-eqn),'' where ``sec'' is the current section number and ``eqn'' is 
the number 
of the equation within that section.  The {\tt eqsecnum} option can be used 
with the manuscript or preprint substyle files.  For example,
\begin{quote}
\verb"\documentstyle[11pt,eqsecnum,aaspp4]"\\
\hspace*{2em}\verb"{article}"
\end{quote}

A \verb"flushrt.sty" file option is available for authors that prefer to have 
their margins left and right justified.
The {\tt aas2pp4} style option is flush right by default, but the {\tt aasms4}
and {\tt aaspp4} style options are ragged right by default.

\begin{quote}
\verb"\documentstyle[11pt,eqsecnum,aaspp4,flushrt]"
\hspace*{2em}\verb"{article}"
\end{quote}


\subsection{Preprint format}

A single-column preprint format can be specified with the {\tt aaspp4}
substyle option.
\begin{quote}
\verb"\documentstyle[11pt,aaspp4]{article}"
\end{quote}
The size of the typeface used is under author control by way of
\LaTeX's {\tt NNpt} article substyles (where {\tt NN} is 10, 11, or 12).
Use of 10 point type is not recommended with the {\tt aaspp4} style.

Authors may wish to adjust vertical spacing within a preprint,
for instance, double-spacing text while single-spacing tables.
Authors who want to alternate between single and double
spacing in the manuscript may use the following commands.
\begin{quote}
\verb"\singlespace"\\
\verb"\doublespace"
\end{quote}
\verb"\singlespace" sets the vertical spacing to a smaller value,
while \verb"\doublespace" causes double-spacing.

\subsection{Two-column format}

The {\tt aas2pp4} substyle has the principal function of setting
up two-column output.
\begin{quote}
\verb"\documentstyle[aas2pp4]{article}"
\end{quote}
Although it is quite obvious, it is important to remember
that text lines are considerably shorter when two of them are typeset
side by side on a page.  Long equations, wide tables and figures, and
the like, may not typeset in this format without some adjustments.
The expenditure of great effort to adapt copy and markup for
two-column pages is probably counterproductive.
Remember that the main
goal of this package at this point is to produce ``correct'' draft
(or referee) format pages;
it is the responsibility of the editors and publishers to
produce publication format papers for the journals.

The {\tt aas2pp4} substyle does not
impose a format for the article's front matter,
although there is often merit in setting the title, author, abstract,
and keyword material on a separate page at full text width.
The author may supply the
\verb"\twocolumn" command wherever desired.
\begin{quote}
\verb"\twocolumn"
\end{quote}
Note that the two-column format begins at the point \verb"\twocolumn"
appears in the text, and if that point is before the front matter,
title and author and so forth will be typeset in two-column mode
along with the rest of the paper; that is how this manual is prepared,
for example.
For purposes of producing ``pretty'' output, it is probably desirable
to put the \verb"\twocolumn" command after the abstract and keywords,
just before the body of the paper.
If \verb"\twocolumn" is not specified explicitly in {\tt aas2pp4}
style documents, the introductory material of the paper will be set
in one-column mode; the first \verb"\section" command (presumably
demarcating the beginning of the main body of the article) will
engage the two-column mode.

\section{Additional documentation}  \label{docs}

The preceding detailed explanation of the markup commands in this
package has certain merit, but many authors will prefer to examine
the sample papers that are included with the style files.
The files of interest are described below.

The file \verb"sample1.tex" is a paper prepared
with the AAS\TeX\ package utilizing a minimal amount of markup.
The most salient thing to observe about this example is that,
apart from the document style declarations, no formatting
instructions are given in the file.

A more comprehensive example requiring nearly all of the capabilities
of the package (in terms of markup as well as formatting)
is in \verb"sample2.tex".
This file is annotated with comments that describe
the purpose of most of the markup.
\verb"sample2.tex" has three tables: one marked up using the 
{\tt deluxetable} environment; another table using the \LaTeX\ table 
environment; and the third table, \verb"samp2tbl.tex", produced as a 
camera-ready table using
either the \verb"apjpt4" or \verb"aj_pt4" style files.

A more complex example of the {\tt deluxetable} environment is presented
as a camera-ready table in the file \verb"complext.tex".

A set of three tables prepared as ``deluxe'' tables are contained in
the files \verb"table1.tex", \verb"table2.tex", and
\verb"table3.tex".  They can be formatted by running \LaTeX\ on
the \verb"samptbls.tex" file, into which they are included.
The document style in \verb"samptbls.tex" is {\tt apjpt4} in the
distribution, but this can be changed to \verb"aj_pt4" to see the
effects of differing requirements among journals.
Page breaks are explicitly indicated in \verb"table1.tex",
and they are set for the \apj\ style; due to the different
type size used in the \aj, the pages break somewhat erratically
when the \verb"aj_pt4" style is used.  It is left as an exercise
to the reader to make sense of this.

This user guide (\verb"aastex.tex")
is also marked up with the AAS\TeX\ package,
although it is not exemplary as a scientific paper.

A number of the markup commands described in the preceding
sections are standard \LaTeX\ commands, and the reader who is
unfamiliar with their syntax is referred to the \LaTeX\
manual (Lamport 1985) or another guide by Hahn (Hahn 1993) for details. 
A crib sheet listing all
the \LaTeX\ commands (and some pertinent plain \TeX\ commands)
with short descriptions of each is published by the \TeX\ Users
Group (Botway and Biemesderfer 1989).

Authors who wish to know the ins and outs of \TeX\ itself
should read the {\it\TeX book} (Knuth 1984), probably more than once.
There is a good deal of information about typography in general
in this source.  Many details of mathematical typography are
discussed in a book by Swanson (1971).

\begin{references}
\reference{Abt} Abt, H.  1990, \apj, 357, 1 (editorial)
\reference{Adobe} Adobe Systems, Inc.  1990, PostScript Language Reference
    Manual, Appendix H (Reading, MA: Addison-Wesley)
\reference{Biemes} Biemesderfer, C., and Hanisch R.  1989, \baas, 21, 780
\reference{Botway} Botway, L., and Biemesderfer, C.  1989,
    {\rm \LaTeX\ Command Summary} (Providence, RI: \TeX\ Users Group)
\reference{H1} Hahn, J. 1993, {\rm \LaTeX\ for Everyone} (Englewood Cliffs, NJ:
Prentice-Hall)
\reference{H2} Hanisch, R., and Biemesderfer, C.  1990, \baas, 22, 829
\reference{Knuth} Knuth, D.  1984, {\rm The \TeX book} (Reading, MA: Addison-Wesley)
\reference{Lamport} Lamport, L.  1985,
    {\rm \LaTeX: A Document Preparation System\/} (Reading, MA: Addison-Wesley)
\reference{Springer1} Springer-Verlag.  1989,
    {\rm Springer-Verlag \TeX\ AA macro package 1989}
    (Springer: Heidelberg)
\reference{Springer2} Springer-Verlag.  1990,
    {\rm Springer-Verlag \LaTeX\ AA macro package 1990}
    (Springer: Heidelberg)
\reference{Swanson} Swanson, E.  1979, {\rm Mathematics into Type} (Providence, RI:
    American Mathematical Society)
\end{references}

\newpage
\appendix
\section{Some Helpful \TeX\ Hints}

In the sections that follow, we review some essential procedures
that must be followed when preparing \TeX\ input.

\subsection{Running text}

Printing is different from typewriting, and \TeX\ is different
from other word processing tools.  This section consists of
reminders (admonitions) about things that require special
attention so that \TeX\ can format the input properly.

\begin{itemize}
\item The ends of words and sentences are marked by white space, and
it does not matter how many spaces are typed; one is as good as 100.
\TeX\ treats the end of a line in the input file as a space.

\item Paragraphs are separated by blank lines.
Do not hyphenate words in the input file;
\TeX\ takes care of hyphenation automatically.
Continue to hyphenate modifiers within a line of text, e.g.,
``author-prepared copy.''

\item Quotation marks should be typed as pairs of opening and
closing single quotes, e.g., {\tt ``quoted text''};
do not use double quotes ({\tt "bad form"}).

\item Do not underline.
In printing, text is emphasized by changing the type style,
usually to {\sl slanted\/} or {\it italic\/} type.

\item A number of common characters are interpreted as commands,
and these must be entered specially, by preceding them with
a backslash (\verb"\"): \$ \& \% \# \{ and \} must be typed
\verb"\$" \verb"\&" \verb"\%" \verb"\#" \verb"\{" and \verb"\}".

\item Authors should refrain from adding vertical or horizontal space.
Concentrate on the content of the document and identifying its
components with the structural markup commands.

\end{itemize}


\subsection{Math}

Mathematical expressions that are part of the running text are
delimited by a single dollar sign (\$),
e.g., \verb"$\pi r^2$" yields $\pi r^2$.
To get the appropriately sized superscript or subscript in the
roman font, use the \verb"\rm" command, e.g., 
\verb"$J_{\rm HF}(t)$" produces $J_{\rm HF}(t)$.

Displayed equations can be delimited in several ways.
The most concise markup is to bracket the equation between
\verb"\[" and \verb"\]" commands,
which is equivalent to placing the formula in a {\tt displaymath}
environment.  These markup commands will produce
{\sl unnumbered} equations.

Numbered equations can be typeset by typing the formula in
an {\tt equation} environment.
A series of related equations that need vertical alignment,
e.g., a derivation for which alignment on the equal sign (=) is desired,
can be typeset in an {\tt eqnarray} environment.

While it is possible for authors to assign their own equation numbers,
it is easier to let \LaTeX\ number them automatically.
By default, \LaTeX\ will number equations sequentially from the
beginning of the paper to the end.

It is sometimes appropriate for equation numbering sequences
to carry through sections of the paper only.
Equations can be numbered ({\it sec-eqn}) by placing the command
\verb"\eqsecnum" in the preamble of the document.

\subsection{Tables}

Tables are notoriously difficult to compose,
and great care and patience are usually required
before tabular information can be typeset satisfactorily.
Tables should be placed in separate table environments,
i.e., the tabular material must be enclosed within
\verb"\begin{table}" and \verb"\end{table}" commands.
Tables should have a title or caption and the correct
number of descriptive column headings.
A single horizontal rule should be set after the column headings
with the \verb"\tableline" command.
Do not insert any other horizontal or vertical
lines in the body of the table.

The {\tt deluxetable} environment is intended to make it easier
to prepare camera-ready tables, although
authors are encouraged to use the {\tt deluxetable} environment
for {\sl all\/} tables.
Authors who are using the deluxetable styles to prepare
camera-ready tables are urged to read the {\tt deluxetable}
section (section \ref{dte}) carefully and to look at the many example files
in the distribution.

Notes in tables should be marked by
\verb"\tablenotemark";
corresponding text should appear in a
\verb"\tablenotetext" command.

\subsection{Cross-referencing}

Cross-referencing equations, tables, and figures
in text depends upon the use of ``keys,'' which
are defined by the user.  The \verb"\label" command is used
to define cross-reference keys for \LaTeX; \verb"\ref" is used to
refer to them.  Keys are simply text strings that
serve to label equations, tables, and  figures, so that
they may be referred to symbolically in the text.
Authors should place \verb"\label" commands immediately
after the markup command that starts the structure being referenced.
For figure and table captions, place the \verb"\label" command {\sl inside}
the \verb"\caption", \verb"\tablecaption" or \verb"\figcaption" commands, 
e.g., \verb"\tablecaption{This is a caption.\label{tab1" \linebreak \verb"}}".
References to page numbers should {\sl not} be made.

\LaTeX\ keeps track of autonumbered counters and cross-reference
information by maintaining an auxiliary file in the same working
directory as the source file.  The auxiliary file will have an
extension of {\tt .aux}.  This file should not be deleted, since
subsequent \LaTeX\ processing uses the auxiliary data to resolve
references, etc.

The auxiliary file mechanism makes it necessary to run \LaTeX\
on a given source file more than once to
ensure that the cross-reference information has been properly resolved.
This is most evident when changes are made that affect the number
or the placement of equations, tables, and the like.
\LaTeX\ will typically issue a warning message that advises
the user to ``rerun to get cross-references right,''
in which case, \LaTeX\ should be run again.
If the error message appears after two successive \LaTeX\ runs,
it is likely that a reference has been made to an undefined label.

\section{AMS Symbols}    \label{ams} 

Authors are permitted to use the American Mathematical
Society extra symbol fonts, in addition to the special symbols provided by 
\LaTeX\ (see below).  The {\tt amssym} substyle file must be
included as an argument to the {\tt documentstyle} command if the AMS symbols
are to be used. 

\begin{quote}
\verb"\documentstyle[aasms4,amssym]{article}"
\end{quote}

The file {\tt ams.ps}, included with the AAS\TeX\ macro package, contains a 
list of the AMS extra symbol fonts that authors might find useful; this is 
a PostScript file that
can be printed to a local PostScript printer---use `lpr' in Unix---and does
not require the AMS symbols to be installed locally.

The AMS symbols are not distributed as part of the
AAS\TeX\ macro package or as part of the \TeX\ distribution, so authors will
need to check with their local \TeX\ administrators to see if these symbols
are available for their use.
The AMS symbols can be obtained from the American Mathematical Society by
anonymous ftp by connecting to {\tt ftp.shsu.edu} and looking in the directory
{\tt tex-archive/fonts/ams}.  The {\tt amssym} style file assumes that the
files {\tt amssym.def} and {\tt amssym.tex} are in the directory in which
the local implementation of \TeX\ looks for input files. 

If the AMS fonts and symbols have not been installed
locally it is still possible to use them in manuscripts that will be
submitted electronically---the local copy of the manuscript will 
not contain these fonts but the copies produced at the
editorial offices and at the Production Offices will be correct.



\section{Special symbols}

The \LaTeX\ language has a wide variety of special symbols
for which markup commands have already been defined.
These range from diacritics to exotic mathematical operators.

This section groups \LaTeX's symbols together more or less
according to function.
Some of these symbols are primarily for use in text;
most of them are mathematical symbols and can only be
used in \LaTeX's math mode.
These tables are excerpted from the \LaTeX\ Command Summary
(Botway 1989).

\begin{table}[h]
\caption{Text-mode accents}
\begin{center}
\begin{tabular}{*{2}{ll@{\hspace{4em}}}ll}
\`{o} & \verb"\`{o}" & \={o} & \verb"\={o}" & \t{oo} & \verb"\t{oo}" \\
\'{o} & \verb"\'{o}" & \.{o} & \verb"\.{o}" & \c{o}  & \verb"\c{o}" \\
\^{o} & \verb"\^{o}" & \u{o} & \verb"\u{o}" & \d{o}  & \verb"\d{o}" \\
\"{o} & \verb#\"{o}# & \v{o} & \verb"\v{o}" & \b{o}  & \verb"\b{o}" \\
\~{o} & \verb"\~{o}" & \H{o} & \verb"\H{o}" & & \\
\end{tabular}
\end{center}
\end{table}

\begin{table}[h]
\caption{National symbols}
\begin{center}
\begin{tabular}{*{2}{ll@{\hspace{4em}}}ll}
\oe & \verb"\oe" & \aa & \verb"\aa" & \l  & \verb"\l" \\
\OE & \verb"\OE" & \AA & \verb"\AA" & \L  & \verb"\L" \\
\ae & \verb"\ae" & \o  & \verb"\o"  & \ss & \verb"\ss" \\
\AE & \verb"\AE" & \O  & \verb"\O"  & & \\
\end{tabular}
\end{center}
\end{table}

\begin{table}[h]
\caption{Miscellaneous symbols}
\begin{center}
\begin{tabular}{*{2}{ll@{\hspace{4em}}}ll}
\dag  & \verb"\dag"  & \S & \verb"\S" & \copyright & \verb"\copyright" \\
\ddag & \verb"\ddag" & \P & \verb"\P" & \pounds    & \verb"\pounds" \\
\#    & \verb"\#"    & \$ & \verb"\$" & \%         & \verb"\%" \\
\&    & \verb"\&"    & \_ & \verb"\_" & & \\
\{    & \verb"\{"    & \} & \verb"\}" & & \\
\end{tabular}
\end{center}
\end{table}

\begin{table}[h]
\caption{Math-mode accents}
\begin{center}
\begin{tabular}{ll@{\hspace{4em}}ll}
$\hat{a}$   & \verb"\hat{a}"   & $\dot{a}$   & \verb"\dot{a}"   \\
$\check{a}$ & \verb"\check{a}" & $\ddot{a}$  & \verb"\ddot{a}"  \\
$\tilde{a}$ & \verb"\tilde{a}" & $\breve{a}$ & \verb"\breve{a}" \\
$\acute{a}$ & \verb"\acute{a}" & $\bar{a}$   & \verb"\bar{a}"   \\
$\grave{a}$ & \verb"\grave{a}" & $\vec{a}$   & \verb"\vec{a}"   \\
\end{tabular}
\end{center}
\end{table}

\begin{table}[h]
\caption{Greek letters (math mode)}
\begin{center}
\begin{tabular}{cl@{\hspace{3em}}cl}
$\alpha$   & \verb"\alpha"   & $\nu$      & \verb"\nu"      \\
$\beta$    & \verb"\beta"    & $\xi$      & \verb"\xi"      \\
$\gamma$   & \verb"\gamma"   & $o$        & \verb"o"        \\
$\delta$   & \verb"\delta"   & $\pi$      & \verb"\pi"      \\
$\epsilon$ & \verb"\epsilon" & $\rho$     & \verb"\rho"     \\
$\zeta$    & \verb"\zeta"    & $\sigma$   & \verb"\sigma"   \\
$\eta$     & \verb"\eta"     & $\tau$     & \verb"\tau"     \\
$\theta$   & \verb"\theta"   & $\upsilon$ & \verb"\upsilon" \\
$\iota$    & \verb"\iota"    & $\phi$     & \verb"\phi"     \\
$\kappa$   & \verb"\kappa"   & $\chi$     & \verb"\chi"     \\
$\lambda$  & \verb"\lambda"  & $\psi$     & \verb"\psi"     \\
$\mu$      & \verb"\mu"      & $\omega$   & \verb"\omega"   \\[4ex]

$\varepsilon$ & \verb"\varepsilon" & $\varsigma$ & \verb"\varsigma" \\
$\vartheta$   & \verb"\vartheta"   & $\varphi$   & \verb"\varphi"   \\
$\varrho$     & \verb"\varrho"     & & \\[4ex]

$\Gamma$  & \verb"\Gamma"  & $\Sigma$   & \verb"\Sigma"   \\
$\Delta$  & \verb"\Delta"  & $\Upsilon$ & \verb"\Upsilon" \\
$\Theta$  & \verb"\Theta"  & $\Phi$     & \verb"\Phi"     \\
$\Lambda$ & \verb"\Lambda" & $\Psi$     & \verb"\Psi"     \\
$\Xi$     & \verb"\Xi"     & $\Omega$   & \verb"\Omega"   \\
$\Pi$     & \verb"\Pi"     & & \\
\end{tabular}
\end{center}
\end{table}

\begin{table}
\caption{Binary operations (math mode)}
\begin{center}
\begin{tabular}{cl@{\hspace{3em}}cl}
$\pm$       & \verb"\pm"       & $\cap$             & \verb"\cap"           \\
$\mp$       & \verb"\mp"       & $\cup$             & \verb"\cup"           \\
$\setminus$ & \verb"\setminus" & $\uplus$           & \verb"\uplus"         \\
$\cdot$     & \verb"\cdot"     & $\sqcap$           & \verb"\sqcap"         \\
$\times$    & \verb"\times"    & $\sqcup$           & \verb"\sqcup"         \\
$\ast$      & \verb"\ast"      & $\triangleleft$    & \verb"\triangleleft"  \\
$\star$     & \verb"\star"     & $\triangleright$   & \verb"\triangleright" \\
$\diamond$  & \verb"\diamond"  & $\wr$              & \verb"\wr"            \\
$\circ$     & \verb"\circ"     & $\bigcirc$         & \verb"\bigcirc"       \\
$\bullet$   & \verb"\bullet"   & $\bigtriangleup$   & \verb"\bigtriangleup" \\
$\div$      & \verb"\div"      & $\bigtriangledown$ & \verb"\bigtriangledown" \\
$\lhd$      & \verb"\lhd"      & $\rhd$             & \verb"\rhd"           \\
$\vee$      & \verb"\vee"      & $\odot$            & \verb"\odot"          \\
$\wedge$    & \verb"\wedge"    & $\dagger$          & \verb"\dagger"        \\
$\oplus$    & \verb"\oplus"    & $\ddagger$         & \verb"\ddagger"       \\
$\ominus$   & \verb"\ominus"   & $\amalg$           & \verb"\amalg"         \\
$\otimes$   & \verb"\otimes"   & $\unlhd$           & \verb"\unlhd"         \\
$\oslash$   & \verb"\oslash"   & $\unrhd$           & \verb"\unrhd"         \\
\end{tabular}
\end{center}
\end{table}

\begin{table}
\caption{Relations (math mode)}
\begin{center}
\begin{tabular}{cl@{\hspace{4em}}cl}
$\leq$        & \verb"\leq"        & $\geq$        & \verb"\geq"        \\
$\prec$       & \verb"\prec"       & $\succ$       & \verb"\succ"       \\
$\preceq$     & \verb"\preceq"     & $\succeq$     & \verb"\succeq"     \\
$\ll$         & \verb"\ll"         & $\gg$         & \verb"\gg"         \\
$\subset$     & \verb"\subset"     & $\supset$     & \verb"\supset"     \\
$\subseteq$   & \verb"\subseteq"   & $\supseteq$   & \verb"\supseteq"   \\
$\sqsubset$   & \verb"\sqsubset"   & $\sqsupset$   & \verb"\sqsupset"   \\
$\sqsubseteq$ & \verb"\sqsubseteq" & $\sqsupseteq$ & \verb"\sqsupseteq" \\
$\in$         & \verb"\in"         & $\ni$         & \verb"\ni"         \\
$\vdash$      & \verb"\vdash"      & $\dashv$      & \verb"\dashv"      \\
$\smile$      & \verb"\smile"      & $\mid$        & \verb"\mid"        \\
$\frown$      & \verb"\frown"      & $\parallel$   & \verb"\parallel"   \\
$\neq$        & \verb"\neq"        & $\perp$       & \verb"\perp"       \\
$\equiv$      & \verb"\equiv"      & $\cong$       & \verb"\cong"       \\
$\sim$        & \verb"\sim"        & $\bowtie$     & \verb"\bowtie"     \\
$\simeq$      & \verb"\simeq"      & $\propto$     & \verb"\propto"     \\
$\asymp$      & \verb"\asymp"      & $\models$     & \verb"\models"     \\
$\approx$     & \verb"\approx"     & $\doteq$      & \verb"\doteq"      \\
              &                    & $\Join$       & \verb"\Join"       \\
\end{tabular}
\end{center}
\end{table}

\begin{table}
\caption{Variable-sized symbols (math mode)}
\begin{center}
\begin{displaymath}
\begin{array}{ccl@{\hspace{2em}}ccl}
\sum & \displaystyle \sum & \hbox{\verb"\sum"} &
	\bigcap & \displaystyle \bigcap & \hbox{\verb"\bigcap"} \\[.7ex]
\prod & \displaystyle \prod & \hbox{\verb"\prod"} &
	\bigcup & \displaystyle \bigcup & \hbox{\verb"\bigcup"} \\[.7ex]
\coprod & \displaystyle \coprod & \hbox{\verb"\coprod"} &
	\bigsqcup & \displaystyle \bigsqcup & \hbox{\verb"\bigsqcup"} \\[.7ex]
\int & \displaystyle \int & \hbox{\verb"\int"} &
	\bigvee & \displaystyle \bigvee & \hbox{\verb"\bigvee"} \\[.7ex]
\oint & \displaystyle \oint & \hbox{\verb"\oint"} &
	\bigwedge & \displaystyle \bigwedge & \hbox{\verb"\bigwedge"} \\[.7ex]
\bigodot & \displaystyle \bigodot & \hbox{\verb"\bigodot"} &
	\bigotimes & \displaystyle \bigotimes & \hbox{\verb"\bigotimes"} \\[.7ex]
\bigoplus & \displaystyle \bigoplus & \hbox{\verb"\bigoplus"} &
	\biguplus & \displaystyle \biguplus & \hbox{\verb"\biguplus"} \\
\end{array}
\end{displaymath}
\end{center}
\end{table}

\begin{table}
\caption{Delimiters (math mode)}
\begin{center}
\begin{tabular}{lc@{\hspace{2em}}lc}
$($            & \verb"("            & $)$            & \verb")" \\
$[$            & \verb"["            & $]$            & \verb"]" \\
$\{$           & \verb"\{"           & $\}$           & \verb"\}" \\
$\lfloor$      & \verb"\lfloor"      & $\rfloor$      & \verb"\rfloor" \\
$\lceil$       & \verb"\lceil"       & $\rceil$       & \verb"\rceil" \\
$\langle$      & \verb"\langle"      & $\rangle$      & \verb"\rangle" \\
$/$            & \verb"/"            & $\backslash$   & \verb"\backslash" \\
$\vert$        & \verb"\vert"        & $\Vert$        & \verb"\Vert" \\
$\uparrow$     & \verb"\uparrow"     & $\Uparrow$     & \verb"\Uparrow" \\
$\downarrow$   & \verb"\downarrow"   & $\Downarrow$   & \verb"\Downarrow" \\
$\updownarrow$ & \verb"\updownarrow" & $\Updownarrow$ & \verb"\Updownarrow" \\
\end{tabular}
\end{center}
\end{table}

\begin{table}
\caption{Function names (math mode)}
\begin{verbatim}
  \arccos   \csc    \ker      \min
  \arcsin   \deg    \lg       \Pr
  \arctan   \det    \lim      \sec
  \arg      \dim    \liminf   \sin
  \cos      \exp    \limsup   \sinh
  \cosh     \gcd    \ln       \sup
  \cot      \hom    \log      \tan
  \coth     \inf    \max      \tanh
\end{verbatim}
\end{table}

\begin{table}
\caption{Arrows (math mode)}
\begin{center}
\begin{tabular}{clcl}
$\leftarrow$ & \verb"\leftarrow" & $\longleftarrow$ & \verb"\longleftarrow" \\
$\Leftarrow$ & \verb"\Leftarrow" & $\Longleftarrow$ & \verb"\Longleftarrow" \\
$\rightarrow$ & \verb"\rightarrow" &
	$\longrightarrow$ & \verb"\longrightarrow" \\
$\Rightarrow$ & \verb"\Rightarrow" &
	$\Longrightarrow$ & \verb"\Longrightarrow" \\
$\leftrightarrow$ & \verb"\leftrightarrow" &
	$\longleftrightarrow$ & \verb"\longleftrightarrow" \\
$\Leftrightarrow$ & \verb"\Leftrightarrow" &
	$\Longleftrightarrow$ & \verb"\Longleftrightarrow" \\
$\mapsto$ & \verb"\mapsto" & $\longmapsto$ & \verb"\longmapsto" \\
$\hookleftarrow$ & \verb"\hookleftarrow" &
	$\hookrightarrow$ & \verb"\hookrightarrow" \\
$\leftharpoonup$ & \verb"\leftharpoonup" &
	$\rightharpoonup$ & \verb"\rightharpoonup" \\
$\leftharpoondown$ & \verb"\leftharpoondown" &
	$\rightharpoondown$ & \verb"\rightharpoondown" \\
$\rightleftharpoons$ & \verb"\rightleftharpoons" &
	$\leadsto$ & \verb"\leadsto" \\
$\uparrow$ & \verb"\uparrow" & $\Updownarrow$ & \verb"\Updownarrow"\\
$\Uparrow$ & \verb"\Uparrow" & $\nearrow$ & \verb"\nearrow"\\
$\downarrow$ & \verb"\downarrow" & $\searrow$ & \verb"\searrow" \\
$\Downarrow$ & \verb"\Downarrow" & $\swarrow$ & \verb"\swarrow"\\
$\updownarrow$ & \verb"\updownarrow" & $\nwarrow$ & \verb"\nwarrow"\\
\end{tabular}
\end{center}
\end{table}

\begin{table}
\caption{Miscellaneous symbols (math mode)}
\begin{center}
\begin{tabular}{cl@{\hspace{3em}}cl}
$\aleph$   & \verb"\aleph"   & $\prime$       & \verb"\prime"       \\
$\hbar$    & \verb"\hbar"    & $\emptyset$    & \verb"\emptyset"    \\
$\imath$   & \verb"\imath"   & $\nabla$       & \verb"\nabla"       \\
$\jmath$   & \verb"\jmath"   & $\surd$        & \verb"\surd"        \\
$\ell$     & \verb"\ell"     & $\top$         & \verb"\top"         \\
$\wp$      & \verb"\wp"      & $\bot$         & \verb"\bot"         \\
$\Re$      & \verb"\Re"      & $\|$           & \verb"\|"           \\
$\Im$      & \verb"\Im"      & $\angle$       & \verb"\angle"       \\
$\partial$ & \verb"\partial" & $\triangle$    & \verb"\triangle"    \\
$\infty$   & \verb"\infty"   & $\backslash$   & \verb"\backslash"   \\
$\Box$     & \verb"\Box"     & $\Diamond$     & \verb"\Diamond"     \\
$\forall$  & \verb"\forall"  & $\sharp$       & \verb"\sharp"       \\
$\exists$  & \verb"\exists"  & $\clubsuit$    & \verb"\clubsuit"    \\
$\neg$     & \verb"\neg"     & $\diamondsuit$ & \verb"\diamondsuit" \\
$\flat$    & \verb"\flat"    & $\heartsuit$   & \verb"\heartsuit"   \\
$\natural$ & \verb"\natural" & $\spadesuit$   & \verb"\spadesuit"   \\
$\mho$     & \verb"\mho"     & & \\
\end{tabular}
\end{center}
\end{table}

\end{document}



