%%%%%%%%%%%%%%%%%%%%%%%%%%%%%%%%%%%%%%%%%%%%%%%%%%%%%%%%%%%
%%%%%%%%%%%%%%%%%%%%%%%%%%%%%%%%%%%%%%%%%%%%%%%%%%%%%%%%%%%
%%%%%%
%%%%%%                     PHYZZM
%%%%%%
%%%%%%       This is built as an extention of the SLAC
%%%%%%   macro PHYZZX created by Vadim Kaplunovsky in
%%%%%%   1984.  As an extention, PHYZZM is fully
%%%%%%   compatible with the PHYZZX macro.  Any major
%%%%%%   command has a corresponding command in PHYZZM.
%%%%%%   Some sections are copied directly from PHYZZX
%%%%%%   to insure this correspondence.  Even though many
%%%%%%   of the macros have been altered to such an extent
%%%%%%   that they barely resemble their PHYZZX counter part,
%%%%%%   they will still behave like the PHYZZX macro.
%%%%%%
%%%%%%      PHYZZM
%%%%%%         Created: 16 July 1986
%%%%%%         by: Mark T. Schutze   mods by B. Ball et.al.
%%%%%%
%%%%%%         5/31/89: D. Williams fixfoots.tex
%%%%%%                  \subsection now works with \unnumberedchapters
%%%%%%                  \doeack{grant no} now generalized
%%%%%%                  \normalspace forced in figs, tables, and contents lists
%%%%%%
%%%%%%         7/24/89: R.A. Dunn--the format for IEEE publications
%%%%%%                  has been added.  To choose this option, type
%%%%%%                  either \IEEE or \ieee at the beginning of the
%%%%%%                  TeX document.
%%%%%%
%%%%%%	       10/16/91:Add \label and related stuff.  B. Ball
%%%%%%	       10/25/91:Add \faxnumber	B. Ball
%%%%%%         8/19/92: Add variable letter and label hsize, vsize.  B. Ball
%%%%%%%%%%%%%%%%%%%%%%%%%%%%%%%%%%%%%%%%%%%%%%%%%%%%%%%%%%%
%%%%%%%%%%%%%%%%%%%%%%%%%%%%%%%%%%%%%%%%%%%%%%%%%%%%%%%%%%%

%\input tex:macros.tex            %%% MTS Make TeX compatible with MTS
%macropackage=phyzzx.tex          %%% MTS Enable the use of Plain TeX
%\input phyzzx.tex                %%% input Phyzzx macros
%                                 %%% remember to reactivate this
%                                 %%%   for MTS use
%\activatefontenv\PGFfiftyfour    %%% MTS activate standard font

\def\unlock{
 \catcode`@=11 }% This allows us to modify PLAIN macros.
\unlock
\def\lock{\catcode`@=12 }% at signs are no longer letters
\def\MYPHYZZM{\input myphyzzm.tex}

%%% special local internal skips
\newskip\@@@@@
\newskip\@@@@@two
\newskip\t@mp
%%%%%
%%%%%         FONT DEFINITIONS AND SETS
%%%%%

     %        \ten<>, \seven<>, \five<> defined in PLAIN TeX
     %            for rm, bf, i, sy
     %            all \ten<> are defined in PLAIN TeX

 \font\twentyfourrm=cmr10               scaled\magstep4
 \font\seventeenrm=cmr10                scaled\magstep3
 \font\fourteenrm=cmr10                 scaled\magstep2
 \font\twelverm=cmr10                   scaled\magstep1
 \font\ninerm=cmr9            \font\sixrm=cmr6
 \font\twentyfourbf=cmbx10              scaled\magstep4
 \font\seventeenbf=cmbx10               scaled\magstep3
 \font\fourteenbf=cmbx10                scaled\magstep2
 \font\twelvebf=cmbx10                  scaled\magstep1
 \font\ninebf=cmbx9            \font\sixbf=cmbx5
 \font\twentyfouri=cmmi10 scaled\magstep4  \skewchar\twentyfouri='177
 \font\seventeeni=cmmi10  scaled\magstep3  \skewchar\seventeeni='177
 \font\fourteeni=cmmi10   scaled\magstep2  \skewchar\fourteeni='177
 \font\twelvei=cmmi10     scaled\magstep1  \skewchar\twelvei='177
 \font\ninei=cmmi9                         \skewchar\ninei='177
 \font\sixi=cmmi6                          \skewchar\sixi='177
 \font\twentyfoursy=cmsy10 scaled\magstep4 \skewchar\twentyfoursy='60
 \font\seventeensy=cmsy10  scaled\magstep3 \skewchar\seventeensy='60
 \font\fourteensy=cmsy10   scaled\magstep2 \skewchar\fourteensy='60
 \font\twelvesy=cmsy10     scaled\magstep1 \skewchar\twelvesy='60
 \font\ninesy=cmsy9                        \skewchar\ninesy='60
 \font\sixsy=cmsy6                         \skewchar\sixsy='60
 \font\twentyfourex=cmex10  scaled\magstep4
 \font\seventeenex=cmex10   scaled\magstep3
 \font\fourteenex=cmex10    scaled\magstep2
 \font\twelveex=cmex10      scaled\magstep1
 \font\elevenex=cmex10      scaled\magstephalf
 \font\twentyfoursl=cmsl10  scaled\magstep4
 \font\seventeensl=cmsl10   scaled\magstep3
 \font\fourteensl=cmsl10    scaled\magstep2
 \font\twelvesl=cmsl10      scaled\magstep1
 \font\ninesl=cmsl9
 \font\twentyfourit=cmti10 scaled\magstep4
 \font\seventeenit=cmti10  scaled\magstep3
 \font\fourteenit=cmti10   scaled\magstep2
 \font\twelveit=cmti10     scaled\magstep1
 \font\twentyfourtt=cmtt10 scaled\magstep4
 \font\twelvett=cmtt10     scaled\magstep1
 \font\twentyfourcp=cmcsc10 scaled\magstep4
 \font\twelvecp=cmcsc10    scaled\magstep1
 \font\tencp=cmcsc10
 \font\hepbig=cmdunh10     scaled\magstep5
 \font\hep=cmdunh10        scaled\magstep0
 \newfam\cpfam
 \font\tenfib=cmr10  % quick fix for a missing font
 \newcount\f@ntkey            \f@ntkey=0
 \def\samef@nt{\relax \ifcase\f@ntkey \rm \or\oldstyle \or\or
          \or\it \or\sl \or\bf \or\tt \or\caps \fi }

 %%%%%
 %%%%%           FONT SETS
 %%%%%
 \def\twentyfourpoint{\relax
     \textfont0=\twentyfourrm           \scriptfont0=\seventeenrm
                                        \scriptscriptfont0=\fourteenrm
     \def\rm{\fam0 \twentyfourrm \f@ntkey=0 }\relax
     \textfont1=\twentyfouri            \scriptfont1=\seventeeni
                                        \scriptscriptfont1=\fourteeni
      \def\oldstyle{\fam1 \twentyfouri\f@ntkey=1 }\relax
     \textfont2=\twentyfoursy           \scriptfont2=\seventeensy
                                        \scriptscriptfont2=\fourteensy
     \textfont3=\twentyfourex           \scriptfont3=\seventeenex
                                        \scriptscriptfont3=\fourteenex
     \def\it{\fam\itfam \twentyfourit\f@ntkey=4 }\textfont\itfam=\twentyfourit
     \def\sl{\fam\slfam \twentyfoursl\f@ntkey=5 }\textfont\slfam=\twentyfoursl
                                        \scriptfont\slfam=\seventeensl
     \def\bf{\fam\bffam \twentyfourbf\f@ntkey=6 }\textfont\bffam=\twentyfourbf
                                        \scriptfont\bffam=\seventeenbf
                                        \scriptscriptfont\bffam=\fourteenbf
     \def\tt{\fam\ttfam \twentyfourtt \f@ntkey=7 }\textfont\ttfam=\twentyfourtt
     \def\caps{\fam\cpfam \twentyfourcp \f@ntkey=8 }
                                        \textfont\cpfam=\twentyfourcp
     \setbox\strutbox=\hbox{\vrule height 23pt depth 8pt width\z@}
     \samef@nt}
 \def\fourteenpoint{\relax
     \textfont0=\fourteenrm          \scriptfont0=\tenrm
     \scriptscriptfont0=\sevenrm
      \def\rm{\fam0 \fourteenrm \f@ntkey=0 }\relax
     \textfont1=\fourteeni           \scriptfont1=\teni
     \scriptscriptfont1=\seveni
      \def\oldstyle{\fam1 \fourteeni\f@ntkey=1 }\relax
     \textfont2=\fourteensy          \scriptfont2=\tensy
     \scriptscriptfont2=\sevensy
     \textfont3=\fourteenex          \scriptfont3=\twelveex
     \scriptscriptfont3=\tenex
     \def\it{\fam\itfam \fourteenit\f@ntkey=4 }\textfont\itfam=\fourteenit
     \def\sl{\fam\slfam \fourteensl\f@ntkey=5 }\textfont\slfam=\fourteensl
     \scriptfont\slfam=\tensl
     \def\bf{\fam\bffam \fourteenbf\f@ntkey=6 }\textfont\bffam=\fourteenbf
     \scriptfont\bffam=\tenbf     \scriptscriptfont\bffam=\sevenbf
     \def\tt{\fam\ttfam \twelvett \f@ntkey=7 }\textfont\ttfam=\twelvett
     \def\caps{\fam\cpfam \twelvecp \f@ntkey=8 }\textfont\cpfam=\twelvecp
     \setbox\strutbox=\hbox{\vrule height 12pt depth 5pt width\z@}
     \samef@nt}

 \def\twelvepoint{\relax
     \textfont0=\twelverm          \scriptfont0=\ninerm
     \scriptscriptfont0=\sixrm
      \def\rm{\fam0 \twelverm \f@ntkey=0 }\relax
     \textfont1=\twelvei           \scriptfont1=\ninei
     \scriptscriptfont1=\sixi
      \def\oldstyle{\fam1 \twelvei\f@ntkey=1 }\relax
     \textfont2=\twelvesy          \scriptfont2=\ninesy
     \scriptscriptfont2=\sixsy
     \textfont3=\twelveex          \scriptfont3=\elevenex
     \scriptscriptfont3=\tenex
     \def\it{\fam\itfam \twelveit \f@ntkey=4 }\textfont\itfam=\twelveit
     \def\sl{\fam\slfam \twelvesl \f@ntkey=5 }\textfont\slfam=\twelvesl
     \scriptfont\slfam=\ninesl
     \def\bf{\fam\bffam \twelvebf \f@ntkey=6 }\textfont\bffam=\twelvebf
     \scriptfont\bffam=\ninebf     \scriptscriptfont\bffam=\sixbf
     \def\tt{\fam\ttfam \twelvett \f@ntkey=7 }\textfont\ttfam=\twelvett
     \def\caps{\fam\cpfam \twelvecp \f@ntkey=8 }\textfont\cpfam=\twelvecp
     \setbox\strutbox=\hbox{\vrule height 10pt depth 4pt width\z@}
     \samef@nt}

 \def\tenpoint{\relax
     \textfont0=\tenrm          \scriptfont0=\sevenrm
     \scriptscriptfont0=\fiverm
     \def\rm{\fam0 \tenrm \f@ntkey=0 }\relax
     \textfont1=\teni           \scriptfont1=\seveni
     \scriptscriptfont1=\fivei
     \def\oldstyle{\fam1 \teni \f@ntkey=1 }\relax
     \textfont2=\tensy          \scriptfont2=\sevensy
     \scriptscriptfont2=\fivesy
     \textfont3=\tenex          \scriptfont3=\tenex
     \scriptscriptfont3=\tenex
     \def\it{\fam\itfam \tenit \f@ntkey=4 }\textfont\itfam=\tenit
     \def\sl{\fam\slfam \tensl \f@ntkey=5 }\textfont\slfam=\tensl
     \def\bf{\fam\bffam \tenbf \f@ntkey=6 }\textfont\bffam=\tenbf
     \scriptfont\bffam=\sevenbf     \scriptscriptfont\bffam=\fivebf
     \def\tt{\fam\ttfam \tentt \f@ntkey=7 }\textfont\ttfam=\tentt
     \def\caps{\fam\cpfam \tencp \f@ntkey=8 }\textfont\cpfam=\tencp
     \setbox\strutbox=\hbox{\vrule height 8.5pt depth 3.5pt width\z@}
     \samef@nt}

%%%%
%%%% Special notations and symbols
%%%%

\def\Lag{{\cal L}}   %%% Lagrangian symbol
\def\SM{Standard\ Model}
\def\qcd{{\rm QCD}}
\def\trouble{{\rm YOU DO NOT SAY}}%quick fix for a jam
 \newtoks\date
 \def\monthname{\relax\ifcase\month 0/\or January\or February\or
    March\or April\or May\or June\or July\or August\or September\or
    October\or November\or December\else\number\month/\fi}
 \date={\monthname\ \number\day, \number\year}
 \def\today{\the\day\ \monthname\ \the\year}
 \def\ie{\hbox{\it i.e.}}     \def\etc{\hbox{\it etc.}}
 \def\eg{\hbox{\it e.g.}}     \def\cf{\hbox{\it cf.}}
 \def\dash{\hbox{---}}        \def\etal{\hbox{\it et.al.}}
 \def\\{\relax\ifmmode\backslash\else$\backslash$\fi}
 \def\nextline{\unskip\nobreak\hskip\parfillskip\break}
 \def\subpar{\vadjust{\allowbreak\vskip\parskip}\nextline}
 \def\journal#1&#2,#3(#4){\unskip \enskip {\sl #1}~{\bf #2}, {\rm #3 (19#4)}}
 \def\cropen#1{\crcr\noalign{\vskip #1}}
 \def\crr{\cropen{10pt}}
 \def\topspace{\hrule height 0pt depth 0pt \vskip}
\def\nullbox#1#2#3{\vbox to #1
     {\vss\vtop to#2
       {\vss\hbox to #3 {}}}}
\def\n@ll{\nullbox{5pt}{3pt}{2pt}}    % NEED TO UNLOCK '@' to use
%\def\undertext#1{\vtop{\hbox{#1}\kern 1pt \hrule}}
\def\UNRTXT#1{\vtop{\hbox{#1}\kern 1pt \hrule}}
\def\undertext#1{\ifvmode\ifinner \UNRTXT{#1}
                         \else    $\hbox{\UNRTXT{#1}}$
                         \fi
                 \else   \ifmmode \hbox{\UNRTXT{#1}}
                         \else    \UNRTXT{#1}
                         \fi
                 \fi }

%%%%%%%
%%%%%%%             HIGH ENERGY NOTATIONS
%%%%%%%

%%%%%
%%%%%   GAMMA MATRICIES
%%%%%

\def\gsmu{\gamma_\mu }  %% subscripted notation
\def\gmu{\gamma^\mu } %% superscripted notation

%%%%%
%%%%%   BRANCHING RATIOS
%%%%%

\def\br#1#2{{\rm BR}\of{#1\to #2}}

%%%%%
%%%%%          Electron volt notations
%%%%%

\def\ev{{\rm e\kern-.1100em V}}
\def\tev{{\rm T\kern-.1000em \ev}}
\def\gev{{\rm G\kern-.1000em \ev}}
\def\mev{{\rm M\kern-.1000em \ev}}
\def\kev{{\rm K\kern-.1000em \ev}}

%%%%%
%%%%%           MASS SCALE NOTATION
%%%%%

\def\csq{/c^2}
\def\gevc{\gev\csq}
\def\mevc{\mev\csq}
\def\evc{\ev\csq}

%%%%%
%%%%%           PARTICLE NOTATIONS
%%%%%

\def\eright{e^-\kern-.3em\low{R}}      %% right handed electron
\def\eleft{e^-\kern-.3em\low{L}}       %% left handed electron
\def\nue{\nu\kern-.3em\low{e}{}}
\def\numu{\nu\low{\mu}}
\def\nutau{\nu\low{\tau}}

%%%%%%%%
%%%%%%%%            MATHEMATIC NOTATION
%%%%%%%%

%%%
%%%           SUB/SUPER SCRIPTS
%%%

\def\low#1{\kern-0.11em\lower0.4em\hbox{$\scriptstyle #1 $}\hskip-0.08em}
\def\up#1{\kern-0.45em\raise0.4em\hbox{$\scriptstyle #1 $}\hskip0.29em}

%%%
%%%             OPERATIONS
%%%

 \def\bra#1{\left\langle #1\right\vert}
 \def\ket#1{\left\vert #1\right\rangle}
 \def\VEV#1{\left\langle #1\right\rangle}
 \def\pidim#1{\mathop{\left( 2\pi\right)^{#1}}\nolimits}
 \def\Tr{\mathop{\rm Tr}\nolimits}
 \let\int=\intop         \let\oint=\ointop
 \def\prop{\mathrel{{\mathchoice{\pr@p\scriptstyle}{\pr@p\scriptstyle}{
                 \pr@p\scriptscriptstyle}{\pr@p\scriptscriptstyle} }}}
 \def\pr@p#1{\setbox0=\hbox{$\cal #1 \char'103$}
    \hbox{$\cal #1 \char'117$\kern-.4\wd0\box0}}
 \let\sec@nt=\sec
 \def\sec{\relax\ifmmode\let\n@xt=\sec@nt\else\let\n@xt\section\fi\n@xt}
 \def\@versim#1#2{\lower0.2ex\vbox{\baselineskip\z@skip\lineskip\z@skip
   \lineskiplimit\z@\ialign{$\m@th#1\hfil##\hfil$\crcr#2\crcr\sim\crcr}}}
\def\lsim{\mathrel{\mathpalette\@versim<}}%Similar or less than%
\def\gsim{\mathrel{\mathpalette\@versim>}}%similar or greater than%
\let\siml=\lsim
\let\simg=\gsim
\def\barr#1{ \overline #1 }
   %%% bar superscript %%%

%%%%
%%%% DERIVATIVE OPERATORS
%%%%

\def\del{{\nabla}}
\def\Differ#1#2{\der{#1}{#2}}
\def\differ#1{\der{}{#1}}
\def\partder#1#2{{\partial #1\over\partial #2}}
\def\part#1#2{\partial^{#1}\kern-.1667em\hbox{$#2$}}
\def\der#1#2{d^{#1}\kern-.1067em\hbox{$#2$}}
\def\dmu{\partial^\mu}          %%% four derivative super
\def\dsmu{{\partial_\mu}}       %%% four derivative sub
\def\coder{{\cal D}}
\def\Dmu{\coder^\mu} \def\Dsmu{\coder_\mu }
\def\delt#1{{\delta\kern-.15em #1 }}

%%%%
%%%%           FOUR VECTOR NOTATIONS.
%%%%
%%%%     NOTE:  FOUR DERIVATIVE OPERATORS ARE ABOVE
%%%%

\def\amu{a^{\mu}}
\def\asmu{a_{\mu}}
\def\Amu{A^{\mu}}
\def\Asmu{A_{\mu}}
\def\Wsmu{W_{\mu }}
\def\Wmu{{W^{\mu }}}
\def\Wmuj{{W_j^\mu }}
\def\Wmuk{{W_k^\mu }}
\def\Wmuip{{W_i^{\prime\mu}}}
\def\Zmu{{Z^\mu }}
\def\Zsmu{{Z_\mu }}
\def\Dmup{{D^{\prime\mu}}}

%%%%
%%%%        OPERATORS
%%%%

%  Box operator
\def\sqr#1#2{{\vcenter{\hrule height.#2pt
    \hbox{\vrule width.#2pt height#1pt \kern#1pt
    \vrule width.#2pt}
    \hrule height.#2pt}}}
\def\bx{\mathchoice\sqr68\sqr68\sqr{4.2}6\sqr{3}6}

%%%%%
%%%%%       SPECIAL FUNCTIONS
%%%%%

\def\vector#1{{\overrightarrow {#1}} }  %% redefine the vector notation
                                        %%   arrow covers entire symbol
\def\cross#1#2{{#1\times #2}}   %% forms the cross product
\def\dotproduct#1#2{{{#1}\cdot {#2}} } %% forms the dot product
\def\trans#1#2#3{#1 \rightarrow\  {#1} \,'=#1 #2 #3}
     %%% this yields the form A --> A' = A B C
     %%% where A(#1) is the function being
     %%% being transformed, B(#2) is the
     %%% operation in the transform,
     %%% C(#3) is the phase shift
\def\transwave#1#2#3{ #1#2 \rightarrow\ #1\,'#2 = #3\, #1 #2}
     %%%  where #1 is the function name
     %%%        #2 is the function variables
     %%%        #3 is the phase shift
\def\gmumat#1#2{\barr{#1}\gmu \kern0.008in {#2}}
     %%%  where #1 is the antiparticle in the Lagrangian
     %%%        #2 is the particle in the Lagrangian
\def\pgmumat#1{\gmumat{#1} {#1}}
\def\absvalue#1{{|#1|}}
\let\absv=\absvalue
\def\absq#1{{{|#1|}^2}}
\def\of#1{\mkern-1.8mu{\left( #1 \right)}}
\def\coeff#1#2{{#1 \over #2 }}
   %% Builds fractions in math modes
\def\indices#1#2{\singlespace \left.\matrix{#1}\right\}= #2}
   %% a group of indicies #1 assigned allowed values #2
   %% where #2 is in matrix notation
\def\rightbracearrow{\raise0.5em\hbox{$|$}\mkern-8.0mu\rightarrow}
   %% builds the construction |->    for nuclear decays
\let\rbracearrow=\rightbracearrow

%%%%%%%
%%%%%%%          GROUP THEORY NOTATIONS
%%%%%%%

%%%%
%%%%             SYMMETRY GROUP NOTATION
%%%%

\def\U#1{U\kern-.25em \left( #1 \right) }
    % Normal group theory notation U(n)
\def\su#1{S\U #1 }
    % Special group theory notation SU(N)

%%%%
%%%%           SYMMETRY FIELD NOTATION
%%%%

%% U(1) field notation

\def\Bmu{B^\mu }   %% superscripted notation
\def\Bsmu{B_\mu }  %% subscripted notation

%% U(2) field notation

\def\Wimu{W^i_\mu} %% subscripted notation
\def\Wmui{W^\mu_i} %% superscripted notation

%% U(3) field notation

\def\Gamu{G_\mu^a } %% subscripted notation
\def\Gmua{G_a^\mu } %% superscripted notation

%% Color Space

\def\colorsinglet{r\barr{r} + g\barr{ g} + b\barr{ b} }

%%%%%%
%%%%%%   These partial remains are from the phyzzx macro
%%%%%%

%%%%             LINE SPACING MACROS
%%%%     COPIED FROM THE SLAC PHYZZX MACRO PACKAGE
%%%%

 \normalbaselineskip = 20pt plus 0.2pt minus 0.1pt
 \normallineskip = 1.5pt plus 0.1pt minus 0.1pt
 \normallineskiplimit = 1.5pt
 \newskip\normaldisplayskip
    \normaldisplayskip = 20pt plus 5pt minus 10pt
 \newskip\normaldispshortskip
    \normaldispshortskip = 6pt plus 5pt
 \newskip\normalparskip
    \normalparskip = 6pt plus 2pt minus 1pt
 \newskip\skipregister
    \skipregister = 5pt plus 2pt minus 1.5pt
 \newif\ifsingl@    \newif\ifdoubl@
 \newif\ifb@@kspace   \b@@kspacefalse
 \newif\iftwelv@    \twelv@true
 \newif\ifp@genum

%%%%
%%%%        LINE SPACING CHOICES
%%%%

 \def\singlespace{\singl@true\doubl@false\b@@kspacefalse\spaces@t}
 \def\doublespace{\singl@false\doubl@true\b@@kspacefalse\spaces@t}
 \def\normalspace{\singl@false\doubl@false\b@@kspacefalse\spaces@t}
 \def\bookspace{\b@@kspacetrue\singl@false\doubl@false\spaces@t}
 \def\tablespace{\singlespace}
 \def\smashspace{\singl@false\doubl@false\b@@kspacefalse\subspaces@t3:8;}

%%%%
%%%%        FONT CHOICES
%%%%

 \def\Tenpoint{\tenpoint\twelv@false\spaces@t}
 \def\Twelvepoint{\twelvepoint\twelv@true\spaces@t}

%%%
%%%       SPACING CONTROLS
%%%

 \def\spaces@t{\relax
       \iftwelv@ \ifsingl@\subspaces@t3:4;\else\subspaces@t1:1;\fi
        \else \ifsingl@\subspaces@t1:2;\else\subspaces@t4:5;\fi \fi
                    %%%%%%original 3:5;
       \ifb@@kspace
          \baselineskip=14pt
       \fi
       \ifdoubl@ \multiply\baselineskip by 5
          \divide\baselineskip by 4 \fi }
 \def\subspaces@t#1:#2;{
       \baselineskip = \normalbaselineskip
       \multiply\baselineskip by #1 \divide\baselineskip by #2
       \lineskip = \normallineskip
       \multiply\lineskip by #1 \divide\lineskip by #2
       \lineskiplimit = \normallineskiplimit
       \multiply\lineskiplimit by #1 \divide\lineskiplimit by #2
       \parskip = \normalparskip
       \multiply\parskip by #1 \divide\parskip by #2
       \abovedisplayskip = \normaldisplayskip
       \multiply\abovedisplayskip by #1 \divide\abovedisplayskip by #2
       \belowdisplayskip = \abovedisplayskip
       \abovedisplayshortskip = \normaldispshortskip
       \multiply\abovedisplayshortskip by #1
         \divide\abovedisplayshortskip by #2
       \belowdisplayshortskip = \abovedisplayshortskip
       \advance\belowdisplayshortskip by \belowdisplayskip
       \divide\belowdisplayshortskip by 2
       \smallskipamount = \skipregister
       \multiply\smallskipamount by #1 \divide\smallskipamount by #2
       \medskipamount = \smallskipamount \multiply\medskipamount by 2
       \bigskipamount = \smallskipamount \multiply\bigskipamount by 4 }
 \def\normalbaselines{ \baselineskip=\normalbaselineskip
    \lineskip=\normallineskip \lineskiplimit=\normallineskip
    \iftwelv@\else \multiply\baselineskip by 4 \divide\baselineskip by 5
      \multiply\lineskiplimit by 4 \divide\lineskiplimit by 5
      \multiply\lineskip by 4 \divide\lineskip by 5 \fi }

%%%%%%
%%%%%%           CHAPTERS, SECTIONS, SUBSECTIONS
%%%%%%
%%%%%%           This section here the labeling
%%%%%%              of chapters, sections and subsections
%%%%%%           This section also builds the table of
%%%%%%              contents
%%%%%%

%%%%
%%%%                  PARAMETERS---LOTS OF THEM
%%%%

 \newskip\tablelineskip           \tablelineskip=0.7in
 \newskip\figurelineskip          \figurelineskip=1.55in
 \newskip\tablelinelength     \tablelinelength=4.5in
 \newskip\tabledotvskip       \tabledotvskip=-0.359in
 \newskip\tabledothskip       \tabledothskip=-0.306in%
 \newskip\bookchapterskip     \bookchapterskip=1.0in
 \newcount\chapternumber    \chapternumber=0
 \newcount\appendixnumber   \appendixnumber=0
 \newcount\sectionnumber    \sectionnumber=0
 \newcount\subsectionnumber \subsectionnumber=0
 \newcount\equanumber       \equanumber=1
 \newcount\problemnumber    \problemnumber=0
 \newcount\figurecount      \figurecount=1
 \newcount\conpage          \conpage=0
 \let\chapterlabel=0
 \newtoks\constyle          \constyle={\Number}
 \newtoks\appendixstyle     \global\appendixstyle={\Alphabetic}
 \newtoks\chapterstyle      \chapterstyle={\Number}
 \newtoks\subsecstyle       \subsecstyle={\alphabetic}
 \newskip\chapterskip       \chapterskip=\bigskipamount
 \newskip\sectionskip       \sectionskip=\medskipamount
 \newskip\headskip          \headskip=8pt plus 3pt minus 3pt
 \newif\ifsp@cecheck
 \newif\iffirst@ppendix       \global\first@ppendixtrue
 \newdimen\chapterminspace    \chapterminspace=15pc
 \newdimen\sectionminspace    \sectionminspace=8pc
 \interlinepenalty=50
 \interfootnotelinepenalty=5000
 \predisplaypenalty=9000
 \postdisplaypenalty=500
 \newwrite\tableconwrite
 \newbox\tableconbox
 \newif\iftabl@conlist
 \newwrite\figwrite
 \newwrite\figurewrite
 \newif\iffigur@list
 \newwrite\eqnwrite
 \newif\if@qlist
 \newif\ifeqlo@d            \eqlo@dfalse
 \newwrite\tablewrite
 \newwrite\tableswrite
 \newif\iftabl@list
 \newwrite\appendixwrite
 \newif\if@ppendix
 \newcount\@ppcharnumber     \@ppcharnumber=64
 \newcount\referencecount \newbox\referencebox
 \newcount\tablecount
 \newif\ifindex      \indexfalse%%% used in the index macros
 \newwrite\indexwrite
 \newbox\indexbox
 \newskip\indexskip
 \newif\ifmanualpageno   \manualpagenofalse
 \newdimen\letterhsize       \letterhsize=6.5in
 \newdimen\lettervsize       \lettervsize=8.5in
 \newdimen\labelhsize        \labelhsize=8.5in
 \newdimen\labelvsize        \labelvsize=11.0in
 \newdimen\letterhoffset     \letterhoffset=0.0in
 \newdimen\lettervoffset     \lettervoffset=0.250in

%%%%
%%%%           LABLE STYLE CONTROLS
%%%%

 \def\Number#1{\number #1}
 \def\makel@bel{\xdef\chapterlabel{%
     \the\chapterstyle{\the\chapternumber}.}}
 \def\sectionlabel{\number\sectionnumber }
 \def\subseclabel{{\the\subsecstyle{\the\subsectionnumber}. }}
 \def\alphabetic#1{\count255='140 \advance\count255 by #1\char\count255}
 \def\Alphabetic#1{\count255=64 \advance\count255 by #1\char\count255}
 \def\Roman#1{\uppercase\expandafter{\romannumeral #1}}
 \def\roman#1{\romannumeral #1}

%%%%
%%%%           PAGING CONTROL MACROS
%%%%    INCLUDING HEADLINES AND FOOTLINES
%%%%

 \countdef\pagenumber=1  \pagenumber=1
 \def\advancepageno{\global\advance\pageno by 1
    \ifnum\pagenumber<0 \global\advance\pagenumber by -1
     \else\global\advance\pagenumber by 1 \fi \global\frontpagefalse }
\def\pagefolio#1{\ifnum#1<0 \romannumeral-#1
            \else \number#1 \fi }

\def\folio{\pagefolio{\pagenumber}}
 \def\pagecontents{
    \ifvoid\topins\else\unvbox\topins\vskip\skip\topins\fi
    \dimen@ = \dp255 \unvbox255
    \ifvoid\footins\else\vskip\skip\footins\footrule\unvbox\footins\fi
    \ifr@ggedbottom \kern-\dimen@ \vfil \fi }
 \def\makeheadline{\vbox to 0pt{ \hfuzz=30pt \skip@=\topskip
       \advance\skip@ by -12pt \advance\skip@ by -2\normalbaselineskip
       \vskip\skip@ \line{\vbox to 12pt{}\the\headline\hfill} \vss
       }\nointerlineskip}
 \def\makefootline{\baselineskip = 1.5\normalbaselineskip
                  \line{\the\footline}}
 \def\nopagenumbers{\p@genumfalse}
 \def\pagenumbers{\p@genumtrue}

%%% Listing break command expansion

 \def\tableconbreak{\break}
 \def\tableconbreakspace{\ }
 \def\tableconbreakfill{\hfill\break}
 \def\tableconbreakfilltab{\hfill\break \hskip\parindent}
 \def\tnullbreak{}

%% listing break command

 \let\conbreak=\tableconbreak
 \let\tbreak=\tableconbreakspace

%%%%%
%%%%% CHAPTER, SECTION, SUBSECTION, ITEM AND POINT MACROS
%%%%%

%%%
%%%         SET UP PARAMETERS FOR PAPER DIVISIONS
%%%

 \def\splitprep{\global\newlinechar=`\^^J}
 \def\splitprepend{\global\newlinechar=-1}


 \def\consection#1{
    \iftabl@conlist
       \splitprep
       \immediate\write\tableconwrite{\string\immediate
                                      \string\spacecheck\sectionminspace
                                     }
       \immediate\write\tableconwrite{\string\vskip0.25in
                         \string\titlestyle{\string\bf\ #1 }%
                         \string\vskip0.0in
                         \string\nullbox{1pt}{1pt}{1pt}%
                                     }
       \splitprepend
    \fi}
 \def\unnumberedchapters{\let\makel@bel=\relax \let\chapterlabel=\relax
             \let\sectionlabel=\relax \let\subseclabel=\relax \equanumber=-1 }
 \def\titlestyle#1{\par\begingroup \interlinepenalty=9999
      \leftskip=0.02\hsize plus 0.23\hsize minus 0.02\hsize
      \rightskip=\leftskip \parfillskip=0pt
      \hyphenpenalty=9000 \exhyphenpenalty=9000
      \tolerance=9999 \pretolerance=9000
      \spaceskip=0.333em \xspaceskip=0.5em
      \iftwelv@\fourteenpoint\else\twelvepoint\fi
    \noindent #1\par\endgroup }
 \def\spacecheck#1{\p@gecheck{#1}%
    \ifsp@cecheck
       \else \vfill\eject
    \fi}
 \def\majorreset{
    \ifnum\figurecount<0\else\global\figurecount=1\fi
    \ifnum\equanumber<0 \else\global\equanumber=1\fi
    \sectionnumber=0 \subsectionnumber=0
    \tablecount=0  \problemnumber=0
    \bookheadline={}%
    \chapterheadline={}%
    }
 \def\chapterreset{\global\advance\chapternumber by 1
                   \majorreset
                   \makel@bel}
 \def\appendflag#1{
        \if@ppendix
        \else
          \global\@ppendixtrue
          \starttable{APPENDICES CALLED}{\appendixout}{\appendixwrite}{6}
        \fi
        \@ddconentry{\appendixwrite}{\noindent{\bf #1}}{1}
                   }

%%%
%%%               CHAPTER MACRO
%%%

 \def\chapter#1{\penalty-300%
    \spacecheck\chapterminspace%
    \chapterreset%
    \ifIEEE%
       \chapterheadline={\tenbf\noindent{\chapterlabel}.\quad #1}%
       \else%
       \chapterheadline={\tenrm\chapterlabel\quad #1}%
      \fi%
    \bookheadline={\tenrm \chapterlabel\quad #1}%
    \ifb@@kstyle%
         \global\FIRSTP@GE%
         \ifodd\pagenumber%
            \else%
               \par\nullbox{1pt}{1pt}{1pt}%
               \vfill\supereject%
         \fi%
         \global\FIRSTP@GE%
         \vtop{\line{\hfill{\ninebf %
               CHAPTER}\hskip12pt{\twentyfourbf\the\chapternumber}}%
         \vglue0.5in%
         \baselineskip=26pt%
         \let\tbreak=\par%
         \let\conbreak=\par%
           \bgroup \everypar={\hfill}%
                   {\twentyfourpoint \bf#1}%
           \egroup}%
         \let\conbreak=\tableconbreakspace%
         \let\tbreak=\tableconbreakspace%
         \bookstyle%
         \vglue\bookchapterskip%
       \else%
          \ifIEEE%
            \par\nobreak\vskip\headskip%
            \vskip\chapterskip%
            \noindent\bf{{\chapterlabel}\  #1}\rm
          \else%
            \par\n@ll%
            \nobreak\vskip\headskip%
            \vskip\chapterskip%
            \titlestyle{\chapterlabel \ #1}%
          \fi%
    \fi%
    \nobreak\vskip\headskip%
    \let\conbreak=\tableconbreak%
    \iftabl@conlist%
      \@ddconentry{\tableconwrite}{\hskip0.0pt plus 0pt minus 0pt
                                   \chapterlabel\quad\ #1}{1}%
    \fi%
    \penalty 30000%
    \wlog{\string\chapter\ \chapterlabel} }%
 \let\chap=\chapter%

%%%
%%%            SECTION MACRO
%%%

 \def\section#1{\par \let\conbreak=\tableconbreakfill%
    \ifnum\the\lastpenalty=30000\else%
    \penalty-200\vskip\sectionskip \immediate\spacecheck\sectionminspace\fi%
    \subsectionnumber=0 %reset the subsection numbering%
    \wlog{\string\section\ \chapterlabel \the\sectionnumber}%
    \global\advance\sectionnumber by 1  \noindent%
    \bookheadline={\tenrm \chapterlabel\sectionlabel\quad #1}%
    \ifb@@kstyle
        {\baselineskip=9.5pt%
        \setbox0=\hbox{\twelvebf\chapterlabel
                         \the\sectionnumber\ }%
        \let\conbreak=\par%
        \let\tbreak=\par%
        \bgroup \everypar={\ }%
           \hskip-\wd0{\twelvepoint\bf\box0\hskip\parindent#1}
        \egroup\par}%
        \let\tbreak=\tableconbreakspace%
        \let\conbreak=\tableconbreakfill%
      \else%
         \ifIEEE%
           \setbox0=\hbox{\it{\chapterlabel\sectionlabel}.}%
           \noindent{\it\box0\quad #1}\par\rm%
         \else%
           \setbox0=\hbox{\caps\chapterlabel \sectionlabel\ }%
           \noindent{\caps\box0\quad #1}\par%
         \fi%
    \fi
    \nobreak\vskip\headskip%
    \let\conbreak=\tableconbreakfill%
    \iftabl@conlist%%
      \@ddconentry{\tableconwrite}{%
            \hskip0.5in\ \chapterlabel\sectionlabel \quad\ #1}{2}%
    \fi%
    \penalty 30000 }%

%%%
%%%         SUBSECTION MACRO
%%%
 \def\subsection#1{\let\conbreak=\tableconbreakfill%
    \par%
    \ifIEEE\subsecstyle={\number}\fi%
    \global\advance\subsectionnumber by 1%
    \ifnum\the\lastpenalty=30000\else \penalty-100\smallskip \fi%
    \wlog{\string\subsection\ \chapterlabel\the\sectionnumber \ %
           \subseclabel}%
    \iftabl@conlist%
       \@ddconentry{\tableconwrite}{%
              \hskip1.0in\ \chapterlabel\sectionlabel\ %
              \subseclabel\quad #1}{3}%
    \fi%
    \ifIEEE%
       \noindent\it\chapterlabel\sectionlabel{.}\the\subsectionnumber{.}\quad%
       \it{#1}\enspace \vadjust{\penalty5000}\rm\par%
     \else
       \noindent\chapterlabel\sectionlabel \ \subseclabel\quad%
       \undertext{#1}\enspace \vadjust{\penalty5000}\fi}%
 \let\subsec=\subsection%

%%%
%%%            APPENDIX MACROS
%%%

 \def\APPENDIX#1#2{%
    \par\penalty-300\vskip\chapterskip%
    \spacecheck\chapterminspace \majorreset%
    \nobreak\vskip\headskip%
    \ifb@@kstyle
         \global\FIRSTP@GE%
         \par\nullbox{1pt}{1pt}{1pt}
         \ifodd\pagenumber% do nothing
            \else \par\nullbox{1pt}{1pt}{1pt}
                  \vfill\supereject            % get rid of even page
         \fi
         \global\FIRSTP@GE%
         \vtop{\line{\hfill{\ninebf %
               APPENDIX}\hskip12pt{\twentyfourbf #1}}%
         \vglue0.5in%
         \baselineskip=26pt%
         \let\tbreak=\par%
         \let\conbreak=\par%
           \bgroup \everypar={\hfill}%
                   {\twentyfourpoint \bf#2}%
           \egroup}%
         \let\conbreak=\tableconbreakspace%
         \let\tbreak=\tableconbreakspace%
         \bookstyle%
         \vglue\bookchapterskip%
      \else
          \immediate\titlestyle{\bf APPENDIX {\bf #1} \break%
                                   {\bf #2}}%
    \fi
    \nobreak\vskip\headskip \penalty 30000%
    \xdef\chapterlabel{\appendixlabel .}%
    \bookheadline={\tenrm \chapterlabel\quad #2}%
    \chapterheadline={\tenrm \chapterlabel\quad #2}%
    \iftabl@conlist%
       \iffirst@ppendix%
           \immediate\write\tableconwrite{%
                        \string\immediate
                        \string\spacecheck\sectionminspace
                                         }%
           \immediate\write\tableconwrite{\nullbox{1pt}{1pt}{1pt}%
                                         }%
           \immediate\write\tableconwrite{\vskip 0.375in
                                          \centerline{\fourteenrm\bf
                                                       APPENDICES}%
                                         }%
           \immediate\write\tableconwrite{\vskip\headskip
                                         }%
           \global\first@ppendixfalse%
       \fi%
       \@ddconentry{\tableconwrite}{%
             {}\chapterlabel\quad\  #2}{1}%
    \fi%
    \wlog{\string\Appendix\ \chapterlabel} }%
 \def\Appendix#1{\APPENDIX{#1}{}}%
 \def\appendix#1{%
      \global\advance\appendixnumber by 1%
      \xdef\appendixlabel{\the\appendixstyle{\the\appendixnumber} }%
      \APPENDIX{\appendixlabel}{#1}}%

%%%
%%%           PROBLEM MACROS
%%%

 \def\Problems{\par\n@ll \penalty-300 \vskip\chapterskip
    \global\problemnumber=0
    \spacecheck\chapterminspace
    \bookheadline={\tenrm  Problems}%
    \nobreak\vskip\headskip%
    \noindent\hskip-0.475in{\twelvepoint\bf\ Problems}%
    \nobreak\vskip\headskip%
    \nobreak\vskip\headskip%
    \iftabl@conlist
       \@ddconentry{\tableconwrite}{%
             \hskip0.563in Problems}{2}%
    \fi
    \penalty 30000
    \wlog{CHAPTER PROBLEMS}%
    }

 \def\problem{\global\advance\problemnumber by 1
         \chapterlabel\the\problemnumber :\quad}

%%%
%%%          FURTHER STUDY MACRO
%%%

 \def\Furtherstudy{\par\n@ll \penalty-300 \vskip\chapterskip%
    \spacecheck\chapterminspace%
    \bookheadline={\tenrm Suggestions for Further Study}%
    \noindent\hskip-0.475in{\twelvepoint\bf\ Suggestions for Further Study}%
    \nobreak\vskip\headskip%
    \iftabl@conlist%
      \@ddconentry{\tableconwrite}{%
             \hskip0.563in Suggestions for Further Study}{2}%
    \fi%
    \nobreak\vskip\headskip%
    \penalty 30000%
    \wlog{FURTHER STUDY} }%

%%%
%%%            INDEX MACROS
%%%

\def\indexon{\ifindex\else
             \indextrue
             \immediate\openout\indexwrite=index.tex
             \fi}
\def\indexoff{\indexfalse
              \closeout\indexwrite}
\def\indexing#1{%% Index call allowing a comment
     \ifindex%
        \conpage=\pagenumber
        \p@gecheck{0pt}%
        \ifsp@cecheck
           \else \advance\conpage by 1 %advance relative page by one
        \fi
        \immediate\write\indexwrite{#1 \the\conpage
                                   }%
     \fi}%
\def\sub{\hskip\parindent}
\def\SI#1#2{\xdef\indexentry{#2}%
         \indexentry%%
         \unskip \indexing{#1,\string\sub\ \indexentry}
         \unskip}%
\def\SII#1#2{\SI{#1}{#2}\ }%
\def\I#1{\xdef\indexentry{#1}%
         \indexentry%% Index call
         \unskip \indexing{\indexentry}%go to the list builder
         \unskip}%
\def\II#1{\I{#1}\ }%
\def\index#1{\unskip \indexing{#1}% go directly to the list builder
            }%
\indexoff %% default setting

%%%
%%%             ACKNOWLEDGEMENTS MACRO
%%%

 \def\ack{\par\penalty-100\medskip \spacecheck\sectionminspace
      \ifIEEE\bf\noindent{Acknowledgements}\rm%
       \else\line{\fourteenrm\hfil ACKNOWLEDGEMENTS\hfil}\fi%
      \nobreak\vskip\headskip }

%%%
%%%           ITEM AND POINT MACRO
%%%

 \def\Textindent#1{\noindent\llap{#1\enspace}\ignorespaces}
 \def\GENITEM#1;#2{\par \hangafter=0 \hangindent=#1
     \Textindent{$ #2 $}\ignorespaces}
 \outer\def\newitem#1=#2;{\gdef#1{\GENITEM #2;}}
 \newdimen\itemsize                \itemsize=30pt
 \newitem\item=1\itemsize;
 \newitem\sitem=1.75\itemsize;     \let\subitem=\sitem
 \newitem\ssitem=2.5\itemsize;     \let\subsubitem\ssitem

 \outer\def\newlist#1=#2&#3&#4;{\toks0={#2}\toks1={#3}%
    \count255=\escapechar \escapechar=-1
    \alloc@0\list\countdef\insc@unt\listcount     \listcount=0
    \edef#1{\par
       \countdef\listcount=\the\allocationnumber
       \advance\listcount by 1
       \hangafter=0 \hangindent=#4
       \Textindent{\the\toks0{\listcount}\the\toks1}}
    \expandafter\expandafter\expandafter
     \edef\c@t#1{begin}{\par
       \countdef\listcount=\the\allocationnumber \listcount=1
       \hangafter=0 \hangindent=#4
       \Textindent{\the\toks0{\listcount}\the\toks1}}
    \expandafter\expandafter\expandafter
     \edef\c@t#1{con}{\par \hangafter=0 \hangindent=#4 \noindent}
    \escapechar=\count255}
 \def\c@t#1#2{\csname\string#1#2\endcsname}
 \newlist\point=\Number&.&1.0\itemsize;
 \newlist\subpoint=(\alphabetic&)&1.75\itemsize;
 \newlist\subsubpoint=(\roman&)&2.5\itemsize;
 \let\spoint=\subpoint             \let\sspoint=\subsubpoint
 \let\spointbegin=\subpointbegin   \let\sspointbegin=\subsubpointbegin
 \let\spointcon=\subpointcon       \let\sspointcon=\subsubpointcon
 \newskip\separationskip   \separationskip=\parskip

%%%%%%%
%%%%%%%              LISTING MACROS
%%%%%%%

%%%  Paramemeters

 \newtoks\temphold
 \newtoks\ta
 \newtoks\tb
 \newtoks\captiontoks
 \newwrite\capwrite
 %\immediate\openout\capwrite=cap_test.tex %%%flag
 \newcount\runner  \runner=0
 \newcount\wordsnum \wordsnum=0
\newif\iflongc@p  \longc@pfalse
\def\longcap{\bgroup\obeyspaces\endlinechar=-1%\catcode`\^^M=\active %
%             \def^^M{\ }
             \global\longc@ptrue}
\def\endlongcap{\egroup\global\longc@pfalse}

 \def\confolio{\pagefolio{\conpage}}
 \def\p@gecheck#1{
        \dimen@=\pagegoal
        \advance\dimen@ by -\pagetotal
        \ifdim\dimen@<#1
           \global\sp@cecheckfalse
           \else\global\sp@cechecktrue
        \fi}
 \def\st@rtt@ble#1#2#3{
                 \message{listing #1:  type
                          \noexpand#2 for list}
                 \global\setbox#3=\vbox{\normalbaselines
                      \titlestyle{\seventeenrm\bf #1} \vskip\headskip}}
 \def\starttable#1#2#3#4{%
     \message{external listing of #1:  type %
              \noexpand#2 for list}%
     \ifcase#4%
           \immediate\openout#3=TABLE_OF_CONTENTS.TEX%  % 0 first list
       \or%
           \immediate\openout#3=FIG_CAP_AND_PAGE.TEX%   % 1 USER CAPTIONS
       \or%
           \immediate\openout#3=FIGURE_CAPTIONS.TEX%    % 2 PUBLICATION
% CAPTIONS
       \or%
           \immediate\openout#3=TABLE_CAPTIONS.TEX%     % 3 PUBLICATION
% CAPTIONS
       \or%
           \immediate\openout#3=TAB_CAP_AND_PAGE.TEX%   % 4 USER CAPTIONS
       \or%
           \immediate\openout#3=EQN_PAGE.TEX%           % 5 USER EQUATIONS
       \or%
           \immediate\openout#3=APP_CALLED.TEX%         % 6 APPENDICES CALLED
       \or%
           \immediate\openout#3=REFERENC.TEX%           % 7 REFERENCES
      \else \immediate\message{You are in big trouble call a %
                                 TeXnician}%
     \fi%
     \ifIEEE
        \ifnum#4=7\immediate\write#3{\string\vbox {\string\normalbaselines%
              \string\bf\noindent References \string\vskip \string\headskip}}
        \fi%
     \else\immediate\write#3{\string\vbox {\string\normalbaselines%
               \string\titlestyle {\string\seventeenrm \string\bf \ #1}
               \string\vskip \string\headskip}}\fi%
 }%
\def\splitcap{^^J
             }%
\let\splitline=\splitcap

\def\untoken#1#2{% #1 is the tok name, #2 are the new tokens to add
     \ta=\expandafter{#1}\tb=\expandafter{#2}%
     \immediate\edef\temphold{\the\ta\the\tb}
     \global\captiontoks=\expandafter{\temphold}}%

\def\getc@pwrite#1{
                   \ifx#1\endcaption%
                      \let\next=\relax%
                      \immediate\write\capwrite{\the\captiontoks}
                      \global\captiontoks={}
                    \else%
                       \untoken{\the\captiontoks}{#1}
                       \ifx#1\space%
                             \advance\runner by 1
                             \advance\wordsnum by1 \let\next=\getc@pwrite
                       \fi%
                    \let\next=\getc@pwrite\fi%
                 \ifnum\runner=10
                       \immediate\message{.}
                       \immediate\write\capwrite{\the\captiontoks}
                       \global\captiontoks={}
                       \runner=0
                 \fi
                       \next}%

\def\c@pwrite#1#2{
               \let\capwrite=#1
               \global\runner=0
               \immediate\message{working on long caption. .}
               \wordsnum=0 \getc@pwrite#2\endcaption
               \immediate\message{The number of words= \number\wordsnum}}

%\def\t@@bl@build@r#1#2#3{\relax%
%         \let\conbreak=\tableconbreakfill
%         \conpage=#3
%         \t@mp=\tablelineskip
%         \multiply\t@mp by #2%%
%         \@@@@@=\tablelinelength \advance\@@@@@ by -\t@mp%
%                                 \advance\@@@@@ by -1in%
%             \@@@@@two=\tablelinelength%
%             \advance\@@@@@two by -1in%
%             \vglue\separationskip
%         \vbox{
%             \vbox{\parshape=3 0pt\@@@@@two%
%                               \t@mp\@@@@@%
%                               \t@mp\@@@@@%
%                   \parfillskip=0pt%
%                   \pretolerance=9000 \tolerance=9999 \hbadness=2000%
%                   \hyphenpenalty=9000 \exhyphenpenalty=9000%
%                   \interlinepenalty=9999%
%                   \tablespace
%                   \vskip\baselineskip\noindent#1{\dotfill \hskip-0.07in
% $\,$}%
%                  }%
%             \parshape=0%
%            \vskip\tabledotvskip\hskip\@@@@@two\hskip\tabledothskip%
%            { \dotfill\quad} {\confolio}%
%         }
%         \vskip-0.10in}%


\def\t@@bl@build@r#1#2{\relax%              #1 multiplier number
         \let\conbreak=\tableconbreakfill % #2 page number
         \conpage=#2
         \t@mp=\tablelineskip
         \multiply\t@mp by #1%%
         \@@@@@=\tablelinelength \advance\@@@@@ by -\t@mp%
                                 \advance\@@@@@ by -1in%
             \@@@@@two=\tablelinelength%
             \advance\@@@@@two by -1in%
             \vglue\separationskip
         \vbox\bgroup
             \vbox\bgroup\parshape=3 0pt\@@@@@two%
                               \t@mp\@@@@@%
                               \t@mp\@@@@@%
                   \parfillskip=0pt%
                   \pretolerance=9000 \tolerance=9999 \hbadness=2000%
                   \hyphenpenalty=9000 \exhyphenpenalty=9000%
                   \interlinepenalty=9999%
                   \tablespace
                   \vskip\baselineskip\raggedright\noindent}
%Place the caption between these commands
\def\endt@@bl@build@r{{\tabledotfill \hskip-0.07in $\,$}%
                  \egroup%
             \parshape=0%
            \vskip\tabledotvskip\hskip\@@@@@two\hskip\tabledothskip%
            { \tabledotfill\quad} {\confolio}%
         \egroup
         \vskip-0.10in}%

 \def\addconentry#1#2{%
     \setbox0=\vbox{\normalbaselines #2}%
     \relax%
     \global\setbox#1=\vbox{\unvbox #1 \vskip 4pt plus 2pt minus 1pt \box0 }}%

% \def\@ddconentry#1#2#3{%
%     \conpage=\pagenumber%
%     \p@gecheck{0pt}%
%     \ifsp@cecheck \else
%         \if\conpage<0
%               \advance\conpage by -1
%            \else
%               \advance\conpage by 1
%         \fi%
%     \fi%
%     \let\conbreak=\tableconbreakfill
%     \splitprep
%     \immediate\write#1
%              {\string\t@@bl@build@r }%
%     \immediate\write#1 {{#2}%
%                        }%
%     \immediate\write#1 {{#3}%
%                         {\the\conpage}%
%                        }%
%     \splitprepend
%}%

\def\@ddconentry#1#2#3{%  #1 where to write
                       %  #2 what
                       %  #3 number
     \conpage=\pagenumber%
     \p@gecheck{0pt}%
     \ifsp@cecheck \else
         \if\conpage<0
               \advance\conpage by -1
            \else
               \advance\conpage by 1
         \fi%
     \fi%
     \let\conbreak=\tableconbreakfill
     \splitprep
     \immediate\write#1
              {\string\t@@bl@build@r}%
     \immediate\write#1{{#3}%
                        {\the\conpage}%
                       }%
     \iflongc@p
         \immediate\c@pwrite{#1}{ #2 }%
       \else
         \immediate\write#1{{#2}}%
     \fi
     \immediate\write#1
              {\string\endt@@bl@build@r}%
     \splitprepend
}

%\def\@ddfigure#1#2#3#4{% #2 -> number #3 -> caption #4 -> Tag
%        \splitprep
%        \immediate\write#1{{\string\vglue\string\separationskip}%
%                          }%
%        \immediate\write#1
%             {\string\par \string\nullbox {1pt}{1pt}{1pt}%
%              }%
%        \immediate\write#1{\string\vbox}%
%        \immediate\write#1{{\string\tablespace%
%                               \string\parshape=2 %
%                                     0pt\string\tablelinelength%
%                                    \string\figurelineskip\string\@@@@@%
%                                 {{#4~#2:\string\quad\ #3}}}}%
%        \immediate\write#1{\string\parshape=0%
%                          }%
%       \splitprepend
%                     }%

\def\@ddfigure#1#2#3#4{% #1 -> where to write
                       % #2 -> figure/table number
                       % #3 -> caption
                       % #4 -> Figure/Table Tag
        \splitprep
        \captiontoks={}
        \immediate\write#1{{\string\vglue\string\separationskip}%
                          }%
        \immediate\write#1
              {\string\par \string\nullbox {1pt}{1pt}{1pt}%
              }%
        \immediate\write#1{\string\vbox}%
        \immediate\write#1{\string\bgroup \string\tablespace}
        \immediate\write#1{\string\parshape=2}
        \immediate\write#1{0pt\string\tablelinelength}
        \immediate\write#1{\string\figurelineskip\string\@@@@@}
        \immediate\write#1{\string\bgroup
                           }
       \iflongc@p
            \immediate\c@pwrite{#1}{#4~#2:\quad\ #3}
          \else
            \immediate\write#1{#4~#2:\quad\ #3}
       \fi
       \immediate\write#1{\string\egroup
                          \string\egroup
                         }%
       \immediate\write#1{\string\parshape=0%
                          }%
       \splitprepend
                     }%

 \def\captionsetup{
        \@@@@@=\tablelinelength \advance\@@@@@ by -\figurelineskip}
 \def\manualpageno{\manualpagenotrue}
 \def\autopageno{\manualpagenofalse}
 \def\dumplist#1{%
     \bookheadline={}%
     \chapterheadline={}%
     \unlock   % allow command characters
     \global\let\conbreak=\tableconbreakfill
     \ifmanualpageno
         \else
            \pagenumber=1
     \fi
     \ifb@@kstyle
        \global\multiply\pagenumber by -1
        \bookheadline={}
     \fi
           %% count in roman numerals if in book format
           %% count in arabic numerals if in paper format
     \vskip\chapterskip  %% skip down page the distance of a chapter title
     \ifcase#1
            \chapterheadline={Contents}
            \bookheadline={Contents}    %      \conout
            \input TABLE_OF_CONTENTS.TEX%      0 list generated this run
       \or
            \chapterheadline={Contents}
            \bookheadline={Contents}         % \Conout
            \input User_table_of_contents.tex% 1 list generated from other runs
       \or
            \chapterheadline={Contents}
            \bookheadline={Contents}         % \CONOUT
            \input User_table_of_contents.tex% 2 list generated from other runs
            \input TABLE_OF_CONTENTS.TEX%        list generated from this run
       \or
           {\obeylines
            \input index.tex%                   3
           }
       \or
           {\obeylines
            \input User_index.tex%              4
           }
       \or
           {\obeylines
            \input User_index.tex%              5
            \input index.tex%
           }
       \or
           \captionsetup
           \input FIGURE_CAPTIONS.TEX%          6
       \or
           \input FIG_CAP_AND_PAGE.TEX%         7
       \or
           \captionsetup
           \input TABLE_CAPTIONS.TEX%           8
       \or
           \input TAB_CAP_AND_PAGE.TEX%         9
       \or
           \input EQN_PAGE.TEX%                10
       \or
           \input APP_CALLED.TEX%              11
       \or
           \input REFERENC.TEX%                12
       \else \immediate\message{ Call a TeXnician, You are in BIG trouble}
     \fi
     \vfill\eject %% fill the page and start a new one
     \lock}
 \def\dumpbox#1{         %% assume that last page is filled
     \ifb@@kstyle
        \multiply\pagenumber by -1
        \bookheadline={}
     \fi
           %% count in roman numerals if in book format
           %% count in arabic numerals if in paper format
     \vskip\chapterskip  %% skip down page the distance of a chapter title
     \unvbox#1           %% download the list
     \vfill\eject}       %% fill the page and start a new one

%%%%
%%%%     LIST GENERATORS
%%%%

 \def\notabledots{\global\let\tabledotfill=\hfill}
 \def\tabledots{\global\let\tabledotfill=\dotfill}
 \def\tableconlist{\global\tabl@conlisttrue
                   \starttable{CONTENTS}{\conout}{\tableconwrite}
                              {0}}
 \def\tableconlistoff{\global\tabl@conlistfalse
                      \immediate\closeout\tableconwrite}

 %%% REFERENCE LIST GENERATED IN REFERENCE ERROR CHECKING
 \def\reflist{\global\referencelisttrue}
 \def\reflistoff{\global\referencelistfalse}

 \def\figurelist{\global\figur@listtrue
                 \starttable{FIGURE CAPTIONS AND PAGES}
                       {\figuresout}{\figurewrite}{1}
                 \starttable{FIGURE CAPTIONS}{\figout}{\figwrite}{2}
                }
 \def\figurelistoff{\global\figur@listfalse}
 \let\killfigures=\figurelistoff

 \def\tablelist{\global\tabl@listtrue
                \starttable{TABLE CAPTIONS}{\tabout}{\tablewrite}{3}
                \starttable{TABLE CAPTIONS AND PAGES}
                           {\tablesout}{\tableswrite}{4}
               }
 \def\tablelistoff{\global\tabl@listfalse}

%%%
%%%       EQUATIONS WITH PAGE NUMBERS
%%%

 \def\equationlist{\global\@qlisttrue
                 \starttable{EQUATIONS}{\eqout}{\eqnwrite}{5}
                 \immediate\message{Producing an external list of equation
                         names in INTERNAL_EQUATION_LIST.TEX}
                 \immediate\openout\equ@tions=internal_equation_list.tex
                  }
 \let\eqlist=\equationlist
 \def\equationlistoff{\global\@qlistfalse}
 \let\eqlistoff=\equationlistoff

%%%
%%%       MAKE LOADABLE LIST OF NAMES
%%%

 \def\produceequations#1{
     \ifnum\equanumber<0
          \immediate\write\equ@tions{\string\xdef \string#1
                  {{\string\rm  (\number-\equanumber)}} }
       \else
          \immediate\write\equ@tions{\string\xdef \string#1
                  {{\string\rm  (\chapterlabel\number\equanumber)}} }
     \fi}

%%%
%%%    LOADING MACRO FOR LIST
%%%

 \def\equ@tionlo@d{ \ifeqlo@d
                       \else \input BOOK_EQUATIONS
                             \eqlo@dtrue
                       \fi}

%%%
%%%           DOWNLOADING THE LISTS
%%%

 \def\dumpall{\vfill\eject\conout\eqout\figout\figuresout
                 \tabout\tablesout
                 \appendixout}
 \def\appendixout{\immediate\closeout\appendixwrite
                  \dumplist{11}
                 }
 \def\conout{\normalspace\immediate\closeout\tableconwrite
             \dumpbox{\tableconbox}
             \dumplist{0}}
 \def\Conout{\dumplist{1}}
 \def\CONOUT{\normalspace\immediate\closeout\tableconwrite
             \dumplist{2}}
 \def\indexout{\immediate\closeout\indexwrite
               \dumplist{3}}
 \def\Indexout{\dumplist{4}}
 \def\INDEXOUT{\immediate\closeout\indexwrite
               \dumplist{5}}
 \def\refout{\immediate\closeout\referencewrite
             \dumplist{12}
%   \par \penalty-400 \vskip\chapterskip
%   \spacecheck\referenceminspace \immediate\closeout\referencewrite
%   \referenceopenfalse
%   \line{\ifIEEE\bf\hfil References\hfil
%         \else\fourteenrm\hfil REFERENCES\hfil\fi%}
%     \vskip\headskip
%   \input referenc.tex
   }
 \def\figout{%\dumpbox{\figbox}
             \immediate\closeout\figwrite
             \dumplist{6}}
 \def\figuresout{%\dumpbox{\figurebox}
             \normalspace\immediate\closeout\figurewrite
             \dumplist{7}}
 \def\FIGOUT{\figout}
 \def\tabout{\immediate\closeout\tablewrite
             \dumplist{8}}
 \def\tablesout{\normalspace\immediate\closeout\tableswrite
          \dumplist{9}}
 \def\TABOUT{\tabout}
 \def\eqout{\immediate\closeout\eqnwrite
          \dumplist{10}
          \immediate\closeout\equ@tions}

%%%%%%
%%%%%%           REFERENCE MACROS
%%%%%%

%%%
%%%           SET UP PARAMETERS
%%%

\newdimen\referenceminspace  \referenceminspace=25pc
\newcount\referencecount     \referencecount=0
\newcount\titlereferencecount     \titlereferencecount=0
\newcount\lastrefsbegincount \lastrefsbegincount=0
\newdimen\refindent     \refindent=30pt
\newif\ifreferencelist       \global\referencelisttrue
\newif\ifreferenceopen       \global\referenceopenfalse
\newwrite\referencewrite
\newwrite\equ@tions
\newtoks\rw@toks

%%%
%%%           REFERENCE MARKING MACROS
%%%

\def\NPrefmark#1{\attach{\scriptscriptstyle [ #1 ] }}
\def\PRrefmark#1{\attach#1}
\def\IEEErefmark#1{ [#1]}
\def\refmark#1{\relax%
     \ifPhysRev\PRrefmark{#1}%
     \else%
        \ifIEEE\IEEErefmark{#1}%
          \else%
            \NPrefmark{#1}%
          \fi%
     \fi}

%%%%%
%%%%%            THE REFERENCE MACROS
%%%%%


\def\REF#1#2{\space@ver{}\refch@ck
   \global\advance\referencecount by 1 \xdef#1{\the\referencecount}%
   \r@fwrit@{#1}{#2}}%
\def\ref#1{\REF\?{#1}\refend}%
\def\Ref#1#2{\REF#1{#2}\refend}%
\def\refend{\global\lastrefsbegincount=\referencecount
            \refsend}%                                 SINGLE REF
\def\REFS#1#2{\REF#1{#2}\global\lastrefsbegincount=\referencecount\relax}%
\def\REFSCON{\REF}%
\def\refs{\REFS\?}%
\def\refscon{\REF\?}%
\def\refsend{\refmark{\count255=\referencecount %    %  MULTIPLE REFS
   \advance\count255 by-\lastrefsbegincount %
   \ifcase\count255 %
       \number\referencecount%                      % single reference
   \or%
      \number\lastrefsbegincount,\number\referencecount% two refs
   \else%
      \number\lastrefsbegincount -\number\referencecount% more than two refs
  \fi}}%

%%%
%%%        ERROR CHECKING
%%%

\def\refch@ck{\ifreferencelist%
                  \ifreferenceopen%
                       \else \global\referenceopentrue%
                           \starttable{REFERENCES}{\refout}{\referencewrite}{7}
                  \fi%
              \fi%
}

%%%
%%%        REFERENCE WRITING
%%%

\def\rw@begin#1\splitout{\rw@toks={#1}\relax%
   \immediate\write\referencewrite{\the\rw@toks}\futurelet\n@xt\rw@next}%
\def\rw@next{\ifx\n@xt\rw@end \let\n@xt=\relax%
      \else \let\n@xt=\rw@begin \fi \n@xt}%
\let\rw@end=\relax%
\let\splitout=\relax%
\def\r@fwrit@#1#2{%
   \splitprep%
   \immediate\write\referencewrite{\ifIEEE\noexpand\refitem{\noindent[#1]}
                                   \else\noexpand\refitem{#1.}\fi}%
   \rw@begin #2\splitout\rw@end \@sf%
   \splitprepend%
                 }%

%%%
%%%          REFERENCE LABELING

\def\refitem#1{\par \hangafter=0 \hangindent=\refindent \Textindent{#1}}%

%%%%%
%%%%%          TITLE REFERENCES
%%%%%

%%%
%%%            TITLE REFERENCE MARKING
%%%

\let\titlerefmark=\refmark%
\def\titlerefend#1{\refmark{#1}}%

%%%%
%%%%          THE MACRO
%%%%

\def\TITLEREF#1#2{\space@ver{}\refch@ck%
   \global\advance\titlereferencecount by 1%
   \xdef#1{\alphabetic{\the\titlereferencecount}}%
   \r@fwrit@{#1}{#2}}%
\def\titleref#1{\TITLEREF\?{#1}\titlerefend{\?}}%

%%%%%%%
%%%%%%%       FIGURE MACROS
%%%%%%%

%%%
%%%       ERROR CHEDKING
%%%

 \def\checkreferror{\ifinner\errmessage{This is a wrong place to DEFINE
     a reference or a figure! You may have your references / figure
     captions screwed up. }\fi}


%%%%%
%%%%%            THE MACROS
%%%%%

% \def\FIG#1#2{\checkreferror
%     \ifnum\figurecount<0
%          \xdef#1{\number-\figurecount}%
%          \advance\figurecount by -1
%       \else
%          \xdef#1{\chapterlabel\number\figurecount}%
%          \advance\figurecount by 1
%     \fi
%     \iffigur@list
%        \@@@@@=\hsize \advance\@@@@@ by -1.219in
%        \@ddconentry{\figurewrite}%
%            {\noindent{\bf#1.}\quad{\string\string\string#1}\quad\ #2}{1}%
%        \@ddfigure{\figwrite}{#1}{#2}{Figure}%
%     \fi     }%

 \def\FIG#1#2{\checkreferror
     \ifnum\figurecount<0
          \global\xdef#1{\number-\figurecount}%
          \global\advance\figurecount by -1
       \else
          \global\xdef#1{\chapterlabel\number\figurecount}%
          \global\advance\figurecount by 1
     \fi
     \iffigur@list
        \@@@@@=\hsize \advance\@@@@@ by -1.219in
        \@ddconentry{\figurewrite}%
            {\noindent
                \bgroup \bf \expandafter{\csname\string#1.\endcsname} \egroup
                \quad{#1}\quad\ #2
            }
            {1}%
        \@ddfigure{\figwrite}{#1}{#2}{Figure}%
     \fi     }%

 \def\FIGURE{\FIG}
 \def\fig#1{\FIG{\?}{#1} \unskip fig.~\? }
 \def\Fig#1{\FIG{\?}{#1} \unskip Fig.~\? }
 \def\Figure#1{\FIG{\?}{#1} \unskip Figure~\? }
 \def\figure#1{\FIG{\?}{#1} \unskip figure~\? }

%%%%%%%
%%%%%%%        TABLE MACROS
%%%%%%%

%%%
%%%       ACCOUNTING
%%%

 \def\nexttable{\checkreferror\global\advance\tablecount by 1}

%%%%%
%%%%%      THE MACROS
%%%%%

% \def\TABLE#1#2{\nexttable\xdef#1{\chapterlabel\the\tablecount}%
%     \iftabl@list
%        \@ddfigure{\tablewrite}{#1}{#2}{Table}%
%        \@ddconentry{\tableswrite}%
%               {\noindent{\bf#1.}\quad{\string\string\string#1}\quad\ #2}{1}%
%     \fi}%

 \def\TABLE#1#2{\nexttable\xdef#1{\chapterlabel\the\tablecount}%
     \iftabl@list
        \@ddfigure{\tablewrite}{#1}{#2}{Table}%
        \@ddconentry{\tableswrite}%
               {\noindent
                   \bgroup \bf \expandafter{\csname\string#1.\endcsname}
\egroup
                   \quad{#1}\quad\ #2}
               {1}%
     \fi}%

 \def\Table#1{\TABLE{\?}{#1} \unskip Table~\? }
 \def\table#1{\TABLE{\?}{#1} \unskip table~\? }

%%%%%%%
%%%%%%%                 EQUATION MACROS
%%%%%%%

%%%%
%%%%       NAMING MACRO
%%%%

% \def\eqname#1{\relax
%     \if@qlist
%       \produceequations{#1}
%     \fi
%     \ifnum\equanumber<0
%           \global\xdef#1{{\rm(\number-\equanumber)}}
%           \global\advance\equanumber by -1
%     \else
%           \global\xdef#1{{\rm(\chapterlabel \number\equanumber)}}
%           \global\advance\equanumber by 1
%     \fi
%     \if@qlist
%       \@ddconentry{\eqnwrite}
%           {\noindent{\bf#1.}\quad{\string\string\string#1}\quad}{1}
%     \fi      }

 \def\eqname#1{\relax
     \if@qlist
       \produceequations{#1}
     \fi
     \ifnum\equanumber<0
           \global\xdef#1{{\rm(\number-\equanumber)}}
           \global\advance\equanumber by -1
     \else
           \global\xdef#1{{\rm(\chapterlabel \number\equanumber)}}
           \global\advance\equanumber by 1
     \fi
     \if@qlist
       \@ddconentry{\eqnwrite}
           {\noindent
             \bgroup \bf \expandafter{\csname\string#1\endcsname.} \egroup
             \quad{#1}\quad}
           {1}
     \fi      }

%%%
%%%          NORMAL MATH MODE
%%%

 \def\eq{\eqname\?\?}
 \def\eqn#1{\eqno\eqname{#1}#1}
 \def\leqn#1{\leqno\eqname{#1}#1}

%%%
%%%          BOXED MATH MODE
%%%

 \def\eqinsert#1{\noalign{\dimen@=\prevdepth \nointerlineskip
    \setbox0=\hbox to\displaywidth{\hfil #1}
    \vbox to 0pt{\vss\hbox{$\!\box0\!$}\kern-0.5\baselineskip}
    \prevdepth=\dimen@}}
 \def\leqinsert#1{\eqinsert{#1\hfill}}
 \def\mideqn#1{\noalign{\eqname{#1}}\eqinsert{#1}}
 \def\midleqn#1{\noalign{\eqname{#1}}\leqinsert{#1}}

%%%
%%%        ALIGNED MATH MODE
%%%

 \def\eqnalign#1{\eqname{#1}#1}

%%%
%%%     EQUATION NUMBERING SCHEMES
%%%

 \def\sequentialequations{\global\equanumber=-1}
 \def\globaleqnumbers{\relax\if\equanumber<0\else\global\equanumber=-1\fi}

%%%
%%%     FIGURE NUMBERING SCHEMES
%%%

 \def\sequentialfigures{\global\figurecount=-1
                       }

\def\globalfigurenumbers{\relax\if\figurecount<0\else\global\figurecount=-1\fi}


%%%%%%%
%%%%%%%       CONTROLLING MACROS FOR THE INSERTION OF UG (and other postscript)
%%%%%%%       figures.
%%%%%%%

 \def\UGbody#1#2#3#4{
      \special {ps::[asis, begin]
         0 SPB
         save
           /showpage {} def
           /initgraphics {} def
           Xpos Ypos translate
           #4 dup scale
           #2 72 mul neg #3 72 mul neg translate
      }
      \special{ps: plotfile #1 asis}
      \special{ps::[asis,end]
          restore
          0 SPE
      }
 }
 \def\UGfigs{
   \input UGinsert
   \input UGrcaption
   \input UGtwoinsert
   \input UGtwocapt
 }

%%%%%%%
%%%%%%%       CONTROLLING MACRO DEFINITIONS FOR LETTERS
%%%%%%%

%%%
%%%                PARAMETERS
%%%

 \newskip\lettertopfil      \lettertopfil = 0pt plus 1.5in minus 0pt
 \newskip\spskip            \setbox0\hbox{\ } \spskip=-1\wd0
 \newskip\signatureskip     \signatureskip=40pt
 \newskip\letterbottomfil   \letterbottomfil = 0pt plus 2.3in minus 0pt
 \newskip\frontpageskip     \frontpageskip=1\medskipamount plus .5fil
 \newskip\headboxwidth      \headboxwidth= 0.0pt
 \newskip\letternameskip    \letternameskip=0pt
 \newskip\letterpush        \letterpush=0pt
 \newbox\headbox
 \newbox\headboxbox
 \newbox\physbox
 \newbox\letterb@x
 \newif\ifse@l              \se@ltrue
 \newif\iffrontpage
 \newif\ifletterstyle
 \newif\ifhe@dboxset        \he@dboxsetfalse
 \newtoks\memoheadline
 \newtoks\letterheadline
 \newtoks\letterfrontheadline
 \newtoks\lettermainheadline
 \newtoks\letterfootline
 \newdimen\holder
 \newdimen\headboxwidth
 \newtoks\myletterheadline  \myletterheadline={define \\myletterheadline}
 \newtoks\mylettername    \mylettername={{\twentyfourbf define}\
\\mylettername}
 \newtoks\phonenumber       \phonenumber={\rm (313) 764-4437}
 \newtoks\faxnumber	    \faxnumber={(313) 763-9694}
 \newtoks\telexnumber       \telexnumber={\tenfib 4320815 UOFM UI}

%%%%
%%%%          LETTER STYLE DEFINITIONS
%%%%

 \def\FIRSTP@GE{\ifvoid255\else\vfill\penalty-2000\fi
                \global\frontpagetrue}
 \def\FRONTPAGE{\ifvoid255\else\vfill\penalty-2000\fi
       \masterreset\global\frontpagetrue
       \global\lastp@g@no=-1 \global\footsymbolcount=0}
 \let\Frontpage=\FRONTPAGE
 \letterfootline={\hfil}
 \lettermainheadline={\hbox to \hsize{
        \rm\ifp@genum page \ \folio\fi
        \hfill\today}}
 \letterheadline{\hfuzz=60pt\iffrontpage\the\letterfrontheadline
      \else\the\lettermainheadline\fi}
 \def\addressee#1{\vskip-0.188in \singlespace
    \ialign to\hsize{\strut ##\hfill\tabskip 0pt plus \hsize \cr #1\hfill\crcr}
    \medskip\noindent\hskip\spskip}
 \def\letterstyle{\global\letterstyletrue%
                  \global\paperstylefalse%
                  \global\b@@kstylefalse%
                  \global\frontpagetrue%
                  \global\singlespace\global\lettersize}
 \def\lettersize{\hsize=\letterhsize \vsize=\lettervsize
                 \hoffset=\letterhoffset \voffset=\lettervoffset
    \headboxwidth=\hsize  \advance\headboxwidth by -1.05in
    \skip\footins=\smallskipamount \multiply\skip\footins by 3}
 \def\telex#1{\telexnumber={\tenfib #1}}
 \def\noseal{\se@lfalse}

%%%
%%%         LETTER HEAD builders
%%%

 \def\myaddress#1{%
         \singlespace
         \global\he@dboxsettrue
         {\hsize=\headboxwidth \setbox0=\vbox{\ialign{##\hfill\cr #1 }}
         \global\setbox\headboxbox=\vbox {\hfuzz=50pt
                  \vfill
                  \line{\hfil\box0\hfil}}}}
 \def\MICH@t#1{\hfuzz=50pt\global\setbox\physbox=\hbox{\it#1}\hfuzz=1pt}
 \def\l@tt@rcheck{\ifhe@dboxset\else
                     \immediate\message{Setting default letterhead.}
                     \immediate\message{For more information consult the PHYZZM
                                       documents.}
                     \UM
                  \fi}
 \def\l@tt@rs#1{\l@tt@rcheck
              \vfill\supereject % Start a new page
              \global\letterstyle
              \global\letterfrontheadline={}
              \global\setbox\headbox=\vbox{{
                  \hbox to\headboxwidth{\hepbig\hfil #1\hfil}
                  \vskip 1pc
                  \unvcopy\headboxbox}}}
 \def\HEAD#1{\singlespace  %set spacing
    \global\letternameskip=22pt plus 0pt minus 0pt
    \global\memoheadline={UM\ #1\ Memorandum}
    \mylettername={\hep #1}
       \myaddress{
   \hep\hfill\hbox{The Harrison M. Randall Laboratory of Physics}\hfill \cr
   \hep\hfill\hbox{500 East University, Ann Arbor, Michigan 48109-1120}\hfill
\cr
                                                 \cr}}

%%%%
%%%%            PUBLICATION TYPES
%%%%


 \def\PDK{\phonenumber={\rm (313) 764-4442}
        \faxnumber={(313) 936-1817}
        \HEAD{High Energy Physics}
        \PUBNUM{PDK} \MICH@t{High Energy Physics}}
 \def\theory{\phonenumber={\rm (313) 763-9698}
        \HEAD{Theoretical Physics}
        \PUBNUM{TH} \MICH@t{Theoretical Physics}}
 \def\solidstate{\phonenumber={\rm (313) 763-2073}
        \faxnumber={(313) 764-2193}
        \HEAD{Solid State Physics}
        \PUBNUM{SS} \MICH@t{Solid State Physics}}
 \def\condensed{\phonenumber={\rm (313) 763-2073}
        \faxnumber={(313) 764-2193}
        \HEAD{Condensed Matter Physics}
        \PUBNUM{CM} \MICH@t{Condensed Matter Physics}}
 \def\highenergy{\phonenumber={\rm (313) 764-4443}
        \faxnumber={(313) 936-1817}
        \HEAD{High Energy Physics}
        \PUBNUM{HE} \MICH@t{High Energy Physics}}
 \def\astro{\HEAD{Astrophysics}
            \PUBNUM{ASTRO} \MICH@t{Astrophysics}}
 \def\Lthree{\phonenumber={\rm (313) 764-4443}
        \faxnumber={(313) 936-1817}
        \HEAD{High Energy Physics}
        \PUBNUM{LEP3} \MICH@t{High Energy Physics}}
 \def\spin{\phonenumber={\rm (313) 936-1027}
        \faxnumber={(313) 936-0794}
	\HEAD{High Energy Spin Physics Group}
	\PUBNUM{SPIN} \MICH@t{High Energy Spin Physics Group}}
 \def\vax{\phonenumber={\rm (313) 936-2426}
	\HEAD{High Energy Physics Computing Services}
	\PUBNUM{VAX} \MICH@t{High Energy Physics Computing Services}}
 \def\ocs{\phonenumber={\rm (313) 764-3348}
	\HEAD{Office of Computing Services}
	\PUBNUM{OCS} \MICH@t{Office of Computing Services}}
 \def\RanBuild{\phonenumber={\rm (313) 764-4443}
        \faxnumber={(313) 936-1817}
        \HEAD{Randall Building Committee}
        \PUBNUM{HE} \MICH@t{Randall Building Committee}}
 \def\UM{\phonenumber={\rm (313) 764-4437}
        \HEAD{Department of Physics}
        \PUBNUM{PHYSICS} \MICH@t{Department of Physics}}
 \let\um=\UM
%%%
%%%           LETTER COMMAND MACROS
%%%

 \def\lettertext{\par\unvcopy\letterb@x\par}
 \def\multiletter{\setbox\letterb@x=\vbox\bgroup
       \everypar{\vrule height 1\baselineskip depth 0pt width 0pt }
       \singlespace \topskip=\baselineskip }
 \def\letterend{\par\egroup}

 \def\letters{\l@tt@rs{The University of Michigan}}
 \def\letter{\wlog{\string\letter}
        \vfill\supereject
        \normalspace       % set spacing for skipping
        \FRONTPAGE
        \nullbox{1pt}{1pt}{1pt}\vskip-0.8750in
        \setbox0 = \vbox{\hfuzz=50pt   % put the letterhead text in box0
           {\hsize=\headboxwidth
	   \unvcopy\headbox\hfill
           \singlespace
           \vskip-0.1in
%
	   \dimen1=2.25truein
	   \setbox1=\hbox{\hep(313) 763-4929}     \advance\dimen1 by \wd1
	   \setbox1=\hbox{\hep\the\phonenumber}   \advance\dimen1 by -1\wd1
%
           \centerline{\the\mylettername\hskip\dimen1{\hep\the\phonenumber}}}}
%
% This is an extract from the \insertUG macro of Dave Nitz.  (If it works,
% don't fix it.)  Put the seal in box1
%
 \ifse@l
   \dimen0=0.8in%                    seal box width (smaller than the real
% seal)
   \setbox1=\hbox to \dimen0{%
    \dimen0=1.0in%                   seal box height
    \vbox to \dimen0{
      \vss
        \UGbody{tex_inputs:umseal.ps}{1.0}{8.31}{1}                 % vax
%        \UGbody{/usr/local/tex/inputs/umseal.ps}{1.0}{8.31}{1}      % sun
%        \UGbody{/progs/tex/inputs/umseal.ps}{1.0}{8.31}{1}          % apollo
     }%
     \hss
    }
 \fi
        \hfuzz=5pt
        \ifse@l
          \hfuzz=15pt\line{\hbox to0pt{}\hskip 0.3in\box1\box0 \hfill}
	\else
	  \line{\hfill\box0\hfill}
	\fi
        \vskip0.35in
        \vskip\letterpush
        \rightline{\today}
        \addressee}
 \let\umletter=\letter

 \def\myletter{\l@tt@rs{\the\myletterheadline}\letter}

 \def\signed#1{\par \penalty 9000 \bigskip \dt@pfalse
   \everycr={\noalign{\ifdt@p\vskip\signatureskip\global\dt@pfalse\fi}}
   \setbox0=\vbox{\singlespace \halign{\tabskip 0pt \strut ##\hfill\cr
    \noalign{\global\dt@ptrue}#1\hfill\crcr}}
   \line{\hskip 0.5\hsize minus 0.5\hsize \box0\hfill} \medskip }
 \def\copies#1{\singlespace\hfill\break
   \line{\nullbox{0pt}{0pt}{\hsize}}
   \setbox0 = \vbox {
     \noindent{\tenrm cc:}\hfill %.2196
     \vskip-0.2115in\hskip0.000in\vbox{\advance\hsize by-\parindent
       \ialign to\hsize{\strut ##\hfill\hfill
                 \tabskip 0pt plus \hsize \cr #1\crcr}}
       \hbox spread\hsize{}\hfill\vfill}
   \line{\box0\hfill}}
 \def\endletter{\nullbox{0pt}{0pt}{\hsize}
       \ifnum\pagenumber=1
                \vskip\letterbottomfil\vfill\supereject
          \else
                \vfill\supereject
       \fi
       \wlog{ENDLETTER}}

%%%%
%%%% LETTER AND MEMO SLACISMS
%%%%

 \def\SLACletter{\FRONTPAGE\SLACHEAD\addressee}

 \def\SLACHEAD{\setbox0=\vbox{\baselineskip=12pt
       \ialign{\tenfib ##\hfil\cr
          \hskip -17pt\tenit Mail Address:\cr
          SLAC, P.O.Box 4349\cr Stanford, California, 94305\cr}}
    \setbox2=\vbox to\ht0{\vfil\hbox{\caps Stanford Linear Accelerator
          Center}\vfil}
    \smallskip \line{\hskip -7pt\box2\hfil\box0}\bigskip}


%%%%%
%%%%%              MEMO DEFINITIONS
%%%%%

 \def\SLACMEMO{\letterstyle\FRONTPAGE \letterfrontheadline={\hfil}
     \line{\quad\fourteenrm SLAC MEMORANDUM\hfil\twelverm\the\date\quad}
     \medskip \memod@f}
 \def\MEMO{\letterstyle\FRONTPAGE \letterfrontheadline={\hfil}
     \l@tt@rcheck
     \line{\fourteenrm \the\memoheadline \hfill
%UM \the\memolabel\ Memorandum\hfill
                \twelverm\the\date}
     \medskip \memod@f}
 \def\memodate#1{\date={#1}\xdef\today{#1}\MEMO}
 \def\memit@m#1{\smallskip \hangafter=0 \hangindent=1in
       \Textindent{\caps #1}}
 \def\memod@f{\xdef\To{\memit@m{To:}}
              \xdef\from{\memit@m{From:}}%
              \xdef\topic{\memit@m{Topic:}}
              \xdef\subject{\memit@m{Subject:}}%
              \xdef\rule{\bigskip\hrule height 1pt\bigskip}}
 %%%%%%%%%%\memod@f

%%%%%
%%%%%               TELE-FAX MACROS
%%%%%

\def\FAX{\letterstyle\FRONTPAGE
    \letterfrontheadline={
          \seventeenrm\hfill\qquad OUTGOING FACSIMILE COVER SHEET\hfill}
    \line{\quad\fourteenrm University of Michigan: Physics\hfil
                                      Fax No. \the\faxnumber\fourteenrm\quad}
    \smallskip
    \line{\quad\fourteenrm Randall Lab\hfil\the\date\quad}
    \medskip \faxd@f}
\def\faxdate#1{\date={#1}\xdef\today{#1}\FAX}
\def\faxd@f{\xdef\numcall{\memit@m{At 'Fax:}}%
            \xdef\numpage{\memit@m{Pages:}}%
            \memod@f}
     %%%%%%%%%%%%\faxd@f

%%%%%
%%%%%             PAPER DEFINITIONS
%%%%%

%%%
%%%     PARAMETERS
%%%

 \newif\ifp@bblock          \p@bblocktrue
 \newif\ifpaperstyle
 \newif\ifb@@kstyle
 \newtoks\paperheadline
 \newtoks\bookheadline
 \newtoks\chapterheadline
 \newtoks\paperfootline
 \newtoks\bookfootline
 \newtoks\Pubnum            \Pubnum={$\caps UM - PHY - PUB -
                                        \the\year - \the\pubnum $}
 \newtoks\pubnum            \pubnum={00}
 \newtoks\pubtype           \pubtype={\tensl Preliminary Version}
 \newcount\yeartest
 \newcount\yearcount

%%%
%%%           STYLES
%%%
 \def\sequentialfootnotes{\global\seqf@@tstrue}
 \def\PHYSREV{\paperstyle\PhysRevtrue\PH@SR@V
     \let\refmark=\attach}
 \def\PH@SR@V{\doubl@true \baselineskip=24.1pt plus 0.2pt minus 0.1pt
              \parskip= 3pt plus 2pt minus 1pt }
 \def\IEEE{\paperstyle\IEEEtrue\I@EE\doublespace\rm\let\refmark=\IEEErefmark%
             \let\unnumberedchapters=\relax}
 \def\I@EE{\baselineskip=24.1pt plus 0.2pt minus 0.1pt
              \parskip= 3pt plus 2pt minus 1pt }

%%%%%
%%%%%             ALLOW GREATER VARIATION OF STYLE DECLARATION
%%%%%
\let\ieee=\IEEE
\let\physrev=\PHYSREV
\let\PhysRev=\PHYSREV

 \def\bookstyle{\b@@kstyletrue%
                \letterstylefalse%
                \paperstylefalse%
                \equ@tionlo@d
                \Tenpoint
                \frenchspacing
                \parskip=0pt
                \bookspace\booksize}
 \def\paperstyle{\paperstyletrue%
                 \b@@kstylefalse%
                 \letterstylefalse%
                 \normalspace}
%
%  the following changed on 4/27/88   B. Ball, J. Chapman
%
%                 \normalspace\papersize}
% \def\papersize{\hsize=35pc\vsize=50pc\hoffset=1pc\voffset=0.375in
%                \skip\footins=\bigskipamount}
 \def\papersize{\hsize=38pc\vsize=50pc\voffset=0.375in
                \skip\footins=\bigskipamount}

\def\booksize{\hsize=29pc\vsize=45pc\hoffset=0.85in\voffset=0.475in\hfuzz=2.5pc
               \itemsize=\parindent}

%%%
%%%           HEAD AND FOOT LINES
%%%

 \paperfootline={\hss\iffrontpage
                         \else \ifp@genum
                                  \hbox to \hsize{\tenrm\hfill\folio\hfill}\hss
                               \fi
                      \fi}
 \paperheadline={\hfil}
 \bookfootline={\hss\iffrontpage
                       \ifp@genum
                              \hbox to \hsize{\tenrm\bf\hfill\folio\hfill}\hss
                       \fi
                     \else
                           \hbox to \hsize{\hfill\hfill}\hss
                     \fi}

%%%%
%%%%                      TITLE PAGE MACROS
%%%%

 \def\titlepage{
    \yeartest=\year \advance\yeartest by -1900
    \ifnum\yeartest>\yearcount
          \global\PUBNUM{UM}
    \fi
    \FRONTPAGE\paperstyle\ifPhysRev\PH@SR@V\fi
    \ifp@bblock\p@bblock\fi}

 \def\nopubblock{\p@bblockfalse}
 \def\endpage{\vfil\break}
 \def\p@bblock{\begingroup \tabskip=\hsize minus \hsize
    \baselineskip=1.5\ht\strutbox \topspace-2\baselineskip
    \halign to\hsize{\strut ##\hfil\tabskip=0pt\crcr
    \the\Pubnum\cr \the\date\cr \the\pubtype\cr}\endgroup}

%%%
%%%      ACTIVE COMMANDS ON TITLE PAGE
%%%

 \def\PUBNUM#1{
     \yearcount=\year
     \advance\yearcount by -1900
     \Pubnum={$\caps UM- #1 - \the\yearcount - \the\pubnum $}}
 %%% DO NOT FORGET TO SET \pubnum={<number>}
 \def\title#1{\vskip\frontpageskip \titlestyle{#1} \vskip\headskip }
 \def\author#1{\vskip\frontpageskip\titlestyle{\twelvecp #1}\nobreak}
 \def\andauthor{\vskip\frontpageskip\centerline{and}\author}
 \def\authors{\vskip\frontpageskip\noindent}
 \def\address#1{\par\noindent\titlestyle{\twelvepoint\it #1}}
     \def\SLAC{\address{Stanford Linear Accelerator Center\break
           Stanford University, Stanford, California, 94305}}
     \def\MICH{\address{University of Michigan,\break
               \unhcopy\physbox\break
               Ann Arbor, Michigan,  48109-1120}}
 \def\andaddress{\par\kern 5pt \centerline{\sl and} \address}
 \def\abstract{\vskip\frontpageskip\centerline{\fourteenrm ABSTRACT}
               \vskip\headskip }
 \def\submit#1{\par\nobreak\vfil\nobreak\medskip
    \centerline{Submitted to \sl #1}}
 \def\doeack#1{\foot{Work supported by the Department of Energy,
       contract $\caps #1$.}}

%%%%%
%%%%%               FOOTNOTE DEFINITIONS
%%%%%

%%%
%%%     PARAMETERS
%%%

 %%
 %%    parameters that are part of TeX
 %%        and don't need to be defined again
 %%
       %\newinsert\footins
       %\skip\footins=\bigskipamount % space added when footnote is present
       %\count\footins=1000 %footnote magnification factor (1 to 1)
       %\dimen\footins=8in % maximum footnotes per page
 %%
 %%

 \newtoks\foottokens       \foottokens={\Tenpoint\singlespace}
 \newdimen\footindent      \footindent=24pt
 \newcount\lastp@g@no	   \lastp@g@no=-1
 \newcount\footsymbolcount \footsymbolcount=0
 \newif\ifPhysRev
 \newif\ifIEEE
 \newif\ifseqf@@ts         \global\seqf@@tsfalse

%%%
%%%    CONTROL SEQUENCES
%%%

 %%
 %%       controls that are part of TeX
 %%           and don't need to be defined again
 %%
      %\def\fo@t{\ifcat\bgroup\noexpand\next \let\next\f@@t
      %       \else\let\next\f@t\fi \next}
      %\def\f@@t{\bgroup\aftergroup\@foot\let\next}
      %\def\f@t#1{#1\@foot}
      %\def\@foot{\strut\egroup}
      %\def\footstrut{\vbpx to\splittopskip{}}
 %%
 %%

 \def\footrule{\dimen@=\prevdepth\nointerlineskip
    \vbox to 0pt{\vskip -0.25\baselineskip \hrule width 0.35\hsize \vss}
    \prevdepth=\dimen@ }
 \def\vfootnote#1{\insert\footins\bgroup  \the\foottokens
    \interlinepenalty=\interfootnotelinepenalty \floatingpenalty=20000
    \splittopskip=\ht\strutbox \boxmaxdepth=\dp\strutbox
    \leftskip=\footindent \rightskip=\z@skip
    \parindent=0.5\footindent \parfillskip=0pt plus 1fil
    \spaceskip=\z@skip \xspaceskip=\z@skip
    \Textindent{$ #1 $}\footstrut\futurelet\next\fo@t}

%%%%
%%%%         ACTIVE USER COMMANDS
%%%%

 \def\footnote#1{\attach{#1}\vfootnote{#1}}
 \def\footattach{\attach\footsymbol}
 \def\foot{\attach\footsymbolgen\vfootnote{^{\footsymbol}}}

%%%
%%%      FOOT SYMBOL CHOICES
%%%

 \let\footsymbol=\star
 \def\footsymbolgen
 {  \relax
   \ifseqf@@ts \seqf@@tgen
   \else
      \ifPhysRev
         \iffrontpage \NPsymbolgen
         \else \PRsymbolgen
         \fi
      \else \NPsymbolgen
      \fi
   \fi
   \footsymbol
 }
 \def\seqf@@tgen{\ifnum\footsymbolcount>0 \global\footsymbolcount=0\fi
       \global\advance\footsymbolcount by -1
       \xdef\footsymbol{\number-\footsymbolcount} }
 \def\NPsymbolgen
 {  \ifnum\footsymbolcount<0 \global\footsymbolcount=0\fi
   {  \iffrontpage \relax
      \else
         \ifnum \lastp@g@no = \pageno
            \relax
         \else
            \global\lastp@g@no = \pageno
            \global\footsymbolcount=0
         \fi
      \fi
   }
   \ifcase\footsymbolcount
      \fd@f\star \or \fd@f\dagger \or \fd@f\ast \or
      \fd@f\ddagger \or \fd@f\natural \or \fd@f\diamond \or
      \fd@f\bullet \or \fd@f\nabla
   \fi
   \global\advance\footsymbolcount by 1
   \ifnum\footsymbolcount>6 \global\footsymbolcount=0\fi
 }
 \def\fd@f#1{\xdef\footsymbol{#1}}
 \def\PRsymbolgen{\ifnum\footsymbolcount>0 \global\footsymbolcount=0\fi
       \global\advance\footsymbolcount by -1
       \xdef\footsymbol{\sharp\number-\footsymbolcount} }
 \def\space@ver#1{\let\@sf=\empty \ifmmode #1\else \ifhmode
    \edef\@sf{\spacefactor=\the\spacefactor}\unskip${}#1$\relax\fi\fi}
 \def\attach#1{\space@ver{\strut^{\mkern 2mu #1} }\@sf\ }

%%%
%%%        HEAD AND FOOT LINE CHOICE
%%%

 \footline={\ifletterstyle \the\letterfootline
               \else \ifpaperstyle \the\paperfootline
                        \else \the\bookfootline
                      \fi
            \fi}
 \headline={\let\conbreak=\tableconbreakspace%
            \let\tbreak=\tableconbreakspace%
            \ifletterstyle \the\letterheadline
               \else \ifpaperstyle \the\paperheadline
                        \else \iffrontpage {}
                                  \else
                                     \setbox0=\hbox{\ {\tenrm\bf \folio}}
                                     \advance\hsize by \wd0
                                     \ifodd\pagenumber
                                          \hbox to \hsize{%
                                              \the\bookheadline\hfill\hfill
                                              \box0
                                                         }%
                                        \else
                                          \hskip-\wd0
                                          \hbox to \hsize{%
                                              \box0\hfill\hfill
                                              \the\chapterheadline
                                              }%
                                     \fi
                              \fi
                     \fi
             \fi}

%%%%%
%%%%%		Macros for making mailing labels.
%%%%%		#1 = vertical size of a single label; if empty, use 1.5in
%%%%%		#2 = number of labels across the page; if empty,  use 3
%%%%%		#3 = distance from left page edge to text start;
%%%%%		     if empty or 0in, then use a default starting
%%%%%		     point for 1.5in x 3
%%%%%		#4 = distance from top of page to text top;
%%%%%		     if empty or 0in, then use a default starting
%%%%%		     point for 1.5in x 3
%%%%%
%%%%%		A standard usage would be:
%%%%%		   \labelsetup{}{}{}{}
%%%%%		   \label{Dr. T.W. Jones\\Dept. of Physics \& Astron.\\
%%%%%		   University College London\\Gower St., London WC1\\
%%%%%		   UNITED KINGDOM }
%%%%%

\newdimen\vlsiz
\newdimen\hlsiz
\def\labelinit#1#2#3#4{
   \def\next{#1} \ifx\next\empty \vlsiz=1.5in \else \vlsiz=#1 \fi
   \hlsiz=\hsize
   \def\next{#2} \ifx\next\empty \divide\hlsiz by 3
                 \else           \divide\hlsiz by #2 \fi
   \advance\hlsiz by -0.01in
   \def\next{#3} \ifx\next\empty \else
   \ifdim#3=0pt \else \hoffset=-0.90in \advance\hoffset by #3 \fi \fi
   \def\next{#4} \ifx\next\empty \else
   \ifdim#4=0pt \else \voffset=-0.90in \advance\voffset by #4 \fi \fi}
\def\labelsetup#1#2#3#4{
   \nopagenumbers
   \singlespace
   \def\\{\ifnum\lastpenalty=1000 \relax\else\hfil\break\fi\ignorespaces}
   \hoffset=-.6in
   \voffset=-.5in
   \vsize=\labelvsize
   \hsize=\labelhsize
   \parskip=0pt
   \parindent=0pt
   \raggedbottom
   \def\label##1{\ifvmode\noindent\fi
      \vbox to \vlsiz{\hsize=\hlsiz\raggedright ##1\par\vss}
      \hskip 0pt plus.25in minus.25in\ignorespaces}
   \labelinit{#1}{#2}{#3}{#4}}

%%%%%%
%%%%%%          FIGURE INSERTION FOR QMS PRINTER
%%%%%%

%%%%     This macro command controls the insertion of figures
%%%%  into the TeX output for the Qms laser printer only!!!!
%%%%  The command asks for #1 the plotfiles name, #2 the
%%%%  verticle size of the plot, and #3 how far from the left
%%%%  margin the plot should be.
%%%%
%%%%    =====>> IMPORTANT: In order to have the plot printed,
%%%%               the plotfile must be in the same directory
%%%%               as the qms file and must be present while
%%%%               the dvi file is being processed.

\def\insertplot#1#2#3{
   \par \
      \vskip0.5in %% Enables plot from upper left cornor
      \hbox{%
         \hskip #3
         \vbox to #2 {
             \special{qms: plotfile #1 relative}
             \vfill
         }%
      }%
}
%%  macro for inserting figures into the text
%%      #1   File where the plot information is located
%%      #2   Verticle size of the plot
%%      #3   How far from left margin to put plot

%%%%%%%
%%%%%%%              MASTER RESET FOR PHYZZEX MACRO
%%%%%%%

%%%%%                   SETS ALL GLOBAL DEFAULTS

 \def\masterreset{\global\pagenumber=1 \global\chapternumber=0
    \global\appendixnumber=0
    \global\equanumber=1 \global\sectionnumber=0 \global\subsectionnumber=0
    \global\referencecount=0 \global\figurecount=1 \global\tablecount=0
    \global\problemnumber=0
    \global\@ppendixfalse
    \ifIEEE\global\setbox\referencebox=\vbox{\normalbaselines
          \noindent{\bf References}\vskip\headskip}%
     \else\global\setbox\referencebox=\vbox{\normalbaselines
          \centerline{\fourteenrm REFERENCES}\vskip\headskip}\fi
   }


%%%%%%
%%%%%%                 ANCIENT COMMANDS THAT ARE NO LONGER USED
%%%%%%

%%%
%%%           ACTIVE COMMAND
%%%

 \def\obsolete#1{\message{Macro \string #1 is obsolete.}}

%%%
%%%          OLD COMMANDS
%%%

 \def\product#1{\obsolete\product \prod_{#1}}
 \def\firstsec#1{\obsolete\firstsec \section{#1}}
 \def\firstsubsec#1{\obsolete\firstsubsec \subsection{#1}}
 \def\thispage#1{\obsolete\thispage \global\pagenumber=#1\frontpagefalse}
 \def\thischapter#1{\obsolete\thischapter \global\chapternumber=#1}
 \def\nextequation#1{\obsolete\nextequation \global\equanumber=#1
    \ifnum\the\equanumber>0 \global\advance\equanumber by 1 \fi}
 \def\BOXITEM{\afterassigment\B@XITEM\setbox0=}
 \def\B@XITEM{\par\hangindent\wd0 \noindent\box0 }

%%%%%%%%%%%%%%%%%%%%%%%%%%%%%%%%%%%%%%%%%%%%%%%
%%%%%%%%%%%%%%%%%%%%%%%%%%%%%%%%%%%%%%%%%%%%%%%

%%%%%
%%%%%             ALLOW THE USER TO INPUT OWN COMMANDS
%%%%%

\def\MYPHYZZM{\input MYPHYZZM.TEX}
\let\myphyzzm=\MYPHYZZM

%%%%%%%%%%%%%%%%%%%%%%%%%%%%%%%%%%%%%%%%%%%%%%%
%%%%%%%%%%%%%%%%%%%%%%%%%%%%%%%%%%%%%%%%%%%%%%%
%%%%%
%%%%%                 SET THE CHOOSEN LOCAL DEFAULTS
%%%%%

 \normalspace                      % set line spacing
 \masterreset
 \pagenumbers                      % default setting --> USE PAGE NUMBERS
 \Twelvepoint                      % default font size
 \figurelistoff                    % default for figure listing is off
 \tableconlistoff                  % default for contents listing is off
 \tablelistoff                     % default for table listing is off
 \reflist                          % default for ref listion is on
                                   %         for compatibility with PHYZZX
 \equationlistoff                  % default for equation listing is off
 \telex{4320815 UOFM UI}           % globally set the telex number
                                   % note: number must be set before the group
 \headboxwidth=6.5in
 \advance\headboxwidth by -1.05in  % Set a default letter-head width
 \UM                               % default publication type
 \paperstyle                       % This is the default
                                   % other styles: \letterstyle
                                   %               \bookstyle
 \tabledots                        % turn on dots for table of contents
                                   %   \notabledots turns them off
 \manualpageno                     % lists are automatically numbered
                                   %   \autopageno select starting page #
 \he@dboxsetfalse                  % do this to ensure proper date on letters
                                   %   and memos
 \lock
 \hfuzz=1pt
 \vfuzz=0.2pt

\def\fmtname{PHYZZM nee PHYZZX}\def\fmtversion{1.9}   % format identifiers
\message{PHYZZM version 1.9}

