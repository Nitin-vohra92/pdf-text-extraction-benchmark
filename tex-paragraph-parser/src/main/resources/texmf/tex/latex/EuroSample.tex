%%%   This is Sample.tex.
%%%   Use of this macro package is not restricted provided
%%%   each use is acknowledged upon publication.
%%%   It is an example of how to typeset your paper 
%%%   using LaTeX 2.09 with `EuroPhys.sty'.
%%%   In preparing your article you are requested to follow these
%%%   guidelines as closely as possible, particularly with regard to the 
%%%   reference list. This will minimize copy editing required and will 
%%%   hasten production process.
%%%   Note, however, that since we aim to maintain our standards for 
%%%   articles published from LaTeX, we reserve the right to make small 
%%%   alterations to clarify and improve the English where necessary and to
%%%   put the article in EPL house style.
%
%%%%%%%%%%%%%%%%%%%%%%%%%%%%%%%%%%%%%%%%%%%%%%%%%%%%%%%%%%%%%%%%%%%%%%%%%%%
%%%%%%%%%%%%%%%%%%%%%%%%%%%%%%%%%%%%%%%%%%%%%%%%%%%%%%%%%%%%%%%%%%%%%%%%%%%
%%%%%%%%%%%%%%%%%%%%%%%%%%%%%%                 %%%%%%%%%%%%%%%%%%%%%%%%%%%%
%%%%%%%%%%%%%%%%%%%           Typeset this file      %%%%%%%%%%%%%%%%%%%%%%
%%%%%%%%%%%%%%%%%    then read other informations in   %%%%%%%%%%%%%%%%%%%%
%%%%%%%%%%%%%%%%%%%             the following        %%%%%%%%%%%%%%%%%%%%%%
%%%%%%%%%%%%%%%%%%%%%%%%%%%%%%                 %%%%%%%%%%%%%%%%%%%%%%%%%%%%
%%%%%%%%%%%%%%%%%%%%%%%%%%%%%%%%%%%%%%%%%%%%%%%%%%%%%%%%%%%%%%%%%%%%%%%%%%%
%%%%%%%%%%%%%%%%%%%%%%%%%%%%%%%%%%%%%%%%%%%%%%%%%%%%%%%%%%%%%%%%%%%%%%%%%%%
%%%   Initial settings
%
%%%   The style to use is EuroPhys.sty. EuroMacr.tex is a file of macros
%%%   that can be useful to correctly typeset your paper.
%
\documentstyle{EuroPhys}
\input EuroMacr
\begin{document}
%
%%%   The headers.
%
%%%   These three macros are to have correct headings in your paper.
%%%   You shall omit all the arguments in the two macros `\euro{}{}{}{}'
%%%   `\Date{}' and fill in `\shorttitle{}'. 
%%%   If there is more than one author in the 
%%%   \shorttitle macro, use the macro \etal after first author's name
%%%   to obtain the correct heading.
%
\euro{29}{6}{1-$\infty$}{1995}
\Date{1 April 1995}
\shorttitle{A. EINSTEIN \etal THE PRINCIPLE OF RELATIVITY ETC.}
%
%%%  The title, the Author(s) and the affiliation(s)
%
%%%   The title is set in bold (initial word only is capitalized).
%%%   Mathematical expressions and formulas within the title shall be left
%%%   in light face. Initial(s) of the first name(s) are followed by the
%%%   author(s)'s last name(s). If the authors have different affiliations,
%%%   the name must be followed by one or more \inst{number} each referring
%%%   to one of the addresses to appear in the following macro \institute.
%%%   Other items like `Present address' or `email' may be added by putting
%%%   a `\footnote' after the last \inst{number}.
%%%   Begin each address with \inst{number}; the end of an address is \\;
%%%   \\ can also be used to break a line.
%
\title{The principle of relativity:\\How to see old things in a new
manner}
\author{A. Einstein\inst{1}\footnote{Present address: unknown.}, H. A. 
    Lorentz\inst{1},
     H. Weyl\inst{2} \And H. Minkowski\inst{2}}
\institute{
     \inst{1} Department of Theoretical Physics -
              Galactic University, M1 Crab Nebula, Universe\\
     \inst{2} Institute of Applied Mathematics - Babel Library, 1.4142 
   Pitagora av.\\
   2.78182 Pseudo-metric, Space-Time}
%
%%%    The `\maketitle' macro needs the following macro:    \rec{}{}
%%%    to be left empty.
%
\rec{1 November 1905}{in final form 1 April 1995}
%
%%%   Physics Abstracts Classification.
%
%%%   There are two macros: the first one `\pacs{}' makes the PACS 
%%%   environment,the second one `\Pacs{}{}{}' can be used for each
%%%   classification you need.
%%%   To create the subject index of the volume it is important to divide
%%%   the classification numbers into the three different arguments like
%%%   in the following examples 
%
\pacs{
\Pacs{02}{40Ky}{Riemannian geometries}
\Pacs{03}{30$+$p}{Special relativity}
\Pacs{04}{20$-$q}{Classical general relativity}
      }
%
\maketitle
%
%%%   ! Don't forget this command to format the title page of your article!
%
%%%   The Abstract
%
\begin{abstract}
Put the text of your abstract here. The abstract should be complete in 
itself (with no tables, figures, equations or reference quotations),
summarizing concisely the aims of your work, not exceeding 750 characters
and in a single paragraph.
\end{abstract}
%
%
%%%   Main text
%
%%%   Sectioning
%
%%%   In EuroPhys there is only ``one'' level of sectioning `\section{}'.
%
Main text begins here.
%
\section{Running head}
If you want you may set a short title to be used in the running
head on odd-numbered pages (only when the title itself cannot be
used as the short title: it should not exceed 40 characters in
length).   
%
\section{Sections and text}
Main text can be divided into one-level sections that are not numbered. 
The text should fit into 6 pages maximum, including formulas, figures and
tables (21,000 equivalent characters excluding title and affiliations).

Note that English spellings are preferred (colour, flavour, behaviour,
tunnelling, artefact, fibre, metre, centre etc.). Compound words 
beginning with ``non'' or ``self'' are hyphenated.

The words: {\it figure, equation\/} and {\it reference\/} should be
abbreviated as fig., eq. (eqs.), ref.

The word {\it table\/} is written in full. The contractions {\it i.e.\/}, 
{\it e.g.\/}, {\it et. al.\/}, should appear in Italic (use
$\backslash${\tt it}) not in Roman.  Units, chemical formulas,
abbreviations should be typeset in Roman while mathematical symbols in
Italic (vectors and tensors are set in boldface Roman (use
$\backslash${\tt bf}); when it is essential to distinguish between them,
sans serif may be used for tensors (use $\backslash${\tt psf}))
$$
%{\bf v}\quad {\psf T}\,.
$$

However, there are some cases in which it is necessary to use a Roman
font\footnote{For most of these cases there are macros ready to use either
built in \LaTeX, or built in the file  `EuroMacr.tex'. The description and 
use of those of EuroPhys  are at the end of this file.}: the differentials
$\drm$, $\Delta$ and the mathematical functions, $\cos,\ \sin, 
\exp,\ \det,\
\ker, \ln;\ \tr, \order$ (for traces and orders); letters or
abbreviations used as sub- or superscripts to variables, but serving
merely as label {\it e.g.\/} $\tau_{\rm a}$ (a = adsorbed) $t_{\rm
out}$, $R_{\rm gyr}$, $\mu_{\rm B}$ etc.

Exponential expressions are clearer in the notation $\exp[\ldots]$
especially the long ones or those containing subscripts or superscripts;
for simple expressions we accept also $e^x$.

It is important to distinguish between $\ln=\log_e$ and $\log=\log_{10}$.

For single fraction in the text, use the {\it solidus\/}, instead of
fraction. Use parentheses whenever necessary to avoid ambiguity, for
example to distinguish between $1/(n-1)$ and $1/n-1$. Exceptions are the
proper fractions $\frac12$, $\frac2{13}$ or $\dif{\ }{x}$ which are
better left in this form.

For usual units, use the standard SI abbreviations, unusual units may
be written in full at least at first place of mention. Footnotes should
be kept to a minimum and numbered consecutively with arabic numerals as
superscript in parentheses (if you use EuroPhys.sty. the \macro{footnote}
macro makes it automatically).
For note added after submission, you can use the notation {\it Additional
Remark\/} in Italic with the text in a smaller type run on.
Use appendices only if vital to the understanding of complex formulas and
in this case treat them as normal sections: {\tt \verb!\section{appendix}!}.

%
%%%   Figures and Tables
%
%%%   There are two environments, one for figures and the other for 
%%%   tables.
%

\section{Figures and Tables}
If you do not use EuroPhys.sty, please put all the captions at the end of 
the paper before the bibliography.
%
\begin{figure}
%
\vbox to 1cm{\vfill\centerline{\fbox{Here is the figure}}\vfill}
%
\caption{Caption of  figure.}
\label{fig1}
\end{figure}
%
\begin{table}
%
\caption{Caption of table. Example of table.}
\[
\begin{tabular}{ccccccc}
\hline
\multicolumn{1}{c}{1.2.95} & \multicolumn{1}{c}{1.7.95} & 
\multicolumn{1}{c}{1.1.96} & \multicolumn{1}{c}{1.7.96} & 
\multicolumn{1}{c}{1.1.97} & \multicolumn{1}{c}{1.7.97} & 31.12.97 \\ 
\hline
\multicolumn{1}{c}{$-10.000.000$} & \multicolumn{1}{c}{} &
\multicolumn{1}{c}{$65.400$} & 
\multicolumn{1}{c}{$65.400$} & \multicolumn{1}{c}{$65.400$} & 
\multicolumn{1}{c}{$65.400$} & $
\begin{array}{r}
65.400 \\ 
+1.111.800
\end{array}
$ \\ \hline
\end{tabular}
\]
\label{Tab1}
\end{table}
%

\section{References}
Literature citations of periodicals \cite{HK}, \cite{S}, books \cite{LS},
\cite{LL}, conference proceedings \cite{GK} and preprints \cite{P} should
be organized according to the following examples after the international
standar ISO document n.~690. Journal names should be abbreviated
according to the list of the Serial Title Word Abbreviations of the ISDS
(International Serial Data System) if available, otherwise full Journal
names are preferable.

If you use EuroPhys.sty, you can use useful macros in the
bibliography environment; their use is described in this file.

\section{Once a paper is accepted}
A message (letter, fax or e-mail) from EPL Editorial office in Geneva
will inform the principal
author that the article has been accepted for publication; at this stage
he is asked to submit immediately the electronic version in the
final form, \lower2pt\vbox{\hbox{exactly matching} \hrule}
the final version accepted in hard copy, to the
production office in {\bf Bo\-lo\-gna}:

\medskip
\centerline{\bf email: SIF@BO.INFN.IT}

\medskip
To ensure portability between the different installations and systems
used it is highly recommended that authors adhere to the following
guidelines:
\medskip

Only {\bf \TeX/\LaTeX} or {\bf Word} or {\bf ASCII} (without formatting
commands) files may be transmitted.
\medskip
\begin{itemize}
\item {\bf \TeX/\LaTeX} files may be transmitted as encoded files or as
simple e-mail files. In the latter case line  lengths should not exceed 80
characters and accented characters should be avoided, {\it i.e.\/} only
plain ASCII files with printable characters (codes between 32 and 128)
should be sent.
Only files that can be typeset with standard format and classical fonts
(CM or AMS fonts) are acceptable. Avoid too personalized style sheets.
\item {\bf Word} and {\bf ASCII} files should be encoded with public
encoding software such as {\tt uuencode} (for DOS and UNIX) and {\tt
binhex} (for Macintosh). Large files should be compressed and/or archived
{\it before\/} encoding, using public software such as {\tt gzip} or {\tt
tar}. Please, note that these files, unlike \TeX/\LaTeX, need partial
retyping.
\item {\bf Graphics}, such as {\bf figures}, {\bf schemas} etc. should be in
separate files. Most packages used to create them will allow to choose a
format to save your file. Adobe Illustrator (AI) Encapsulated Postscript
is highly recommended; Encapsulated Postscript, Postscript, TIFF (tagged
image file format) and PICT are acceptable as well.
Sending native application files could help to better reproduce your
figures according to EPL house style. In this case, please specify which
software was used.
\item Figures incorporated into Word files are also acceptable.
\item Large files should be compressed and/or archived with public
software, then encoded.
\item {\bf Important:} unnecessary text (figure number, caption, author or
file name) {\it should not be included in or around figures\/}. These
indications should be handwritten on back of the paper version. 
\item Different parts  of a figure,(a), b), etc.) should be in {\it
separate files\/}.
\item {\it If you do not use EPS format\/}, to indicate different parts of a
diagram use differente types of hatching; avoid gray scales, which might
be difficult to distinguish after reduction or may completely disappear
during the printing process.

\end{itemize}


%
%%%   In the acknowledgments, use the following macro  before and instead 
%%%   of  ``Acknoledgments''
%
\stars
%
%%%   Bibliography environment begins here. You can use the macros \Name{},
%%%   \And, \Book{} or \Review{}, \Vol{}, \Year{} and \Page{}, to type your
%%%   references.
%
\vskip-12pt
\begin{thebibliography}{99}
%
\bibitem{HK}
\Name{Huckenstein B. \And Kramer B.} \Review{Phys. Rev. Lett.} \Vol{64}
\Year{1990} \Page{1437}.
%
\bibitem{S}
\Name{G. W. Scherer} \Review{J. Amer. Ceram. Soc.} \Vol{74} \Year{1991}  
 \Page{1523}.
%
\bibitem{LS}
\Name{Lawrie I. D. \And Sarbach S.} in \Book{Phase Transition and
Critical Phenomena} edited by \Name{C. Domb \And J. L. Lebowitz}
\Vol{9} (Academic Press, London) \Year {1984} p\Page{65-68}.
%
\bibitem{LL}
\Name{Landau L. D. \And   Lifshitz E. M.}
\Book{Course of Theoretical Physics} \Vol{6} and {\bf 8} (Pergamon Press,   
New York, N.Y.) \Year{1947}; \Name{Leusieur M.}   \Book{Turbulence in
Fluids} (Kluwer Academic Publishers) \Year{1990}; \Name{Moreau R.}
\Book{Magnetohydrodynamics} (Kluwer Academic Publishers) \Year{1992}.
%
\bibitem{GK}
\Name{Gawedzki K. \And Kupiainen A.} in \Book{Critical Phenomena, Random
Systemsm Gauge Theories, Les Houches 1984} edited by \Name{K.
Osterwalder \And R. Stora} Part I (North-Hollandm Amsterdam) \Year{1986},
p\Page{185-293}.
% 
\bibitem{P}
\Name{Polyakov A.}
PUPT-1341, hep-th/9209046 Preprint, \Year{1992};\quad
PUPT-1369, hep-th/9212145 Preprint, 1992.
%
\end{thebibliography}
\end{document}
%%%%%%%%%%%%%%%%%%%%%%%%%%%%% Useful macros %%%%%%%%%%%%%%%%%%%%%%%%%%%%%%%
%
%% All the following macros are defined in EuroMacr.tex file.
%
%%%%%%%%%%%%%%%%    \etal            %%%%%%%%%%%%%%%%%%%%%%%%%%%%%%%%%%%%%%
%
%% Use the  macro `\etal' to obtain an italicized et al. in the  short 
%% title.      !!!! Attention, it works for short title only!!!!
%
%%%%%%%%%%%%%%%%     \And            %%%%%%%%%%%%%%%%%%%%%%%%%%%%%%%%%%%%%%
%
%% To obtain a roman `and' into an environment  with a different font
%
%%%%%%%%%%%%%%%%     \dfrac{}{}      %%%%%%%%%%%%%%%%%%%%%%%%%%%%%%%%%%%%%%
%
%% To obtain a fraction in displaystyle, the first entry is the 
%% numerator and the second the denominator.
%
%% Use the following macros to obtain differentials and  derivatives 
%% with a roman ``d''  
%% according to EPL house style:
%
%%%%%%%%%%%%%%%%     \drm            %%%%%%%%%%%%%%%%%%%%%%%%%%%%%%%%%%%%%%
%
%% to obtain a roman `d',
%
%%%%%%%%%%%%%%%%    \dif{}{}         %%%%%%%%%%%%%%%%%%%%%%%%%%%%%%%%%%%%%%
%
%% to obtain the first derivative with roman d, the first argument is the 
%% argument of the numerator and the second the argument of 
%% the denominator.
%
%% Besides the default macros for trigonometric and hyperbolic functions 
%% there are also the following
%
%%%%%%%%%%%%%%%%     \tg  \ctg  \arctg  \sec  \cosec  \tgh  \ctgh  %%%%%%%%
%
%% Then there are other macros for diffrente functions and operators
%
%%%%%%%%%%%%%%%%  \order  \tr  \Div  \Rot  \Grad    %%%%%%%%%%%%%%%%%%%%%%%
%
%% The following macro can be used to insert an `EPS figure'
%% into the `figure' environment.
%% The first argument is the figure name, the second the scaling 
%% percentage(1000 is one to one), the third is the scaled figure height
%% and the fourth the scaled figure width.
%
%%%%%%%%%%%%%%%%     \centerinsert#1#2#3#4     %%%%%%%%%%%%%%%%%%%%%%%%%%%%
%
%%%%%%%%%%%%%%%%     \stars          %%%%%%%%%%%%%%%%%%%%%%%%%%%%%%%%%%%%%%
%
%%To be used  before and instead of ``Acknoledgments''
%
%% Use the following macros for  references. Their meaning is obvious.
%
%%%%%%%%%%%%%%%%     \Name{}         %%%%%%%%%%%%%%%%%%%%%%%%%%%%%%%%%%%%%%
%%%%%%%%%%%%%%%%     \Review{}       %%%%%%%%%%%%%%%%%%%%%%%%%%%%%%%%%%%%%%
%%%%%%%%%%%%%%%%     \Book{}         %%%%%%%%%%%%%%%%%%%%%%%%%%%%%%%%%%%%%%
%%%%%%%%%%%%%%%%     \Vol{}          %%%%%%%%%%%%%%%%%%%%%%%%%%%%%%%%%%%%%%
%%%%%%%%%%%%%%%%     \Year{}         %%%%%%%%%%%%%%%%%%%%%%%%%%%%%%%%%%%%%%
%%%%%%%%%%%%%%%%     \Page{}         %%%%%%%%%%%%%%%%%%%%%%%%%%%%%%%%%%%%%%
%
%
%%%%%%%%%%%%%%%%%%%%%%%%%%%%%%%%% End of Macros %%%%%%%%%%%%%%%%%%%%%%%%%%%
