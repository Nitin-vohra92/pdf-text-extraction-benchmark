%%% ======================================================================
%%%  @LaTeX-file{
%%%     filename        = "manaps.tex",
%%%     version         = "3.0",
%%%     date            = "November 10, 1992",
%%%     ISO-date        = "1992.11.10",
%%%     time            = "15:41:54.18 EST",
%%%     author          = "American Physical Society",
%%%     contact         = "Christopher B. Hamlin",
%%%     address         = "APS Publications Liaison Office
%%%                        500 Sunnyside Blvd.
%%%                        Woodbury, NY 11797",
%%%     telephone       = "(516) 576-2390",
%%%     FAX             = "(516) 349-7817",
%%%     email           = "mis@aps.org (Internet)",
%%%     supported       = "yes",
%%%     archived        = "pinet.aip.org/pub/revtex,
%%%                        Niord.SHSU.edu:[FILESERV.REVTEX]",
%%%     keywords        = "REVTeX, version 3.0, input guide,
%%%                        American Physical Society",
%%%     codetable       = "ISO/ASCII",
%%%     checksum        = "20517 2271 13076 91856",
%%%     docstring       = "This is the American Physical Society chapter
%%%                        in the input guide for REVTeX 3.0.
%%%
%%%                        The checksum field above contains a CRC-16
%%%                        checksum as the first value, followed by the
%%%                        equivalent of the standard UNIX wc (word
%%%                        count) utility output of lines, words, and
%%%                        characters.  This is produced by Robert
%%%                        Solovay's checksum utility."
%%% }
%%% ======================================================================
% start of file manaps.tex
%
%   This file is part of the APS files in the REVTeX 3.0 distribution.
%   Version 3.0 of REVTeX, November 10, 1992.
%
%   Copyright (c) 1992 The American Physical Society.
%
%   See the REVTeX 3.0 README file for restrictions and more information.
%
%



\documentstyle[twocolumn,aps]{revtex}
% Macros for the various macro package names, etc.
\def\SNG{{\em Physical Review Style and Notation Guide}}
\def\LUG {{\em \LaTeX{} User's Guide \& Reference Manual}}
% \def\RUG {{\em \REVTeX{} User's Guide \& Reference Manual}}
\def\btt#1{{\tt$\backslash$\string#1}}%
\def\REVTeX{REV\TeX}
\def\AmS{{\protect\the\textfont2
        A\kern-.1667em\lower.5ex\hbox{M}\kern-.125emS}}
\def\AmSLaTeX{\AmS-\LaTeX}
\def\BibTeX{\rm B{\sc ib}\TeX}
\makeatletter
% run page numbers by "chapter"
\def\thepage{1-\@arabic\c@page}
% these page numbers need a bit more width
\def\@pnumwidth{2em}
\makeatother
\begin{document}
\title{REV\TeX{} Information for APS Authors\\[1pc]
Instructions for preparing compuscripts to be submitted to APS journals\\
in the \REVTeX{} 3.0 format}
\maketitle




\tableofcontents


\makeatletter
\global\@specialpagefalse
\def\@oddhead{\REVTeX{} 3.0\hfill Released November 10, 1992}
\let\@evenhead\@oddhead
% run page numbers by "chapter", with copyright for first page
\def\@oddfoot{\reset@font\rm\hfill \thepage\hfill
\ifnum\c@page=1
  \llap{\protect\copyright{} 1992
  American Physical Society}%
\fi
} \let\@evenfoot\@oddfoot
\makeatother


\section{Introduction}
\label{sec:intro}

In 1987, the Council of The American Physical Society authorized acceptance
of \TeX-formatted author-prepared compuscripts to be submitted to {\em
Physical~Review A, B, C,} and {\em D\/} in machine-readable form.

It is essential that author-prepared input be consistent and standardized
so that the compuscripts can become part of the normal production
procedures. It is only by routinely handling author-prepared files that
this method of production will be economically feasible.

This input guide contains basic instructions for keyboarding compuscripts
using the \REVTeX{} macro package, which works in a \LaTeX{} environment.
This guide is part of the ``\REVTeX{} compuscript toolbox;'' other items in
the toolbox are the \REVTeX{} macro package (several style files),
bibliography tools, a test file, a README file (which contains details
regarding installation and copying of the style files), and a file for the
\SNG. For correct notation and style practices, authors should read the
\SNG, furnished with the toolbox; authors also should be guided by recent
issues of the journals.

Compuscripts that do not comply closely with these instructions will not be
used directly in the production process.

You may need to supplement this input guide with the standard documentation
available for \LaTeX: namely, the \LUG{}, by Leslie Lamport, published by
Addison Wesley. This manual assumes some familiarity with \LaTeX;
specifically the article style. \REVTeX{} version 3.0 is based on article
style. The notations \verb+#1+, \verb+#2+, etc.\ are used throughout this
manual to denote user-supplied arguments. Commands will be shown in their
full form; i.e., with their mandatory arguments.


The electronic submission and compuscript production programs were extended
to {\em Physical Review Letters\/} in 1992. The toolbox, including this
input guide and the \SNG{}, apply to PRL; additional details of the PRL
program can be found in an information booklet that appears at the back of
some issues of the journal, and which is available electronically from the
Editorial Offices (inquire at \verb+prltex@aps.org+).

{\bf Contents of this Guide.} In Sec.\ \ref{sec:gensubmit} we describe
procedures for making an original submission to the Editorial Offices.


Section \ref{sec:filesubmit} discusses the specifications for file
submittal (electronic mail or floppy disk).

Section \ref{sec:process}  briefly describes author proofs.

In Sec.\ \ref{sec:instruct} we describe in detail several aspects of
compuscript creation: the input of front matter and how to switch from
galley to preprint style; paragraph formatting, hyphenation, quotes, and
text-width issues; section headings; basic input for in-text math and
displayed equations; special character issues; footnotes and references;
figure captions; and tables.

In Sec.\ \ref{sec:xrefs} we have described the various numbering and
cross-referencing features of \REVTeX{} using the available commands; these
features can be used to label and cross-reference equations, figures,
tables, references, and section heads.

Section \ref{sec:fonts} discusses font selection schemes, support for
AMSFonts, and extra symbols available in \REVTeX{} 3.0

Section \ref{sec:upgrade} itemizes the differences between v2.x and v3.0 of
\REVTeX{}, for current users of v2.x, along with differences between
\LaTeX{} article style and \REVTeX{} 3.0.

Authors who have never participated in the author-prepared program may wish
to read Sec.\ \ref{sec:fastfacts} and  Sec.\ \ref{sec:macros} at this
point. There we give a brief overview of the author-prepared program and
answer some frequently asked author questions.

Section \ref{sec:contacts} lists the people and places to contact if you
have questions about any services described herein.  To obtain the most
proper and expedient answer to your question, please consult this section
before calling or e-mailing the APS.

Appendix A gives a complete listing of symbols and characters available
with \REVTeX{} v3.0.

Appendix B is a handy  list of some necessary commands that will be found
in a \REVTeX{} compuscript.  This command list supplements the \LUG{}.

Output for Appendixes A and B can be obtained by running the file
manend.tex through \LaTeX.

Unless specified in Section \ref{sec:contacts}, questions regarding this
input guide should be directed to Christopher B.\ Hamlin, APS Publications
Liaison Office, 500 Sunnyside Blvd., Woodbury, NY 11797.
%


{\bf Note on participating journals}.
The compuscript program is open to all authors in {\em Physical Review A,
B, C, D, E,} and {\em Physical Review Letters}.



\section{General Compuscript Submission Procedures}
\label{sec:gensubmit}

When the \REVTeX{} compuscript (preprint style) is ready to go, it should
should be submitted to the Editorial Offices in conformance with the
printed procedures in the first issue of each volume of {\em Physical
Review A, B, C, D, E,} and {\em L}. The following procedures are those
unique to compuscript processing.

Papers intended for the compuscript production program should be submitted
and resubmitted by electronic mail or by DOS-formatted floppy disk. (See
details below.)

Earlier in the compuscript program, compuscripts were often submitted in
conventional paper form, with the electronic file sent to the editorial
office only after acceptance for publication. Experience has shown,
however, that the program works best if the electronic file is supplied at
the outset. If a compuscript is submitted in conventional paper form, the
author will be asked to supply it at once in electronic form.

Paper copies of a compuscript, generated in the editorial office from the
author's file, are sent to referees. During the review process,
correspondence from the Editor to an author is generally in paper form via
conventional mail. Unless requested otherwise by the author, the manuscript
and original figures are not returned.

When a compuscript is changed in response to recommendations and criticisms
from the review process, or to present other corrections and author
revisions, the full file should be resubmitted. Please include a summary of
the changes made, and state whether or not figures have been modified. Any
new or revised figures should be forwarded as described below.

Avoid resubmission by conventional means (paper, postal mail); continue to
use one of the electronic modes.


\subsection{Electronic mail submissions}

Submissions and inquiries should be sent via electronic mail to the
appropriate address given below.
\begin{quasitable}
\begin{tabular}{@{\hspace{.5in}}ll@{\hspace{.5in}}}
        & Internet \\
Journal & address \\
\hline
PR A--E & \verb+prtex@aps.org+ \\
PR Letters & \verb+prltex@aps.org+
\end{tabular}
\end{quasitable}
Files must be accompanied by a cover message stating to which journal the
paper is submitted and must provide a conventional mailing address. The
manuscript accession code should be included for resubmissions. The cover
message can be in the form of comments at the head of the file(s).

The file must be in ASCII containing no control codes, with line lengths of
80 characters or less. The file should produce double-spaced output (three
lines per inch and at least 6 mm white space between lines).

Editorial processing of an electronic-mail submission cannot begin until at
least review-quality copies of the figures are received. Sending
photoreproducible journal-quality figures immediately by overnight mail
will meet this need. Alternatively, review-quality figures  can be
forwarded by FAX to 516-924-5294, while the ``originals'' are sent by
conventional or overnight mail as soon as possible. Please mark the FAX
transmission as being part of an electronic-mail submission. Our processing
of figures sent via electronic mail (e.g., PostScript formatted) is not yet
totally reliable.

Include the copyright transfer form with the original figures or forward it
by FAX.

If the transmission of the computer file and the figures has been
successful, you will receive an acknowledgment from us by electronic mail
and be informed of the manuscript's code number. If we encounter any
problems, we will contact you immediately, again by electronic mail, and
inform you of the problem. If you do not hear from us within 24 hours, you
can assume that we never received your file.

Copies of the manuscript {\em must not\/} be sent by conventional mail
unless there has been an unsuccessful transmission. For the initial
submission of a compuscript, a confirmation copy of the output generated at
the Editorial Offices is sent to the author by conventional mail.

\subsection{Floppy disk submissions}

Compuscripts may also be submitted or resubmitted on a DOS-formatted floppy
disk. The disk should be accompanied by a cover letter stating to which
journal the paper is being submitted, and a conventional mailing address
should be provided. A single copy of the manuscript should also be
enclosed. This single copy will only be used for processing if we are
unable to generate output from the disk submitted.

Floppy disks should be standard 5$\frac14$- or 3$\frac12$-inch diameter,
and should be mailed in a protective envelope to avoid possible damage to
the files.  For the initial submission of a compuscript, a confirmation
copy of the output generated at the Editorial Offices is sent to the author
by conventional mail.

\section{File Transmission Specifications}
\label{sec:filesubmit}

Please see Sec.\ \ref{sec:gensubmit} for important submittal procedures.


Compuscripts should be transmitted (1) via electronic mail or (2) on a
DOS-formatted floppy disk. The floppy should be labeled to indicate the
name of the author, the name of the file on the disk, and the return
address of the submitting author.

The electronic submission may be as a single large file (preferred), e.g.,
BB1010, or as a series of separately named files, e.g., BB1010, BB1010caps,
BB1010tabs. The code number of the compuscript does not have to be used as
the file name or names; smithpaper, smithcaps, smithfigs, etc., will
suffice.

Upon receipt of the file(s) by the editorial office, the contents will be
concatenated and assigned the appropriate code number. The compuscript file
will then be evaluated for various potential problems and compliance with
the input instructions.  The author may be advised of technical problems,
and asked to submit a proper file.

All e-mail receipts will be acknowledged. For the initial submission of a
compuscript, a confirmation copy of the output generated at the Editorial
Offices is sent to the author via conventional mail.

\section{Production of Author Proofs}
\label{sec:process}

When/if a manuscript is accepted for publication, the manuscript and the
compuscript file will be forwarded from the Editorial Offices to the
production group. Standard  procedures for  production will be in effect,
and the author will  receive the correspondence associated with routine
manuscript publication. The manuscript will be returned to the author with
the page proofs.  The manuscript will be thoroughly marked to indicate
where file changes were needed, and the author will be responsible for a
thorough reading of the proofs.

{\bf Note}: For {\em Physical Review Letters}, proofs are not normally
sent. If requested by the author, page copies are sent by FAX.

Although the author has assumed the responsibilities of keyboarding and
proofreading, the production staff will still perform a complete check of
the article and will insert editorial style changes. The changes are made
to the file at the production site. If their number is very high it may be
more efficient for the production process to proceed in the conventional
manner, and this is done. We do not ask the author to make the changes.

Page proofs should be carefully checked.  Return the marked manuscript and
the corrected proofs to the address indicated on the cover sheet; the
necessary corrections will be  handled by the production staff. The
electronic file should {\em not\/} be sent at this stage.

\section{Compuscript Instructions}
\label{sec:instruct}

Author-prepared compuscripts should include the following parts in this
order:  title, author, address, abstract, suggested PACS numbers (use
current Physics and Astronomy Classification Scheme), main manuscript body,
references, figure captions, and tables.  The production staff will add
verified PACS numbers and manuscript receipt date. Specific instructions
pertaining to various parts of the compuscript are listed below as well as
a short annotated example of compuscript input.  Formatting commands
(macros) are indicated where necessary.

Compuscripts should not contain any author-defined macros. Macros which are
simple text substitutions can be ``expanded'' by the author before
submittal. More complicated macros can create problems when the file is
edited for production and should also be avoided. See Sec.\
\ref{sec:macros} for further information.

This manual assumes some familiarity with \LaTeX; specifically the article
style. \REVTeX{} version 3.0 is based on article style. The notations
\verb+#1+, \verb+#2+, etc.\ are used throughout this manual to denote
user-supplied arguments. Commands will be shown in their full form; i.e.,
with their mandatory arguments.

Authors should also print the file apssamp.tex and compare the input for
further instruction and detailed examples. This guide and the sample file
depend upon each other to cover all features of \REVTeX. The file
template.aps may be copied to another name to use as a basis for creating a
new \REVTeX{} file.


\subsection{Galley style and preprint style}

The \REVTeX{} macro package has been developed to accomodate the preprint
needs of the author as well as the production needs of the APS. If you use
\REVTeX{} to prepare a manuscript for submission to the Editorial Offices
and participation in the author-prepared program, please follow these
steps: (a) Review the galley-format output (which is the default style),
paying particular attention to the physical presentation of equations,
tables, and column width (guidelines for equation breaking and layout are
described later and in the \SNG). (b) In the case of papers subject to
length restrictions, estimate the overall length by directly measuring the
journal text. Add in the space that will be occupied by any figures (at
their final reduction). On the journal page, the two-column area available
for text and figures is nominally 9.5 in.\ (24 cm) deep. (c) Switch to
preprint style (see below) and review the output, which is at a larger type
size. Check that equations and tables remain satisfactory. A \REVTeX{}
facility for ``squeezing'' preprint-style tables is described later. (d)
Submit the preprint-style file electronically to the Editorial Offices.

{\bf Galley style}, which is needed for production of the page that goes to
the printer, is produced by the following front matter:
\begin{verbatim}
\documentstyle[aps]{revtex}
\begin{document}
\end{verbatim}
Galley style activates the width-changing commands and centers equations by
default. Commands to place equations flush left or flush right are enabled
in galley style, but are not normally needed.

The editors of {\em Physical Review\/} require the traditional form of
manuscript for the review process: large typeface for legibility and ease
of review; adequate space between typed lines and wide margins, for editor
and reviewer marks and comments. This {\bf preprint style} can be obtained
with the following front matter:
\begin{verbatim}
\documentstyle[preprint,aps]{revtex}
\begin{document}
\end{verbatim}
Preprint style gives a constant-width output with equations centered.

\subsection{Style options}

The main style is \verb+revtex+, and \verb+aps+ is a mandatory style option
for papers to be submitted to the APS. Other style options include
\verb+eqsecnum+ (to number equations by section), \verb+preprint+ (to get
double-spaced output for submission purposes), \verb+tighten+ (to get
single-spaced output with the preprint style), and \verb+amsfonts+ and
\verb+amssymb+ (see Sec.\ \ref{sec:fonts}). There are also style options
for each APS journal: \verb+pra+, \verb+prb+, \verb+prc+, \verb+prd+,
\verb+pre+, and \verb+prl+, for {\it Physical Review\/} {\it A}, {\it B},
{\it C}, {\it D}, {\it E}, and {\it Letters}, respectively. \verb+pra+ is
the default. The \verb+prb+ option gives superscript reference citations,
as is the style for {\it Physical Review B}. The \verb+prl+ option yields
the slightly different line spacing of {\em Letters\/} (use for accurate
length estimates). Other than this, there are no substantial differences in
the journal options. Please do not use the \verb+prb+ option unless you
will be submitting to {\it Physical Review B}. The \verb+floats+ style
option enables \LaTeX{}-style floating figures and tables---it is for an
author's personal use, and is {\it not\/} for use with files to be
submitted to the APS. All files submitted to the APS should have figures
and tables at the end of the file. Other arrangements may not be
accommodated by the compuscript program.

\subsection{Front matter}
\label{sec:front}

The typical file will start off with a \LaTeX{} documentstyle line and
begin the document:
\begin{verbatim}
\documentstyle[pra,aps]{revtex}
\begin{document}
\end{verbatim}
Next comes the \verb+\draft+ command, which instructs \REVTeX{} to print
out the suggested PACS numbers from the \verb+\pacs{#1}+ command:
\begin{verbatim}
\draft
\end{verbatim}
The title is entered with the \verb+\title{#1}+ command:
\begin{verbatim}
\title{The title of the paper goes here}
\end{verbatim}
Now the author(s) and address(es) are entered:
\begin{verbatim}
\author{Jackson P. Jones}
\address{321 Main Street, Everville,
          Illinois 12345-6789}
\end{verbatim}
The \verb+\author{#1}+ and \verb+\address{#1}+ commands may be repeated as
a pair. Bottom-of-page footnotes to the author may be set using \LaTeX's
normal \verb+\thanks{#1}+ command. \LaTeX's normal footnote commands are
also enabled for use in \REVTeX{}. Bottom-of-page footnotes of any kind
should only be used by authors publishing in a journal that allows such
footnotes (e.g., {\em Physical Review C\/} and {\em D\/}). Use recent
issues of the particular journal as a guide. If unsure, use endnotes
instead of footnotes (see Sec.\ \ref{sec:endnotes}.).


The \verb+\date{#1}+ command can optionally be entered if the author wishes
to have dates print on the manuscript. \verb+#1+ represents the date of
receipt at the Editorial Offices. This date will be inserted at the
production site. Using \verb+\today+ will cause \LaTeX{} to insert the
current date whenever the file is run:
\begin{verbatim}
\date{\today}
\end{verbatim}
The \verb+\maketitle+ command must be entered just before the abstract.
Don't forget this command or the title, author(s), address(es), and date
will not print!
\begin{verbatim}
\maketitle
\end{verbatim}
Now enter the abstract in the abstract environment:
\begin{verbatim}
\begin{abstract}
In this paper we show the result of . . .
\end{abstract}
\end{verbatim}
The final piece of the front matter is the \verb+\pacs{#1}+ command. This
should be included even if \verb+#1+ is empty.
\begin{verbatim}
\pacs{23.23.+x, 56.65.Dy}
\end{verbatim}

\subsection{Section headings}

Section headings are input as in \LaTeX. The output is similar, with a few
extra features.

Four levels of headings are provided\break in \REVTeX{}:
\verb+\section{#1}+, \verb+\subsection{#1}+, \verb+\subsubsection{#1}+, and
\verb+\paragraph{#1}+. Use the star form of the command to suppress the
automatic numbering; e.g.,
\begin{verbatim}
     \section*{Introduction}
\end{verbatim}

To label a section heading for cross referencing use the \verb+\label{#1}+
command {\em after\/} the heading; e.g.,
\begin{verbatim}
     \section{Introduction}
     \label{sec:intro}
\end{verbatim}

All text in the \verb+\section{#1}+ command is automatically set uppercase.
If a lowercase letter is needed, just use \verb+\lowercase{x}+. For
example, to use ``He'' for helium in a \verb+\section{#1}+ command, type
\verb+H\lowercase{e}+ in \verb+#1+.

The \verb+\appendix+ command signals that all following sections are
appendixes, so \verb+\section{#1}+ after \verb+\appendix+ will set
\verb+#1+ as an appendix heading. \verb+#1+ may be empty. If only one
appendix is used, use a \verb+\section*{#1}+ command to suppress the
appendix letter in the section heading.

Use \verb+\protect\\+ to force a line break in a section heading. (Fragile
commands must be protected in section headings and captions, and \verb+\\+
is a fragile command.)

{\bf Note\/}: For {\em Physical Review Letters,} if there are to be section
headings, use only the fourth-level type, \verb+\paragraph{#1}+. Use the
``star form'' of the command (\verb+\paragraph*{#1}+) to avoid the
numbering that is normally attached [(a), (b), $\ldots$].


\subsection{Text}

{\bf Paragraphs} always begin with a blank input line. {\bf Do not
hyphenate} words at the end of a line; \TeX{} will do this.  Continue to
hyphenate modifiers within a line of text, e.g., ``author-prepared copy.''

{\bf Use curly quotes} for quotation marks around quoted text ({\tt
``xxx''}) not straight quotes ({\tt "xxx"}).  [For opening quotes, this is
two octal 140 characters (hex 60); for closing quotes, this is  two octal
047 (hex 27) characters.]

There are two commands that control the width of the text across the page
in the galley style; \verb+\narrowtext+ will set the column width to
3$\case3/8$ in., and the \verb+\widetext+ command will set the text 7 in.\
wide.  The \verb+\widetext+ command is needed to set very long equations.
See the section on displayed math, below.

Neither \verb+\narrowtext+ or \verb+\widetext+ have any effect on the
output if the front matter calls for the preprint style.  The preprint
style is a uniform 6.5 in.\ throughout.

Don't use \verb+\smallskip+, \verb+\bigskip+, or any other vertical motion
commands. Horizontal motion commands should be unnecessary as well.

\LaTeX's normal footnote commands are enabled for use in \REVTeX{}, but
should only be used by authors publishing in a journal that allows
bottom-of-page footnotes.  Use recent issues of the particular journal as a
guide.


\subsection{Math in text}

{\em Physical Review\/} uses the normal delimiter \$ for any \TeX{} in-line
math, e.g.,
\begin{quote}
     {the quantity $a^{z}$}
\end{quote}
is obtained from the input
\begin{verbatim}
     the quantity $a^{z}$
\end{verbatim}

\TeX{} will assume that you want a superscript or subscript to consist of
the first {\em token\/} (generally a single character or command) following
the \^{} or \_, {\em unless\/} you use curly brackets to delimit a
subscript/superscript. It is safest to use the curly brackets if unsure. We
have followed this convention in this guide. Again, don't use any vertical
motion commands in math; horizontal motion commands should be unnecessary
as well.

\subsection{Text in math}
\label{sec:textinmath}

There are times when an author needs to insert text into math. The
\verb+\rm+ command only switches to the Roman font for math letters. It
does not, for example, let you print a normal text hyphen: \verb+${\rm
e-p}$+ gives ``$\rm e-p$''. Using an \verb+\mbox{#1}+ will give you normal
text, including a hyphen, but will not scale correctly in superscripts:
\verb+$x_{\mbox{e-p}}$+ gives ``$x_{\mbox{e-p}}$''. The \verb+\text{#1}+
command will solve both problems. It gives you regular text {\em and\/}
scales correctly in superscripts: \verb+$y=x \text{ for } x_{\text{e-p}}$+
gives ``$y=x \text{ for } x_{\text{e-p}}$''.


\subsection{Displayed equations}

The  most common (and preferred) type of displayed equation in {\em
Physical Review\/} is a narrow, {\bf single-line equation, with an equation
number on the same line}. Try to set as many equations as you can in this
way.

Equations are now normally set centered in the column width for APS style
with \REVTeX{}. Setting the equation number is taken care of by
\REVTeX{}---the number will be set below the equation if necessary.
Breaking the equation into multiline format may be necessary for very long
equations. The \verb+eqnarray+ environment is used for this purpose.

Break at a sign of relation or an operator sign; the sign (e.g., $=$, $+$,
$\times$) begins the next line of the equation. Specify the proper
(leftmost) alignment of the line following a break (using \verb+&&+). A
multiline equation centers as a unit. Use a separate \verb+equation+ or
\verb+eqnarray+ environment (\verb+\begin{#1}+-\verb+\end{#1}+ command
pair) for {\em each\/} single-line equation or multiline equation. Short
displayed equations that can appear together on a single line may be placed
in one equation environment.

If an equation needs to be broken into more than four lines, it should be
set in a wide column for ease of reading, using the \verb+\widetext+
command. The author should return to \verb+\narrowtext+ as soon as possible
after one or more very long equations, but short pieces of narrow text
and/or math between nearly contiguous wide sections should be left wide and
incorporated into the surrounding wide sections.



In apssamp.tex, we have illustrated how to obtain each of the above.


\subsubsection{Numbering displayed equations}

The \REVTeX{} macro package allows two methods for numbering equations: you
can assign your own equation numbers or you can allow \REVTeX{} to number
for you.

Use the command \verb+\eqnum{#1}+ to number your own equations. You can
also use this command to produce a specific equation number not normally
obtainable; $(1')$, for example. Numbers assigned by \verb+\eqnum{#1}+ are
completely independent of the automatic numbering.

For automatically numbered single-line and multiline equations, use the
{\tt equation} and {\tt eqnarray} environments. You can use the
\verb+\[+,\verb+\]+ commands and the {\tt eqnarray$\ast$} environment for
unnumbered single-line and multiline equations, respectively. The command
\verb+\nonumber+ will suppress the numbering on a single line of an
eqnarray.

If you wish a series of equations to be a lettered sequence, e.g., (3a),
(3b), and (3c), just include the equation(s) or eqnarray(s) within the {\tt
mathletters} environment.

Finally, to have \REVTeX{} number equations by section, use the {\tt
eqsecnum} style option.

See apssamp.tex to see examples of how all this works.

\subsubsection{Cross-referencing displayed equations}

Authors will probably not cross-reference every equation in text. When a
numbered equation needs to be referred to in text by its number, the
\verb+\label{#1}+ and \verb+\ref{#1}+ commands should be used. The
\verb+\label{#1}+ command is used within the equation or the eqnarray line
to be referenced:

\smallskip

\leftline{\bf input:}
\begin{verbatim}
\begin{equation}
A=B \label{pauli}
\end{equation}
 ... It follows from Eq.\ (\ref{pauli})
that this is the case ...
\end{verbatim}
\leftline{\bf output:}
\begin{equation}
A=B \label{pauli}
\end{equation}
 ... It follows from Eq.\ (\ref{pauli})  that this
is the case ...

\smallskip

Please note the parentheses surrounding the command. They are necessary for
proper output. You can also label individual lines in an eqnarray. Numbers
produced with \verb+\eqnum{#1}+ can also be cross-referenced: just follow
the \verb+\eqnum{#1}+ command with a \verb+\label{#1}+ command.

Using a \verb+\label{#1}+ after \verb+\begin{mathletters}+ will allow you
to reference the {\em general\/} number of the equations in the
\verb+mathletters+ environment. For example, if
\begin{verbatim}
\begin{mathletters}
\label{allequations} % notice location
\begin{equation}
E=mc^2,\label{equationa}
\end{equation}
\begin{equation}
E=mc^2,\label{equationb}
\end{equation}
\begin{equation}
E=mc^2,\label{equationc}
\end{equation}
\end{mathletters}
\end{verbatim}


\noindent
gives the output
\smallskip\hrule\smallskip
\begin{mathletters}
\label{allequations}
\begin{equation}
E=mc^2,\label{equationa}
\end{equation}
\begin{equation}
E=mc^2,\label{equationb}
\end{equation}
\begin{equation}
E=mc^2,\label{equationc}
\end{equation}
\end{mathletters}
\smallskip\hrule\smallskip
\noindent
then \verb+Eq.\ (\ref{allequations})+ gives ``Eq.\ (\ref{allequations})''.


{\bf Note:} incorrect cross-referencing will result if \verb+\label{#1}+ is
used in an unnumbered single line equation (i.e., within the \verb+\[+ and
\verb+\]+ commands), or if \verb+\label{#1}+ is used on a line of an
eqnarray that is not being numbered (i.e., a line that has a
\verb+\nonumber+).

Please see Sec.\ \ref{sec:xrefs} for further information about
cross-referencing.

\subsection{Special characters}

Authors should avoid the use of specially designed ``define characters''
and choose symbols from those shown in the \LUG{}. There is no guarantee
that a specially designed definition will produce the desired results at
the production installation. If a special symbol is required and not listed
in the \LUG, please request special consideration in the cover letter
accompanying the file submittal.  The copyeditor will make note of it and
the production staff will attempt to accommodate the author.  Unusual
characters are subject to approval by the editor.

See Appendix A for a list of normal \LaTeX{} symbols, a list of symbols
available when the \verb+amsfonts+ and \verb+amssymb+ options are used, and
a list of extra symbols made available by \REVTeX.

\subsection{Endnotes and references}
\label{sec:endnotes}

The list of references should appear after the main body of the paper.
Please refer to the \SNG{} and recent issues of {\em Physical Review\/} for
current style. apssamp.tex shows examples of a variety of reference
entries, e.g., byline, journal, book, and private communication. Remember
to include a space (or hyphen) between author-name initials and between
initial and surname.

References will be listed in the reference section using the
\verb+\bibitem{#1}+ command, and they will be cited in text using the
\verb+\cite{#1}+ command.

A cite command that has a list of references will be output with
consecutive reference numbers collapsed; e.g., [1,2,3,5,7,8,9] will be
output as [1--3,5,7--9]. No ordering will be done, so [1,3,2,4] will be
output as [1,3,2,4]. If you use a \verb+\cite{#1}+ command with a long list
of tags, you may need to split the list over more than one line. Use a \%
character immediately following a comma to make sure that you do not get
unwanted spaces:
\begin{verbatim}
  . . . as shown in \cite{a,b,c,d,e,f,g,h,i,j,%
  k,l,m,n,o,p,q,r,s,t,u,v,w,x,y,z}
\end{verbatim}
Note the \% inserted after the comma on the first line. This ensures that
the entire list will be processed correctly.


A byline endnote and the first reference cited may  appear in the reference
section like this:

\smallskip

\begin{verbatim}
\begin{references}
\bibitem[*]{AAAuth}Present Address: Brookhaven
National Laboratory, Upton, New York, 11973.
\bibitem{tal82}Y. Tal and L. J. Bartolotti,
J. Chem. Phys. {\bf 76}, 4056 (1982).
\end{references}
\end{verbatim}

\smallskip

The \verb+[*]+ represents an optional, author-specified endnote symbol. If
an endnote symbol is not present, \REVTeX{} will assign the next available
reference number.

\verb+AAAuth+ and \verb+tal82+ are tags; they can be any string of letters
and numbers that you will easily associate with the reference.  The tag
will be used in text to tell \TeX{} what reference you want to cite. See
the example below.

\bigskip

% due to order of examples, we use \nocite so reftest will think
% this document is OK
\nocite{AAAuth}


\leftline{\bf input:}
\begin{verbatim}
This has been noted previously \cite{tal82}.
\end{verbatim}
\leftline{\bf output:}
\begin{quote}
This has been noted previously \cite{tal82}.
\end{quote}
\smallskip
Input for an author name with a byline endnote
is similar, but the output is different:

\bigskip

\leftline{\bf input:}
\begin{verbatim}
\author{A. A. Author\cite{AAAuth}}
\end{verbatim}
\leftline{\bf output:}
\bigskip
\begin{center} % we have to fake this.
A. A. Author$^{*}$
\end{center}

\bigskip

(See the \SNG{} for details on proper usage of byline endnotes.) Output
(galley style) in the reference section for the endnote and reference
samples above will look like this:

\newpage

\begin{references}
\bibitem[*]{AAAuth}Present Address: Brookhaven
National Laboratory, Upton, New York, 11973.
\bibitem{tal82}Y. Tal and L. J. Bartolotti,
J. Chem. Phys. {\bf 76}, 4056 (1982).
\end{references}

\smallskip


Since {\em Physical Review B\/} still uses superscript reference citations,
authors will need a special command to get on-line citations when the
\verb+prb+ style option is in effect. The command \verb+\onlinecite{#1}+
can be used for this purpose. For example, if the \verb+prb+ style option
is in effect, \verb+Ref.\ \onlinecite{tal82}+ will give the output ``Ref.\
1''.


It should be mentioned that the normal \LaTeX{} \verb+thebibliography+
environment will also work in \REVTeX{}.

\bigskip


There are also two tools for creating reference sections: prsty.bst and
reftest.tex.

prsty.bst is a \BibTeX{} style file that will output references in {\em
Physical Review\/} style. You should now be able to use the normal
\LaTeX/\BibTeX{} commands (\verb+\bibliographystyle{#1}+ and
\verb+\bibliography{#1}+) in lieu of typing in the references environment
by hand.  If you do this, you must of course {\em make sure\/} that you
keep the correct references with the main file when you submit it. For the
sake of simplicity, it is better if the Editorial Offices receive a single
file, especially in the case of an electronic submission. With these
concerns in mind, it is better to just comment out the two bibliography
commands and input the .bbl file directly into your main file just before
submitting it. It should run correctly this way. Please {\em do not\/} send
 .bib or .bst files to the Editorial Offices.


There is also a tool for authors that prepare their bibliographies by hand.
It is called reftest.tex. It will check to make sure that you (1) have no
uncited references, (2) have no undefined citations, and (3) have your
references in the same order as your citations. These are all requirements
in {\it Physical Review\/} style. This can only work if you use \LaTeX{}'s
\verb+\bibitem{#1}+ and \verb+\cite{#1}+ mechanisms. You just need to run
reftest through \LaTeX{}. For example, suppose you want to check the
references for the file test.tex. You would first run test.tex through
\LaTeX{} as usual. This creates an up-to-date auxiliary file, which is what
reftest uses to analyze your references. Then run \LaTeX{} on reftest. You
will be prompted for the name of the file you wish to check (without the
extension). Answer \verb+test+ at the prompt (not \verb+test.tex+ or
\verb+test.aux+). You will receive some messages on your screen and in the
log file (reftest.log) that tell you if there are any problems. You can
also preview or print the file reftest.dvi. If your references are out of
order, the correct order will be given only in reftest.dvi, not through
messages on the screen. Using reftest, an author can put the citations in
the correct order once, after writing the paper, by using the correct order
printed by reftest.


\bigskip

{\bf Quick guide to references:}

\begin{itemize}
\item The \verb+\bibitem{#1}+ command begins a reference.

\item  References should be listed in the reference section in the order in
which they are first cited in the text. (See next item.)

\item  References will automatically be numbered by \REVTeX{} in the order
in which they occur in the reference section, unless the author provides
his/her own label.

\item  The \verb+#1+ in \verb+\bibitem{#1}+ is a tag; it can be any string
of letters and numbers that you will easily associate with the reference.
This tag will be used in text (with the \verb+\cite{#1}+ command) to tell
\REVTeX{} what reference you want to cite.

\item  Endnotes to the byline should precede the references and should not
be numbered.  Provide a label in square brackets: e.g.,
\verb+\bibitem[*]{byline}+. For {\em Physical Review A, B, C, D, E,} and
{\em L,} use \verb+[*]+, \verb+[\dag]+,  \verb+[\ddag]+, \verb+[\S]+,
\verb+[**]+, \verb+[\dag\dag]+,  \verb+[\ddag\ddag]+, \verb+[\S\S]+ in the
order listed. This conforms to the requirements detailed in the \SNG{}.

\end{itemize}



\subsection{Figure captions}

Figure captions are a part of the compuscript and should appear after the
references. They should be input sequentially in the order in which they
are cited in the text; \LaTeX{}  will label and number the captions FIG.~1,
FIG.~2, etc.

Note below the use of the \verb+\label{#1}+ command; this is used to
cross-reference figures in text. The \verb+\label{#1}+ command should be
inserted inside or after the figure caption, but before the end of the
figure environment.

\smallskip

\leftline{\bf input:}
\begin{verbatim}
\begin{figure}
\caption{Text of first caption.}
\label{fig1}
\end{figure}

\begin{figure}
\caption{This is the second caption:
comparison of the differential cross
sections for the subprocess $qg \rightarrow
qggg$ of our approximation (dotted line)
versus the approximation of Maxwell together
with the use of the effective structure
function approximation(solid line).}
\label{fig2}
\end{figure}

\end{verbatim}

\leftline{\bf output:}

\begin{figure}
\caption{Text of first caption.}
\label{fig1}
\end{figure}

\begin{figure}
\caption{This is the second caption:
comparison of the differential cross sections
for the subprocess $qg \rightarrow qggg$ of our
approximation (dotted line) versus the
approximation of Maxwell together with the use
of the effective structure function
approximation(solid line).}
\label{fig2}
\end{figure}


\bigskip

Figures are cited in text with the use of  the \verb+\ref{#1}+ command:

\smallskip

\leftline{\bf input:}
\begin{verbatim}
 ...It can be seen from Fig.\ \ref{fig1} that
the data are inconsistent with this
conclusion...
\end{verbatim}

\leftline{\bf output:}

\bigskip

 ...It can be seen from Fig.\ \ref{fig1} that
the data are inconsistent with this conclusion...

\bigskip

\noindent
Further information on cross-referencing can be found in
Sec.~\ref{sec:xrefs}.

Figures in \REVTeX{} do not normally float. The \verb+floats+ style option
restores floating behavior for figures and tables. (This option has been
added for the author's personal use. It should not be used in any file
destined for submission to the APS.)

Figures and illustrations are submitted as originals or glossy prints.
Follow the rules in the \SNG{}  for style and specifications.

\subsection{Tables}
\label{sec:tables}

Tables are a part of the compuscript and should appear at the end of the
file.  Every table must have a complete caption and the correct number of
descriptive column headings.  Tables may be narrow (8.6~cm or 3.4~in.\
wide), medium (14~cm or 5.5~in.\ wide), or wide (17.8~cm or  7.0~in.\
wide), in galley style. An example of each appears in the sample
compuscript input. Tables will be sized at the production site to be set
narrow, medium, or wide (according to the number of columns, type of
material, etc.). (When using the preprint style for \REVTeX{}, all tables
will be set 6.5~in.\ wide.  \REVTeX{} will ignore \verb+\narrowtext+,
\verb+\mediumtext+, and \verb+\widetext+ commands if the front matter of
the file calls for the preprint style.)

Each table must begin with \verb+\begin{table}+, and end with
\verb+\end{table}+. Follow current {\em Physical Review\/} style concerning
placement of table lines.  (See examples in this guide and recent issues of
{\em Physical Review}.)  The table commands will set double horizontal
lines appearing at the beginning and end of the table; a single horizontal
rule should be set after the column headings with the use of the
\verb+\hline+ command. Extra sets of column headings within the table will
require another \verb+\hline+ to separate the headings from the column
entries. Do not insert any other horizontal or vertical lines in the body
of the table.

Since tables are automatically numbered, the \verb+\label{#1}+ command is
used with the \verb+\ref{#1}+ command to cite tables in text.  The
\verb+\label{#1}+ command should appear after the \verb+\caption{#1}+
command and before the \verb+\end{table}+ command.
\bigskip

{\bf Some special table considerations:}
\begin{itemize}

\item {\em Numerical columns\/} should align on the decimal point (or
decimal points if more than one is is present). A new column specifier,
``d,'' has been added. This should be used for simple numeric data with a
{\em single\/} decimal point. Material without a decimal point is simply
centered. Notes: entries that start with a decimal point (e.g.,
\verb+.003+) will not be aligned by the decimal point; you should add a
prezero to align the number correctly (e.g., \verb+0.003+). Additionally,
the entry is typeset in separate parts separated by any decimal point(s)
present, so parts of the entry to the left and right of a decimal point
must be able to be typeset separately. For example, \verb+$-1.23$+ will not
work in a \verb+d+ column. You will get a ``missing \$'' error because
\verb+$-1+ is typeset separately from \verb+23$+. Use instead
\verb+$-$1.23+. If multiple decimal points are present then the last is
used for alignment. To escape from the \verb+d+ column use
\verb+\multicolumn+ as usual. See apssamp.tex for examples.

\item Use \$ delimiters for all math in a table (no displayed equation
commands).

\item {\em  Footnotes\/} in a table must be labeled a, b, c, etc. Tablenote
commands that act just like regular footnotes have been added.  See
apssamp.tex for examples and explanations of use.

\item {\em Extra wide tables\/}
that will not fit into the 17.8-cm or 7.0-in.\ designation can be
manipulated by the production staff to produce a turned table that will
appear lengthwise on a page.  A cover letter requesting this special
handling should accompany file submittal. The author can use the
\verb+\squeezetable+ command with tables that do not fit on the page. This
command will make the fonts in the body of the table smaller, allowing
larger tables to fit onto the page.

\end{itemize}

Tables in \REVTeX{} do not normally float. The \verb+floats+ style option
restores floating behavior for figures and tables. Tables in \REVTeX{} will
break across pages if they are more than a full page in length, unless the
\verb+floats+ option has been selected. (The \verb+floats+ option has been
added for the author's personal use. It should not be used in any file
destined for submission to the APS.)


Authors should consider the feasibility of depositing extensive tabular
material in the Physics Auxiliary Publication Service of the American
Institute of Physics.  This material will usually be included in the
microfilm edition of the {\em Physical~Review}. For details, please write
to the Editorial Offices.


\section{Cross-referencing}
\label{sec:xrefs}

\REVTeX{} has built-in features for labeling and cross-referencing of
section headings, equations, tables, and figures. This section contains a
simplified explanation of cross-referencing features.  The format for using
these features with section headings, equations, tables, and figures is
discussed in the appropriate section.

Cross-referencing depends upon the use of ``tags,'' which are defined by
the user.  The \verb+\label{#1}+ command is used to identify tags for
\REVTeX . Tags are strings of characters that serve to label section
headings, equations, tables, and  figures, so that you don't need to know
what number \REVTeX{} has assigned to the item in order to talk about it in
text.

You will need to \LaTeX{} the original file more than once to ensure that
the tags have been properly linked to appropriate numbers.  If you add any
tags, you will need to \LaTeX{} more than once in subsequent work sessions:
\LaTeX{} will display an error message that ends with {\tt ... Rerun to get
cross-references right}. If you see that message, \LaTeX{} the file again.
If the error message appears after two \LaTeX ings, please check your
labels; you probably have referred to an item in text without tagging the
item.

You may not need to know (or care to know) all about what \REVTeX{} is
doing for autonumbering; however, you may want to know that when you
\LaTeX{} the file for the first time, an auxiliary file with the {\tt .aux}
filename extension will be created that connects numbers with their tags.
Subsequent \LaTeX ing accesses the auxiliary file to put the proper number
in the text.

\section{Fonts}
\label{sec:fonts}

Fonts are complicated. \REVTeX{} has been set up to give good results on
all \LaTeX{} installations, but no guarantee can be given that you will be
able to access all the font options---memory and font restrictions vary in
\TeX{} implementations and computers.

\subsection{Font selection schemes (OFSS and NFSS)}
\label{sec:onfss}

The font-selection scheme (FSS) that \LaTeX{} normally uses is somewhat
limited. This is known as the Old Font Selection Scheme (OFSS). A new
scheme has been written and distributed by Frank Mittelbach and Rainer
Sch\"opf---the so-called New Font Selection Scheme (NFSS). When you run a
file with \REVTeX{} you will see a message telling you which FSS you are
running on, right after the file aps.sty is read in. Several font problems
have been addressed in \REVTeX{} itself so it is not very important which
FSS you use, since \REVTeX{} has been written to run equally well on either
FSS. It should be said that, in general, an NFSS installation is more
capable than an OFSS installation.

At this time we are only supporting the NFSS if it is used with
oldlfont.sty. This makes the NFSS use the same font-selection macros as the
OFSS. If you use the NFSS and \verb+${\bf ABC}$+ gives boldface letters,
then you are using oldlfont.sty. Here is the output of \verb+${\bf ABC}$+:
${\bf ABC}$.

\subsection[Using bold symbols in math:
\btt{bbox$\{\#1\}$} and the {\protect\tt amsfonts} option]%
{Using bold symbols in math:\protect\\
\btt{bbox$\{\#1\}$} and the {\protect\tt amsfonts} option}
\label{sec:bboxamsfonts}

If you use regular \LaTeX{} with the OFSS, then you will probably get
incorrectly sized letters if you use \verb+\bf+ in a superscript. For
example, both letters in the output from \verb+${\bf x}^{\bf x}$+ are the
same size. This problem has been corrected if you run \REVTeX{} on the
OFSS.

There are also problems if you try to get bold math symbols in \LaTeX{}.
The solution given by the \LUG{} is to use \verb+\mbox{\boldmath$#1$}+
where \verb+#1+ is the symbol to be set bold. There are problems with this
approach. On the average \LaTeX{} setup with the OFSS you only use this for
\verb+\cal+, lowercase Greek letters, curly brackets, and other
miscellaneous symbols. You will not be able to get these characters in the
correct size in a superscript, either. If you use the NFSS you will also be
able to set digits, uppercase Greek letters, parentheses, and square
brackets in boldface using \verb+\boldmath+, but you will still not get
them in the correct size for superscripts.

The \REVTeX{} command \verb+\bbox{#1}+ will make \verb+#1+ bold in math
mode, but it will first make sure that it is the correct size, even in
superscripts. If the correct font in the correct size is not available then
you get \verb+#1+ at the correct size in lightface and \LaTeX{} will issue
a warning that says ``\verb+No+ \verb+\boldmath+ \verb+typeface+ \verb+in+
\verb+this+ \verb+size+ \verb+. . .+''.

So the \verb+\bbox{#1}+ command will give bold output of \verb+#1+ in math
mode. You can use it to get bold greek characters---upper- and
lowercase---and other symbols. It is still easier to use \verb+\bf+ to get
upright Latin letters in boldface. How much comes out bold and in how many
places you get bold output depends on how many fonts you have installed.
Using the \verb+amsfonts+ option will automatically use the extra AMS
Computer Modern math and symbol fonts for bold in superscripts and smaller
sizes, if you have installed the AMSFonts. The following will come out bold
in bboxes: normal math italic letters, numbers, Greek letters (uppercase
and lowercase), small bracketing and operators, and \verb+\cal+.

If you use only on-line bold math symbols there is no advantage to using
\verb+\mbox{\boldmath$#1$}+. If you use superscript bold math symbols then
you need the AMSFonts installed and the \verb+amsfonts+ style option to see
bold, but the symbols will be in the correct size and will come out bold at
the APS. Overall, it seems better to use \verb+\bbox{#1}+ everywhere.

Note that \verb+\bbox{#1}+ is a fragile command.

\subsection[Extra typefaces in math: {\protect\tt amsfonts} option]%
{Extra typefaces in math:\protect\\ {\protect\tt amsfonts} option}

In addition to the extra bold capabilities you get in math with the
\verb+amsfonts+ option, you also gain access to the Fraktur and Blackboard
Bold typefaces. You select these with normal font-switching commands:
\verb+${\frak G}$+ gives a Fraktur ``G'' and \verb+${\Bbb Z}$+ gives a
Blackboard Bold ``Z''. Fraktur will become bold in a bbox; there is no bold
version of Blackboard Bold. If you have the AMSFonts installed and have the
\verb+amsfonts+ option selected, example output can be found in Appendix A.


\subsection[Extra symbols in math: {\protect\tt amssymb} option]%
{Extra symbols in math:\protect\\ {\protect\tt amssymb} option}

Many new symbols are available to you if you have the AMSFonts installed.
The \verb+amssymb+ style option gives you all the font capabilities of the
\verb+amsfonts+ style option and further defines the commands to get the
symbols shown in Appendix A. See Appendix A for examples of the symbols and
instructions on use. These characters will scale correctly in different
areas of the paper and in superscripts. Note that the symbols and typefaces
in Appendix A will not be printed unless you have the AMSFonts installed
and have selected either the \verb+amsfonts+ or \verb+amssymb+ style
option.


\subsection{AMSFonts}
\label{AMSFonts}


The AMSFonts are fonts that were developed and are now made available free
of charge by the American Mathematical Society. The METAFONT source files
for \vadjust{\penalty-10000} these fonts are freely available, as are
precompiled .pk files, for those with Internet ftp capabilities. There are
two style options that can be used to access the AMSFonts: \verb+amsfonts+
and \verb+amssymb+.  Not distributed with \REVTeX{} are the files
amsfonts.sty and amssymb.sty from the AMS's \AmSLaTeX{} distribution. These
files are called in by \REVTeX{} to give you access to the AMSFonts when
the NFSS is in effect; \REVTeX{} itself will do the work necessary to allow
access when the OFSS is in effect.

The \verb+amsfonts+ option will define the \verb+\frak+ and \verb+\Bbb+
commands to switch to the Fraktur and Blackboard Bold fonts, respectively.
Fraktur characters will come out bold in a bbox, Blackboard Bold will not.
The \verb+amsfonts+ option also adds support for bold math letters and
symbols in smaller sizes in galley style and in superscripts when a
\verb+\bbox{#1}+ is used.  For example, \verb+$^{\bbox{\pi}}$+ gives a bold
lowercase pi in the superscript position. \verb+amssymb+ gives the
capabilities of the \verb+amsfonts+ option and additionally defines many
new characters for use in math. Here are the fonts you need to have
installed for the \verb+amsfonts+ and \verb+amssymb+ options:
\begin{enumerate}
\item msam5, msam6, msam7, msam8, msam9, and msam10 at their normal
(unmagnified) sizes, and msam10 at magsteps
 $\frac12$,1,2.
\item msbm5, msbm6, msbm7, msbm8, msbm9, and msbm10 at their normal
(unmagnified) sizes, and msbm10 at magsteps
 $\frac12$,1,2.
\item eufm5, eufm6, eufm7, eufm8, eufm9, and eufm10 at their normal
(unmagnified) sizes, and eufm10 at magsteps
 $\frac12$,1,2.
\item eufb5, eufb6, eufb7, eufb8, eufb9, and eufb10 at their normal
(unmagnified) sizes, and eufb10 at magsteps
 $\frac12$,1,2.
\item cmmib5, cmmib6, cmmib7, cmmib8, cmmib9, and cmmib10 at their normal
(unmagnified) sizes, and cmmib10 at magsteps
 $\frac12$,1,2.
\item cmbsy5, cmbsy6, cmbsy7, cmbsy8, cmbsy9, and cmbsy10 at their normal
(unmagnified) sizes, and cmbsy10 at magsteps
 $\frac12$,1,2.
\end{enumerate}

The following table shows only the  \REVTeX{} requirements for a {\em
minimal\/} AMSFonts installation; i.e., one that will function correctly at
normal sizes. ``Normal sizes'' means the sizes one gets automatically in
\REVTeX{} without using \LaTeX's explicit size-changing commands. (It may
be worth installing the fonts at larger sizes if you use the NFSS, for uses
other than \REVTeX{}.)

\newpage


\begin{table}
\caption{Minimum fonts and resolutions required for a 300-dpi installation
of AMSFonts for use of the {\tt amsfonts} and {\tt amssymb} style options
at normal \REVTeX{} sizes.}
\begin{tabular}{@{\hspace{.4in}}ll@{\hspace{.4in}}}
Font & Resolution(s) \\
\hline
  msam5 & 300 \\
  msam6 & 300 \\
  msam7 & 300 \\
  msam8 & 300 \\
  msam9 & 300 \\
  msam10 & 300,329,360,432 \\
  msbm5 & 300 \\
  msbm6 & 300 \\
  msbm7 & 300 \\
  msbm8 & 300 \\
  msbm9 & 300 \\
  msbm10 & 300,329,360,432 \\
  eufm5 & 300 \\
  eufm6 & 300 \\
  eufm7 & 300 \\
  eufm8 & 300 \\
  eufm9 & 300 \\
  eufm10 & 300,329,360,432 \\
  eufb5 & 300 \\
  eufb6 & 300 \\
  eufb7 & 300 \\
  eufb8 & 300 \\
  eufb9 & 300 \\
  eufb10 & 300,329,360,432 \\
  cmbsy5 & 300 \\
  cmbsy6 & 300 \\
  cmbsy7 & 300 \\
  cmbsy8 & 300 \\
  cmbsy9 & 300 \\
  cmbsy10 & 300,329,360,432 \\
  cmmib5 & 300 \\
  cmmib6 & 300 \\
  cmmib7 & 300 \\
  cmmib8 & 300 \\
  cmmib9 & 300 \\
  cmmib10 & 300,329,360,432 \\
\end{tabular}
\end{table}

\vskip-.5pc

\REVTeX{} does not support the use of the extra Euler fonts (the AMSFonts
starting with \verb+eur+ or \verb+eus+) or the Cyrillic fonts (the AMSFonts
starting with \verb+w+).

In addition, if you are using the NFSS you will need to have the files
amsfonts.sty and amssymb.sty from the \AmSLaTeX{} distribution. See Sec.\
\ref{sec:onfss} for an explanation of what the NFSS is.



\section{Installing \REVTeX{} 3.0}
\label{sec:upgrade}


%%%%%%%%%%%%%%%%%%%%%%% REVTeX DISTRIBUTION %%%%%%%%%%%%%%%%%%%%%%%

Files in the complete REVTeX 3.0 distribution:

  (a) the general files:
\begin{verbatim}
      README
      revtex.sty
      manintro.tex
      manend.tex
\end{verbatim}

\newpage

  (b) APS-specific files
\begin{verbatim}
      aps.sty
      aps10.sty
      aps12.sty
      prabib.sty
      prbbib.sty
      template.aps
      manaps.tex
      apssamp.tex
      prsty.bst
      reftest.tex
\end{verbatim}



%%%%%%%%%%%%%%%%%%%%%%% GETTING STARTED %%%%%%%%%%%%%%%%%%%%%%%

You must have \TeX{} and \LaTeX{} running to use these macros. All macros
run on a standard \LaTeX{} format.

Installing to just test the macros can be accomplished by copying all the
files into an unused directory, then changing to that directory. All the
files and facilities of \REVTeX{} should then be usable while you are in
that directory.

A more permanent installation would usually mean installing the .sty files
and reftest.tex files into the directory where \TeX{} usually looks for
input files, the .bst files where \BibTeX{} usually looks for its style
files, and the the rest of the files into a \REVTeX{} working directory or
a public directory, depending on your system setup. Care should be taken to
avoid writing over files already present. The files in the \REVTeX{}
package have been named to help avoid this problem, but there can be no
guarantee that our filenames are unique (e.g., README).


If you are upgrading an older version of \REVTeX{} ($<$3.0), the files from
the old version should be deleted first. Here is a list of the older files:

\begin{verbatim}
      readme
(make sure you only delete the REVTeX readme!)
      revtex.sty
      aps.sty
      aps10.sty
      aps12.sty
      preprint.sty
      eqsecnum.sty
      smplea.tex
      smpleb.tex
      smplec.tex
      apguide.tex
\end{verbatim}

The next thing you need to do is run \LaTeX{} on the appropriate user
guide(s). Run manintro.tex, manaps.tex, and manend.tex. Collecting the
output together in this order will provide a complete user guide for APS
authors. Running manaip.tex and manosa.tex will give you the chapters for
AIP and OSA authors.

If needed, consult a local \LaTeX{} user or system support person for
information on how to run \LaTeX{} and print its output on your local
system.


%%%%%%%%%%%%%%%%%%%%%%% ERROR REPORTS %%%%%%%%%%%%%%%%%%%%%%%

Before you report an error please check that
\begin{itemize}
  \item the error isn't caused by obsolete versions of other software.
    \LaTeX{} from 1986 is a good candidate.

  \item you use an original version of the package.
\end{itemize}


If you think you have found a genuine bug please report it, together
with the following information:
\begin{itemize}
  \item version of the \REVTeX{} file(s)

  \item version (date!) of your \LaTeX{}

  \item a short test file showing the behavior with all unnecessary
    coding removed. The log file showing the problem might also help.
\end{itemize}


Errors should be reported to the support person listed in
Sec.\ \ref{sec:contacts}.






\subsection[Differences between \REVTeX{} v3.0 and \LaTeX{} article style]%
{Differences between \REVTeX{} v3.0 and LaTeX{} article style}
\label{sec:ltor}

\begin{enumerate}
\item The documentstyle is different. The main style is \verb+revtex+, and
\verb+aps+ is a mandatory style option for APS authors. Other style options
are \verb+eqsecnum+ (to number equations by section), \verb+preprint+ (to
get double-spaced output for submission purposes), \verb+tighten+ (to get
single-spaced output with the preprint option), and \verb+amsfonts+ and
\verb+amssymb+ (for extra font capabilities, see Sec.\ \ref{sec:fonts}).
There are also style options for each APS journal: \verb+pra+, \verb+prb+,
\verb+prc+, \verb+prd+, \verb+pre+, and \verb+prl+, for {\it Physical
Review\/} {\it A}, {\it B}, {\it C}, {\it D}, {\it E}, and {\it Letters},
respectively. \verb+pra+ is the default. The \verb+prb+ option gives
superscript reference citations, as is the style for {\it Physical Review
B}. Other than this, there are no substantial differences in the journal
options. Please do not use the \verb+prb+ option unless you will be
submitting to {\it Physical Review B}. The \verb+prl+ option yields a
slightly different line spacing for accurate PRL length estimates. The
\verb+floats+ style option enables \LaTeX{}-style floating figures and
tables---it is {\it not\/} for use with files to be submitted to the APS.

\item The beginning of a file is different in \REVTeX. The top of a typical
paper might look like (cf.\ template.aps)
\begin{verbatim}
\documentstyle[pra,aps]{revtex}
\begin{document}
\draft
\title{Title here}
\author{Author(s) here}
\address{Address(es) here}
\author{Another author(s) here}
\address{Another address(es) here}
\date{\today}
\maketitle
\begin{abstract}
Abstract here.
\end{abstract}
\pacs{PACS numbers here}
\end{verbatim}
Note the \verb+\draft+, \verb+\address{#1}+, and \verb+\pacs{#1}+ commands.
See Sec.\ \ref{sec:front} for details.

\item Figures and tables are input the same as in \LaTeX{}. Tables can be
over 1 page long and will break automatically across pages. Figures and
tables do not float unless you use the \verb+floats+ option. As an
experiment, floats can be re-enabled by use of the \verb+floats+ style
option. You {\em cannot\/} use this for files that you submit to the APS;
it is added for your personal use. Floating tables and figures will not
break across pages. All tables expand to fill the column width.

\item The \verb+\text{#1}+ command will print \verb+#1+ as regular text
output in math. In particular, you will get hyphens instead of minus signs.
Used in a superscript, you will get the correct size. See Sec.\
\ref{sec:textinmath}.

\item Using a \verb+\label{#1}+ after \verb+\begin{mathletters}+ will allow
you to reference the {\em general\/} number of the equations in the
\verb+mathletters+ environment. For example, if
\begin{verbatim}
\begin{mathletters}
\label{alleqs}  % observe location
\begin{equation}
E=mc^2,\label{eqa}
\end{equation}
\begin{equation}
E=mc^2,\label{eqb}
\end{equation}
\begin{equation}
E=mc^2,\label{eqc}
\end{equation}
\end{mathletters}
\end{verbatim}
gives the output
\smallskip\hrule\smallskip
\begin{mathletters}
\label{alleqs}
\begin{equation}
E=mc^2,\label{eqa}
\end{equation}
\begin{equation}
E=mc^2,\label{eqb}
\end{equation}
\begin{equation}
E=mc^2,\label{eqc}
\end{equation}
\end{mathletters}
\smallskip\hrule\smallskip
then \verb+Eq.\ (\ref{alleqs})+ gives ``Eq.\ (\ref{alleqs})''.

\item When you use the \verb+\bf+ command in superscripts, you will get a
correctly sized character.


\item There are commands \verb+\tablenote{#1}+, \verb+\tablenotetext{#1}+,
and \verb+\tablenotemark{#1}+. These commands work in direct analogy to the
regular footnoting commands in \LaTeX{}. They should be used only in
tables, and the notes will come out at the bottom of the table in which
they appear. See apssamp.tex for instructions and examples.

\item There is a new letter for specifying columns in tabular environments.
Using \verb+d+ in a tabular specification will create a column centered on
the decimal points of the entries. See Sec.\ \ref{sec:tables} for details,
apssamp.tex for examples.


\item Extra diacritics are available: \verb+\tensor+ (double-headed
overarrow), \verb+\overdots+ (triple overdots), \verb+\overstar+ (star),
\verb+\overcirc+ (circle), \verb+\loarrow+ (left-going overarrow), and
\verb+\roarrow+ (right-going overarrow). They scale correctly in
superscripts. See Appendix A for examples.

\item \verb+\case{#1}{#2}+ will give text-style fractions (smaller) in
display-style math.

\item There is a \BibTeX{} style file, \verb+prsty.bst+, that can be used
to prepare bibliographies automatically, as explained in Lamport's book.

\item There is also a tool for authors that prepare their bibliographies by
hand. It is called reftest.tex. It will check to make sure that you (1)
have no uncited bibitems, (2) have no undefined citations, and (3) have
your bibitems in the same order as your citations. These are all
requirements in {\it Physical Review\/} style. See Sec.\
\ref{sec:endnotes}.

\item The \LaTeX{} command \verb+\extracolsep{#1}+ sets extra intercolumn
spacing, but this extra spacing has already been set in \REVTeX{} to allow
the columns in the table to expand out to fill the text width. Therefore,
\verb+\extracolsep{#1}+ will not work in \REVTeX{}. Use the \verb+@{#1}+
command to specify spacing between two adjacent columns. See Appendix C.9.2
of Lamport for a full explanation of \verb+@{#1}+. An example has been
given in apssamp.tex.

\item We have tried to make this version of \REVTeX{} as compatible as
possible with \LaTeX{}, including features we (APS) do not require at this
time such as \verb+\twocolumn+, table of contents, etc. As an experiment,
floats can be re-enabled by use of the \verb+floats+ style option. You {\em
cannot\/} use this for files that you submit to the APS; it is added for
your personal use. Please let us know of any other incompatibilities.

\item You should be able to use either the OFSS or NFSS with these macros,
with similar results (the NFSS may give you more fonts in some areas,
depending what your setup is). At this time we are only supporting the NFSS
with \verb+oldlfont.sty+ active. Please let us know if you have any
problems in this area. And don't worry if you don't know what the OFSS and
NFSS are. (See Sec.\ \ref{sec:fonts} if you are curious.) \end{enumerate}



\subsection{Differences between \REVTeX{} v3.0 and \REVTeX{} v$<$3.0}
\label{sec:2to3}

Here are the differences between \REVTeX{} v3.0 and versions $<$3.0.
\begin{enumerate}
\item The \verb+\documentstyle+ is different. The main style is now
\verb+revtex+, and \verb+aps+ is a mandatory style option for APS
authors. Other
style options are \verb+eqsecnum+ (to number equations by section),
\verb+preprint+ (to get double-spaced output for submission purposes),
\verb+tighten+ (to get single-spaced output with the preprint option),
and \verb+amsfonts+ and \verb+amssymb+ (see below).
There are also style options for each APS journal: \verb+pra+, \verb+prb+,
\verb+prc+, \verb+prd+,
\verb+pre+, and \verb+prl+, for {\it Physical
Review\/} {\it A}, {\it B}, {\it C}, {\it D}, {\it E}, and {\it Letters},
respectively. \verb+pra+ is the default. The \verb+prb+
option gives superscript
reference citations, as is the style for {\it Physical Review B}.
Other than this, there are no substantial differences in the journal options.
Please do not use the \verb+prb+ option unless you
will be submitting to {\it Physical Review B}.
The \verb+prl+ option yields a slightly different line spacing for accurate
PRL length estimates.
The \verb+floats+ style option enables \LaTeX{}-style floating figures and
tables---it is {\it not\/} for use with files to be submitted to the
APS.

\item The beginning of a file is different in v3.0. It is more like \LaTeX{}'s
article style.
See Sec.\ \ref{sec:front} and template.aps for details.

\item The \verb+\tightenlines+ command is now called \verb+\tighten+.
The \verb+tighten+ style option can be used to get a single-spaced
preprint.


\item You can now use the
\verb+\label{#1}+ command
after \verb+\begin{mathletters}+. This allows you to
reference the {\em general\/} number of the equations in the
\verb+mathletters+ environment. For example, if
\begin{verbatim}
\begin{mathletters}
\label{eq:all} % note location
\begin{equation}
E=mc^2,\label{eq:a}
\end{equation}
\begin{equation}
E=mc^2,\label{eq:b}
\end{equation}
\begin{equation}
E=mc^2,\label{eq:c}
\end{equation}
\end{mathletters}
\end{verbatim}
gives the output
\smallskip\hrule\smallskip
\begin{mathletters}
\label{eq:all}
\begin{equation}
E=mc^2,\label{eq:a}
\end{equation}
\begin{equation}
E=mc^2,\label{eq:b}
\end{equation}
\begin{equation}
E=mc^2,\label{eq:c}
\end{equation}
\end{mathletters}
\smallskip\hrule\smallskip
then \verb+Eq.\ (\ref{eq:all})+ gives ``Eq.\ (\ref{eq:all})''.


\item The \verb+\nonum+ command is no longer used. You can get unnumbered
section headings by using the ``star form'' of the command; e.g.,
\begin{verbatim}
\section*{Unnumbered section}
\end{verbatim}
This is normal \LaTeX{} practice. Appendix headings used to be set with the
\verb+\appendix{#1}+ and \verb+\unletteredappendix{#1}+ commands. The
normal \LaTeX{} convention is now used: When you wish to start the
appendix(es) use the \verb+\appendix+ command (no argument). Then
\verb+\section{#1}+ will give an appendix heading and \verb+\section*{#1}+
gives an unlettered appendix. All numbering, labeling, and
cross-referencing remain the same.

\item Figure captions should be input as in \LaTeX{}.
The syntax is exactly the same as in \LaTeX{}; e.g.,
\begin{verbatim}
\begin{figure}
\caption{Here is the caption.\label{xxx}}
\end{figure}
\end{verbatim}
Note that the label can be inside the caption or after it, as long as it is
inside the figure environment and does not come before the caption.

\item All the normal \LaTeX{} rules about fragile commands in moving
arguments apply. Especially in figure and table captions, where \REVTeX{}
users have not had to worry until now. An error of
\begin{verbatim}
  ! Argument of \@caption has an extra }
\end{verbatim}
generally indicates that a command in a caption must be preceded by
the \verb+\protect+ command.
\verb+\protect+ will need to be used much more often, unfortunately.
Specifically, \verb+\\+ and \verb+\ref+, along with other fragile commands,
will need to be protected in section heads and captions.

\item Equations and eqnarrays center in the column width, instead of
appearing indented. The \verb+\FL+ and \verb+\FR+ commands still flush
equations left and right, repectively, in the column, but should be needed
only occasionally. This is how the equations will appear in the journal.
This is a style decision, {\em not\/} a programming decision.

\item The AMSFonts are supported through the \verb+amsfonts+ and
\verb+amssymb+ style options.  These work similar on both \LaTeX's normal
font macros (OFSS) and the so-called NFSS. You do not need to know what the
OFSS and NFSS are. (See Sec.\ \ref{sec:fonts} if you are curious.)


The \verb+amsfonts+ option will define the \verb+\frak+ and \verb+\Bbb+
commands to switch to the Fraktur and Blackboard Bold fonts, respectively.
Fraktur characters will come out bold in a bbox (see below), Blackboard
Bold will not. The \verb+amsfonts+ option also adds support for bold math
letters and symbols in smaller sizes in galley style and in superscripts
when a bbox is used.  For example, \verb+$^{\bbox{\pi}}$+ gives a bold
lowercase pi in the superscript position. \verb+\bbox{#1}+ is explained
below.

The \verb+amssymb+ option gives all the capabilities of the \verb+amsfonts+
option, but also defines names for all the extra symbols in the AMSFonts.

See Sec.\ \ref{sec:fonts} for font details.

\item The \verb+\bbox{#1}+ command will give bold output of \verb+#1+ in
math mode. You can use it to get bold greek characters---upper- and
lowercase---and other symbols. It is still easier to use \verb+\bf+ to get
upright Latin letters in boldface. How much comes out bold and in how many
places you get bold output depends on how many fonts you have installed.
Using the \verb+amsfonts+ option will automatically enable bold Fraktur in
a bbox and will use the extra AMS Computer Modern math and symbol fonts for
bold in superscripts and smaller sizes. The following will come out bold in
bboxes: normal math italic letters, numbers, Greek letters (uppercase and
lowercase), small bracketing and operators, \verb+\cal+, and \verb+\frak+.
See Sec.\ \ref{sec:bboxamsfonts}.

\item The \verb+\text{#1}+ command will print \verb+#1+ as regular text
output in math. In particular, you will get hyphens instead of minus signs.
Used in a superscript, you will get the correct size. See Sec.\
\ref{sec:textinmath}.

\item When you use the \verb+\bf+ command in superscripts, you will get a
correctly sized character.

\item The \verb+/+ in the \verb+\case+ command is now optional. Either
\verb+\case{1}/{2}+ or \verb+\case{1}{2}+ will work. Our hope is that the
latter, more usual construct will become the norm.


\item There are commands \verb+\tablenote{#1}+, \verb+\tablenotetext{#1}+,
and \verb+\tablenotemark{#1}+. These commands work in direct analogy to the
regular footnoting commands in \LaTeX{}. They should be used only in
tables, and the notes will come out at the bottom of the table in which
they appear. See apssamp.tex for instructions and examples.


\item There is a new letter for specifying columns in tabular environments.
Using \verb+d+ in a tabular specification will create a column centered on
the decimal points of the entries. See Sec.\ \ref{sec:tables} for details;
see apssamp.tex for examples. \verb+\dec+ and \verb+\setdec+ should not be
used.

\item The symbols $\lesssim,\gtrsim$ were called \verb+\alt+,\verb+\agt+ in
previous versions of \REVTeX{}. These symbols are now called
\verb+\lesssim+ and \verb+\gtrsim+, for compatibility with normal AMSFonts
notation.

\item Extra diacritics are available: \verb+\tensor+ (double-headed
overarrow), \verb+\overdots+ (triple overdots), \verb+\overstar+ (star),
\verb+\overcirc+ (circle), \verb+\loarrow+ (left-going overarrow), and
\verb+\roarrow+ (right-going overarrow). They scale correctly in
superscripts. See Appendix A for examples.

\item There is a \BibTeX{} style file, \verb+prsty.bst+, that can be used
to prepare bibliographies automatically, as explained in Lamport's book.

\item There is also a tool for authors that prepare their bibliographies by
hand. It is called reftest.tex. It will check to make sure that you (1)
have no uncited bibitems, (2) have no undefined citations, and (3) have
your bibitems in the same order as your citations. These are all
requirements in {\it Physical Review\/} style. This can only work if you
use \LaTeX{}'s \verb+\bibitem+ and \verb+\cite+ mechanisms. See Sec.\
\ref{sec:endnotes}.

\item We have tried to make this version of \REVTeX{} as compatible as
possible with \LaTeX{}, including features we (APS) do not require at this
time such as \verb+\twocolumn+, table of contents, etc. As an experiment,
floats can be re-enabled by use of the \verb+floats+ style option. You {\em
cannot\/} use this for files that you submit to the APS; it is added for
your personal use. Please let us know of any incompatibilities.

\item You should be able to use either the OFSS or NFSS with these macros,
with similar results (the NFSS may give you more fonts in some areas,
depending what your setup is). At this time we are only supporting the NFSS
with \verb+oldlfont.sty+ active. Please let us know if you have any
problems in this area. And don't worry if you don't know what the OFSS and
NFSS are. (See Sec.\ \ref{sec:fonts} if you are curious.)
\end{enumerate}



\subsection{Running an older \REVTeX{} file under \REVTeX{} v3.0}

If you have version 2.x files that you wish to run with v3.0, use the
\verb+version2+ style option.

If you have a file formatted with version 1.x, first delete the optional
arguments to the bibitem commands. That is, \verb+\bibitem[1]{firstref}+
would become \verb+\bibitem{firstref}+, etc. Then use the \verb+version2+
style option to run the paper. This means a documentstyle line of
\begin{verbatim}
\documentstyle[version2,aps]{revtex}
\end{verbatim}


\subsection{Converting a \REVTeX{} v2.x file to \REVTeX{} v3.0}

To change a file over from v2.x to run with v3.0, you need to make the
following changes. If you have a file formatted with version 1.x, first
delete the optional arguments to the \verb+\bibitem+ commands. That is,
\verb+\bibitem[1]{firstref}+ would become \verb+\bibitem{firstref}+, etc.
Then make the following changes.

\smallskip
  \hrule \nobreak\smallskip
\begin{verbatim}
\documentstyle[revtex]{aps}
\end{verbatim}
  becomes
\begin{verbatim}
\documentstyle[aps]{revtex}
\end{verbatim}
The \verb+preprint+ and \verb+eqsecnum+ options work the same as before.
\smallskip \hrule \nobreak\smallskip
\begin{verbatim}
\tightenlines
\end{verbatim}
  becomes
\begin{verbatim}
\tighten
\end{verbatim}
\smallskip \hrule \nobreak\smallskip
\begin{verbatim}
$\alt$ and $\agt$
\end{verbatim}
  become
\begin{verbatim}
$\lesssim$ and $\gtrsim$
\end{verbatim}
  \smallskip \hrule  \smallskip
\begin{verbatim}
\begin{title}
XXX
\end{title}
\end{verbatim}
  becomes
\begin{verbatim}
\title{XXX}
\end{verbatim}
  \smallskip \hrule  \smallskip
\begin{verbatim}
\begin{instit}
XXX
\end{instit}
\end{verbatim}
  becomes
\begin{verbatim}
\address{XXX}
\end{verbatim}
  \smallskip \hrule  \smallskip
\begin{verbatim}
\receipt{XXX}
\end{verbatim}
  becomes
\begin{verbatim}
\date{XXX}
\end{verbatim}
  \smallskip \hrule  \smallskip
  \verb+\maketitle+~should~be~added~just~before the \verb+\begin{abstract}+
  command is entered.
  \smallskip \hrule  \smallskip
\begin{verbatim}
\figure{XXX\label{YYY}}
\end{verbatim}
  becomes
\begin{verbatim}
\begin{figure}
\caption{XXX\label{YYY}}
\end{figure}
\end{verbatim}
  \smallskip \hrule  \smallskip
Tablenotes should be converted over to run correctly. The first three
\verb+\tablenotes{#1}+ commands would be treated as follows:
\begin{verbatim}
\tablenotes{$^{\rm a}$Tablenote a.}
\tablenotes{$^{\rm b}$Tablenote b.}
\tablenotes{$^{\rm c}$Tablenote c.}
\end{verbatim}
becomes
\begin{verbatim}
\tablenotetext[1]{Tablenote a.}
\tablenotetext[2]{Tablenote b.}
\tablenotetext[3]{Tablenote c.}
\end{verbatim}
Further tablenotes should be treated similarly.
This should allow older tables to run correctly. There is much greater
capability in the new tablenote macros, however. See apssamp.tex for
examples and explanations.
  \smallskip \hrule  \smallskip
\begin{verbatim}
\nonum\section{XXX}
\end{verbatim}
  becomes
\begin{verbatim}
\section*{XXX}
\end{verbatim}
  \smallskip \hrule  \smallskip
Note that there should only be one \verb+\unletteredappendix+
command in any file.
\begin{verbatim}
\unletteredappendix{XXX}
\end{verbatim}
  becomes
\begin{verbatim}
\appendix
\section*{XXX}
\end{verbatim}
  \smallskip \hrule  \smallskip
Note that there may be more than one \verb+\appendix+
command in any file. The first occurrence should be treated as follows:
\begin{verbatim}
\appendix{XXX}
\end{verbatim}
  becomes
\begin{verbatim}
\appendix
\section{XXX}
\end{verbatim}
Subsequent occurrences should be treated as follows:
\begin{verbatim}
\appendix{XXX}
\end{verbatim}
  becomes
\begin{verbatim}
\section{XXX}
\end{verbatim}
\nobreak  \smallskip \hrule \nobreak \smallskip





\nobreak
\section{Fast Facts for New Participants}
\label{sec:fastfacts}

Since 1980 the American Physical Society has been accepting computer files
from authors and using those files (``compuscripts'') in the preparation of
their papers. In 1987 a research and development effort was launched to
expand this service to include \TeX-formatted compuscripts. We have been
publishing \TeX{} compuscripts since 1988.

\medskip
{\bf Benefits to Author: Reduced Proofreading}. Since final output is
composed from virtually the same file that produced the manuscript,
proofreading time can be reduced significantly.

\medskip
{\bf How to Qualify: Use \REVTeX{} or  \LaTeX{} Macros}. The APS has
developed a ``compuscript toolbox,'' which is composed of macros, a {\em
Physical Review Input Guide for \TeX{} Author-Prepared Compuscripts}, a
\SNG{}, and complete instructions on how to prepare a manuscript for the
{\em Physical Review}.

\medskip


{\bf Which Journals are Participants?} Papers submitted to {\em Physical
Review A, B, C, D, E,} or {\em Letters\/} may qualify.

\medskip
{\bf Media for Compuscripts}. We can process files received via electronic
mail or on DOS-formatted floppy disk.

\medskip
{\bf How to Participate}.  Make the original submission of your \REVTeX{}
compuscript to the Editorial Offices via electronic mail or DOS-formatted
floppy disk. We will contact you to confirm file qualification. During the
review process, resubmit in one of the electronic modes; do {\em not\/}
resubmit by conventional means.

\medskip
{\bf Obtaining the Toolbox and More Information}. Contact

\medskip
Christopher B. Hamlin

APS Liaison Office

500 Sunnyside Boulevard

Woodbury, New York~~11797

\medskip
Telephone: (516) 576-2390

FAX: (516) 349-7817

E-mail address: {\tt mis@aps.org} (Internet)

\leavevmode\phantom{E mail address: }{\tt mis@apsedoff} (Bitnet)


\section{Common Author Questions}
\label{sec:macros}

{\bf Page Charges.}
Historically, page charges for compuscripts have fluctuated. The Council
has, at various points over the ten-year history of the program, voted for
reduced page charges, the elimination of page charges, and full page
charges for compuscripts.

In a current three-year pilot program, publication charges have been
suspended for papers accepted after 30 June 1992 for publication in {\em
Physical Review C\/} as compuscripts, or accepted after that date for
publication in {\em Physical Review D\/} (whether as compuscripts or via
conventional production). Publication charges continue to apply to
compuscripts in the other journals.

\bigskip
{\bf Macros.}
Some authors use specialized definitions, or macros, in their files.  These
definitions serve different purposes:  some macros save the author from
typing a long character string repetitively (Type 1), and  some macros act
as commands to the \TeX{} program (Type 2).

\smallskip
{\bf How macros become problematic.} Type-1 macros enable the author to
define a frequently occurring string of characters as a shorter string, in
order to save typing time. These macros become problematic at production
stage when the frequently occurring string needs stylistic or grammatical
changes. At that point, the production of the compuscript requires either
(1) evaluation by a staff member who is fluent in macro construction,
because the macro will need to changed, or (2) additional attention by a
staff member who will change every occurrence of the string manually in the
file.

Neither one of the two alternatives is in the spirit of the compuscript
program: the author of the compuscript will need to proofread the galleys
very carefully, and production time/cost has become inflated by processing
as a compuscript!

Type-2 macros enable the author to give commands to the \TeX\ program.
Authors need to do this when the macro package they are using does not
contain a command that they need.

Type-2 macros frequently occur in \LaTeX{} compuscripts. This is because
the macros do not provide for certain elements of {\em Physical Review}
style; for instance, letters in equation numbers. Authors who are using
\LaTeX{} to compose their compuscripts would need to develop a command that
would number their equations (1a), (1b), etc.

Type-2 macros should not occur in \REVTeX{} compuscripts. The \REVTeX{}
macros ideally represent a complete command set, allowing the author to do
anything that {\em Physical Review\/} style allows.

Since our first release of \REVTeX{}, some authors have given us feedback
on macros that they would like included in the \REVTeX{} package, as well
as changes they would like made.  Some of these suggestions are feasible
(accurate double spacing) and some are not (the ability to draw many
horizontal lines within the tabular environment). Relevant suggestions have
been incorporated into \REVTeX{}.  The APS will review compuscripts
containing Type-2 macros when authors feel that they have found a
deficiency in \REVTeX{}.

{\bf Authors should remove all macros from their compuscripts.} Type-1
macros are easy to remove, with the aid of a word processor which is
equipped to do global substitutions. Type-2 macros are not easy to remove;
authors may not be able to do so and therefore will be ineligible for the
compuscript program.

This represents the current policy for compuscript page charges and macro
usage, which is subject to change. If you have any questions regarding
these issues please contact the authors of this guide.

\section{Troubleshooting and other questions}

This section is intended to help authors with problems  and common
questions that arise when using \REVTeX{}.

{\bf Question: Where are the appendixes that are mentioned in the text of
this document?} You need to run the file manend.tex through \LaTeX. Also,
you may elect to try a documentstyle with either the \verb+amsfonts+ or
\verb+amssymb+ option selected, if you have the AMSFonts installed and the
correct \AmSLaTeX{} files (if needed). See Sec.\ \ref{sec:fonts} for
details on these options.

{\bf Question: \REVTeX{} types out information about the NFSS (or OFSS).
What does this mean?} This is simply information to let you know which FSS
you are running on. Normally this information is not important. (See Sec.\
\ref{sec:fonts} if you are curious.)


{\bf Question: How do I get lowercase letters in the \verb+\section{#1}+
command?} All text in the \verb+\section{#1}+ command is automatically set
uppercase. If a lowercase letter is needed, just use \verb+\lowercase{x}+.
For example, to use ``He'' for helium in a \verb+\section{#1}+ command,
type \verb+H\lowercase{e}+ in \verb+#1+. This also works in math mode:
\verb+$\lowercase{e}^2$+ in a \verb+\section{#1}+ command will output
$e^2$.

{\bf Problem: I am getting error messages on the lines of my
\verb+\section{#1}+,
\verb+\subsection{#1}+,
\verb+\subsubsection{#1}+, or
\verb+\caption{#1}+ commands, and I can't understand why!}
You may have a so-called ``fragile'' command in a section heading or
caption. This is solved in \LaTeX{} by immediately preceding the fragile
command with \verb+\protect+. Some common fragile commands include:
\begin{verbatim}
  \cite{#1}
  \onlinecite{#1}
  \ref{#1}
  \sqrt{#1}
  \openone
  \lesssim
  \gtrsim
  \\
  \newline
  \bbox{#1}
\end{verbatim}
So, if you have one of these commands, or another fragile command (check
Lamport's book), just precede them with \verb+\protect+ and try running
the file again.
For example, if you have
\begin{verbatim}
\section{The next result: $\sqrt{-1}$}
\end{verbatim}
just change it to
\begin{verbatim}
\section{The next result: $\protect\sqrt{-1}$}
\end{verbatim}

{\bf Problem: I have tables that do not fit into the preprint width.} Try
putting the \verb+\squeezetable+ command right after the
\verb+\begin{table}+ command. This will reduce the size of the type in the
body of the table, thus allowing more data to fit.

{\bf Problem: \TeX{} (or my device driver) runs out of font space.} Try
removing the \verb+amsfonts+ and \verb+amssymb+ style options. \TeX{}
implementations vary, and some implementations will be unable to provide
the resources needed to run these options.

{\bf Problem: \TeX{} runs out of string space (\verb+pool_size+ is too
small).} Remove the \verb+amssymb+ style option. It defines hundreds of
symbol names. Some \TeX{} implementations will be unable to provide the
resources needed to run this option.

{\bf Problem: (a) The text immediately following an equation is
``outdented''. That is, indented into the margin. (b) I get a \verb+missing
}+ error in the references, but the input is OK. If I let \TeX{} run
through, the output is OK, too.} \REVTeX{} is having a bad interaction with
an older version of \LaTeX{}. Upgrading to a newer \LaTeX{} has cured these
problems in the past.

{\bf Problem: One of my equations (or more) is not being cross-referenced
correctly.} Make sure that you have run \LaTeX{} at least twice since the
equation numbering was last disturbed by an input change. Also note that
incorrect cross-referencing will result if \verb+\label{#1}+ is used in an
unnumbered single line equation (i.e., within the \verb+\[+ and \verb+\]+
commands), or if \verb+\label{#1}+ is used on a line of an eqnarray that is
not being numbered (i.e., a line that has a \verb+\nonumber+).

{\bf Problem: I get a \LaTeX{} message at the end of the run that tells me
that the references may have changed, no matter how many times I run
\LaTeX{}.} Make sure that you have not used the same tag to label two
different things. This will produce this effect, but will also produce a
warning during the run and is therefore easy to detect. Also make sure that
you have not used the same tag for two different bibitems. That is, make
sure that two different \verb+\bibitem{#1}+ commands do not use the same
text for \verb+#1+. You will probably {\em not\/} get a warning for this,
so this a  more subtle error.

\penalty-10000


\section{Contacts}
\label{sec:contacts}

\begin{table}
\begin{tabular}{p{3.2in}}
\noalign{\vskip.75pc}
{\bf \REVTeX{} questions/technical support\tablenotemark[1]}\par
\vskip4pt\hrule width2.5in\vskip4pt
C.\ Hamlin, {\tt mis@aps.org} (Internet)\par
\leavevmode\phantom{C.\ Hamlin, }{\tt mis@apsedoff} (Bitnet)\par
\\[.75pc]
{\bf Electronic mail submissions\tablenotemark[2]}\par
\vskip4pt\hrule width2.5in\vskip4pt
{\em Physical Review A,B,C,D,E\/}: {\tt prtex@aps.org} (Internet)\par
{\em Physical Review Letters\/}: {\tt prltex@aps.org} (Internet)\par
\\[.75pc]
{\bf Manuscript status queries\tablenotemark[2]}\par
\vskip4pt\hrule width2.5in\vskip4pt
{\em Physical Review A\/}: {\tt pra@aps.org} (Internet)\par
{\em Physical Review B\/}: {\tt prb@aps.org} (Internet)\par
{\em Physical Review C\/}: {\tt prc@aps.org} (Internet)\par
{\em Physical Review D\/}: {\tt prd@aps.org} (Internet)\par
{\em Physical Review E\/}: {\tt pre@aps.org} (Internet)\par
{\em Physical Review L\/}: {\tt prl@aps.org} (Internet)\par
\\[.75pc]
{\bf \mbox{Policy questions\tablenotemark[1]}\par}
\vskip4pt\hrule width2.5in\vskip4pt
C.\ Hamlin \par
\\[.75pc]
{\bf \REVTeX{} project development questions\tablenotemark[1]}\par
\vskip4pt\hrule width2.5in\vskip4pt
C.\ Hamlin\par
\end{tabular}
\tablenotetext[1]{American Physical Society Liaison Office, 500 Sunnyside
Boulevard, Woodbury NY 11797-2999.}
\tablenotetext[2]{Editorial Offices, 1 Research Road, Ridge, NY 11961.}
\end{table}

\acknowledgments

We wish to acknowledge the support of the author community in using
\REVTeX{}, offering suggestions and encouragement, and testing new
versions.



\end{document}

% end of file manaps.tex
