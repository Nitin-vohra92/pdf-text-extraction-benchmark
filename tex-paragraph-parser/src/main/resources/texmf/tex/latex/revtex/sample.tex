%%% ======================================================================
%%%  @LaTeX-file{
%%%     filename        = "sample.tex",
%%%     version         = "3.0",
%%%     date            = "October 20, 1992",
%%%     ISO-date        = "1992.10.20",
%%%     time            = "15:41:54.18 EST",
%%%     author          = "Optical Society of America",
%%%     contact         = "Frank E. Harris",
%%%     address         = "Optical Society of America
%%%                        2010 Massachusetts Ave., N.W.
%%%                        Washington, D.C.  20036-1023",
%%%     email           = "fharris@pinet.aip.org (Internet)",
%%%     telephone       = "(202) 416-1903",
%%%     FAX             = "(202) 416-6120",
%%%     supported       = "yes",
%%%     archived        = "pinet.aip.org/pub/revtex,
%%%                        Niord.SHSU.edu:[FILESERV.REVTEX]",
%%%     keywords        = "REVTeX, version 3.0, sample, Optical
%%%                        Society of America",
%%%     codetable       = "ISO/ASCII",
%%%     checksum        = "42145 2097 11698 87138",
%%%     docstring       = "This is a longer sample of REVTeX use under
%%%                        REVTeX 3.0 (release of November 10, 1992).
%%%
%%%                        The checksum field above contains a CRC-16
%%%                        checksum as the first value, followed by the
%%%                        equivalent of the standard UNIX wc (word
%%%                        count) utility output of lines, words, and
%%%                        characters.  This is produced by Robert
%%%                        Solovay's checksum utility."
%%% }
%%% ======================================================================
%%%%%%%%%%%%%%%%%% file sample.tex %%%%%%%%%%%%%%%%%%%%
%                                                     %
%   Copyright (c) Optical Society of America, 1992.   %
%                                                     %
%%%%%%%%%%%%%%%%%% October 20, 1992 %%%%%%%%%%%%%%%%%%%
%
\documentstyle[osa,manuscript]{revtex}
\newcommand{\MF}{{\large{\manual META}\-{\manual FONT}}}
\newcommand{\manual}{rm}        % Substitute rm (Roman) font.
\newcommand\bs{\char '134 }     % add backslash char to \tt font %
%
\begin{document}
%\setcounter{page}{38}

\section*{ V. EXCERPTS FROM OSA MANUSCRIPTS}

{\bf (File: sample.tex)}
\vskip.5in

\baselineskip = .5\baselineskip  % single space
Manuscript excerpts from {\it Journal of the Optical Society of America A}
(JOSA A), {\it Journal of the Optical Society of America B} (JOSA B),
and {\it Applied Optics} are presented in this document, sample.tex.
Each manuscript has been ``\TeX ed'' with the REV\TeX 3.0 macros added
to the original manuscript.  Each manuscript was also cut
to about 20\% of its original length. \\ \vskip.25in

Even numbered pages present the manuscript output, as produced by REV\TeX
3.0.  Odd numbered pages show the ASCII input required to produce
the output shown on the previous page, for an OSA manuscript submission. \\
\vskip.25in

More complete versions (approximately 40\% complete) of these sample
manuscripts are available in the files josaa.tex, josab.tex,
and aplop.tex.  Josaa.tex, josab.tex, and aplop.tex have not
been modified to display input on facing pages.
The full articles are available in the OSA
journals.  They are: \\  \vskip.35in
\begin{quote}
1. R. J. Sasiela, ``Strehl ratios with various types of
anisoplanatism,'' \josaa  {\bf 9,} 1398--1405 (1992). \\  \vskip.1in
2. W. Zhao and E. Bourkoff, ``Generation, propagation, and
amplification of dark solitons,'' \josab  {\bf 9,}
1134--1144 (1992). \\         \vskip.25in
3. J. P. Pratt and  V. P. Heuring, ``Designing digital optical
computing systems: power distribution and cross talk,'' \ao
{\bf 31,} 4657--4661 (1992). \\
\end{quote}   \vskip.25in

{\it The Optical Society of America expresses its appreciation to the
authors listed above for their permission to reuse the material
in this way.}
\baselineskip = 2\baselineskip
\begin{center}
{\small  \copyright\ Optical Society of America, 1992.}
\end{center}
\newpage


%%%%%%%%%%%%%%%%%%%%%%%%%%%%%%%%
\begin{center}{\Large \bf Strehl ratios with various types
of anisoplanatism} \\
\vskip.25in
{Richard J. Sasiela}

{\it
Lincoln Laboratory, Massachusetts Institute of Technology,
Lexington,  Massachusetts 02173-9108} %
\end{center}


%\title{Strehl ratios with various types of anisoplanatism}
%\author{Richard J. Sasiela }
%\address{Lincoln Laboratory, Massachusetts Institute of Technology,
%Lexington,  Massachusetts 02173-9108} %
%\maketitle
%%%%%%%%%%%%%%%%%%%%%%%%%%%%%%%%%

\begin{abstract}                % DON'T CHANGE THIS LINE
There are many ways in which the paths of two waves through
turbulence  can become separated, thereby leading to anisoplanatic
effects.  Among  these are a parallel path separation, an angular
separation, one caused  by a time delay, and one that is due to
differential refraction at two  wavelengths.  All these effects can
be treated in the same manner.   Gegenbauer polynomials are used to
obtain an approximation for the  Strehl ratio for these
anisoplanatic effects, yielding a greater range  of applicability
than the Mar\'{e}chal approximation.
\end{abstract}

\section{ INTRODUCTION}
Adaptive-optics systems are  used to correct images of objects.
These systems work by measuring the  phase distortion on a
downpropagating wave called a beacon and applying  the negative of that
phase to a deformable mirror.  If this is done  well, then the
image of the beacon is close to diffraction limited; and  if a
laser beam is projected along the corrected path, it will have
propagation characteristics approaching those of a wave propagating
in  vacuum.  It is not possible to make a perfect correction; one
of the  major error sources is due to the fact  that the rays of
the object to  be imaged or the laser beam to be propagated are
along a path displaced  from that of the beacon.  A measurement of
this degradation is the  Strehl ratio, which is the ratio of the
intensity of the actual beam on  axis to that of a
diffraction-limited beam.

\newpage
\baselineskip = .5\baselineskip  % single space the verbatim
\begin{verbatim}

\documentstyle[osa,manuscript]{revtex}  % DON'T CHANGE %
\newcommand{\MF}{{\large{\manual META}\-{\manual FONT}}}
\newcommand{\manual}{rm}        % Substitute rm (Roman) font.
\newcommand\bs{\char '134 }     % add backslash char to \tt font %
%
\begin{document}                % INITIALIZE - DONT CHANGE % %  %

\title{Strehl ratios with various types of anisoplanatism}

\author{Richard J. Sasiela }

\address{Lincoln Laboratory, Massachusetts Institute of Technology,
Lexington,  Massachusetts 02173-9108} %

\maketitle
\begin{abstract}                % DON'T CHANGE THIS LINE
There are many ways in which the paths of two waves through
turbulence  can become separated, thereby leading to anisoplanatic
effects.  Among  these are a parallel path separation, an angular
separation, one caused  by a time delay, and one that is due to
differential refraction at two  wavelengths.  All these effects can
be treated in the same manner.   Gegenbauer polynomials are used to
obtain an approximation for the  Strehl ratio for these
anisoplanatic effects, yielding a greater range  of applicability
than the Mar\'{e}chal approximation.
\end{abstract}

\section{ INTRODUCTION}
Adaptive-optics systems are  used to correct images of objects.
These systems work by measuring the  phase distortion on a
downpropagating wave called a beacon and applying  the negative of that
phase to a deformable mirror.  If this is done  well, then the
image of the beacon is close to diffraction limited; and  if a
laser beam is projected along the corrected path, it will have
propagation characteristics approaching those of a wave propagating
in  vacuum.  It is not possible to make a perfect correction; one
of the  major error sources is due to the fact  that the rays of
the object to  be imaged or the laser beam to be propagated are
along a path displaced  from that of the beacon.  A measurement of
this degradation is the  Strehl ratio, which is the ratio of the
intensity of the actual beam on  axis to that of a
diffraction-limited beam.

\end{verbatim}
\newpage
\baselineskip = 2\baselineskip  % double space the text

This displacement can  have several causes.  The receiving and the
transmitting apertures may  be displaced from each other owing to
misalignment or vignetting of the  beams.  The paths can be
separated in angle, for instance, when the  object to be imaged is
different from the beacon.  The correction is  applied with a time
delay after the measurements.  In this time the  turbulence is
displaced by winds and slewing of the telescope.  The  paths may be
separated because the beacon and the imaging wavelengths  differ,
in which case refraction operates differently on the two waves.
All the  effects are typically present simultaneously.

These  anisoplanatisms have been treated separately in the
past\cite{1,2,3,4,5,6,7}; \ldots

\section{ STREHL RATIO WITH ANISOPLANATISM}
\label{SR}
For a perfect correction the  paths of the beacon signal and the
imaging or projected laser should be  the same.  In general, this
is not possible to achieve, and there is a  degradation in
performance caused by time delays, displacement of the  two paths
by translation and angle, and differences in wavelength of the
beacon and the measurement or projecting systems.

The effects of
displacement, angular mispointing, time delay, and atmospheric
dispersion can each be treated as an anisoplanatic effect.  In
fact, if  all the effects are present simultaneously, they can be
combined to get  a total offset of the measurement from the imaging
paths.  In this  section the effect of a general displacement on
the Strehl ratio is  determined.

The Strehl ratio (SR) for a circular aperture \cite{7} from  the
Huygens--Fresnel approximation is  \begin{eqnarray}{\rm  SR}
={1 \over {2\pi }}\int {{\rm d}\bbox  \alpha }\,K(\alpha )\,\exp
\,\left[ {-{{{\cal D}\left( {\bbox \alpha } \right)}  \over 2}}
\right].\end{eqnarray}  The integral is over a circular aperture of
unit radius,  ${\cal D}( {\bbox \alpha } )$  is the structure
function, and  $K(\alpha )$  is a  factor times the optical
transfer function given by  \begin{eqnarray}K(\alpha )={{16} \over
\pi }\left[ {\cos ^{- 1}(\alpha )-\alpha \left( {1-\alpha ^2}
\right)^{1/ 2}} \right]\,U(1- \alpha ),\end{eqnarray}

\newpage
\baselineskip = .5\baselineskip  % single space the verbatim
\begin{verbatim}

This displacement can  have several causes.  The receiving and the
transmitting apertures may  be displaced from each other owing to
misalignment or vignetting of the  beams.  The paths can be
separated in angle, for instance, when the  object to be imaged is
different from the beacon.  The correction is  applied with a time
delay after the measurements.  In this time the  turbulence is
displaced by winds and slewing of the telescope.  The  paths may be
separated because the beacon and the imaging wavelengths  differ,
in which case refraction operates differently on the two waves.
All the  effects are typically present simultaneously.

These  anisoplanatisms have been treated separately in the
past\cite{1,2,3,4,5,6,7}; \ldots

\section{ STREHL RATIO WITH ANISOPLANATISM}
\label{SR}
For a perfect correction the  paths of the beacon signal and the
imaging or projected laser should be  the same.  In general, this
is not possible to achieve, and there is a  degradation in
performance caused by time delays, displacement of the  two paths
by translation and angle, and differences in wavelength of the
beacon and the measurement or projecting systems.

The effects of
displacement, angular mispointing, time delay, and atmospheric
dispersion can each be treated as an anisoplanatic effect.  In
fact, if  all the effects are present simultaneously, they can be
combined to get  a total offset of the measurement from the imaging
paths.  In this  section the effect of a general displacement on
the Strehl ratio is  determined.

The Strehl ratio (SR) for a circular aperture \cite{7} from  the
Huygens--Fresnel approximation is  \begin{eqnarray}{\rm  SR}
={1 \over {2\pi }}\int {{\rm d}\bbox  \alpha }\,K(\alpha )\,\exp
\,\left[ {-{{{\cal D}\left( {\bbox \alpha } \right)}  \over 2}}
\right].\end{eqnarray}  The integral is over a circular aperture of
unit radius,  ${\cal D}( {\bbox \alpha } )$  is the structure
function, and  $K(\alpha )$  is a  factor times the optical
transfer function given by  \begin{eqnarray}K(\alpha )={{16} \over
\pi }\left[ {\cos ^{- 1}(\alpha )-\alpha \left( {1-\alpha ^2}
\right)^{1/ 2}} \right]\,U(1- \alpha ),\end{eqnarray}

\end{verbatim}
\newpage
\baselineskip = 2\baselineskip  % double space the text

where
$U\left(  x \right)$  is the unit step function defined as
\begin{eqnarray} U( x )&=&1\,\,\,\,\,\,\,\,{\rm for}\,\,\,\,x\ge
0\,,  \nonumber \\   U( x )&=&0\,\,\,\,\,\,\,\,{\rm
for}\,\,\,\,x<0\,\,.    \end{eqnarray}

To find  the Strehl ratio, one must first determine the structure
function.  It  was found by Fried\cite{4}  for angular
anisoplanatism.  If the source  is collimated and a general
displacement is introduced, his expression  for a wave propagating
from ground to space becomes
\begin{eqnarray}
{\cal D}({\alpha
\kern 1ptD} )&=& 2(2.91)\,{k_0}^2\int\limits_{\,\,\; 0}^{\,\,\,\,\,\;
\infty}   {\rm d}z\,{C_n}^2(z)\left[ {( {\alpha \kern 1ptD}  )^{5/
3}+d^{5/ 3}(z)}\right.  \nonumber\\
&&\left.
{-{\slantfrac{1}{2}}\,\left| {{\bbox \alpha} \kern 1ptD+{\bbox
d}(z)\,} \right|^{5/ 3} -{\textstyle \slantfrac{1}{2}}\left|
{\,{\bbox \alpha}  \kern 1ptD-{\bbox d}(z)\,} \right|^{5 / 3}}
\right],
\end{eqnarray}
where  ${C_n}^2(z)$  is the turbulence
strength as a function of altitude;  $k_0=2\kern 1pt\pi / \lambda
,$  where $\lambda $ is the wavelength  of operation; $D$ is the
aperture diameter; and  ${\bbox d}(z)$   is the vector displacement
of the two paths.

The sums of the terms in  brackets almost cancel, thus causing
difficulties if one tries to  evaluate this integral numerically.
The terms in the absolute-value  sign are equal to
\begin{eqnarray}\left| {\,{\bbox \alpha}  \kern 1ptD\pm {\bbox
d}(z)\,} \right|^{5/ 3}=\left[ {\left( {\alpha \kern  1ptD}
\right)^2\pm 2\alpha \kern 1ptD\,d(z)\cos \left( \varphi
\right)+d^2(z)} \right]^{5/ 6},\end{eqnarray}  where  is the angle
between  ${\bbox \alpha} $  and  ${\bbox d}( z )$ .    This
expression can be simplified and  the numerical difficulties can be
eliminated by using Gegenbauer  polynomials.\cite{8}  Their
generating function is  \begin{eqnarray}\left( {1-2ax+a^2}
\right)^{-\lambda }=\sum\limits_{p=0}^\infty  {{C_p}^\lambda
(x)\,a^p}. \end{eqnarray}   These functions are sometimes referred
to as ultraspherical functions because they are a generalization of
the Legendre polynomials  $P_n(t)$ , whose generating function is
\begin{eqnarray}\left( {1- 2ax+a^2} \right)^{-1/
2}=\sum\limits_{p=0}^\infty  {P_p(x)\,a^p}.\end{eqnarray}      The
Gegenbauer polynomials with the cosine of a variable as the
argument are given in Eq. (8.934  \#2) of Ref. \onlinecite{8}  and
can be rewritten as  \begin{eqnarray}{C_p}^\lambda \left[ {\cos
\left( \varphi   \right)} \right]=\sum\limits_{m=0}^p
{}{{\Gamma\,\left[ {\lambda +m}  \right]\,\Gamma\,\left[ {\lambda
+p-m} \right]\cos \left[ {(p-2m)\varphi }  \right]} \over
{m!\,(p-m)!\,\left( {\Gamma\,\left[ \lambda  \right]}
\right)^2}},\end{eqnarray}

\newpage
\baselineskip = .5\baselineskip  % single space the verbatim
\begin{verbatim}

where
$U\left(  x \right)$  is the unit step function defined as
\begin{eqnarray} U( x )&=&1\,\,\,\,\,\,\,\,{\rm for}\,\,\,\,x\ge
0\,,  \nonumber \\   U( x )&=&0\,\,\,\,\,\,\,\,{\rm
for}\,\,\,\,x<0\,\,.    \end{eqnarray}

To find  the Strehl ratio, one must first determine the structure
function.  It  was found by Fried\cite{4}  for angular
anisoplanatism.  If the source  is collimated and a general
displacement is introduced, his expression  for a wave propagating
from ground to space becomes
\begin{eqnarray}
{\cal D}({\alpha
\kern 1ptD} )&=& 2(2.91)\,{k_0}^2\int\limits_{\,\,\; 0}^{\,\,\,\,\,\;
\infty}   {\rm d}z\,{C_n}^2(z)\left[ {( {\alpha \kern 1ptD}  )^{5/
3}+d^{5/ 3}(z)}\right.  \nonumber\\
&&\left.
{-{\slantfrac{1}{2}}\,\left| {{\bbox \alpha} \kern 1ptD+{\bbox
d}(z)\,} \right|^{5/ 3} -{\textstyle \slantfrac{1}{2}}\left|
{\,{\bbox \alpha}  \kern 1ptD-{\bbox d}(z)\,} \right|^{5 / 3}}
\right],
\end{eqnarray}
where  ${C_n}^2(z)$  is the turbulence
strength as a function of altitude;  $k_0=2\kern 1pt\pi / \lambda
,$  where $\lambda $ is the wavelength  of operation; $D$ is the
aperture diameter; and  ${\bbox d}(z)$   is the vector displacement
of the two paths.

The sums of the terms in  brackets almost cancel, thus causing
difficulties if one tries to  evaluate this integral numerically.
The terms in the absolute-value  sign are equal to
\begin{eqnarray}\left| {\,{\bbox \alpha}  \kern 1ptD\pm {\bbox
d}(z)\,} \right|^{5/ 3}=\left[ {\left( {\alpha \kern  1ptD}
\right)^2\pm 2\alpha \kern 1ptD\,d(z)\cos \left( \varphi
\right)+d^2(z)} \right]^{5/ 6},\end{eqnarray}  where  is the angle
between  ${\bbox \alpha} $  and  ${\bbox d}( z )$ .    This
expression can be simplified and  the numerical difficulties can be
eliminated by using Gegenbauer  polynomials.\cite{8}  Their
generating function is  \begin{eqnarray}\left( {1-2ax+a^2}
\right)^{-\lambda }=\sum\limits_{p=0}^\infty  {{C_p}^\lambda
(x)\,a^p}. \end{eqnarray}   These functions are sometimes referred
to as ultraspherical functions because they are a generalization of
the Legendre polynomials  $P_n(t)$ , whose generating function is
\begin{eqnarray}\left( {1- 2ax+a^2} \right)^{-1/
2}=\sum\limits_{p=0}^\infty  {P_p(x)\,a^p}.\end{eqnarray}      The
Gegenbauer polynomials with the cosine of a variable as the
argument are given in Eq. (8.934  \#2) of Ref. \onlinecite{8}  and
can be rewritten as  \begin{eqnarray}{C_p}^\lambda \left[ {\cos
\left( \varphi   \right)} \right]=\sum\limits_{m=0}^p
{}{{\Gamma\,\left[ {\lambda +m}  \right]\,\Gamma\,\left[ {\lambda
+p-m} \right]\cos \left[ {(p-2m)\varphi }  \right]} \over
{m!\,(p-m)!\,\left( {\Gamma\,\left[ \lambda  \right]}
\right)^2}},\end{eqnarray}
\end{verbatim}
\newpage
\baselineskip = 2\baselineskip  % double space the text

where - $\Gamma\left[ x \right]$  is
the gamma function.  A particular Gegenbauer  polynomial that is
required is  \begin{eqnarray}{C_2}^{-5/ 6}\left[ {\cos (\varphi )}
\right]={\textstyle{\slantfrac{5}{6}}}\left[ {1- {\textstyle{
\slantfrac{1}{3}}}\cos ^2\left( \varphi  \right)} \right].
\end{eqnarray}   For  $\alpha \kern 1ptD>d(z)$ , the terms in the
structure function can  be expanded in Gegenbauer polynomials.  The
zeroth- and all odd-order  terms cancel.  When the summation index
is changed by the substitution  $p\to 2\kern 1ptp$  the result is
\begin{eqnarray} {\cal D}(\alpha \kern
1ptD)=2(2.91)\,{k_0}^2\int\limits_{\,\,\, 0}^{\,\,\,\,\,\,\infty}{\rm
d}z\,{C_n}^2(z) \left\{ {d^{5/  3}(z)- (\alpha \kern 1ptD)^{5/
3}\sum\limits_{p=1}^\infty  {{C_{2p}}^{- 5/ 6}\,\left[ {\cos \left(
\varphi  \right)} \right]}\,\left[ {{{d(z)}  \over {\alpha \kern
1ptD}}} \right]^{2p}} \right\}.\end{eqnarray} It is this  canceling
of the first two terms of the power series that would cause
numerical difficulties.

Define a distance moment as
\begin{eqnarray}d_m\equiv  2.91\,{k_0}^2\int\limits_{\,\,\,
0}^{\,\,\,\,\,\,\infty}{\rm d}z\,{C_n}^2(z)\,d^m(z) \end{eqnarray}
and a phase variance as  \begin{eqnarray}{\sigma _\varphi}^2=d_{5/
3}.\end{eqnarray}    Unlike the calculation for Strehl ratio for
uncorrected  turbulence and for corrected turbulence with tilt
jitter, an exact  analytical solution cannot be found for
anisoplanatism.  Fortunately,  for adaptive-optics systems, the
Strehl ratio should be fairly high by  design, which requires the
structure function to be small.  This  assumption allows one to
retain only the first term of the Gegenbauer  expansion to give
\begin{eqnarray}{\cal D}(\alpha \kern  1ptD)=2{\sigma
_\varphi}^2-2x,\end{eqnarray} where
\begin{eqnarray}x=d_{2}\left[ {1-
{\textstyle{\slantfrac{1}{3}}}\cos ^2\left( \varphi  \right)}
\right]{\slantfrac{5}{6}}(\alpha \kern 1ptD)^{-1/ 3}.\end{eqnarray}
 We justify this single-term approximation below by showing that it
produces a result close to the exact result. \\      \ldots \\

\newpage
\baselineskip = .5\baselineskip  % single space the verbatim
\begin{verbatim}
where - $\Gamma\left[ x \right]$  is
the gamma function.  A particular Gegenbauer  polynomial that is
required is  \begin{eqnarray}{C_2}^{-5/ 6}\left[ {\cos (\varphi )}
\right]={\textstyle{\slantfrac{5}{6}}}\left[ {1- {\textstyle{
\slantfrac{1}{3}}}\cos ^2\left( \varphi  \right)} \right].
\end{eqnarray}   For  $\alpha \kern 1ptD>d(z)$ , the terms in the
structure function can  be expanded in Gegenbauer polynomials.  The
zeroth- and all odd-order  terms cancel.  When the summation index
is changed by the substitution  $p\to 2\kern 1ptp$  the result is
\begin{eqnarray} {\cal D}(\alpha \kern
1ptD)=2(2.91)\,{k_0}^2\int\limits_{\,\,\, 0}^{\,\,\,\,\,\,\infty}{\rm
d}z\,{C_n}^2(z) \left\{ {d^{5/  3}(z)- (\alpha \kern 1ptD)^{5/
3}\sum\limits_{p=1}^\infty  {{C_{2p}}^{- 5/ 6}\,\left[ {\cos \left(
\varphi  \right)} \right]}\,\left[ {{{d(z)}  \over {\alpha \kern
1ptD}}} \right]^{2p}} \right\}.\end{eqnarray} It is this  canceling
of the first two terms of the power series that would cause
numerical difficulties.

Define a distance moment as
\begin{eqnarray}d_m\equiv  2.91\,{k_0}^2\int\limits_{\,\,\,
0}^{\,\,\,\,\,\,\infty}{\rm d}z\,{C_n}^2(z)\,d^m(z) \end{eqnarray}
and a phase variance as  \begin{eqnarray}{\sigma _\varphi}^2=d_{5/
3}.\end{eqnarray}    Unlike the calculation for Strehl ratio for
uncorrected  turbulence and for corrected turbulence with tilt
jitter, an exact  analytical solution cannot be found for
anisoplanatism.  Fortunately,  for adaptive-optics systems, the
Strehl ratio should be fairly high by  design, which requires the
structure function to be small.  This  assumption allows one to
retain only the first term of the Gegenbauer  expansion to give
\begin{eqnarray}{\cal D}(\alpha \kern  1ptD)=2{\sigma
_\varphi}^2-2x,\end{eqnarray} where
\begin{eqnarray}x=d_{2}\left[ {1-
{\textstyle{\slantfrac{1}{3}}}\cos ^2\left( \varphi  \right)}
\right]{\slantfrac{5}{6}}(\alpha \kern 1ptD)^{-1/ 3}.\end{eqnarray}
 We justify this single-term approximation below by showing that it
produces a result close to the exact result. \\      \ldots \\

\end{verbatim}
\newpage
\baselineskip = 2\baselineskip  % double space the text

The
Strehl ratio with the six term approximation is
\begin{eqnarray}{\rm   SR} \approx  {{\exp \left( {-\sigma
_\varphi} ^2 \right)} \over {2\pi }}\int {\rm d{\bbox  \alpha}
\,K(\alpha )\,}\kern-.5em\left( {1+x+{{x^2} \over 2}+{{x^3} \over 6}+{{x^4}
\over {24}}+{{x^5} \over {120}}} \right).\end{eqnarray}  If just
the  first term in the last parenthetical expression  is retained,
the result is equivalent to  the extended Mar\'{e}chal
approximation.  It is shown below that the six-term  approximation
is best for aperture sizes normally encountered.   The angle
integral for the $n$th term, after use of the binomial theorem,  is
proportional to  \begin{eqnarray}\Phi (n)={1 \over {2\pi
}}\int\limits_{\,\,\, 0}^{\,\,\,\,\,\,\,\, 2\pi } {\rm d}\varphi \,\left[
{1-\slantfrac{1}{3}} \cos ^2\left( \varphi  \right) \right]^n={1
\over {2\pi  }}\sum\limits_{m=0}^n {\left( \begin{array}{c} n \\
n-m\end{array}  \right)}\,3^{-m}\int\limits_{\,\,\, 0}^{\,\,\,\,\,\, 2\pi
} {\rm d\varphi }\, \cos ^{2m}\left( \varphi
\right),\end{eqnarray}

\begin{eqnarray}\left(
\begin{array}{c} n \\ n-m \end{array} \right)={{n!} \over {\left(
{n-m} \right)!\,\,m!}}.\end{eqnarray}  Equation (4.641 \# 4) in
Gradshteyn  and Ryzhik\cite{8} is
\begin{eqnarray}\int\limits_{\,\,\, 0}^{\,\,\,\,\,\, \pi /  2}{\rm
d\varphi \,}\cos ^{2m}\left( \varphi  \right)={{\pi (2m-1)!!} \over
{2(2m)!!}},\end{eqnarray}   where
\begin{eqnarray}(2m-1)!!&=&(2m-1)(2m-3)\ldots (3)(1), \\
(2m)!!&=&(2m)(2m-2)\ldots (4)(2).\end{eqnarray}  With these
relations, the angle integral is equal to
\begin{eqnarray}\Phi (n)=1-\sum\limits_{m=1}^n {\left(
\begin{array}{c}n \\  n-m \end{array} \right)}\,3^{-m}{{(2m-1)!!}
\over {(2m)!!}}.\end{eqnarray}  The values of interest to us are
$\Phi (0) = 1$, $\Phi (1) = 0.8333$, $\Phi (2) = 0.7083$, $ \Phi
(3) = 0.6134$, $\Phi (4) = 0.5404$, and  $\Phi (5) = 0.4836$.   The
aperture integration for the $n$th term is proportional to
\begin{eqnarray}Y(n)=\int\limits_{\,\,\, 0}^{\,\,\,\,\,\, 1} {\rm d\alpha
\,}\alpha ^{1-n/ 3}K(\alpha ).\end{eqnarray}

\newpage
\baselineskip = .5\baselineskip  % single space the verbatim
\begin{verbatim}

The
Strehl ratio with the six term approximation is
\begin{eqnarray}{\rm   SR} \approx  {{\exp \left( {-\sigma
_\varphi} ^2 \right)} \over {2\pi }}\int {\rm d{\bbox  \alpha}
\,K(\alpha )\,}\kern-.5em\left( {1+x+{{x^2} \over 2}+{{x^3} \over 6}+{{x^4}
\over {24}}+{{x^5} \over {120}}} \right).\end{eqnarray}
If just
the  first term in the last parenthetical expression  is retained,
the result is equivalent to  the extended Mar\'{e}chal
approximation.  It is shown below that the six-term  approximation
is best for aperture sizes normally encountered.   The angle
integral for the $n$th term, after use of the binomial theorem,  is
proportional to
\begin{eqnarray}\Phi (n)={1 \over {2\pi
}}\int\limits_{\,\,\, 0}^{\,\,\,\,\,\,\,\, 2\pi } {\rm d}\varphi \,\left[
{1-\slantfrac{1}{3}} \cos ^2\left( \varphi  \right) \right]^n={1
\over {2\pi  }}\sum\limits_{m=0}^n {\left( \begin{array}{c} n \\
n-m\end{array}  \right)}\,3^{-m}\int\limits_{\,\,\, 0}^{\,\,\,\,\,\, 2\pi
} {\rm d\varphi }\, \cos ^{2m}\left( \varphi
\right),\end{eqnarray}
\begin{eqnarray}\left(
\begin{array}{c} n \\ n-m \end{array} \right)={{n!} \over {\left(
{n-m} \right)!\,\,m!}}.\end{eqnarray}
Equation (4.641 \# 4) in
Gradshteyn  and Ryzhik\cite{8} is
\begin{eqnarray}\int\limits_{\,\,\, 0}^{\,\,\,\,\,\, \pi /  2}{\rm
d\varphi \,}\cos ^{2m}\left( \varphi  \right)={{\pi (2m-1)!!} \over
{2(2m)!!}},\end{eqnarray}
where
\begin{eqnarray}(2m-1)!!&=&(2m-1)(2m-3)\ldots (3)(1), \\
(2m)!!&=&(2m)(2m-2)\ldots (4)(2).\end{eqnarray}
With these
relations, the angle integral is equal to
\begin{eqnarray}\Phi (n)=1-\sum\limits_{m=1}^n {\left(
\begin{array}{c}n \\  n-m \end{array} \right)}\,3^{-m}{{(2m-1)!!}
\over {(2m)!!}}.\end{eqnarray}
The values of interest to us are
$\Phi (0) = 1$, $\Phi (1) = 0.8333$, $\Phi (2) = 0.7083$, $ \Phi
(3) = 0.6134$, $\Phi (4) = 0.5404$, and  $\Phi (5) = 0.4836$.   The
aperture integration for the $n$th term is proportional to
\begin{eqnarray}Y(n)=\int\limits_{\,\,\, 0}^{\,\,\,\,\,\, 1} {\rm d\alpha
\,}\alpha ^{1-n/ 3}K(\alpha ).\end{eqnarray}

\end{verbatim}
\newpage
\baselineskip = 2\baselineskip  % double space the text


\section{ DISPLACEMENT ANISOPLANATISM}
\label{da}
In the simplest case of displacement  anisoplanatism, which was
treated in Section \ref{SR}, the displacement is  constant along
the propagation direction.  The terms to use to find the  Strehl
ratio are  \begin{eqnarray}  d(z)&=&d  ,  \\
d_{\,2}&=&2.91\,k_0^2\,\mu _0\,d^2  ,     \\ E&=&6.88\,\left( {{d
\over D}} \right)^2\left(  {{D \over {r_o}}} \right)^{5/3}  ,
 \\ \sigma _\varphi ^2&=&2.91\,k_0^2\,\mu _0\,d^{5/3}=6.88\, \left(
{{d \over {r_o}}} \right)^{5/3}  .  \end{eqnarray}

The Strehl
ratios are plotted in Figs.~\ref{f5}  and ~\ref{f10}.

\section{ ANGULAR ANISOPLANATISM}
\label{aa}
When the propagation beam is offset by a  constant angle from the
direction along which turbulence is measured,  the effect is called
angular anisoplanatism.\cite{4} \ldots

\section{ TIME DELAY}
\label{td}
If there is a time delay  between when turbulence is measured and
when a correction is applied to the deformable mirror, there is  a
degradation in performance.\cite{7}  This effect is not often
thought of  as an anisoplanatic effect; however, it can be treated
as such.  ...

\begin{eqnarray}   d(z)&=&v(z)\tau   ,  \\
d_2&=&2.91\,k_0^2\int\limits_{\,\,\, 0}^{\,\,\,\,\,\, L} {\rm
d}z\,{C_n}^2(z)\,v^2(z)\,\tau ^2=\left(  {\tau / \tau _2} \right)^2
,   \\    E&=&{{\tau ^2} \over {\tau _2^2D^{1/ 3}}}  ,  \\  \sigma
_\varphi ^2&=&2.91\,k_0^2\int\limits_{\,\,\, 0}^{\,\,\,\,\,\, L}  {\rm
d}z\,{C_n}^2(z)\,v^{5/ 3}(z)\,\tau ^{5/ 3}=\left( {\tau / \tau _{5/
3}} \right)^{5/ 3}  ,   \end{eqnarray}
\newpage
\baselineskip = .5\baselineskip  % single space the verbatim
\begin{verbatim}

\section{ DISPLACEMENT ANISOPLANATISM}
\label{da}
In the simplest case of displacement  anisoplanatism, which was
treated in Section \ref{SR}, the displacement is  constant along
the propagation direction.  The terms to use to find the  Strehl
ratio are  \begin{eqnarray}  d(z)&=&d  ,  \\
d_{\,2}&=&2.91\,k_0^2\,\mu _0\,d^2  ,     \\ E&=&6.88\,\left( {{d
\over D}} \right)^2\left(  {{D \over {r_o}}} \right)^{5/3}  ,
 \\ \sigma _\varphi ^2&=&2.91\,k_0^2\,\mu _0\,d^{5/3}=6.88\, \left(
{{d \over {r_o}}} \right)^{5/3}  .  \end{eqnarray}

The Strehl
ratios are plotted in Figs.~\ref{f5}  and ~\ref{f10}.

\section{ ANGULAR ANISOPLANATISM}
\label{aa}
When the propagation beam is offset by a  constant angle from the
direction along which turbulence is measured,  the effect is called
angular anisoplanatism.\cite{4} \ldots

\section{ TIME DELAY}
\label{td}
If there is a time delay  between when turbulence is measured and
when a correction is applied to the deformable mirror, there is  a
degradation in performance.\cite{7}  This effect is not often
thought of  as an anisoplanatic effect; however, it can be treated
as such.  ...

\begin{eqnarray}   d(z)&=&v(z)\tau   ,  \\
d_2&=&2.91\,k_0^2\int\limits_{\,\,\, 0}^{\,\,\,\,\,\, L} {\rm
d}z\,{C_n}^2(z)\,v^2(z)\,\tau ^2=\left(  {\tau / \tau _2} \right)^2
,   \\    E&=&{{\tau ^2} \over {\tau _2^2D^{1/ 3}}}  ,  \\  \sigma
_\varphi ^2&=&2.91\,k_0^2\int\limits_{\,\,\, 0}^{\,\,\,\,\,\, L}  {\rm
d}z\,{C_n}^2(z)\,v^{5/ 3}(z)\,\tau ^{5/ 3}=\left( {\tau / \tau _{5/
3}} \right)^{5/ 3}  ,   \end{eqnarray}

\end{verbatim}
\newpage
\baselineskip = 2\baselineskip  % double space the text


where the temporal moment is
defined as  \begin{eqnarray}  1/ \tau _m^{5/
3}=2.91\,k_0^2\int\limits_{\,\,\, 0}^{\,\,\,\,\,\, L}  {\rm
d}z\,{C_n}^2(z)\,v^m(z)  .   \end{eqnarray}
\ldots

\section{ CHROMATIC ANISOPLANATISM}
\label{ca}
If the beacon beam that senses the  turbulence has a wavelength
different from that of the laser beam that  is sent out, then the
two beams will follow different paths through the  atmosphere
because of the dispersive properties of the atmosphere. \ldots

\section{ COMBINED DISPLACEMENT}
\label{cd}
If there are several anisoplanatic  effects present, with each not
decreasing the Strehl ratio much, it is a  common practice to
multiply the Strehl ratios for the individual effects  to get a
combined Strehl ratio. \ldots
\begin{eqnarray}
{\rm \pmb{d}}_t(z)={\rm  \pmb{d}}+{ \rm \bbox{
\theta}} \kern 1ptz+{\rm \pmb{v}}(z)\tau +{\rm  \pmb{d}}_c(z)  ,
\end{eqnarray} where chromatic displacement is given in Eq. (50).
The two terms  necessary for calculating the Strehl ratio are
\ldots

\newpage
\baselineskip = .5\baselineskip  % single space the verbatim
\begin{verbatim}
where the temporal moment is
defined as  \begin{eqnarray}  1/ \tau _m^{5/
3}=2.91\,k_0^2\int\limits_{\,\,\, 0}^{\,\,\,\,\,\, L}  {\rm
d}z\,{C_n}^2(z)\,v^m(z)  .   \end{eqnarray}
\ldots

\section{ CHROMATIC ANISOPLANATISM}
\label{ca}
If the beacon beam that senses the  turbulence has a wavelength
different from that of the laser beam that  is sent out, then the
two beams will follow different paths through the  atmosphere
because of the dispersive properties of the atmosphere. \ldots

\section{ COMBINED DISPLACEMENT}
\label{cd}
If there are several anisoplanatic  effects present, with each not
decreasing the Strehl ratio much, it is a  common practice to
multiply the Strehl ratios for the individual effects  to get a
combined Strehl ratio. \ldots
\begin{eqnarray}
{\rm \pmb{d}}_t(z)={\rm  \pmb{d}}+{ \rm \bbox{
\theta}} \kern 1ptz+{\rm \pmb{v}}(z)\tau +{\rm  \pmb{d}}_c(z)  ,
\end{eqnarray} where chromatic displacement is given in Eq. (50).
The two terms  necessary for calculating the Strehl ratio are
\ldots

\end{verbatim}
\newpage
\baselineskip = 2\baselineskip  % double space the text


\section{ SUMMARY}
\label{Su}
An approximate expression for the Strehl ratio that is  easily
evaluated for any turbulence distribution was derived.  It  applies
for various anisoplanatic effects.  This expression was shown to
give much better agreement with the exact answer than the extended
Marechal approximation.  The zenith dependence is included in the
formula.  This approximation was applied to parallel path
displacements,  angular offsets, time-delay induced offsets, and
offsets owing to  refractive effects that vary with wavelength.
Examples for each type of  anisoplanatism at various zenith angles
were evaluated.

The  Strehl ratio in the presence of several effects was examined.
It was  shown that, depending on the direction of the relative
displacements,  one can get a cancellation or an enhancement of the
effect of the  displacements.  Therefore it is possible for there
to be little  reduction in the Strehl ratio if there is little net
path displacement.   If the displacements are in the same
direction, the Strehl ratio is less  than the product of the Strehl
ratios of the individual terms.

\acknowledgments This research was  sponsored by the Strategic
Defense Initiative Organization through the  U.S. Department of the
Air Force.
\newpage
\baselineskip = .5\baselineskip  % single space the verbatim
\begin{verbatim}

\section{ SUMMARY}
\label{Su}
An approximate expression for the Strehl ratio that is  easily
evaluated for any turbulence distribution was derived.  It  applies
for various anisoplanatic effects.  This expression was shown to
give much better agreement with the exact answer than the extended
Marechal approximation.  The zenith dependence is included in the
formula.  This approximation was applied to parallel path
displacements,  angular offsets, time-delay induced offsets, and
offsets owing to  refractive effects that vary with wavelength.
Examples for each type of  anisoplanatism at various zenith angles
were evaluated.

The  Strehl ratio in the presence of several effects was examined.
It was  shown that, depending on the direction of the relative
displacements,  one can get a cancellation or an enhancement of the
effect of the  displacements.  Therefore it is possible for there
to be little  reduction in the Strehl ratio if there is little net
path displacement.   If the displacements are in the same
direction, the Strehl ratio is less  than the product of the Strehl
ratios of the individual terms.

\acknowledgments This research was  sponsored by the Strategic
Defense Initiative Organization through the  U.S. Department of the
Air Force.

\end{verbatim}
\newpage
\baselineskip = 2\baselineskip  % double space the text


\begin{references}
\bibitem{1}  J. Belsher and D. Fried, ``Chromatic refraction
induced pseudo  anisoplanatism,'' tOSC Rep. TR-433 (Optical
Sciences Co.,  Placentia, Calif., 1981).
\bibitem{2}  B. L. Ellerbroek and P. H. Roberts,  ``Turbulence
induced angular separation errors; expected values for the  SOR-2
experiment,'' tOSC Rep.  TR-613 (Optical Sciences Co.,  Placentia,
Calif., 1984).
\bibitem{3}  D. L. Fried, ``Differential angle of  arrival: theory,
evaluation, and measurement feasibility,'' Radio  Sci. {\bf 10,}
71-76 (1975).
\bibitem{4}  D. Fried, ``Anisoplanatism in adaptive  optics,''
\josa {\bf 72,} 52-61 (1982).  \bibitem{5}  D. Korff, G. Druden,
and  R. P. Leavitt, ``Isoplanicity: the translation invariance of
the  atmospheric Green's function,'' \josa {\bf 65,} 1321-1330
(1975).
\bibitem{6}   J. H. Shapiro, ``Point-ahead limitation on
reciprocity tracking,''  \josa {\bf 65,} 65-68 (1975).
\bibitem{7}  G. A. Tyler, ``Turbulence-induced  adaptive-optics
performance degradation: evaluation in the time domain,''  \josaa
{\bf  1,} 251-262 (1984).
\bibitem{8}  I. S. Gradshteyn and I. M.  Ryzhik, {\it Table of
Integrals, Series, and Products}  (Academic, New York, 1980).
\bibitem{9}  V. I. Tatarski, {\it The Effects Of The Turbulent
Atmosphere  On Wave Propagation} (U. S. Department of Commerce,
Washington, D.C., 1971).
\bibitem{10}  R. E.  Hufnagel, {\it Optical Propagation through
Turbulence} (Optical Society of America, Washington, D. C., 1974).
\end{references}

\newpage
\baselineskip = .5\baselineskip  % single space the verbatim
\begin{verbatim}

\begin{references}
\bibitem{1}  J. Belsher and D. Fried, ``Chromatic refraction
induced pseudo  anisoplanatism,'' tOSC Rep. TR-433 (Optical
Sciences Co.,  Placentia, Calif., 1981).
\bibitem{2}  B. L. Ellerbroek and P. H. Roberts,  ``Turbulence
induced angular separation errors; expected values for the  SOR-2
experiment,'' tOSC Rep.  TR-613 (Optical Sciences Co.,  Placentia,
Calif., 1984).
\bibitem{3}  D. L. Fried, ``Differential angle of  arrival: theory,
evaluation, and measurement feasibility,'' Radio  Sci. {\bf 10,}
71-76 (1975).
\bibitem{4}  D. Fried, ``Anisoplanatism in adaptive  optics,''
\josa {\bf 72,} 52-61 (1982).  \bibitem{5}  D. Korff, G. Druden,
and  R. P. Leavitt, ``Isoplanicity: the translation invariance of
the  atmospheric Green's function,'' \josa {\bf 65,} 1321-1330
(1975).
\bibitem{6}   J. H. Shapiro, ``Point-ahead limitation on
reciprocity tracking,''  \josa {\bf 65,} 65-68 (1975).
\bibitem{7}  G. A. Tyler, ``Turbulence-induced  adaptive-optics
performance degradation: evaluation in the time domain,''  \josaa
{\bf  1,} 251-262 (1984).
\bibitem{8}  I. S. Gradshteyn and I. M.  Ryzhik, {\it Table of
Integrals, Series, and Products}  (Academic, New York, 1980).
\bibitem{9}  V. I. Tatarski, {\it The Effects Of The Turbulent
Atmosphere  On Wave Propagation} (U. S. Department of Commerce,
Washington, D.C., 1971).
\bibitem{10}  R. E.  Hufnagel, {\it Optical Propagation through
Turbulence} (Optical Society of America, Washington, D. C., 1974).
\end{references}

\end{verbatim}
\newpage
\baselineskip = 2\baselineskip  % double space the text

\begin{figure}
\caption{ Comparison of the Mar\'{e}chal and the two- to six-term
approximations  with the exact value of the Strell ratio, for an
anisoplanatic displacement, for $D/r_0$  equal to 1.}\label{f1}
\end{figure}

\begin{figure}
\caption{ Comparison of the Mar\'{e}chal and the two- to six-term
approximations  with the exact value of the Strell ratio, for an
anisoplanatic displacement, for $D/r_0$  equal to 5. } \label{f5}
\end{figure}
\begin{figure}
\caption{ Comparison of the Mar\'{e}chal and the two- to six-term
approximations  with the exact value of the Strell ratio, for an
anisoplanatic displacement, for $D/r_0$  equal to 10. } \label{f10}
\end{figure}
\begin{figure}
\caption{Strehl ratio for angular anisoplanatic error at zenith,
for  various turbulence models, versus separation angle for a 0.6-m
system.   Upper-altitude turbulence has a strong effect on the
Strehl ratio.}
\label{faaz}
\end{figure}
\begin{figure}
\caption{ Strehl ratio for angular anisoplanatism at $30^{\circ}$
for a 0.6-m system.}
\label{faa30}
\end{figure}
\begin{figure}
\caption{ Strehl ratio versus time delay at zenith for a 0.6-m
system.}
\label{ftdz}
\end{figure}
\begin{figure}
\caption{ Strehl ratio versus time delay for a 0.6-m system at
$30^{\circ}$ zenith angle.     Strehl ratio  at $30^{\circ}$ for a
0.6-m system. }
\label{ftd30}
\end{figure}
\begin{figure}
\caption{ Difference ($\times 10^6$) in refractive index between
$0.5 \, \mu \rm m$ and other wavelengths.}\label{fri}
\end{figure}

\newpage
\baselineskip = .5\baselineskip  % single space the verbatim
\begin{verbatim}

\begin{figure}
\caption{ Comparison of the Mar\'{e}chal and the two- to six-term
approximations  with the exact value of the Strell ratio, for an
anisoplanatic displacement, for $D/r_0$  equal to 1.}\label{f1}
\end{figure}

\begin{figure}
\caption{ Comparison of the Mar\'{e}chal and the two- to six-term
approximations  with the exact value of the Strell ratio, for an
anisoplanatic displacement, for $D/r_0$  equal to 5. } \label{f5}
\end{figure}
\begin{figure}
\caption{ Comparison of the Mar\'{e}chal and the two- to six-term
approximations  with the exact value of the Strell ratio, for an
anisoplanatic displacement, for $D/r_0$  equal to 10. } \label{f10}
\end{figure}
\begin{figure}
\caption{Strehl ratio for angular anisoplanatic error at zenith,
for  various turbulence models, versus separation angle for a 0.6-m
system.   Upper-altitude turbulence has a strong effect on the
Strehl ratio.}
\label{faaz}
\end{figure}
\begin{figure}
\caption{ Strehl ratio for angular anisoplanatism at $30^{\circ}$
for a 0.6-m system.}
\label{faa30}
\end{figure}
\begin{figure}
\caption{ Strehl ratio versus time delay at zenith for a 0.6-m
system.}
\label{ftdz}
\end{figure}
\begin{figure}
\caption{ Strehl ratio versus time delay for a 0.6-m system at
$30^{\circ}$ zenith angle.     Strehl ratio  at $30^{\circ}$ for a
0.6-m system. }
\label{ftd30}
\end{figure}
\begin{figure}
\caption{ Difference ($\times 10^6$) in refractive index between
$0.5 \, \mu \rm m$ and other wavelengths.}\label{fri}
\end{figure}

\end{verbatim}
\newpage
\baselineskip = 2\baselineskip  % double space the text

\begin{table}
\caption{Values of $T_2$ and  $T_{5/3}$  to Solve for the Chromatic
Displacement for Various  Turbulence Models for a Wavelength of 0.5
$\mu \rm m$}
\begin{tabular}{lcc}
Model&$T_2$\tablenote{The units of $T_2$ are $m^{1/3}$.}&
$T_{5/3}$\tablenote{$T_{5/3}$  is dimensionless.} \\ \tableline
SLC-Day&$2.71 \, \times \, 10^{-6}$&$2.00 \, \times \, 10^{-7}$\\
HV-21&$6.16 \, \times \, 10^{-6}$&$3.60 \, \times \, 10^{-7}$\\
HV-54&$3.40 \, \times \, 10^{-5}$&$1.87 \, \times \, 10^{-6}$\\
HV-72&$5.95 \, \times \, 10^{-5}$&$3.25 \, \times \, 10^{-6}$\\
\end{tabular}
\end{table}

%\end{document}


\newpage
\baselineskip = .5\baselineskip  % single space the verbatim
\begin{verbatim}

\begin{table}
\caption{Values of $T_2$ and  $T_{5/3}$  to Solve for the Chromatic
Displacement for Various  Turbulence Models for a Wavelength of 0.5
$\mu \rm m$}
\begin{tabular}{lcc}
Model&$T_2$\tablenote{The units of $T_2$ are $m^{1/3}$.}&
$T_{5/3}$\tablenote{$T_{5/3}$  is dimensionless.} \\ \tableline
SLC-Day&$2.71 \, \times \, 10^{-6}$&$2.00 \, \times \, 10^{-7}$\\
HV-21&$6.16 \, \times \, 10^{-6}$&$3.60 \, \times \, 10^{-7}$\\
HV-54&$3.40 \, \times \, 10^{-5}$&$1.87 \, \times \, 10^{-6}$\\
HV-72&$5.95 \, \times \, 10^{-5}$&$3.25 \, \times \, 10^{-6}$\\
\end{tabular}
\end{table}

\end{document}


\end{verbatim}
\newpage
\baselineskip = 2\baselineskip  % double space the text NEW DOCUMENT!!!
\setcounter{eqletter}{0}       \setcounter{equation}{0}
\setcounter{section}{0}        \setcounter{subsection}{0}
\setcounter{subsubsection}{0}  \setcounter{figure}{0}
\begin{center}{\Large \bf Generation, propagation, and
amplification of dark solitons} \\
\vskip.5in
{W. Zhao and E. Bourkoff}

{\it Department of Electrical and Computer Engineering,
The University of South Carolina,
Columbia, South Carolina, 29208}
\end{center}
\vskip.5in
\begin{abstract}                % DON'T CHANGE THIS LINE
The technique for generating dark solitons with constant background
using guided-wave Mach--Zehnder interferometers is further examined.
Under optimal conditions, a reduction of 30\% in both the input optical power
and the driving voltage can be achieved, as compared with the case of
complete modulation.  Dark solitons are also found to experience compression
through amplification.  When the gain coefficient is small, adiabatic
amplification is possible.  Raman amplification can be used as the gain
mechanism for adiabatic amplification, in addition to being used for
loss-compensation.  The frequency and time shifts caused by intrapulse
stimulated Raman scattering are both found to be a factor of 2 smaller
than those for bright solitons.  Finally, the propagation properties of
even dark pulses are described quantitatively.
\end{abstract}


\section{ INTRODUCTION}
\label{INT}
Nonlinear optical pulses can propagate in dispersive fibers in the form of
bright and dark solitons under certain conditions,
as first described by Zakharov and Shabat in 1972\cite{ZA}
and in 1973,\cite{ZB}
respectively.
They are stationary solutions of the initial boundary value problem of the
nonlinear Schr{$\rm\ddot o$}dinger equation (NLSE).\cite{SA}

\newpage
\baselineskip = .5\baselineskip  % single space the verbatim
\begin{verbatim}
\documentstyle[osa,manuscript]{revtex}  % DON'T CHANGE
%
%
\newcommand{\MF}{{\large{\manual META}\-{\manual FONT}}}
\newcommand{\manual}{rm}        % Substitute rm (Roman) font.
\newcommand\bs{\char '134 }     % add backslash char to \tt font
%
%
\begin{document}                % INITIALIZE - DONT CHANGE
%
%
%
\title{Generation, propagation, and amplification of dark solitons}
%
\author{W. Zhao and E. Bourkoff}
%
\address{Department of Electrical and Computer Engineering,
The University of South Carolina,
Columbia, South Carolina, 29208}
%
\maketitle
\begin{abstract}                % DON'T CHANGE THIS LINE
The technique for generating dark solitons with constant background
using guided-wave Mach--Zehnder interferometers is further examined.
Under optimal conditions, a reduction of 30\% in both the input optical power
and the driving voltage can be achieved, as compared with the case of
complete modulation.  Dark solitons are also found to experience compression
through amplification.  When the gain coefficient is small, adiabatic
amplification is possible.  Raman amplification can be used as the gain
mechanism for adiabatic amplification, in addition to being used for
loss-compensation.  The frequency and time shifts caused by intrapulse
stimulated Raman scattering are both found to be a factor of 2 smaller
than those for bright solitons.  Finally, the propagation properties of
even dark pulses are described quantitatively.
\end{abstract}

\section{ INTRODUCTION}
\label{INT}
Nonlinear optical pulses can propagate in dispersive fibers in the form of
bright and dark solitons under certain conditions,
as first described by Zakharov and Shabat in 1972\cite{ZA}
and in 1973,\cite{ZB}
respectively.
They are stationary solutions of the initial boundary value problem of the
nonlinear Schr{$\rm\ddot o$}dinger equation (NLSE).\cite{SA} \ldots

\end{verbatim}
\newpage
\baselineskip = 2\baselineskip  % double space the text

In the anomalous dispersion regime of the fiber, under the boundary
condition $  u( z, t = \pm \infty ) = 0 $, there exists
a class of particle-like, stationary solutions
called bright solitons.\cite{HA}
In the normal dispersion region, under the boundary condition
$ | u( z, t = \pm \infty ) | = $constant,
one can obtain another class of stationary solutions,
which are called dark solitons, since a dip occurs at the
center of the pulse.\cite{HB} \ldots



In the following discussions, we adopt the normalization convention
used in Agrawal's book.\cite{AB}
We normalize the field amplitude $A$ (optical power $P_0 = A^2 $)
into $u$ by
\begin{eqnarray*}
u = \left( { 2 \pi n_2 {\tau_0}^2 }\over
{ \lambda A_{\rm eff} | \beta_2 | } \right)^{1/2} A ,
\end{eqnarray*}
where $A_{\rm eff }$ is the effective area of the propagating
mode, $n_2 = 3.2\times 10^{-16}$cm$^2 /$W is the nonlinear optical
Kerr coefficient of the silica
fiber, and $ \beta_{2} $ is a parameter describing
the group velocity dispersion of
fiber, \ldots




\section{GENERATION OF DARK SOLITONS}
\label{GDS}
In our earlier work\cite{ZBD,ZBE} we discussed the possibility
of using an integrated Mach--Zehnder interferometer
(MZI) to generate dark solitons with constant background. \ldots

 \ldots Therefore the pulse after the MZI,
when properly biased, can have the form
\begin{eqnarray}
u (0,t)  =  a\, {\rm sin} [ \delta \pi /2\, {\rm tanh} (t) ],
\label{E1}
\end{eqnarray}


\section{PROPAGATION AND AMPLIFICATION}
\label{PAA}
As discussed in Section \ref{GDS}, when smaller values of
$ \delta $ are used, pulses of better
characteristics are obtained.  This can be seen in Fig. 1(d),  where
$ a  =  1.33 $ and a pure fundamental dark soliton is
generated. \ldots .

\newpage
\baselineskip = .5\baselineskip  % single space the verbatim
\begin{verbatim}

In the anomalous dispersion regime of the fiber, under the boundary
condition $  u( z, t = \pm \infty ) = 0 $, there exists
a class of particle-like, stationary solutions
called bright solitons.\cite{HA}
In the normal dispersion region, under the boundary condition
$ | u( z, t = \pm \infty ) | = $constant,
one can obtain another class of stationary solutions,
which are called dark solitons, since a dip occurs at the
center of the pulse.\cite{HB} \ldots

In the following discussions, we adopt the normalization convention
used in Agrawal's book.\cite{AB}
We normalize the field amplitude $A$ (optical power $P_0 = A^2 $)
into $u$ by
\begin{eqnarray*}
u = \left( { 2 \pi n_2 {\tau_0}^2 }\over
{ \lambda A_{\rm eff} | \beta_2 | } \right)^{1/2} A ,
\end{eqnarray*}
where $A_{\rm eff }$ is the effective area of the propagating
mode, $n_2 = 3.2\times 10^{-16}$cm$^2 /$W is the nonlinear optical
Kerr coefficient of the silica
fiber, and $ \beta_{2} $ is a parameter describing
the group velocity dispersion of
fiber, \ldots

\section{GENERATION OF DARK SOLITONS}
\label{GDS}
In our earlier work\cite{ZBD,ZBE} we discussed the possibility
of using an integrated Mach--Zehnder interferometer
(MZI) to generate dark solitons with constant background. \ldots

 \ldots Therefore the pulse after the MZI,
when properly biased, can have the form
\begin{eqnarray}
u (0,t)  =  a\, {\rm sin} [ \delta \pi /2\, {\rm tanh} (t) ],
\label{E1}
\end{eqnarray}

\section{PROPAGATION AND AMPLIFICATION}
\label{PAA}
As discussed in Section \ref{GDS}, when smaller values of
$ \delta $ are used, pulses of better
characteristics are obtained.  This can be seen in Fig. 1(d),  where
$ a  =  1.33 $ and a pure fundamental dark soliton is
generated. \ldots .

\end{verbatim}
\newpage
\baselineskip = 2\baselineskip  % double space the text

We first examine the solution of a modified NLSE with a constant gain:
\begin{eqnarray}
i u_{z} - {1/2} u_{tt} + |u|^2 u  =  i \Gamma u,
\label{E2}
\end{eqnarray}
where $\Gamma $ is assumed to be a constant,
appropriate for the Raman amplification
under strong pumping without depletion.  The
solution of a similar equation to Eq. (\ref{E2}),
but \ldots
\begin{mathletters}
\begin{eqnarray}
t'  &=&  t e^{ \Gamma z }, \label{E4}  \\
z'  &=&  { e^{2 \Gamma z } - 1 \over  2 \Gamma },  \label{E5}  \\
u   &=&  v e^{ \Gamma z } .                          \label{E6}
\end{eqnarray}
\end{mathletters}
Under this transformation,  the NLSE has the new form
\begin{eqnarray}
i v_{z'} -\slantfrac{1}{2} v_{ t' t' } - |v|^2 v  &=&
- { \Gamma t' \over 2 \Gamma z' + 1} v_{t'}. \label{E7}
\end{eqnarray}
The solution of Eq. (\ref{E2}) when $\Gamma $ = 0 is well known and
has the form ${\rm exp} [i \sigma (z,t) ] \kappa \tanh \kappa t $,
where $\kappa $ is the form factor and the phase variable satisfies
$ \partial \sigma / \partial z  =  \kappa^2 $.\cite{ZA}
Therefore, when the right-hand-side of
Eq.(\ref{E7}) is zero, an exact solution for $v(z',t)$ can be
obtained from Eq. (\ref{E7}).
On the other hand, when $z \rightarrow \infty $ and hence
$z' \rightarrow \infty $ or $ \Gamma \rightarrow 0$,  the
right-hand side of Eq. (\ref{E7}) becomes infinitely small.
Under these conditions, we can treat the right-hand
side of Eq. (\ref{E7}) as a perturbation to the NLSE.
\ldots
\begin{eqnarray}
u(z,t)&=&{\rm exp}\left( i{e^{2\Gamma z}-1 \over 2\Gamma}\right)
e^{\Gamma z} \, {\rm tanh} (te^{\Gamma z}),
\label{E8}                                \\
\Gamma&=&g(e^{-2\Gamma_pz} + e^{-2\Gamma_p(L-z)}) - \Gamma_s,
\label{E9} \\
g&=&{\Gamma_p(\Gamma_s + \beta)L \over {\rm sinh}(\Gamma_pL)}
e^{\Gamma_pL} , \label{E10} \\
\kappa(z) &=& \kappa_0 \, {\rm exp}(\beta z). \label{E11}
\end{eqnarray}

\section{EFFECTS OF INTRAPULSE STIMULATED RAMAN SCATTERING}
\label{EIS}
The properties of dark solitons considered thus far are
based on the simplified propagation equation (\ref{E2}).
\ldots

\newpage
\baselineskip = .5\baselineskip  % single space the verbatim
\begin{verbatim}
We first examine the solution of a modified NLSE with a constant gain:
\begin{eqnarray}
i u_{z} - {1/2} u_{tt} + |u|^2 u  =  i \Gamma u,
\label{E2}
\end{eqnarray}
where $\Gamma $ is assumed to be a constant,
appropriate for the Raman amplification
under strong pumping without depletion.  The
solution of a similar equation to Eq. (\ref{E2}),
but \ldots
\begin{mathletters}
\begin{eqnarray}
t'  &=&  t e^{ \Gamma z }, \label{E4}  \\
z'  &=&  { e^{2 \Gamma z } - 1 \over  2 \Gamma },  \label{E5}  \\
u   &=&  v e^{ \Gamma z } .                          \label{E6}
\end{eqnarray}
\end{mathletters}
Under this transformation,  the NLSE has the new form
\begin{eqnarray}
i v_{z'} -\slantfrac{1}{2} v_{ t' t' } - |v|^2 v  &=&
- { \Gamma t' \over 2 \Gamma z' + 1} v_{t'}. \label{E7}
\end{eqnarray}
The solution of Eq. (\ref{E2}) when $\Gamma $ = 0 is well known and
has the form ${\rm exp} [i \sigma (z,t) ] \kappa \tanh \kappa t $,
where $\kappa $ is the form factor and the phase variable satisfies
$ \partial \sigma / \partial z  =  \kappa^2 $.\cite{ZA}
Therefore, when the right-hand-side of
Eq.(\ref{E7}) is zero, an exact solution for $v(z',t)$ can be
obtained from Eq. (\ref{E7}).
On the other hand, when $z \rightarrow \infty $ and hence
$z' \rightarrow \infty $ or $ \Gamma \rightarrow 0$,  the
right-hand side of Eq. (\ref{E7}) becomes infinitely small.
Under these conditions, we can treat the right-hand
side of Eq. (\ref{E7}) as a perturbation to the NLSE.
\ldots
\begin{eqnarray}
u(z,t)&=&{\rm exp}\left( i{e^{2\Gamma z}-1 \over 2\Gamma}\right)
e^{\Gamma z} \, {\rm tanh} (te^{\Gamma z}),
\label{E8}                                \\
\Gamma&=&g(e^{-2\Gamma_pz} + e^{-2\Gamma_p(L-z)}) - \Gamma_s,
\label{E9} \\
g&=&{\Gamma_p(\Gamma_s + \beta)L \over {\rm sinh}(\Gamma_pL)}
e^{\Gamma_pL} , \label{E10} \\
\kappa(z) &=& \kappa_0 \, {\rm exp}(\beta z). \label{E11}
\end{eqnarray}

\section{EFFECTS OF INTRAPULSE STIMULATED RAMAN SCATTERING}
\label{EIS}
The properties of dark solitons considered thus far are
based on the simplified propagation equation (\ref{E2}).
\ldots

\end{verbatim}
\newpage
\baselineskip = 2\baselineskip  % double space the text

\ldots The energies of these
separating solitons are distributed in such way to ensure conservation of
momentum.                                             \ldots

\begin{eqnarray}
iu_z - {1/2}u_{tt}+|u|^2u &=& \tau_d{\partial |u|^2 \over \partial t}u,
\label{E12}
\end{eqnarray}

\section{EVEN  DARK PULSES}
\label{EDP}
Even dark pulses,\cite{KA,WA} which are symmetric functions of time
centered around the pulse, can be simply generated by driving
the MZI with a short electric pulse. \ldots

If we define the amplitudes of the secondary soliton pairs as
\begin{eqnarray}
\kappa_n  =  \kappa_0 - \Delta_{n} ,  \label{E16}
\end{eqnarray}
then the $n$th order secondary pulse shape (n = 1, 2, 3, \ldots )
has the form
\begin{eqnarray}
u_n (z,t)  =  \kappa_{0}{(\lambda_n - i \nu_n )^2 - \nu_n
\,{\rm exp} [ 2 \nu_n (t-t_{n0} - \lambda_{n} z)] \over  1 +
\nu_n\, {\rm exp} [ 2 \nu_n (t-t_{n0} - \lambda_n z)]} e^{iz},
\label{E17}
\end{eqnarray}
\ldots

\section{CONCLUSIONS}
We have discussed the possibility of using the waveguide Mach--Zehnder
interferometer to generate a variety of dark solitons under constant
background.  Under optimal
operation,  30\% less input power and driving voltage are required
than for complete modulation.  The
generated solitons can have good pulse quality and stimulated Raman
scattering process can be utilized to compensate for fiber loss and even
to amplify and compress the dark solitons. \ldots


\acknowledgments

The authors  thank the reviewers for their constructive comments.
This research was supported by National Science Foundation grant
ECS-91960-64.

\newpage
\baselineskip = .5\baselineskip  % single space the verbatim
\begin{verbatim}

\ldots The energies of these
separating solitons are distributed in such way to ensure conservation of
momentum.                                             \ldots

\begin{eqnarray}
iu_z - {1/2}u_{tt}+|u|^2u &=& \tau_d{\partial |u|^2 \over \partial t}u,
\label{E12}
\end{eqnarray}
\ldots

\section{EVEN  DARK PULSES}
\label{EDP}
Even dark pulses,\cite{KA,WA} which are symmetric functions of time
centered around the pulse, can be simply generated by driving
the MZI with a short electric pulse. \ldots

If we define the amplitudes of the secondary soliton pairs as
\begin{eqnarray}
\kappa_n  =  \kappa_0 - \Delta_{n} ,  \label{E16}
\end{eqnarray}
then the $n$th order secondary pulse shape (n = 1, 2, 3, \ldots )
has the form
\begin{eqnarray}
u_n (z,t)  =  \kappa_{0}{(\lambda_n - i \nu_n )^2 - \nu_n
\,{\rm exp} [ 2 \nu_n (t-t_{n0} - \lambda_{n} z)] \over  1 +
\nu_n\, {\rm exp} [ 2 \nu_n (t-t_{n0} - \lambda_n z)]} e^{iz},
\label{E17}
\end{eqnarray}
\ldots

\section{CONCLUSIONS}
We have discussed the possibility of using the waveguide Mach--Zehnder
interferometer to generate a variety of dark solitons under constant
background.  Under optimal
operation,  30\% less input power and driving voltage are required
than for complete modulation.  The
generated solitons can have good pulse quality and stimulated Raman
scattering process can be utilized to compensate for fiber loss and even
to amplify and compress the dark solitons. \ldots


\acknowledgments

The authors  thank the reviewers for their constructive comments.
This research was supported by National Science Foundation grant
ECS-91960-64.

\end{verbatim}
\newpage
\baselineskip = 2\baselineskip  % double space the text

\begin{references}
\bibitem{ZA}
V. E. Zakharov and A. B. Shabat, ``Exact theory of two-dimensional
self-focusing and one-dimensional self-modulation of waves in nonlinear
media,'' Sov. Phys. JETP {\bf 5,} 364--372 (1972).
\bibitem{ZB}
V. E. Zakharov and A. B. Shabat, ``Interaction between solitons in
a stable medium,'' Sov. Phys. JETP {\bf 37,} 823--828 (1973).
\bibitem{SA}
J. Satruma and N. Yajima, ``Initial value problems of one-dimensional
self-phase modulation of nonlinear waves in dispersive media,'' Progr.
Theor. Phys. Suppl. {\bf 55,} 284--305 (1974).
\bibitem{HA}
A. Hasegawa and F. Tappert, ``Transmission of stationary nonlinear
optical pulses in dispersive dielectric fibers. I. Anomalous
dispersion,'' Appl. Phys. Lett. {\bf 23,} 142 (1973).
\bibitem{HB}
A. Hasegawa and F. Tappert, ``Transmission of stationary nonlinear
optical pulses in dispersive dielectric fibers. II. Normal
dispersion,'' Appl. Phys. Lett. {\bf 23,} 172 (1973).
\bibitem{AB}
G. P. Agrawal, {\it Nonlinear Fiber Optics,} Chapt. 5 (Academic,
Boston, 1989).
\bibitem{ZBD}
W. Zhao and E. Bourkoff, ``Generation of dark solitons under cw background
using waveguide EO modulators,'' Opt. Lett. {\bf 15,} 405--407 (1990).
\bibitem{ZBE}
W. Zhao and E. Bourkoff, ``Dark solitons: generation, propagation,
and amplification'', {\it OSA Annual Meeting,} Vol. 18 of 1989 OSA
Technical Digest Series (Optical
Society of America, Washington, D.C., 1989), p. 185.
\bibitem{KA}
D. Kr$\rm {\ddot o}$kel, N. J. Halas, G. Giuliani, and D. Grischkowsky,
``Dark-pulse propagation in optical fibers,''
Phys. Rev. Lett. {\bf 60,} 29--32 (1988).
\bibitem{WA}
A. M. Weiner, J. P. Heritage, R. J. Hawkins, R. N. Thurston, E. M.
Kirschner, D. E. Leaird, and W. J. Tomlinson, ``Experimental
observation of the fundamental dark soliton in optical fibers,''
Phys. Rev. Lett. {\bf 61,} 2445--2448 (1988).
\end{references}

\newpage
\baselineskip = .5\baselineskip  % single space the verbatim
\begin{verbatim}

\begin{references}
\bibitem{ZA}
V. E. Zakharov and A. B. Shabat, ``Exact theory of two-dimensional
self-focusing and one-dimensional self-modulation of waves in nonlinear
media,'' Sov. Phys. JETP {\bf 5,} 364--372 (1972).
\bibitem{ZB}
V. E. Zakharov and A. B. Shabat, ``Interaction between solitons in
a stable medium,'' Sov. Phys. JETP {\bf 37,} 823--828 (1973).
\bibitem{SA}
J. Satruma and N. Yajima, ``Initial value problems of one-dimensional
self-phase modulation of nonlinear waves in dispersive media,'' Progr.
Theor. Phys. Suppl. {\bf 55,} 284--305 (1974).
\bibitem{HA}
A. Hasegawa and F. Tappert, ``Transmission of stationary nonlinear
optical pulses in dispersive dielectric fibers. I. Anomalous
dispersion,'' Appl. Phys. Lett. {\bf 23,} 142 (1973).
\bibitem{HB}
A. Hasegawa and F. Tappert, ``Transmission of stationary nonlinear
optical pulses in dispersive dielectric fibers. II. Normal
dispersion,'' Appl. Phys. Lett. {\bf 23,} 172 (1973).
\bibitem{AB}
G. P. Agrawal, {\it Nonlinear Fiber Optics,} Chapt. 5 (Academic,
Boston, 1989).
\bibitem{ZBD}
W. Zhao and E. Bourkoff, ``Generation of dark solitons under cw background
using waveguide EO modulators,'' Opt. Lett. {\bf 15,} 405--407 (1990).
\bibitem{ZBE}
W. Zhao and E. Bourkoff, ``Dark solitons: generation, propagation,  and
amplification'', {\it OSA Annual Meeting,} Vol. 18 of 1989 OSA Technical
Digest Series (Optical Society of America, Washington, D.C., 1989), p. 185.
\bibitem{KA}
D. Kr$\rm {\ddot o}$kel, N. J. Halas, G. Giuliani, and D. Grischkowsky,
``Dark-pulse propagation in optical fibers,''
Phys. Rev. Lett. {\bf 60,} 29--32 (1988).
\bibitem{WA}
A. M. Weiner, J. P. Heritage, R. J. Hawkins, R. N. Thurston, E. M.
Kirschner, D. E. Leaird, and W. J. Tomlinson, ``Experimental
observation of the fundamental dark soliton in optical fibers,''
Phys. Rev. Lett. {\bf 61,} 2445--2448 (1988).
\end{references}

\end{verbatim}
\newpage
\baselineskip = 2\baselineskip  % double space the text


\begin{figure}
\caption{The dark solitons generated by the waveguide
Mach-Zehnder interferometer.  The amplitude of the input cw
light is chosen to be $ a =  \pi /2 $ for (a)-(c).  The
parameter $ \delta $ is (a) 0.8, (b) 0.5, and (c) 0.2.  Part (d) is the case
of optimal operation when $ a =  1.33 $, and $ \delta  =  0.7 $.  In all
cases, the output pulse shapes are plotted as solid curves while
the dashed curves are input pulse shapes.  The pulses shown here are at a
propagation distance of $ z  =  4 $.}
\end{figure}
\begin{figure}
\caption{
Dark solitons under constant gain.  Pulse shapes (solid) when $\Gamma$=0.05
(a) and 1(b), after certain propagation distance, $\Gamma$z=1.6, as
compared to input pulse shapes (dashed). (c): The pulse duration,
relative to its input,  as a function of $\Gamma z$ at various $\Gamma$.
The solid curve is the adiabatic approximation obtained by perturbation
method. Three values of $\Gamma$ are used: $\Gamma$ = 0.05 (dotted);
0.2 (dash-dotted); and 1 (dashed). Negative $\Gamma$z depicts the
case of loss.}
\end{figure}
\begin{figure}
\caption{
The pulse shapes of amplified dark solitons. (a) $ \delta  =  0.5 $,
$ \beta  =  2 ln 1.05 $, $ \Gamma_p L  =  2 $, after 8 amplifying cycles
(solid); (b) $ \delta  =  0.5 $, $ \beta  =  2 ln 1.02 $, $ \Gamma_p L
=  2 $, after 16 amplifying cycles (solid); (c) $ \delta  =  0.5 $,
$ \beta  =  2 ln 1.02 $, $ \Gamma_p L  =  0.5 $, after 16 amplifying
cycles (solid); (d) The input pulse is the same as in Fig. 1(c),
$ \beta  =  2 ln 1.05 $, after 8 amplification periods (solid).  The
input pulse shapes are plotted as dashed curves.}
\end{figure}
\begin{figure}
\caption{
(a) The shape of a fundamental dark soliton after a propagation distance
of 40 (solid). The normalized time delay $ \tau_d  =  0.01 $.  The
dashed curve is the input pulse shape. (b) The trace of the soliton (solid)
as a function of propagation distance for the situation described by (a).
The dotted curve represents the case for a fundamental bright soliton
under similar conditions.}
\end{figure}

\newpage
\baselineskip = .5\baselineskip  % single space the verbatim
\begin{verbatim}

\begin{figure}
\caption{The dark solitons generated by the waveguide
Mach-Zehnder interferometer.  The amplitude of the input cw
light is chosen to be $ a =  \pi /2 $ for (a)-(c).  The
parameter $ \delta $ is (a) 0.8, (b) 0.5, and (c) 0.2.  Part (d) is the case
of optimal operation when $ a =  1.33 $, and $ \delta  =  0.7 $.  In all
cases, the output pulse shapes are plotted as solid curves while
the dashed curves are input pulse shapes.  The pulses shown here are at a
propagation distance of $ z  =  4 $.}
\end{figure}
\begin{figure}
\caption{
Dark solitons under constant gain.  Pulse shapes (solid) when $\Gamma$=0.05
(a) and 1(b), after certain propagation distance, $\Gamma$z=1.6, as
compared to input pulse shapes (dashed). (c): The pulse duration, relative
to its input,  as a function of $\Gamma z$ at various $\Gamma$.
The solid curve is the adiabatic approximation obtained by perturbation
method. Three values of $\Gamma$ are used: $\Gamma$ = 0.05 (dotted);
0.2 (dash-dotted); and 1 (dashed). Negative $\Gamma$z depicts the case
of loss.}
\end{figure}
\begin{figure}
\caption{
The pulse shapes of amplified dark solitons. (a) $ \delta  =  0.5 $,
$ \beta  =  2 ln 1.05 $, $ \Gamma_p L  =  2 $, after 8 amplifying cycles
(solid); (b) $ \delta  =  0.5 $, $ \beta  =  2 ln 1.02 $, $ \Gamma_p L
=  2 $, after 16 amplifying cycles (solid); (c) $ \delta  =  0.5 $,
$ \beta  =  2 ln 1.02 $, $ \Gamma_p L  =  0.5 $, after 16 amplifying
cycles (solid); (d) The input pulse is the same as in Fig. 1(c),
$ \beta  =  2 ln 1.05 $, after 8 amplification periods (solid).  The
input pulse  shapes are plotted as dashed curves.}
\end{figure}
\begin{figure}
\caption{
(a) The shape of a fundamental dark soliton after a propagation distance
of 40 (solid). The normalized time delay $ \tau_d  =  0.01 $.  The dashed
curve is the input pulse shape. (b) The trace of the soliton (solid)
as a function of propagation distance for the situation described by (a).
The dotted curve represents the case for a fundamental bright soliton
under similar conditions.}
\end{figure}

\end{verbatim}
\newpage
\baselineskip = 2\baselineskip  % double space the text



\begin{table}
\caption{Amplitudes of  Secondary Even Dark Pulses}
\begin{tabular}{cccccr}
&&Input Pulse Shape&&&\\
\cline{2-4}
$\Delta_n$Values&$\kappa_0|{\rm tanh}t|$&$\kappa_0[1-{\rm exp}(-t^2/
{\tau_g}^2)]^{1/2}$&$\kappa_0[1-{\rm sech}(t/\tau_s)]$&Avg.&Range\\
\tableline
$\Delta_1$&0.34&0.30&0.21&0.28&$\pm 25\%$ \\
$\Delta_2$&1.56&1.41&1.26&1.41&$\pm 11\%$ \\
$\Delta_3$&2.47&2.26&2.28&2.34&$\pm 6\%$ \\
$\Delta_4$&3.52&3.25&3.31&3.36&$\pm 6\%$ \\
$\Delta_5$&4.45&4.26&4.42&4.38&$\pm 6\%$ \\
$\Delta_6$&5.52&5.35&5.50&5.50&$\pm 5\%$ \\
\end{tabular}
\end{table}

%\end{document}

\newpage
\baselineskip = .5\baselineskip  % single space the verbatim
\begin{verbatim}

\begin{table}
\caption{Amplitudes of  Secondary Even Dark Pulses}
\begin{tabular}{cccccr}
&&Input Pulse Shape&&&\\
\cline{2-4}
$\Delta_n$Values&$\kappa_0|{\rm tanh}t|$&$\kappa_0[1-{\rm exp}(-t^2/
{\tau_g}^2)]^{1/2}$&$\kappa_0[1-{\rm sech}(t/\tau_s)]$&Avg.&Range\\
\tableline
$\Delta_1$&0.34&0.30&0.21&0.28&$\pm 25\%$ \\
$\Delta_2$&1.56&1.41&1.26&1.41&$\pm 11\%$ \\
$\Delta_3$&2.47&2.26&2.28&2.34&$\pm 6\%$ \\
$\Delta_4$&3.52&3.25&3.31&3.36&$\pm 6\%$ \\
$\Delta_5$&4.45&4.26&4.42&4.38&$\pm 6\%$ \\
$\Delta_6$&5.52&5.35&5.50&5.50&$\pm 5\%$ \\
\end{tabular}
\end{table}

\end{document}

\end{verbatim}
\newpage
\baselineskip = 2\baselineskip  % double space the text NEW DOCUMENT!!!
\setcounter{eqletter}{0}       \setcounter{equation}{0}
\setcounter{section}{0}        \setcounter{subsection}{0}
\setcounter{subsubsection}{0}  \setcounter{figure}{0}
\begin{center}{\Large \bf Designing digital optical
computing systems: power distribution and cross talk}
\vskip.5in
{Jonathan P. Pratt and      Vincent P. Heuring}

{\it
When this work was performed, both the authors were with the
Boulder     Optoelectronic Computing Systems Center      and
Department of Electrical and Computer Engineering, University of
Colorado,      Campus Box 425, Boulder, Colorado 80309-0425. They
are now with  the  Department of Radiology, University of  Colorado
Health Sciences Center, Box  A034, 4200 East Ninth Avenue, Denver,
Colorado 80262. }
\end{center}
\vskip.5in

\begin{abstract}
  Complex optical computer designs must implicitly or explicitly
allow       for power budgeting, to compensate for cross talk and
loss in both       devices and interconnections.  We develop
algorithms for       calculating the system  cross talk
and power loss in optical systems,       using a graph-theoretic
model.  Devices are modeled as directed       graphs with nodes
representing inputs and outputs, and edges are       weighted with
the power relationships between nodes.  Systems are       modeled
by interconnecting the individual device graphs in a manner
that reflects the connectivity of the system.  A system's power
   budget is efficiently computed by a depth-first search of its
graph.       The algorithms have been incorporated into an optical
computer-aided design system that       is now being used to
design a bit-serial optical computer       containing hundreds of
components.

 Key words: Optical computing, optical systems, optical
communications, power loss, cross talk, graphs. \end{abstract}

\newpage
\baselineskip = .5\baselineskip  % single space the verbatim
\begin{verbatim}

\documentstyle[osa,aplop,manuscript]{revtex}  % DON'T CHANGE %
\newcommand{\MF}{{\large{\manual META}\-{\manual FONT}}}
\newcommand{\manual}{rm}        % Substitute rm (Roman) font.
\newcommand\bs{\char '134 }     % add backslash char to \tt font %
%
\begin{document}                % INITIALIZE - DONT CHANGE

\title{Designing digital optical computing systems:      power
distribution and cross talk}

\author{Jonathan P. Pratt and      Vincent P. Heuring}

\address{
When this work was performed, both the authors were with the
Boulder     Optoelectronic Computing Systems Center      and
Department of Electrical and Computer Engineering, University of
Colorado,      Campus Box 425, Boulder, Colorado 80309-0425. They
are now with  the  Department of Radiology, University of  Colorado
Health Sciences Center, Box  A034, 4200 East Ninth Avenue, Denver,
Colorado 80262. }

\maketitle

\begin{abstract}
  Complex optical computer designs must implicitly or explicitly
allow       for power budgeting, to compensate for cross talk and
loss in both       devices and interconnections.  We develop
algorithms for       calculating the system  cross talk
and power loss in optical systems,       using a graph-theoretic
model.  Devices are modeled as directed       graphs with nodes
representing inputs and outputs, and edges are       weighted with
the power relationships between nodes.  Systems are       modeled
by interconnecting the individual device graphs in a manner
that reflects the connectivity of the system.  A system's power
   budget is efficiently computed by a depth-first search of its
graph.       The algorithms have been incorporated into an optical
computer-aided design system that       is now being used to
design a bit-serial optical computer       containing hundreds of
components.

 Key words: Optical computing, optical systems, optical
communications, power loss, cross talk, graphs. \end{abstract}

\end{verbatim}
\newpage
\baselineskip = 2\baselineskip  % double space the text


  \section{ Introduction}
          We describe a technique that facilitates the design
 of digital optical computers and other complex optical circuitry,
     such as optical communications systems.  Although there has
been       some discussion in the literature of power budgeting in
optical       systems,\cite{1,2} the treatment has been limited to
relatively       uncomplicated applications, \ldots



   \section{Power Loss and Cross Talk in the System}

\subsection{ Introduction}
Appropriate signal levels must be maintained in any digital
optical system that uses signal level thresholds to encode
transmitted information.  Usually a high-level signal represents a
     logic 1 and a low-level signal represents a logic 0.  In these
     systems the device characteristics of importance are power
loss and       cross talk.

\ldots

\subsection{Power Levels and Correct Device Operation}
Here we discuss the type of power information desired from a
system model.  Since the objective is to find weak points in the
    system power flow, only power extremes are considered.  Power
     extremes are the cross talk and signal levels obtained when
the worst       possible combinations of device states and input
power levels are  assumed.

\ldots        The weakest 1 arriving at the
detection point under all conditions from all possible paths
to the point is defined as $P_{1\rm min}$, and similarly, the
strongest 0 is defined as $ P_{0\rm max}$.   Proper
device operation can be ensured if the following relations
are met:
\begin{equation}      P_{0\rm max} < P_{S2} < P_{D}  <
P_{S1} < P_{1\rm min}.\label{p0} \end{equation}

\newpage
\baselineskip = .5\baselineskip  % single space the verbatim
\begin{verbatim}

  \section{ Introduction}
          We describe a technique that facilitates the design
 of digital optical computers and other complex optical circuitry,
 such as optical communications systems.  Although there has
been some discussion in the literature of power budgeting in
optical systems,\cite{1,2} the treatment has been limited to
relatively uncomplicated applications, \ldots



   \section{Power Loss and Cross Talk in the System}

\subsection{ Introduction}
Appropriate signal levels must be maintained in any digital
optical system that uses signal level thresholds to encode
transmitted information.  Usually a high-level signal represents a
logic 1 and a low-level signal represents a logic 0.  In these
systems the device characteristics of importance are power
loss and cross talk.

\ldots

\subsection{Power Levels and Correct Device Operation}
Here we discuss the type of power information desired from a
system model.  Since the objective is to find weak points in the
system power flow, only power extremes are considered.  Power
extremes are the cross talk and signal levels obtained when
the worst possible combinations of device states and input
power levels are assumed.

\ldots        The weakest 1  arriving at the
detection point under all conditions from all possible paths
to the point is defined as $P_{1\rm min}$, and similarly, the
strongest 0 is defined as $ P_{0\rm max}$.  Proper
device operation can be  ensured if the following relations
are met:
\begin{equation}      P_{0\rm max} < P_{S2} < P_{D}  <
P_{S1} < P_{1\rm min}.\label{p0} \end{equation}

\end{verbatim}
\newpage
\baselineskip = 2\baselineskip  % double space the text


It is also
desirable to have information about $ P_{\rm max} $, the maximum
    power level that can occur at the inputs to a given device.  A
power       detector may provide erroneous results when the power
of a logic 1    arriving at a detection point is too large;
that is, when $ P_{\rm max} $  exceeds $ P_D $  by some large
amount.  A second and more important reason       for computing $
P_{\rm max} $  is that it makes the major contribution to
cross talk, as discussed below.  Knowledge of the       power
triple $ P_{0\rm max}, P_{1\rm min}, $ and $ P_{\rm max} $
at each device in a system       permits the tracking of power
levels throughout the entire system.




\subsection{ Modeling the Device}
Here we discuss the means for calculating the power triples
$ P_{0\rm max},\kern.5em P_{1\rm min}, $ and $ P_{\rm max}$
  at the outputs of a given device, given the       values of the
triples at each of its inputs.  \ldots

\ldots The power triple for the $j$th output of a device is
computed       from the input triples and the coupling terms as
follows: \begin{eqnarray}                           P_{1\rm
min}({\rm out})_j & = & \begin{array}[t]{c}{\rm min} \\[-15pt]
{s\in\rm states} \end{array}\, \{ \,  \begin{array}[t]{c}{\rm min}
\\[-15pt] {\rm inputs}\, i \end{array} \;  [P_{1\rm min}({\rm in})_i
- L_{ij}(s)]\}, L_{ij}(s) \; \in \;  {\rm
loss},\label{p1min} \\ P_{0\rm max}({\rm out})_j & =&
\begin{array}[t]{c}{\rm max} \\[-15pt] {s\in\rm states}
\end{array}\, \sum_{{\rm inputs}\, i}\left\{
\begin{array}{l}P_{\rm max}{\rm (in)}_i - L_{ij}(s)\;  ,\;
L_{ij}(s)\in {\rm cross\, talk},\label{p0max} \\  P_{0\rm max}({\rm
in})_i - L_{ij}(s)\; ,\; L_{ij}(s)\in {\rm loss},   \end{array}
\right. \\ P_{\rm max}({\rm out})_j &=& \begin{array}[t]{c}{\rm
max} \\[-15pt] {s\in\rm states} \end{array}\, \sum_{{\rm inputs}\,
i}  P_{\rm max}{\rm (in)}_i - L_{ij}(s) . \label{pmax}
\end{eqnarray}

\newpage
\baselineskip = .5\baselineskip  % single space the verbatim
\begin{verbatim}

It is also
desirable to have information about $ P_{\rm max} $, the maximum
    power level that can occur at the inputs to a given device.  A
power       detector may provide erroneous results when the power
of a logic 1    arriving at a detection point is too large;
that is, when $ P_{\rm max} $  exceeds $ P_D $  by some large
amount.  A second and more important reason       for computing $
P_{\rm max} $  is that it makes the major contribution to
cross talk, as discussed below.  Knowledge of the       power
triple $ P_{0\rm max}, P_{1\rm min}, $ and $ P_{\rm max} $
at each device in a system       permits the tracking of power
levels throughout the entire system.




\subsection{ Modeling the Device}
Here we discuss the means for calculating the power triples
$ P_{0\rm max},\kern.5em P_{1\rm min}, $ and $ P_{\rm max}$
  at the outputs of a given device, given the       values of the
triples at each of its inputs.  \ldots

\ldots The power triple for the $j$th output of a device is
computed       from the input triples and the coupling terms as
follows: \begin{eqnarray}                           P_{1\rm
min}({\rm out})_j & = & \begin{array}[t]{c}{\rm min} \\[-15pt]
{s\in\rm states} \end{array}\, \{ \,  \begin{array}[t]{c}{\rm min}
\\[-15pt] {\rm inputs}\, i \end{array} \;  [P_{1\rm min}({\rm in})_i
- L_{ij}(s)]\}, L_{ij}(s) \; \in \;  {\rm
loss},\label{p1min} \\ P_{0\rm max}({\rm out})_j & =&
\begin{array}[t]{c}{\rm max} \\[-15pt] {s\in\rm states}
\end{array}\, \sum_{{\rm inputs}\, i}\left\{
\begin{array}{l}P_{\rm max}{\rm (in)}_i - L_{ij}(s)\;  ,\;
L_{ij}(s)\in {\rm cross\, talk},\label{p0max} \\  P_{0\rm max}({\rm
in})_i - L_{ij}(s)\; ,\; L_{ij}(s)\in {\rm loss},   \end{array}
\right. \\ P_{\rm max}({\rm out})_j &=& \begin{array}[t]{c}{\rm
max} \\[-15pt] {s\in\rm states} \end{array}\, \sum_{{\rm inputs}\,
i}  P_{\rm max}{\rm (in)}_i - L_{ij}(s) . \label{pmax}
\end{eqnarray}

\end{verbatim}
\newpage
\baselineskip = 2\baselineskip  % double space the text


Equation (\ref{p1min}) states that the power of
the minimum 1 emerging from the   {\it j}th output of the device will
be the minimum over all possible       states of the minimum over
all possible inputs having loss terms of       the minimum 1's
arriving at those inputs minus the loss terms.        Equation
(\ref{p0max}) states that the power of the       maximum 0 emerging
from the {\it j}th output of the device will be the  maximum over
all possible states of the sum of the inputs \ldots



\subsection{ Modeling the System}
In this section we extend the applicability of the device  graph model
to complete systems. \ldots

\section{Discussion}
The technique described above is indispensable in designing
complex optical systems whose components have significant
nonidealities.  It has been incorporated into a digital optical
computer-assisted design    system,  HATCH,\cite{10} where it has
proven invaluable in the design of optical  counters
and an optical delay line memory system.  It is now being
used in designing a bit-serial optical computer now       under
construction in our laboratories. \ldots


\acknowledgments
This research was supported by the National Science Foundation
Engineering Research Centers      program  under  grant   CDR
8622236 and by  the Colorado       Advanced Technology Institute.

\newpage
\baselineskip = .5\baselineskip  % single space the verbatim
\begin{verbatim}

Equation (\ref{p1min}) states that the power of
the minimum 1 emerging from the   {\it j}th output of the device will
be the minimum over all possible       states of the minimum over
all possible inputs having loss terms of       the minimum 1's
arriving at those inputs minus the loss terms.        Equation
(\ref{p0max}) states that the power of the       maximum 0 emerging
from the {\it j}th output of the device will be the  maximum over
all possible states of the sum of the inputs \ldots



\subsection{ Modeling the System}
In this section we extend the applicability of the device  graph model
to complete systems. \ldots

\section{Discussion}
The technique described above is indispensable in designing
complex optical systems whose components have significant
nonidealities.  It has been incorporated into a digital optical
computer-assisted design    system,  HATCH,\cite{10} where it has
proven invaluable in the design of optical  counters
and an optical delay line memory system.  It is now being
used in designing a bit-serial optical computer now       under
construction in our laboratories. \ldots


\acknowledgments
This research was supported by the National Science Foundation
Engineering Research Centers      program  under  grant   CDR
8622236 and by  the Colorado       Advanced Technology Institute.

\end{verbatim}
\newpage
\baselineskip = 2\baselineskip  % double space the text




\begin{references}
\bibitem{1}  J. C. Palais, {\it Fiber Optic Communications},  2nd
ed.  (Prentice-Hall, Englewood Cliffs, N.J., 1988), pp. 270-271.
\bibitem{2}  E. E. Bash and H. A. Carnes, ``Digital optical
communications  systems,''in {\it Fiber Optics,} J. Daly, ed. (CRC
          Press, Boca Raton, Fla., 1987), pp. 153-154.
\bibitem{3}
V. P. Heuring, H. F. Jordan, and J. P. Pratt, "A bit serial
 architecture for optical computing," \ao {\bf 31,} 3213-3224
     (1992). \bibitem{4}  V. P. Heuring, H. F. Jordan, and J. P.
Pratt, "Bit serial optical            computer design," in {\it
Optical Computing 1988,} P. Chavel, J. W.            Goodman, and
G. Robin, eds., \pspie {\bf 963,} 346-353 (1988). \bibitem{5} V. P.
Heuring and  J. P. Pratt,  "Designing continuous dataflow
  optical computing systems, I.\ synchronization," in {\it OSA
Annual            Meeting,} Vol.\ 15 of 1990 OSA Technical Digest
Series           (Optical Society of America, Washington, D.C.,
1990), paper TuUU2. \bibitem{6}  A. F. Benner, H. F. Jordan, and V.
P. Heuring, "Digital optical            computer with optically
switched directional coupler,"  Opt.\ Eng.\             {\bf  30,}
1936-1941 (1991).
\bibitem{7}  J. P. Tremblay, P. G. Sorenson, {\it
The Theory and Practice of            Compiler Writing}
 (McGraw-Hill, New York, 1987), pp. 635-640. \bibitem{8}  C. Berge,
{\it Graphs}  (Elsevier, New York, 1985), pp. 143-152. \bibitem{9}
W. M. Waite and G. Goos, {\it Compiler Construction}
(Springer-Verlag, New York, 1987), pp. 398-399.
\bibitem{10} J. P.
Pratt, "HATCH instruction manual,"  OCS Tech. Rep.            89-31
(Optoelectronic Computing Systems Center, University of
Colorado,  Boulder, Colo., 1989).
\end{references}

\newpage
\baselineskip = .5\baselineskip  % single space the verbatim
\begin{verbatim}

\begin{references}
\bibitem{1}  J. C. Palais, {\it Fiber Optic Communications},  2nd
ed.  (Prentice-Hall, Englewood Cliffs, N.J., 1988), pp. 270-271.
\bibitem{2}  E. E. Bash and H. A. Carnes, ``Digital optical
communications  systems,''in {\it Fiber Optics,} J. Daly, ed. (CRC
          Press, Boca Raton, Fla., 1987), pp. 153-154.
\bibitem{3}
V. P. Heuring, H. F. Jordan, and J. P. Pratt, "A bit serial
 architecture for optical computing," \ao {\bf 31,} 3213-3224
     (1992). \bibitem{4}  V. P. Heuring, H. F. Jordan, and J. P.
Pratt, "Bit serial optical            computer design," in {\it
Optical Computing 1988,} P. Chavel, J. W.            Goodman, and
G. Robin, eds., \pspie {\bf 963,} 346-353 (1988). \bibitem{5} V. P.
Heuring and  J. P. Pratt,  "Designing continuous dataflow
  optical computing systems, I.\ synchronization," in {\it OSA
Annual            Meeting,} Vol.\ 15 of 1990 OSA Technical Digest
Series           (Optical Society of America, Washington, D.C.,
1990), paper TuUU2. \bibitem{6}  A. F. Benner, H. F. Jordan, and V.
P. Heuring, "Digital optical            computer with optically
switched directional coupler,"  Opt.\ Eng.\             {\bf  30,}
1936-1941 (1991).
\bibitem{7}  J. P. Tremblay, P. G. Sorenson, {\it
The Theory and Practice of            Compiler Writing}
 (McGraw-Hill, New York, 1987), pp. 635-640. \bibitem{8}  C. Berge,
{\it Graphs}  (Elsevier, New York, 1985), pp. 143-152. \bibitem{9}
W. M. Waite and G. Goos, {\it Compiler Construction}
(Springer-Verlag, New York, 1987), pp. 398-399.
\bibitem{10} J. P.
Pratt, "HATCH instruction manual,"  OCS Tech. Rep.            89-31
(Optoelectronic Computing Systems Center, University of
Colorado,  Boulder, Colo., 1989).
\end{references}

\end{verbatim}
\newpage
\baselineskip = 2\baselineskip  % double space the text

\begin{figure}
\caption{ Power fluctuations at a detection point.} \end{figure}

\begin{figure}
\caption{General device model.} \end{figure}
\begin{figure} \caption{Modeling a lithium niobate switch.}
\end{figure}
\begin{figure}
\caption{Modeling device loss and cross talk.} \end{figure}

 \begin{figure}
\caption{Optical circuit.} \end{figure}
\begin{figure} \caption{Graph model of optical circuit.}
\end{figure}

\newpage
\baselineskip = .5\baselineskip  % single space the verbatim
\begin{verbatim}

\begin{figure}
\caption{ Power fluctuations at a detection point.} \end{figure}

\begin{figure}
\caption{General device model.} \end{figure}
\begin{figure} \caption{Modeling a lithium niobate switch.}
\end{figure}
\begin{figure}
\caption{Modeling device loss and cross talk.} \end{figure}

 \begin{figure}
\caption{Optical circuit.} \end{figure}
\begin{figure} \caption{Graph model of optical circuit.}
\end{figure}

\end{verbatim}
\newpage
\baselineskip = 2\baselineskip  % double space the text


\begin{table}
\caption{Minimum Signal Powers}
\begin{tabular}{cc}
Vertex&$P_{1\rm min} $ (dBm) \\ \tableline
  1 &           0   \\                             2 &          -3
 \\                             3 &          -5   \\
              4 &          -5   \\                             5 &
        -8   \\                             6 &          -11  \\
                          7 &          -8
\end{tabular}
\end{table}

\newpage
\baselineskip = .5\baselineskip  % single space the verbatim
\begin{verbatim}
\begin{table}
\caption{Minimum Signal Powers}
\begin{tabular}{cc}
Vertex&$P_{1\rm min} $ (dBm) \\ \tableline
  1 &  0   \\  2 &  -3 \\   3 &   -5   \\
  4 & -5   \\  5 &  -8 \\   6 &   -11  \\
  7 & -8
\end{tabular}
\end{table}

\end{document}
\end{verbatim}

\end{document}
%%%%% file sample.tex %%%%%
