%These macros are used for most LBL preprints. I am not sure how widely
%they are used elsewhere.
%  This produces documents of a size suitable for sending to a journal.
%
\def\journal{\topmargin .3in	\oddsidemargin .5in
	\headheight 0pt	\headsep 0pt
	\textwidth 5.625in % 1.2 preprint size  %6.5in
	\textheight 8.25in % 1.2 preprint size 9in
	\marginparwidth 1.5in
	\parindent 2em
	\parskip .5ex plus .1ex		\jot = 1.5ex}
%
%	The default is set to be journal!
\journal
\def\baselinestretch{1.2}
\catcode`\@=11
\def\marginnote#1{}
%%%%%%%%%%%%%%%%%%%%%%%%%%%%%%%%%%%%%%%%%%%%%%%%%%%%%%%%%%%%%%%%%%%
% 	The time macros where written by Jon Yamron
%
\newcount\hour
\newcount\minute
\newtoks\amorpm
\hour=\time\divide\hour by60
\minute=\time{\multiply\hour by60 \global\advance\minute by-\hour}
\edef\standardtime{{\ifnum\hour<12 \global\amorpm={am}%
	\else\global\amorpm={pm}\advance\hour by-12 \fi
	\ifnum\hour=0 \hour=12 \fi
	\number\hour:\ifnum\minute<10 0\fi\number\minute\the\amorpm}}
\edef\militarytime{\number\hour:\ifnum\minute<10 0\fi\number\minute}
%%%%%%%%%%%%%%%%%%%%%%%%%%%%%%%%%%%%%%%%%%%%%%%%%%%%%%%%%%%%%%%%%%%%%%%
\def\draftlabel#1{{\@bsphack\if@filesw {\let\thepage\relax
   \xdef\@gtempa{\write\@auxout{\string
      \newlabel{#1}{{\@currentlabel}{\thepage}}}}}\@gtempa
   \if@nobreak \ifvmode\nobreak\fi\fi\fi\@esphack}
	\gdef\@eqnlabel{#1}}
\def\@eqnlabel{}
\def\@vacuum{}
\def\draftmarginnote#1{\marginpar{\raggedright\scriptsize\tt#1}}
%
\def\draft{\oddsidemargin -.5truein
	\def\@oddfoot{\sl preliminary draft \hfil
	\rm\thepage\hfil\sl\today\quad\militarytime}
	\let\@evenfoot\@oddfoot	\overfullrule 3pt
	\let\label=\draftlabel
	\let\marginnote=\draftmarginnote
   \def\@eqnnum{(\theequation)\rlap{\kern\marginparsep\tt\@eqnlabel}%
\global\let\@eqnlabel\@vacuum}  }
%
%	This defines the preprint style which is to be imprinted in
%	landscape mode. The command \preprint precedes the begin
%	document command.
%
\def\preprint{\twocolumn\sloppy\flushbottom\parindent 2em
	\leftmargini 2em\leftmarginv .5em\leftmarginvi .5em
	\oddsidemargin -.5in	\evensidemargin -.5in
	\columnsep .4in	\footheight 0pt
	\textwidth 10in	\topmargin  -.4in
	\headheight 12pt \topskip .4in
	\textheight 7.1in \footskip 0pt
	\def\@oddhead{\thepage\hfil\addtocounter{page}{1}\thepage}
	\let\@evenhead\@oddhead	\def\@oddfoot{}	\def\@evenfoot{} }
%
%	This sets the default for World Scientific proceedings or
%	metric size proceedings contributions.
\def\proceedings{\pagestyle{empty}
	\oddsidemargin .26in \evensidemargin .26in
	\topmargin .27in	\textwidth 145mm
	\parindent 12mm	\textheight 225mm
	\headheight 0pt	\headsep 0pt
	\footskip 0pt	\footheight 0pt}
%
%	This causes equations to be numbered by section
\def\numberbysection{\@addtoreset{equation}{section}
	\def\theequation{\thesection.\arabic{equation}}}
\def\underline#1{\relax\ifmmode\@@underline#1\else
	$\@@underline{\hbox{#1}}$\relax\fi}
%
\def\titlepage{\@restonecolfalse\if@twocolumn\@restonecoltrue\onecolumn
     \else \newpage \fi \thispagestyle{empty}\c@page\z@
	\def\thefootnote{\fnsymbol{footnote}} }
\def\endtitlepage{\if@restonecol\twocolumn \else \newpage \fi
	\def\thefootnote{\arabic{footnote}}
	\setcounter{footnote}{0}}  %\c@footnote\z@ }
%
\catcode`@=12
\relax
%
%	This defines the figure caption environment
%
\def\figcap{\section*{Figure Captions\markboth
	{FIGURECAPTIONS}{FIGURECAPTIONS}}\list
	{Figure \arabic{enumi}:\hfill}{\settowidth\labelwidth{Figure 999:}
	\leftmargin\labelwidth
	\advance\leftmargin\labelsep\usecounter{enumi}}}
\let\endfigcap\endlist \relax
\def\tablecap{\section*{Table Captions\markboth
	{TABLECAPTIONS}{TABLECAPTIONS}}\list
	{Table \arabic{enumi}:\hfill}{\settowidth\labelwidth{Table 999:}
	\leftmargin\labelwidth
	\advance\leftmargin\labelsep\usecounter{enumi}}}
\let\endtablecap\endlist \relax
\def\reflist{\section*{References\markboth
	{REFLIST}{REFLIST}}\list
	{[\arabic{enumi}]\hfill}{\settowidth\labelwidth{[999]}
	\leftmargin\labelwidth
	\advance\leftmargin\labelsep\usecounter{enumi}}}
\let\endreflist\endlist \relax
%
%
%	The publist environment is ideal for publications.
%	If functions very similar to enumerate but it accepts an
%	optional argument which sets the counter to begin at a
%	specified number.
%	The form \begin{publist} starts the counter at 1.
%	The command \end{publist} will not reset the counter.
%	One will continuously label the publication's list.
%	To reset the counter to any number such as 1 or 15 use
%	the form with the optional argument.
%	The form \begin{publist}[15] starts the counter at 15.
%
%
\makeatletter
\newcounter{pubctr}
\def\publist{\@ifnextchar[{\@publist}{\@@publist}}
\def\@publist[#1]{\list
	{[\arabic{pubctr}]\hfill}{\settowidth\labelwidth{[999]}
	\leftmargin\labelwidth
	\advance\leftmargin\labelsep
	\@nmbrlisttrue\def\@listctr{pubctr}
	\setcounter{pubctr}{#1}\addtocounter{pubctr}{-1}}}
\def\@@publist{\list
	{[\arabic{pubctr}]\hfill}{\settowidth\labelwidth{[999]}
	\leftmargin\labelwidth
	\advance\leftmargin\labelsep
	\@nmbrlisttrue\def\@listctr{pubctr}}}
\let\endpublist\endlist \relax
\makeatother
%
%this is better than \cdot
\def\mdot{\hskip -.1cm \cdot \hskip -.1cm}
%
% This lines up the \not better
\def\slash{\not\!}
%
%This redefinition of \section reduces the size of the font for the
%title of the section and eliminates the tendency to hyphenate words
%in the title
%
\catcode`\@=11
\def\section{\@startsection {section}{1}{0pt}{-3.5ex plus -1ex minus
 -.2ex}{2.3ex plus .2ex}{\raggedright\large\bf}}
\catcode`\@=12
%
%This macro (math-boldface) gives bold-face for lower case Greek letters.
%e.g. \mbf{\gamma} give lower case, bold, gamma
\def\mbf#1{\hbox{\boldmath $#1$}}
%
% these written by bob cahn
% ******************************
%	the stuff below defines \eqalign and \eqalignno in such a
%	way that they will run on Latex
\newskip\humongous \humongous=0pt plus 1000pt minus 1000pt
\def\caja{\mathsurround=0pt}
\def\eqalign#1{\,\vcenter{\openup1\jot \caja
	\ialign{\strut \hfil$\displaystyle{##}$&$
	\displaystyle{{}##}$\hfil\crcr#1\crcr}}\,}
\newif\ifdtup
\def\panorama{\global\dtuptrue \openup1\jot \caja
	\everycr{\noalign{\ifdtup \global\dtupfalse
	\vskip-\lineskiplimit \vskip\normallineskiplimit
	\else \penalty\interdisplaylinepenalty \fi}}}
\def\eqalignno#1{\panorama \tabskip=\humongous
	\halign to\displaywidth{\hfil$\displaystyle{##}$
	\tabskip=0pt&$\displaystyle{{}##}$\hfil
	\tabskip=\humongous&\llap{$##$}\tabskip=0pt
	\crcr#1\crcr}}
%	The oldref and fig macros are for formatting
%	references and figure lists at the end of the paper.
%	If you type \oldref{1}Dirac, P.A.M. you will get
%	[1] Dirac, P.A.M.
%	Same goes for \fig except you get Figure 2.1
\def\oldrefledge{\hangindent3\parindent}
\def\oldreffmt#1{\rlap{[#1]} \hbox to 2\parindent{}}
\def\oldref#1{\par\noindent\oldrefledge \oldreffmt{#1}
	\ignorespaces}
\def\figledge{\hangindent=1.25in}
\def\figfmt#1{\rlap{Figure {#1}} \hbox to 1in{}}
\def\fig#1{\par\noindent\figledge \figfmt{#1}
	\ignorespaces}
%
%
% 	This defines et al., i.e., e.g., cf., etc.
\def\ie{\hbox{\it i.e.}}	\def\etc{\hbox{\it etc.}}
\def\eg{\hbox{\it e.g.}}	\def\cf{\hbox{\it cf.}}
\def\etal{\hbox{\it et al.}}
\def\dash{\hbox{---}}
%	common physics symbols
\def\cok{\mathop{\rm cok}}
\def\tr{\mathop{\rm tr}}
\def\Tr{\mathop{\rm Tr}}
\def\Im{\mathop{\rm Im}}
\def\Re{\mathop{\rm Re}}
\def\bR{\mathop{\bf R}}
\def\bC{\mathop{\bf C}}
\def\lie{\hbox{\it \$}}	% fancy L for the Lie derivative
\def\partder#1#2{{\partial #1\over\partial #2}}
\def\secder#1#2#3{{\partial^2 #1\over\partial #2 \partial #3}}
\def\bra#1{\left\langle #1\right|}
\def\ket#1{\left| #1\right\rangle}
\def\VEV#1{\left\langle #1\right\rangle}
\let\vev\VEV
\def\gdot#1{\rlap{$#1$}/}
\def\abs#1{\left| #1\right|}
\def\pri#1{#1^\prime}
\def\ltap{\raisebox{-.4ex}{\rlap{$\sim$}} \raisebox{.4ex}{$<$}}
\def\gtap{\raisebox{-.4ex}{\rlap{$\sim$}} \raisebox{.4ex}{$>$}}
% \contract is a differential geometry contraction sign _|
\def\contract{\makebox[1.2em][c]{
	\mbox{\rule{.6em}{.01truein}\rule{.01truein}{.6em}}}}
\def\half{{1\over 2}}
\def\beq{\begin{equation}}
\def\eeq{\end{equation}}
\def\ul{\underline}
\def\bea{\begin{eqnarray}}
% double-headed superior arrow added 9.2.86
\def\lrover#1{
	\raisebox{1.3ex}{\rlap{$\leftrightarrow$}} \raisebox{ 0ex}{$#1$}}
%
% commutator added 11.14.86
\def\com#1#2{
	\left[#1, #2\right]}
%
\def\eea{\end{eqnarray}}
%these written by orlando alvarez
% ************************************************************
%	The following macros were written by Chris Quigg.
%	They create bent arrows and can be used to write
%	decays such as pi --> mu + nu
%		 	       --> e nu nubar
%
\def\bentarrow{\:\raisebox{1.3ex}{\rlap{$\vert$}}\!\rightarrow}
\def\longbent{\:\raisebox{3.5ex}{\rlap{$\vert$}}\raisebox{1.3ex}%
	{\rlap{$\vert$}}\!\rightarrow}
\def\onedk#1#2{
	\begin{equation}
	\begin{array}{l}
	 #1 \\
	 \bentarrow #2
	\end{array}
	\end{equation}
		}
\def\dk#1#2#3{
	\begin{equation}
	\begin{array}{r c l}
	#1 & \rightarrow & #2 \\
	 & & \bentarrow #3
	\end{array}
	\end{equation}
		}
\def\dkp#1#2#3#4{
	\begin{equation}
	\begin{array}{r c l}
	#1 & \rightarrow & #2#3 \\
	 & & \phantom{\; #2}\bentarrow #4
	\end{array}
	\end{equation}
		}
\def\bothdk#1#2#3#4#5{
	\begin{equation}
	\begin{array}{r c l}
	#1 & \rightarrow & #2#3 \\
	 & & \:\raisebox{1.3ex}{\rlap{$\vert$}}\raisebox{-0.5ex}{$\vert$}%
	\phantom{#2}\!\bentarrow #4 \\
	 & & \bentarrow #5
	\end{array}
	\end{equation}
		}
%
%	End of Quigg macros
\hyphenation{anom-a-ly}
\hyphenation{comp-act-ifica-tion}
%
%abbreviated journal names
%
\def\ap#1#2#3{           {\it Ann. Phys. (NY) }{\bf #1}, #2 (19#3)}
\def\apj#1#2#3{          {\it Astrophys. J. }{\bf #1}, #2 (19#3)}
\def\apjl#1#2#3{         {\it Astrophys. J. Lett. }{\bf #1}, #2 (19#3)}
\def\app#1#2#3{          {\it Acta Phys. Polon. }{\bf #1}, #2 (19#3)}
\def\ar#1#2#3{     {\it Ann. Rev. Nucl. and Part. Sci. }{\bf #1}, #2 (19#3)}
\def\com#1#2#3{          {\it Comm. Math. Phys. }{\bf #1}, #2 (19#3)}
\def\ib#1#2#3{           {\it ibid. }{\bf #1}, #2 (19#3)}
\def\nat#1#2#3{          {\it Nature (London) }{\bf #1}, #2 (19#3)}
\def\nc#1#2#3{           {\it Nuovo Cim.  }{\bf #1}, #2 (19#3)}
\def\np#1#2#3{           {\it Nucl. Phys. }{\bf #1}, #2 (19#3)}
\def\pl#1#2#3{           {\it Phys. Lett. }{\bf #1}, #2 (19#3)}
\def\pr#1#2#3{           {\it Phys. Rev. }{\bf #1}, #2 (19#3)}
\def\prep#1#2#3{         {\it Phys. Rep. }{\bf #1}, #2 (19#3)}
\def\prl#1#2#3{          {\it Phys. Rev. Lett. }{\bf #1}, #2 (19#3)}
\def\pro#1#2#3{          {\it Prog. Theor. Phys. }{\bf #1}, #2 (19#3)}
\def\rmp#1#2#3{          {\it Rev. Mod. Phys. }{\bf #1}, #2 (19#3)}
\def\sp#1#2#3{           {\it Sov. Phys.-Usp. }{\bf #1}, #2 (19#3)}
\def\sjn#1#2#3{           {\it Sov. J. Nucl. Phys. }{#1}, #2 (19#3)}
\def\srv#1#2#3{           {\it Surv. High Energy Phys. }{#1}, #2 (19#3)}
\def\tp{these proceedings}
\def\zp#1#2#3{           {\it Zeit. fur Physik }{\bf #1}, #2 (19#3)}
%
% These are modifications to the eqnarray* environment that allows equation
% numbers to be inserted by hand.  The format is
%
%	\begin{eqnarray*}
%		x & = & y \eqno{(2.1a)} \\
%		y & = & z               \\
%		  & = & x \eqno{(2.1b)} \\
%		z & = & x \eqno{(2.1c)}
%	\end{eqnarray*}
%
% This produces
%			x = y                              (2.1a)
%			y = z
%			  = x                              (2.1b)
%			z = x                              (2.1c)
%
% You need not put an equation number on every line.  The argument of \eqno
% must be enclosed in braces.
\catcode`\@=11
\def\eqnarray{\stepcounter{equation}\let\@currentlabel=\theequation
\global\@eqnswtrue
\global\@eqcnt\z@\tabskip\@centering\let\\=\@eqncr
\gdef\@@fix{}\def\eqno##1{\gdef\@@fix{##1}}%
$$\halign to \displaywidth\bgroup\@eqnsel\hskip\@centering
  $\displaystyle\tabskip\z@{##}$&\global\@eqcnt\@ne
  \hskip 2\arraycolsep \hfil${##}$\hfil
  &\global\@eqcnt\tw@ \hskip 2\arraycolsep $\displaystyle\tabskip\z@{##}$\hfil
   \tabskip\@centering&\llap{##}\tabskip\z@\cr}

\def\@@eqncr{\let\@tempa\relax
    \ifcase\@eqcnt \def\@tempa{& & &}\or \def\@tempa{& &}
      \else \def\@tempa{&}\fi
     \@tempa \if@eqnsw\@eqnnum\stepcounter{equation}\else\@@fix\gdef\@@fix{}\fi
     \global\@eqnswtrue\global\@eqcnt\z@\cr}

\catcode`\@=12
%
% below defines boldface greek
%
% DEFINE A NEW FAMILY: SEE THE TeXBook, Exercise 17.20 and its answer
\font\tenbifull=cmmib10 % bold math italic
\font\tenbimed=cmmib10 scaled 800
\font\tenbismall=cmmib10 scaled 666
\textfont9=\tenbifull \scriptfont9=\tenbimed
\scriptscriptfont9=\tenbismall
\def\bmit{\fam9 }
% The bold versions of the lower-case Greek letters.
\def\boldalpha{\fam=9{\mathchar"710B } }
\def\boldbeta{\fam=9{\mathchar"710C } }
\def\boldgamma{\fam=9{\mathchar"710D } }
\def\bolddelta{\fam=9{\mathchar"710E } }
\def\boldepsilon{\fam=9{\mathchar"710F } }
\def\boldzeta{\fam=9{\mathchar"7110 } }
\def\boldeta{\fam=9{\mathchar"7111 } }
\def\boldtheta{\fam=9{\mathchar"7112 } }
\def\boldiota{\fam=9{\mathchar"7113 } }
\def\boldkappa{\fam=9{\mathchar"7114 } }
\def\boldlambda{\fam=9{\mathchar"7115 } }
\def\boldmu{\fam=9{\mathchar"7116 } }
\def\boldnu{\fam=9{\mathchar"7117 } }
\def\boldomicron{\fam=9{\mathchar"716F } }
\def\boldxi{\fam=9{\mathchar"7118 } }
\def\boldpi{\fam=9{\mathchar"7119 } }
\def\boldrho{\fam=9{\mathchar"711A } }
\def\boldsigma{\fam=9{\mathchar"711B } }
\def\boldtau{\fam=9{\mathchar"711C } }
\def\boldupsilon{\fam=9{\mathchar"711D } }
\def\boldphi{\fam=9{\mathchar"711E } }
\def\boldchi{\fam=9{\mathchar"711F } }
\def\boldpsi{\fam=9{\mathchar"7120 } }
\def\boldomega{\fam=9{\mathchar"7121 } }
\def\boldvarepsilon{\fam=9{\mathchar"7122 } }
\def\boldvartheta{\fam=9{\mathchar"7123 } }
\def\boldvarpi{\fam=9{\mathchar"7124 } }
\def\boldvarrho{\fam=9{\mathchar"7125 } }
\def\boldvarsigma{\fam=9{\mathchar"7126 } }
\def\boldvarphi{\fam=9{\mathchar"7127 } }
% DEFINE BOLD VERSIONS OF THE UPPER-CASE GREEK LETTERS:
% We need to use Family 6 (\bffam)
\def\boldGamma{\fam=6{\mathchar"7000 } }
\def\boldDelta{\fam=6{\mathchar"7001 } }
\def\boldTheta{\fam=6{\mathchar"7002 } }
\def\boldLambda{\fam=6{\mathchar"7003 } }
\def\boldXi{\fam=6{\mathchar"7004 } }
\def\boldPi{\fam=6{\mathchar"7005 } }
\def\boldSigma{\fam=6{\mathchar"7006 } }
\def\boldUpsilon{\fam=6{\mathchar"7007 } }
\def\boldPhi{\fam=6{\mathchar"7008 } }
\def\boldPsi{\fam=6{\mathchar"7009 } }
\def\boldOmega{\fam=6{\mathchar"700A } }
% Bold SLANTED uppercase Greek letters:
\def\boldmitOmega{\fam=9{\mathchar"700A } }
\def\boldmitGamma{\fam=9{\mathchar"7000 } }
\def\boldmitDelta{\fam=9{\mathchar"7001 } }
\def\boldmitTheta{\fam=9{\mathchar"7002 } }
\def\boldmitLambda{\fam=9{\mathchar"7003 } }
\def\boldmitXi{\fam=9{\mathchar"7004 } }
\def\boldmitPi{\fam=9{\mathchar"7005 } }
\def\boldmitSigma{\fam=9{\mathchar"7006 } }
\def\boldmitUpsilon{\fam=9{\mathchar"7007 } }
\def\boldmitPhi{\fam=9{\mathchar"7008 } }
\def\boldmitPsi{\fam=9{\mathchar"7009 } }
\def\boldmitOmega{\fam=9{\mathchar"700A } }


\relax

% double.tex - from Jackie Damrau, TUGBoat, Volume 11 (1990),
% No. 1, page 85
% Switch to doublespacing
\def\double{
        \renewcommand{\baselinestretch}{2}
        \large
        \normalsize
        }

% single.tex - from Jackie Damrau, TUGBoat, Volume 11 (1990),
% No. 1, page 85
% Switch to singlespacing
\def\single {
                \renewcommand{\baselinestretch}{1}
                \large
                \normalsize
                }

