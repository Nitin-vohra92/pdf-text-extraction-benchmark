%%%%%%%%%%%%%%%%%%%%%%%%%%%%%%%%%%%%%%%%%%
% djnlx.tex
%
% version as supplied by Arovas 4/15/88
%
%%%%%%%%%%%%%%%%%%%%%%%%%%%%%%%%%%%%%%%%%%
%%                             DJNLX.TEX
%%
%%      This is an extended version of JNL.TEX Version 0.3 (as of 6/12/85).
%%
%%     This is a set of TeX 82 macros designed to produce scientific
%%     papers with a minimum of fuss and using as much of plain.tex as
%%     possible.  The user need only know what is in the TeXbook, and
%%     the macros under ``user definitions'' below.  Also, the user
%%     definitions are intended to be as simple as possible, so that
%%     the user may change them as desired.
%%
%%
%%  Font definitions suitable for the IMAGEN (Written by Tony Kennedy).
%%
%%  Boldface mathematic italic (\mib) is included. Sans serif, boldface
%%  sans serif, sans serif italic, "Q"-sans serif, "Q"-sans serif italic,
%%  "MC"-sans serif, Dunhill style, and "capitals and small capitals" style
%%  fonts can be included from external files by a single control sequence.
%%  Have a look at the JNLX.DOC file for further information and at the
%%  JNLXAMP.DVI file for examples.
%%
%%  (by Ulrich Kettler 5/8/86)(last change 6/10/86).
%%  (eleven point font by Chris Stanton)
%% Define a whole menagerie of pseudo-12pt fonts
%%
  \font\twelverm=cmr10 scaled 1200       \font\twelvei=cmmi10 scaled 1200
  \font\twelvesy=cmsy10 scaled 1200      \font\twelveex=cmex10 scaled 1200
  \font\twelvebf=cmbx10 scaled 1200      \font\twelvesl=cmsl10 scaled 1200
  \font\twelvett=cmtt10 scaled 1200      \font\twelveit=cmti10 scaled 1200
  \font\twelvemib=cmmib10 scaled 1200
  \font\elevenmib=cmmib10 scaled 1095
  \font\tenmib=cmmib10
  \font\eightmib=cmmib10 scaled 800
  \font\sixmib=cmmib10 scaled 667
%  Define a whole menagerie of pseudo-11pt fonts
\font\elevenrm=cmr10 scaled 1095    \font\eleveni=cmmi10 scaled 1095
\font\elevensy=cmsy10 scaled 1095   \font\elevenex=cmex10 scaled 1095
\font\elevenbf=cmbx10 scaled 1095   \font\elevensl=cmsl10 scaled 1095
\font\eleventt=cmtt10 scaled 1095   \font\elevenit=cmti10 scaled 1095
%
% other fonts
%
\font\seventeenrm=cmr10 scaled \magstep3
\font\seventeenbf=cmbx10 scaled \magstep3
\font\seventeeni=cmmi10 scaled \magstep3
\font\seventeenex=cmex10 scaled \magstep3
\font\seventeensl=cmsl10 scaled \magstep3
\font\seventeensy=cmsy10 scaled \magstep3
\font\seventeenit=cmti10 scaled \magstep3
\font\seventeenmib=cmmib10 scaled \magstep3
\font\fourteenrm=cmr10 scaled\magstep2
\font\fourteenbf=cmbx10 scaled \magstep2
\font\hugestrm=cmr10 scaled \magstep5
\font\hugestbf=cmbx10 scaled \magstep5
\font\tencp=cmcsc10
\font\twelvecp=cmcsc10 scaled \magstep1
\font\seventeencp=cmcsc10 scaled \magstep3
\newfam\cpfam%


%  Define a whole menagerie of pseudo-8pt fonts

\font\eightrm=cmr8 \font\eighti=cmmi8
\font\eightsy=cmsy8 \font\eightbf=cmbx8

%  Define a whole menagerie of pseudo-6pt fonts

\font\sixrm=cmr6 \font\sixi=cmmi6
\font\sixsy=cmsy6 \font\sixbf=cmbx6

\skewchar\eleveni='177   \skewchar\elevensy='60
\skewchar\elevenmib='177  \skewchar\seventeensy='60
\skewchar\seventeenmib='177
\skewchar\seventeeni='177

\newfam\mibfam%

%  elevenpoint

\def\elevenpoint{\normalbaselineskip=12.2pt
  \abovedisplayskip 12.2pt plus 3pt minus 9pt
  \belowdisplayskip 12.2pt plus 3pt minus 9pt
  \abovedisplayshortskip 0pt plus 3pt
  \belowdisplayshortskip 7.1pt plus 3pt minus 4pt
  \smallskipamount=3.3pt plus1.1pt minus1.1pt
  \medskipamount=6.6pt plus2.2pt minus2.2pt
  \bigskipamount=13.3pt plus4.4pt minus4.4pt
  \def\rm{\fam0\elevenrm}          \def\it{\fam\itfam\elevenit}%
  \def\sl{\fam\slfam\elevensl}     \def\bf{\fam\bffam\elevenbf}%
  \def\mit{\fam 1}                 \def\cal{\fam 2}%
  \def\tt{\eleventt}
  \def\mib{\fam\mibfam\elevenmib}%
  \textfont0=\elevenrm   \scriptfont0=\eightrm   \scriptscriptfont0=\sixrm
  \textfont1=\eleveni    \scriptfont1=\eighti    \scriptscriptfont1=\sixi
  \textfont2=\elevensy   \scriptfont2=\eightsy   \scriptscriptfont2=\sixsy
  \textfont3=\elevenex   \scriptfont3=\elevenex  \scriptscriptfont3=\elevenex
  \textfont\itfam=\elevenit
  \textfont\slfam=\elevensl
  \textfont\bffam=\elevenbf \scriptfont\bffam=\eightbf
  \scriptscriptfont\bffam=\sixbf
  \textfont\mibfam=\elevenmib
  \scriptfont\mibfam=\eightmib
  \scriptscriptfont\mibfam=\sixmib
  \def\xrm{\textfont0=\elevenrm\scriptfont0=\eightrm
      \scriptscriptfont0=\sixrm}
  \normalbaselines\rm}

  \skewchar\twelvei='177   \skewchar\twelvesy='60
  \skewchar\twelvemib='177
%
%  twelvepoint
%
\def\twelvepoint{\normalbaselineskip=12.4pt
  \abovedisplayskip 12.4pt plus 3pt minus 9pt
  \belowdisplayskip 12.4pt plus 3pt minus 9pt
  \abovedisplayshortskip 0pt plus 3pt
  \belowdisplayshortskip 7.2pt plus 3pt minus 4pt
  \smallskipamount=3.6pt plus 1.2pt minus 1.2pt
  \medskipamount=7.2pt plus 2.4pt minus 2.4pt
  \bigskipamount=14.4pt plus 4.8pt minus 4.8pt
  \def\rm{\fam0\twelverm}          \def\it{\fam\itfam\twelveit}%
  \def\sl{\fam\slfam\twelvesl}     \def\bf{\fam\bffam\twelvebf}%
  \def\mit{\fam 1}                 \def\cal{\fam 2}%
  \def\tt{\twelvett}%
  \def\mib{\fam\mibfam\twelvemib}%

  \textfont0=\twelverm   \scriptfont0=\tenrm     \scriptscriptfont0=\sevenrm
  \textfont1=\twelvei    \scriptfont1=\teni      \scriptscriptfont1=\seveni
  \textfont2=\twelvesy   \scriptfont2=\tensy     \scriptscriptfont2=\sevensy
  \textfont3=\twelveex   \scriptfont3=\twelveex  \scriptscriptfont3=\twelveex
  \textfont\itfam=\twelveit
  \textfont\slfam=\twelvesl
  \textfont\bffam=\twelvebf
  \textfont\mibfam=\twelvemib       \scriptfont\mibfam=\tenmib
                                             \scriptscriptfont\mibfam=\eightmib


  \def\xrm{\textfont0=\twelverm\scriptfont0=\tenrm
      \scriptscriptfont0=\sevenrm\rm}
\normalbaselines\rm}

%      tenpoint

\def\tenpoint{\normalbaselineskip=12pt
  \abovedisplayskip 12pt plus 3pt minus 9pt
  \belowdisplayskip 12pt plus 3pt minus 9pt
  \abovedisplayshortskip 0pt plus 3pt
  \belowdisplayshortskip 7pt plus 3pt minus 4pt
  \smallskipamount=3pt plus1pt minus1pt
  \medskipamount=6pt plus2pt minus2pt
  \bigskipamount=12pt plus4pt minus4pt
  \def\rm{\fam0\tenrm}            \def\it{\fam\itfam\tenit}%
  \def\sl{\fam\slfam\tensl}       \def\bf{\fam\bffam\tenbf}%
  \def\mit{\fam 1}                \def\tt{\fam\ttfam\tentt}%
  \def\cal{\fam 2}               \def\mib{\fam\mibfam\tenmib}%

  \textfont0=\tenrm   \scriptfont0=\sevenrm   \scriptscriptfont0=\fiverm
  \textfont1=\teni    \scriptfont1=\seveni    \scriptscriptfont1=\fivei
  \textfont2=\tensy   \scriptfont2=\sevensy   \scriptscriptfont2=\fivesy
  \textfont3=\tenex   \scriptfont3=\tenex     \scriptscriptfont3=\tenex
  \textfont\itfam=\tenit        \textfont\slfam=\tensl
  \textfont\bffam=\tenbf        \scriptfont\bffam=\sevenbf
                                       \scriptscriptfont\bffam=\fivebf
  \textfont\mibfam=\tenmib      \scriptfont\mibfam=\eightmib
                                       \scriptscriptfont\mibfam=\sixmib

  \def\xrm{\textfont0=\tenrm\scriptfont0=\sevenrm
       \scriptscriptfont0=\fiverm\rm}
\normalbaselines\rm}

%%
%%     Change of the internal codes for lowercase Greek letters
%%     in order to use {\mib\alpha}... for boldface "alpha"... .
%%
\mathchardef\alpha="710B
\mathchardef\beta="710C
\mathchardef\gamma="710D
\mathchardef\delta="710E
\mathchardef\epsilon="710F
\mathchardef\zeta="7110
\mathchardef\eta="7111
\mathchardef\theta="7112
\mathchardef\kappa="7114
\mathchardef\lambda="7115
\mathchardef\mu="7116
\mathchardef\nu="7117
\mathchardef\xi="7118
\mathchardef\pi="7119
\mathchardef\rho="711A
\mathchardef\sigma="711B
\mathchardef\tau="711C
\mathchardef\phi="711E
\mathchardef\chi="711F
\mathchardef\psi="7120
\mathchardef\omega="7121
\mathchardef\varepsilon="7122
\mathchardef\vartheta="7123
\mathchardef\varrho="7125
\mathchardef\varphi="7127

%%     Change of the internal codes of the uppercase Greek letters, in order
%%     to get mathematic italic "Gamma", by typing "\Gamma", and
%%     to get roman "Gamma", by typing "{\rm\Gamma}".
%%     This is of advantage in physics, because letters which are
%%     representing physical quantities are ALWAYS printed italic.
%%
\def\physgreek{
\mathchardef\Gamma="7100
\mathchardef\Delta="7101
\mathchardef\Theta="7102
\mathchardef\Lambda="7103
\mathchardef\Xi="7104
\mathchardef\Pi="7105
\mathchardef\Sigma="7106
\mathchardef\Upsilon="7107
\mathchardef\Phi="7108
\mathchardef\Psi="7109
\mathchardef\Omega="710A}

%%     Various internal macros
%%

\def\beginlinemode{\endmode
  \begingroup\parskip=0pt \obeylines\def\\{\par}\def\endmode{\par\endgroup}}
\def\beginparmode{\endmode
  \begingroup \def\endmode{\par\endgroup}}
\let\endmode=\par
{\obeylines\gdef\
{}}
\def\singlespace{\baselineskip=\normalbaselineskip}
\def\oneandathirdspace{\baselineskip=\normalbaselineskip
  \multiply\baselineskip by 4 \divide\baselineskip by 3}
\def\oneandahalfspace{\baselineskip=\normalbaselineskip
  \multiply\baselineskip by 3 \divide\baselineskip by 2}
\def\doublespace{\baselineskip=\normalbaselineskip \multiply\baselineskip by 2}
\def\triplespace{\baselineskip=\normalbaselineskip \multiply\baselineskip by 3}
\nopagenumbers
\newcount\firstpageno
\firstpageno=2
\headline={\ifnum\pageno<\firstpageno{\hfil}\else{\hfil\elevenrm\folio\hfil}\fi}
\let\rawfootnote=\footnote             % We must set the footnote style
%%%%%%%%%%%%%%%%%%%%%%%%%%%%%%%%%%%%%%%%%%%%%%%%%%%%%%%%%%%%%%%
%\def\footnote#1#2{{\oneandahalfspace\parindent=0pt
%\rawfootnote{#1}{#2}}}
%%%%%%%%%%%%%%%%%%%%%%%%%%%%%%%%%%%%%%%%%%%%%%%%%%%%
\def\footnote#1#2{{\singlespace\parindent=0pt
\rawfootnote{#1}{#2}}}
%
\def\raggedcenter{\leftskip=4em plus 12em \rightskip=\leftskip
  \parindent=0pt \parfillskip=0pt \spaceskip=.3333em \xspaceskip=.5em
  \pretolerance=9999 \tolerance=9999
  \hyphenpenalty=9999 \exhyphenpenalty=9999 }
\def\dateline{\rightline{\ifcase\month\or
  January\or February\or March\or April\or May\or June\or
  July\or August\or September\or October\or November\or December\fi
  \space\number\year}}
\def\received{\vskip 3pt plus 0.2fill
 \centerline{\sl (Received\space\ifcase\month\or
  January\or February\or March\or April\or May\or June\or
  July\or August\or September\or October\or November\or December\fi
  \qquad, \number\year)}}

%%
%%     Page layout, margins, font and spacing (feel free to change)
%%

%\hsize=5.9truein   %for Physica Scripta
\hsize=6.5truein
%\hoffset=0.6truein  %for Physica Scripta
\hoffset=0.0truein
\vsize=8.9truein
\voffset=0truein
\hfuzz=0.1pt
\vfuzz=0.1pt
\parskip=\medskipamount
\overfullrule=0pt      % delete the nasty little black boxes for overfull box

%%
%%     The user definitions for major parts of a paper (feel free to change)
%%

\def\preprintno#1{
 \rightline{\rm #1}}   % Preprint number at upper right of title page

\def\title                     %  Title on title page
  {\null\vskip 3pt plus 0.1fill
   \beginlinemode \doublespace \raggedcenter \bf}

\def\author                    %  Author(s) name(s)  on title page
  {\vskip 6pt plus 0.2fill \beginlinemode
   \singlespace \raggedcenter}

\def\affil        % Affiliations (can intermix with \author)
  {\vskip 6pt plus 0.1fill \beginlinemode
   \oneandahalfspace \raggedcenter \it}

\def\abstract                  % Begin abstract
  {\vskip 6pt plus 0.3fill \beginparmode
   \doublespace \narrower }%%%%%smg change 8/89%%%ABSTRACT: }

\def\summary                   % same as abstract
  {\vskip 3pt plus 0.3fill \beginparmode
   \doublespace \narrower SUMMARY: }

\def\pacs#1
  {\vskip 3pt plus 0.2fill PACS numbers: #1}

\def\endtitlepage              % End title page, begin body of paper
  {\endpage                    %       This subsumes \body
   \body}

\def\body                      % Begin text body;  can be used to end
  {\beginparmode}              % \title, \author, \affil, \abstract,
                               % \reference, or \figurecaption modes

\def\head#1{                   % Head;  NOTE enclose the text in {}
  \filbreak\vskip 0.5truein    %  e.g., \head{I. Introduction}
  {\immediate\write16{#1}
   \raggedcenter \uppercase{#1}\par}
   \nobreak\vskip 0.25truein\nobreak}

\def\subhead#1{
     \vskip 0.25truein
      \leftline{\undertext{\bf{#1}}}
       \nobreak}

\def\subsubhead#1{
     \vskip 0.20truein
      \leftline{\undertext{\sl{#1}}}
       \nobreak}

\def\subsubsubhead#1{
 \vskip 0.20truein
  {\it {#1} }
   \nobreak}

%%%%%%%%%%%%%%%%%%%%%%%%%%%%%%%%%%%%%%%%%%%%%%%%%%%%%%%%%%%%%%%%%%%%%%%%%%%
%
% SMG ADDITIONS
%
\def\lefthead#1{                   % Head;  NOTE enclose the text in {}
  \vskip 0.5truein                 %  e.g., \head{I. Introduction}
  {\immediate\write16{#1}
   \leftline  {\uppercase{\bf #1}}\par}
   \nobreak\vskip 0.25truein\nobreak}
%
%%%%%%  SMG SMG   works if put after input smgjnl but before input
%%%%%%  reforder
\def\inlinerefs{
  \gdef\refto##1{ [##1]}                % Reference in text []
\gdef\refis##1{\indent\hbox to 0pt{\hss##1.~}} % Ref list numbers.  %SMG
\gdef\journal##1, ##2, ##3, 1##4##5##6{ % Journal reference.  Comma sets
    {\sl ##1~}{\bf ##2}, ##3 (1##4##5##6)}}    % off: name, vol, page, year
%
\def\keywords#1
  {\vskip 3pt plus 0.2fill Keywords: #1}
%
\gdef\figis#1{\indent\hbox to 0pt{\hss#1.~}} %figure numbers set up
%

\def\figurecaptions     %%% Begin references -- basic format is Phys Rev
  {\head{Figure Captions}    % HEAD will capitalize this
   \beginparmode
   \interlinepenalty=10000%   added by SMG  8/29/88 %%%%%%%%%%%
   \frenchspacing \parindent=0pt \leftskip=1truecm
   \parskip=8pt plus 3pt \everypar{\hangindent=\parindent}}

%
%%%%%%%%%%%%%%%%%%%%%%%%%%%%%%%%%%%%%%%%%%%%%%%%%%%%%%%%%%%%%%%%%%%%%%%%%%%%
%
%\def\refto#1{[#1]}          % For references in text INLINE  SMG
%above superseded hopefully by \def inlinerefs.
\def\refto#1{$^{#1}$}          % For references in text as superscript

%smg fixed a possible bug s Rev ran off onto the line below where it
%     should have
\def\references       % Begin references -- basic format is Phys Rev
  {\head{References}           % I.e., volume, page, year (space after commas).
   \beginparmode
   \frenchspacing \parindent=0pt \leftskip=1truecm
   \interlinepenalty=10000%%%%%          added by smg 8/29/88 %%%%%%%
   \parskip=8pt plus 3pt \everypar{\hangindent=\parindent}}

\gdef\refis#1{\indent\hbox to 0pt{\hss#1.~}} % Ref list numbers.  %SMG
%\gdef\refis#1{\indent\hbox to 0pt{\hss[#1]~}} % Ref list numbers.

\gdef\journal#1, #2, #3, 1#4#5#6{              % Journal reference.  Comma sets
    {\sl #1~}{\bf #2}, #3 (1#4#5#6)}          % off: name, vol, page, year

\def\refstylenp{               % Nucl Phys(or Phys Lett) ref style: V, Y, P
  \gdef\refto##1{ [##1]}                               % Reference in text []
  \gdef\refis##1{\indent\hbox to 0pt{\hss##1)~}}      % Ref list numbers)
  \gdef\journal##1, ##2, ##3, ##4 {                    % Journal reference
     {\sl ##1~}{\bf ##2~}(##3) ##4 }}

\def\refstyleprnp{             % Input like pr, output like np!!
  \gdef\refto##1{ [##1]}                               % Reference in text []
  \gdef\refis##1{\indent\hbox to 0pt{\hss##1)~}}      % Ref list numbers)
  \gdef\journal##1, ##2, ##3, 1##4##5##6{              % Journal reference
    {\sl ##1~}{\bf ##2~}(1##4##5##6) ##3}}

\def\pr{\journal Phys. Rev., }

\def\pra{\journal Phys. Rev. A, }

\def\prb{\journal Phys. Rev. B, }

\def\prc{\journal Phys. Rev. C, }

\def\prd{\journal Phys. Rev. D, }

\def\prl{\journal Phys. Rev. Lett., }

\def\jmp{\journal J. Math. Phys., }

\def\revmp{\journal Rev. Mod. Phys., }

\def\cmp{\journal Comm. Math. Phys., }

\def\np{\journal Nucl. Phys., }

\def\pla{\journal Phys. Lett. A, }

\def\phyl{\journal Phys. Lett., }

\def\cpl{\jounal Chem. Phys. Lett., }

\def\annp{\journal Ann. Phys. (N.Y.), }

\def\masc{\journal Math. Scand., }

\def\dmj{\journal Duke Math J., }

\def\nuovc{\journal Il Nuovo Cimento, }

\def\jpc{\journal J. Phys. C: Solid State Phys., }

\def\prsl{\journal Proc. Roy. Soc. London, }

\def\jpsj{\journal J. Phys. Soc. Japan., }

\def\cpam{\journal Comm. Pure and App. Math., }

\def\bullam{\journal Bull. Am. Phys. Soc., }

\def\helvacta{\journal Helv. Phys. Acta, }

\def\europhys{\journal Euro. Phys. Lett., }

\def\jlowt{\journal J. Low Temp. Phys., }

\def\jpcm{\journal J. Phys.: Condensed Matter, }

\def\endreferences{\body}

%
% superseded by SMG additions
%
%\def\figurecaptions            % Begin figure captions
%  {\endpage
%   \beginparmode
%   \head{Figure Captions}
%  \parskip=24pt plus 3pt \everypar={\hangindent=4em}
%}

\def\endfigurecaptions{\body}

\def\endpage                   %  Eject a page
  {\vfill\eject}

\def\endpaper                  %  Ways to say goodbye
  {\endmode\vfill\supereject}
\def\endjnlx
  {\endpaper}
\def\endit
  {\endpaper\end}



%%
%%     Various little user definitions
%%

\def\ref#1{Ref.[#1]}                   %       for inline references
\def\Ref#1{Ref.[#1]}                   %       ditto
\def\Refs#1{Refs.[#1]}                 %       ditto
\def\tab#1{Tab.[#1]}                   %       ditto
\def\Tab#1{Tab.[#1]}                   %       ditto
\def\fig#1{Fig.[#1]}
\def\Fig#1{Fig.[#1]}
\def\figs#1{Figs.[#1]}
\def\Figs#1{Figs.[#1]}
\def\Equation#1{Equation [#1]}         % For citation of equation numbers
\def\Equations#1{Equations [#1]}       %       ditto
\def\Eq#1{Eq. (#1)}                     %       ditto
\def\eq#1{Eq. (#1)}                     %       ditto
\def\Eqs#1{Eqs. (#1)}                   %       ditto
\def\eqs#1{Eqs. (#1)}                   %       ditto
\def\frac#1#2{{\textstyle{{\strut #1} \over{\strut #2}}}}
\def\half{{\textstyle{ 1\over 2}}}
\let\ts=\thinspace
\def\eg{{\it e.g.\/}}
\def\Eg{{\it E.g.\/}}
\def\ie{{\it i.e.\/}\ }
\def\Ie{{\it I.e.}}
\def\etal{{\it et al.\/}}
\def\etc{{\it etc.\/}}
\def\sla{\raise.15ex\hbox{$/$}\kern-.57em}
\def\leaderfill{\leaders\hbox to 1em{\hss.\hss}\hfill}
\def\twiddle{\lower.9ex\rlap{$\kern-.1em\scriptstyle\sim$}}
\def\bigtwiddle{\lower1.ex\rlap{$\sim$}}
\def\gtwid{\mathrel{\raise.3ex\hbox{$>$\kern-.75em\lower1ex\hbox{$\sim$}}}}
\def\ltwid{\mathrel{\raise.3ex\hbox{$<$\kern-.75em\lower1ex\hbox{$\sim$}}}}
\def\square{\kern1pt\vbox{\hrule height 1.2pt\hbox{\vrule width 1.2pt\hskip 3pt
   \vbox{\vskip 6pt}\hskip 3pt\vrule width 0.6pt}\hrule height 0.6pt}\kern1pt}
\def\ucsb{Department of Physics\\University of California\\
Santa Barbara CA 93106}
\def\ucsd{Department of Physics\\University of California\\
La Jolla, CA 92093}
%%%%%%%%%%%%%%%%%%%%%%%%%%%%%%%%%%%%%%%%%%%%%%%%%%%%%%
%
\def\IU{Department of Physics\\Indiana University\\
Bloomington, IN 47405}
%
%%%%%%%%%%%%%%%%%%%%%%%%%%%%%%%%%%%%%%%%%%%%%%%%%%
\def\begintable{\offinterlineskip\hrule}
\def\tablespace{height 2pt&\omit&&\omit&&\omit&\cr}
\def\tablerule{\tablespace\noalign{\hrule}\tablespace}
\def\endtable{\hrule}
\def\prim{{\scriptscriptstyle{\prime}}}


%%%%%%%%%%%%%%%%%%%%%%%%%%%%%%%%%%%%%%%%%%
%%%%%%%%%%%%
% smgdefs.tex
%
% 5/17/88 modifications of Daniel's defs.
%%%%%%%%%%%%%%%%%%%%%%%%%%%%%%%%%%%%%%%%%%%%%
\def\nsf{DMR-8802383}

%%%%%%%%%%%%%%%%%%%%%%%%%%%%%%%%%%%%%%%%%%
%\input /usr/rm064/smg/suntex/DJNLX/smgjnl.tex
%
\physgreek
%
%commented out because duplicated in djnlx and smgjnl:
%%%%%%\def\refto#1{$^{#1}$}
\def\sss#1{{\scriptscriptstyle #1}}
\def\ss#1{{\scriptstyle #1}}
\def\ssr#1{{\sss{\rm #1}}}
\def\lala{\langle\!\langle}
\def\rara{\rangle\!\rangle}
\def\intl{\int\limits}
\def\fnb{\tenpoint\vskip -0.82cm\oneandahalfspace\noindent}
\def\fne{\vskip -0.4cm}
\def\kT{k_{\sss\rmB}T}
\def\dsl{\raise.15ex\hbox{$/$}\kern-.57em\hbox{$\partial$}}
\def\nsl{\raise.15ex\hbox{$/$}\kern-.57em\hbox{$\nabla$}}
\def\gtwid{\,{\raise.3ex\hbox{$>$\kern-.75em\lower1ex\hbox{$\sim$}}}\,}
\def\ltwid{\,{\raise.3ex\hbox{$<$\kern-.75em\lower1ex\hbox{$\sim$}}}\,}
\def\undr{\raise.3ex\hbox{$\sim$\kern-.75em\lower1ex\hbox{$|\vec x|\to\infty$}}}
\def\sq{\vbox {\hrule height 0.6pt\hbox{\vrule width 0.6pt\hskip 3pt\vbox{\vskip
 6pt}\hskip 3pt \vrule width 0.6pt}\hrule height 0.6pt}}
\def\half{\frac{1}{2}}
\def\third{\frac{1}{3}}
\def\fourth{\frac{1}{4}}
\def\eighth{\frac{1}{8}}
\def\[{\left [}
\def\]{\right ]}
\def\({\left (}
\def\){\right )}
\def\cbl{\left\{}
\def\cbr{\right\}}


\def\twid#1{{\tilde {#1}}}
\def\xhat{{\hat{\rbfx}}}
\def\yhat{{\hat{\rbfy}}}
\def\zhat{{\hat{\rbfz}}}
\def\khat{{\hat{\rbfk}}}
\def\rhat{{\hat{\rbfr}}}
\def\thhat{{\hat{\bftheta}}}

% calligraphic
\def\cA{{\cal A}}
\def\cB{{\cal B}}
\def\cC{{\cal C}}
\def\cD{{\cal D}}
\def\cE{{\cal E}}
\def\cF{{\cal F}}
\def\cG{{\cal G}}
\def\cH{{\cal H}}
\def\cI{{\cal I}}
\def\cJ{{\cal J}}
\def\cK{{\cal K}}
\def\cL{{\cal L}}
\def\cM{{\cal M}}
\def\cN{{\cal N}}
\def\cO{{\cal O}}
\def\cP{{\cal P}}
\def\cQ{{\cal Q}}
\def\cR{{\cal R}}
\def\cS{{\cal S}}
\def\cT{{\cal T}}
\def\cU{{\cal U}}
\def\cV{{\cal V}}
\def\cW{{\cal W}}
\def\cX{{\cal X}}
\def\cY{{\cal Y}}
\def\cZ{{\cal Z}}

% math boldface
\def\bfA{{\mib A}}
\def\bfB{{\mib B}}
\def\bfC{{\mib C}}
\def\bfD{{\mib D}}
\def\bfE{{\mib E}}
\def\bfF{{\mib F}}
\def\bfG{{\mib G}}
\def\bfH{{\mib H}}
\def\bfI{{\mib I}}
\def\bfJ{{\mib J}}
\def\bfK{{\mib K}}
\def\bfL{{\mib L}}
\def\bfM{{\mib M}}
\def\bfN{{\mib N}}
\def\bfO{{\mib O}}
\def\bfP{{\mib P}}
\def\bfQ{{\mib Q}}
\def\bfR{{\mib R}}
\def\bfS{{\mib S}}
\def\bfT{{\mib T}}
\def\bfU{{\mib U}}
\def\bfV{{\mib V}}
\def\bfW{{\mib W}}
\def\bfX{{\mib X}}
\def\bfY{{\mib Y}}
\def\bfZ{{\mib Z}}
\def\bfa{{\mib a}}
\def\bfb{{\mib b}}
\def\bfc{{\mib c}}
\def\bfd{{\mib d}}
\def\bfe{{\mib e}}
\def\bff{{\mib f}}
\def\bfg{{\mib g}}
\def\bfh{{\mib h}}
\def\bfi{{\mib i}}
\def\bfj{{\mib j}}
\def\bfk{{\mib k}}
\def\bfl{{\mib l}}
\def\bfm{{\mib m}}
\def\bfn{{\mib n}}
\def\bfo{{\mib o}}
\def\bfp{{\mib p}}
\def\bfq{{\mib q}}
\def\bfr{{\mib r}}
\def\bfs{{\mib s}}
\def\bft{{\mib t}}
\def\bfu{{\mib u}}
\def\bfv{{\mib v}}
\def\bfw{{\mib w}}
\def\bfx{{\mib x}}
\def\bfy{{\mib y}}
\def\bfz{{\mib z}}

% roman boldface
\def\rbfA{{\bf A}}
\def\rbfB{{\bf B}}
\def\rbfC{{\bf C}}
\def\rbfD{{\bf D}}
\def\rbfE{{\bf E}}
\def\rbfF{{\bf F}}
\def\rbfG{{\bf G}}
\def\rbfH{{\bf H}}
\def\rbfI{{\bf I}}
\def\rbfJ{{\bf J}}
\def\rbfK{{\bf K}}
\def\rbfL{{\bf L}}
\def\rbfM{{\bf M}}
\def\rbfN{{\bf N}}
\def\rbfO{{\bf O}}
\def\rbfP{{\bf P}}
\def\rbfQ{{\bf Q}}
\def\rbfR{{\bf R}}
\def\rbfS{{\bf S}}
\def\rbfT{{\bf T}}
\def\rbfU{{\bf U}}
\def\rbfV{{\bf V}}
\def\rbfW{{\bf W}}
\def\rbfX{{\bf X}}
\def\rbfY{{\bf Y}}
\def\rbfZ{{\bf Z}}
\def\rbfa{{\bf a}}
\def\rbfb{{\bf b}}
\def\rbfc{{\bf c}}
\def\rbfd{{\bf d}}
\def\rbfe{{\bf e}}
\def\rbff{{\bf f}}
\def\rbfg{{\bf g}}
\def\rbfh{{\bf h}}
\def\rbfi{{\bf i}}
\def\rbfj{{\bf j}}
\def\rbfk{{\bf k}}
\def\rbfl{{\bf l}}
\def\rbfm{{\bf m}}
\def\rbfn{{\bf n}}
\def\rbfo{{\bf o}}
\def\rbfp{{\bf p}}
\def\rbfq{{\bf q}}
\def\rbfr{{\bf r}}
\def\rbfs{{\bf s}}
\def\rbft{{\bf t}}
\def\rbfu{{\bf u}}
\def\rbfv{{\bf v}}
\def\rbfw{{\bf w}}
\def\rbfx{{\bf x}}
\def\rbfy{{\bf y}}
\def\rbfz{{\bf z}}

% greek boldface
\def\bfalpha{{\mib\alpha}}
\def\bfbeta{{\mib\beta}}
\def\bfgamma{{\mib\gamma}}
\def\bfdelta{{\mib\delta}}
\def\eps{\epsilon}
\def\ve{\varepsilon}
\def\bfeps{{\mib\eps}}
\def\bfve{{\mib\ve}}
\def\bfzeta{{\mib\zeta}}
\def\bfeta{{\mib\eta}}
\def\bftheta{{\mib\theta}}
\def\vth{\vartheta}
\def\bfvth{{\mib\vth}}
\def\bfiota{{\mib\iota}}
\def\bfkappa{{\mib\kappa}}
\def\bflambda{{\mib\lambda}}
\def\bfmu{{\mib\mu}}
\def\bfnu{{\mib\nu}}
\def\bfxi{{\mib\xi}}
\def\bfpi{{\mib\pi}}
\def\bfvarpi{{\mib\varpi}}
\def\bfrho{{\mib\rho}}
\def\vrh{\varrho}
\def\bfvrh{{\mib\vrh}}
\def\bfsigma{{\mib\sigma}}
\def\vsig{\varsigma}
\def\bfvsig{{\mib\varsigma}}
\def\bftau{{\mib\tau}}
\def\ups{\upsilon}
\def\bfups{{\mib\ups}}
\def\bfphi{{\mib\phi}}
\def\vphi{\varphi}
\def\bfvphi{{\mib\vphi}}
\def\bfchi{{\mib\chi}}
\def\bfpsi{{\mib\psi}}
\def\bfomega{{\mib\omega}}
\def\bfGamma{{\mib\Gamma}}
\def\bfDelta{{\mib\Delta}}
\def\bfTheta{{\mib\Theta}}
\def\bfLambda{{\mib\Lambda}}
\def\bfXi{{\mib\Xi}}
\def\bfPi{{\mib\Pi}}
\def\bfSigma{{\mib\Sigma}}
\def\Ups{\Upsilon}
\def\bfUps{{\mib\Ups}}
\def\bfPhi{{\mib\Phi}}
\def\bfPsi{{\mib\Psi}}
\def\bfOmega{{\mib\Omega}}

% greek unslanted (for use in conjunction with \physgreek)
\def\rmGamma{{\rm\Gamma}}
\def\rmDelta{{\rm\Delta}}
\def\rmTheta{{\rm\Theta}}
\def\rmLambda{{\rm\Lambda}}
\def\rmXi{{\rm\Xi}}
\def\rmPi{{\rm\Pi}}
\def\rmSigma{{\rm\Sigma}}
\def\rmUps{{\rm\Upsilon}}
\def\rmPhi{{\rm\Phi}}
\def\rmPsi{{\rm\Psi}}
\def\rmOmega{{\rm\Omega}}

% roman
\def\rmA{{\rm A}}
\def\rmB{{\rm B}}
\def\rmC{{\rm C}}
\def\rmD{{\rm D}}
\def\rmE{{\rm E}}
\def\rmF{{\rm F}}
\def\rmG{{\rm G}}
\def\rmH{{\rm H}}
\def\rmI{{\rm I}}
\def\rmJ{{\rm J}}
\def\rmK{{\rm K}}
\def\rmL{{\rm L}}
\def\rmM{{\rm M}}
\def\rmN{{\rm N}}
\def\rmO{{\rm O}}
\def\rmP{{\rm P}}
\def\rmQ{{\rm Q}}
\def\rmR{{\rm R}}
\def\rmS{{\rm S}}
\def\rmT{{\rm T}}
\def\rmU{{\rm U}}
\def\rmV{{\rm V}}
\def\rmW{{\rm W}}
\def\rmX{{\rm X}}
\def\rmY{{\rm Y}}
\def\rmZ{{\rm Z}}
\def\rma{{\rm a}}
\def\rmb{{\rm b}}
\def\rmc{{\rm c}}
\def\rmd{{\rm d}}
\def\rme{{\rm e}}
\def\rmf{{\rm f}}
\def\rmg{{\rm g}}
\def\rmh{{\rm h}}
\def\rmi{{\rm i}}
\def\rmj{{\rm j}}
\def\rmk{{\rm k}}
\def\rml{{\rm l}}
\def\rmm{{\rm m}}
\def\rmn{{\rm n}}
\def\rmo{{\rm o}}
\def\rmp{{\rm p}}
\def\rmq{{\rm q}}
\def\rmr{{\rm r}}
\def\rms{{\rm s}}
\def\rmt{{\rm t}}
\def\rmu{{\rm u}}
\def\rmv{{\rm v}}
\def\rmw{{\rm w}}
\def\rmx{{\rm x}}
\def\rmy{{\rm y}}
\def\rmz{{\rm z}}

\def\dby#1{{d\over d#1}}
\def\pby#1{{\partial\over\partial #1}}
\def\undertext#1{$\underline{\hbox{#1}}$}
\def\Tr{\mathop{\rm Tr}}
\def\ket#1{{\,|\,#1\,\rangle\,}}
\def\bra#1{{\,\langle\,#1\,|\,}}
\def\braket#1#2{{\,\langle\,#1\,|\,#2\,\rangle\,}}
\def\expect#1#2#3{{\,\langle\,#1\,|\,#2\,|\,#3\,\rangle\,}}
\def\ketB#1{{\,|\,#1\,)\,}}
\def\braB#1{{\,(\,#1\,|\,}}
\def\braketB#1#2{{\,(\,#1\,|\,#2\,)\,}}
\def\expectB#1#2#3{{\,(\,#1\,|\,#2\,|\,#3\,)\,}}
\def\bargfo#1#2#3{{\,\lala\,#1\,|\,#2\,|\,#3\,\rara\,}}
\def\pz{{\partial}}
\def\pzb{{\bar{\partial}}}
\def\delby#1{{\delta\over\delta#1}}
\def\zb{{\bar z}}
\def\zbar{\zb}
\def\zdot{{\dot z}}
\def\zbdot{{\dot{\bar z}}}
\def\nd{^{\vphantom{\dagger}}}
\def\yd{^\dagger}
\def\impi{\int_{-\infty}^{\infty}}
\def\izpi{\int_{0}^{\infty}}
\def\izbdt{\int_{0}^{\beta}\!d\tau}
\def\dvec#1#2{{#1_1,\ldots,#1_{\s #2}}}
\def\vph{\vphantom{\sum_i}}
\def\bvph{\vphantom{\sum_N^N}}
\def\ppsc#1#2#3{{\partial #1\over\partial #2}\bigg|_{\scriptstyle #3}}
\def\csch{\,{\rm csch\,}}
\def\ctnh{\,{\rm ctnh\,}}
\def\ctn{\,{\rm ctn\,}}
\def\fc#1#2{{#1\over#2}}
\def\grp#1#2#3{#1_{\ss{#2}},\ldots,#1_{\ss{#3}}}
\def\fdag{{${}^{\dagger}$}}
\def\fddag{{${}^{\ddagger}$}}
\def\and{a^{\phantom\dagger}}
\def\bnd{b^{\phantom\dagger}}
\def\cnd{c^{\phantom\dagger}}
\def\adag{a^\dagger}
\def\bdag{b^\dagger}
\def\cdag{c^\dagger}

%%%%%%%%%%%%%%%%%%%%%Daniel's inexact differential%%%%%%%%%%
%
\def\id{\raise.72ex\hbox{$-$}\kern-.85em\hbox{$d$}\,}
%
%%%%%%%%%%%%%%%%%%%%%%%%%%%%%%%%%%%%%%%%%%%%%%%%%%%%%%%%%%%%
\def\ppsctwo#1#2#3{{\partial^2 #1\over\partial #2^2}\bigg|_{\scriptstyle #3}}
%

%addition copied from IUTEX in order to make aps abstract form work:
%
\def\nextline{\unskip\nobreak\hskip\parfillskip\break}
%From sg@denali.physics.indiana.edu Mon Sep  4 14:36:19 1989
%To: smg@smg
%Subject: my version of reforder.tex
%
%%		REFORDER.TEX			6/7/85	Doug E.
%%					(mods:	3/25/87 R.G.Palmer)
%%
%%	This macro package is intended for use with JNL.
%%	It will automatically order and sort the references in a paper
%%	by order of first citation.(!!)  To use, say \input reforder
%%	after \input jnl (and after all definitions of \refto etc.,
%%	in particular after any use of the \refstyleXX macros),
%%	but before any use of \refto etc.  Use \refto{} (or \ref{} and
%%	\Ref{}) to cite references in the text.  Use \refis{} to supply
%%	the references,  SKIP A LINE after each reference.  Open the
%%	reference listing with \references and close it with \endreferences.
%%
%%	REFORDER depends on the
%%	JNL macros \refto{}, \ref{}, \Ref{} to identify citation of references.
%%	REFORDER also contains a macro \cite{} which can be used to cite
%%	references; e.g., ``Reference \cite{19} blah...'' will produce
%% 	output ``Reference 19 blah''.  Multiple citations can be separated
%%	by commas.  E.g., \refto{24,26,27} and \cite{3,7}
%%	are legal.  Also legal is \refto{3-7}, which expands to mean the same
%%	as \refto{3,4,5,6,7}.  Reference ``numbers'' can in general be any
%%	alphanumeric string; e.g. BjorkenAndDrell is perfectly OK used in
%%	the form\ref{BjorkenAndDrell};  such strings should contain no blanks.
%%
%%	If you have your own pet macros to cite references such as, e.g.,
%%	\def\referpet#1{$^(#1)$)}, you can bring it to the attention of
%%	REFORDER so all \referpet's will be properly cited simply by
%%	declaring ``\citeall\referpet'' once near the beginning, after
%%	\referpet is defined and
%%	after \input reforder.  This has the effect of redefining the macro
%%	as e.g., \def\referpet#1{$^(\cite{#1})$}.  (Such \citeall'ed macros
%%	must have exactly one argument #1, as in \referpet.)  See e.g.,
%%	the end of this file where \refto, \ref and \Ref are \citeall'ed.
%%	
%%	REFORDER depends on the macro \refis{} to supply each reference.
%%      \refis{} can be used to supply a reference anywhere in the paper
%%	after its first citation.  The macro \endreferences actually triggers
%%	sorting and listing of references.  Skip a line after a reference
%%	listing (or, alternatively, end each listing in \par).
%%
%%	Use \ignoreuncited after \input reforder if you wish to ignore
%%	references that are supplied but not cited.  This is particularly
%%	useful if you maintain a master file of references (each supplied
%%	with \refis{}) but only use a subset of these in a given paper.
%%      Include your reference file (with \input) between \references
%%      and \endreferences.
%%
%%	Use \referencefile after \input reforder if you want an ordered
%%      source listing of the references in file <name>.ref
%%
%%	The \reftorange macro can be used to produce a superscript
%%	reference range, like $^{10-15}$.  (The \refto macro always
%%	lists the references one by one, even for e.g. \refto{10-15}).
%%	Use e.g. \reftorange{10}{11-14}{15} -- the references in the
%%	middle group are cited but only 10-15 appears in the text.
%%      Note that \reftorange does NOT check for increasing order.

\catcode`@=11
\newcount\r@fcount \r@fcount=0
\newcount\r@fcurr
\immediate\newwrite\reffile
\newif\ifr@ffile\r@ffilefalse
\def\w@rnwrite#1{\ifr@ffile\immediate\write\reffile{#1}\fi\message{#1}}

\def\writer@f#1>>{}
\def\referencefile{%			  Stuff to write .REF file
  \r@ffiletrue\immediate\openout\reffile=\jobname.ref%
  \def\writer@f##1>>{\ifr@ffile\immediate\write\reffile%
    {\noexpand\refis{##1} = \csname r@fnum##1\endcsname = %
     \expandafter\expandafter\expandafter\strip@t\expandafter%
     \meaning\csname r@ftext\csname r@fnum##1\endcsname\endcsname}\fi}%
  \def\strip@t##1>>{}}
\let\referencelist=\referencefile

\def\citeall#1{\xdef#1##1{#1{\noexpand\cite{##1}}}}
\def\cite#1{\each@rg\citer@nge{#1}}	% Variable No. of args, separated by ","

\def\each@rg#1#2{{\let\thecsname=#1\expandafter\first@rg#2,\end,}}
\def\first@rg#1,{\thecsname{#1}\apply@rg}	% each@ag is a general purpose
\def\apply@rg#1,{\ifx\end#1\let\next=\relax%	  variable no. of arg. macro.
\else,\thecsname{#1}\let\next=\apply@rg\fi\next}% args separated by commas

\def\citer@nge#1{\citedor@nge#1-\end-}	% Check for M-N range (M and N numbers)
\def\citer@ngeat#1\end-{#1}
\def\citedor@nge#1-#2-{\ifx\end#2\r@featspace#1 % Single argument
  \else\citel@@p{#1}{#2}\citer@ngeat\fi}	% M-N range of arguments
\def\citel@@p#1#2{\ifnum#1>#2{\errmessage{Reference range #1-#2\space is bad.}%
    \errhelp{If you cite a series of references by the notation M-N, then M and
    N must be integers, and N must be greater than or equal to M.}}\else%
 {\count0=#1\count1=#2\advance\count1 by1\relax\expandafter\r@fcite\the\count0,%
  \loop\advance\count0 by1\relax%	  Loop from M to N
    \ifnum\count0<\count1,\expandafter\r@fcite\the\count0,%
  \repeat}\fi}

\def\r@featspace#1#2 {\r@fcite#1#2,}	% Eat spaces at beginning or end of arg
\def\r@fcite#1,{\ifuncit@d{#1}%		  Cite individual reference
    \newr@f{#1}%
    \expandafter\gdef\csname r@ftext\number\r@fcount\endcsname%
                     {\message{Reference #1 to be supplied.}%
                      \writer@f#1>>#1 to be supplied.\par}%
 \fi%
 \csname r@fnum#1\endcsname}
\def\ifuncit@d#1{\expandafter\ifx\csname r@fnum#1\endcsname\relax}%
\def\newr@f#1{\global\advance\r@fcount by1%
    \expandafter\xdef\csname r@fnum#1\endcsname{\number\r@fcount}}

\let\r@fis=\refis			% Save old \refis, redefine
\def\refis#1#2#3\par{\ifuncit@d{#1}%      Use two params #2 #3 to strip blank
   \newr@f{#1}%
   \w@rnwrite{Reference #1=\number\r@fcount\space is not cited up to now.}\fi%
  \expandafter\gdef\csname r@ftext\csname r@fnum#1\endcsname\endcsname%
  {\writer@f#1>>#2#3\par}}

\def\ignoreuncited{%   redefine \refis if ignoring uncited references
   \def\refis##1##2##3\par{\ifuncit@d{##1}%
     \else\expandafter\gdef\csname r@ftext\csname r@fnum##1\endcsname\endcsname%
     {\writer@f##1>>##2##3\par}\fi}}

\def\r@ferr{\endreferences\errmessage{I was expecting to see
\noexpand\endreferences before now;  I have inserted it here.}}
\let\r@ferences=\references
\def\references{\r@ferences\def\endmode{\r@ferr\par\endgroup}}

\let\endr@ferences=\endreferences
\def\endreferences{\r@fcurr=0%		  Save old \endreferences, redefine
  {\loop\ifnum\r@fcurr<\r@fcount%	  Loop over refnum and produce text
    \advance\r@fcurr by 1\relax\expandafter\r@fis\expandafter{\number\r@fcurr}%
    \csname r@ftext\number\r@fcurr\endcsname%
  \repeat}\gdef\r@ferr{}\endr@ferences}

% Save old \endpaper, redefine it to write parting message.

\let\r@fend=\endpaper\gdef\endpaper{\ifr@ffile
\immediate\write16{Cross References written on []\jobname.REF.}\fi\r@fend}

\catcode`@=12

\def\reftorange#1#2#3{$^{\cite{#1}-\setbox0=\hbox{\cite{#2}}\cite{#3}}$}

\citeall\refto		% These macros will generate citations
\citeall\ref		%
\citeall\Ref		%


%   see SMG's notes at the end
%
%
%
%
%%              EQNORDER.TEX                    11/05/85        Doug E.
%%
%%      This macro package is intended for use with JNL.
%%      It will automatically order and sort the equations in a paper
%%      by order of appearance.  To use, say \input eqnorder
%%      after \input jnl (and after all definitions of \eqno etc.,
%%      but before any use of \eqno etc.  Use \() to cite equations
%%      in the text.  Use \eqno() or \tag to put the numbers on displayed
%%      equations; or use &() with \eqalignno{} as explained in the TeXBOOK.
%%
%%      EQNORDER depends on the macro \() to refer to equations in the
%%      text; use it as Equation \() or Eq. \() or Eqs. \(), etc.
%%      EQNORDER also contains a macro \call{} which can be used to refer
%%      equations; e.g., ``Equation \call{19} blah...'' will produce
%%      output ``Equation 19 blah''.  Multiple citations must be separated
%%      by commas.  E.g., \(24,26,27) and \call{3,7} are legal.  A sequence
%%      of equation numbers can be referred to by, e.g., \(3-7) which means
%%      the same as (3,4,5,6,7).
%%
%%      Equation ``numbers'' can actually be any alphanumeric string;
%%      e.g., equation \tag Schroedinger $$ can be referred to by
%%      \(Schroedinger).  In fact, if you expect to renumber the equations,
%%      it is actually easier and less confusing to tag them with names
%%      rather than numbers.
%%
%%      There is one big rule:  You cannot refer to an equation before you
%%      display it.  There is a limited loophole:  You can refer to the
%%      first, second or third equation number just below where you are
%%      as \(+1), \(+2), or \(+3).  In the same way, the equation number
%%      just above can be called \(0), and the three preceding numbers
%%      \(-1), \(-2), \(-3).
%%
%%      TeX keeps the equation number as a count \tagnumber.  \tagnumber
%%      is initially 0, and it is incremented by 1 just BEFORE it is used
%%      to tag a displayed equation.  You are free to reset \tagnumber,
%%      which you can do just by writing e.g. \tagnumber=23.
%%
%%      If you label a displayed equation with a null number, \tag $$
%%      or \eqno() or &(), an incremented \tagnumber will be generated for
%%      the equation, but the only way to refer to that equation is
%%      via the \(+n) or \(0) or \(-n) notation.
%%
%%      In long papers, the author often numbers the equations anew
%%      in each section in the style \tag 6.1 $$, \tag 6.2 $$, and so
%%      forth;  the equations are then referred to by \(6.1) etc.
%%      One reason for doing this is to minimize chaos when equations
%%      have to be renumbered --- but this is what EQNORDER already does!
%%      If you still want to use such a style, just declare \taghead{6.}
%%      for example at the beginning of Section 6.  The effect of \taghead
%%      is to reset \tagnumber to 0, and to save the argument of \taghead
%%      to so that it can be put in front of each equation number in the
%%      output.
%%
%%      Sometimes a sequence of displayed equations is labelled with
%%      the same number (e.g., 25) and then sublabeled a,b,c,d...  Use
%%      the form \tag 25 a$$, \tag 25 b$$,... or \eqno(25 a)$$,
%%      \eqno(25 b)$$,... to put the numbers on such a sequence;  note
%%      the space.  Such equations can be referred to in the text either
%%      as \(25) or \(25 a).  Also such constructions as \(25 a,25 b,26)
%%      are legal.  Again, note carefully the position of the space.  The
%%      effect of the space is to mark the end of the equation number that
%%      TeX keeps track of;  the following string (a or b or ...) is just
%%      put out without modification.  Thus constructions like \tag 25 ' $$
%%      and \tag 25 '' $$ are legal.
%%
%%      If you have your own pet macros to call equations such as, e.g.,
%%      \def\eqnpet#1{($#1$))}, you can bring it to the attention of
%%      REFORDER so all \eqnpet's will be properly calld simply by
%%      declaring ``\callall\eqnpet'' after \eqnpet is defined and
%%      after \input eqnorder.  This has the effect of redefining the macro
%%      as e.g., \def\eqnpet#1{($\call{#1}$)}.  (Such \callall'ed macros
%%      must have exactly one argument #1, as in \eqnpet.)
 %%%%%%%%%%%%%%%%%%%%%%%%%%%%%%%%%%%%%%%%%%%%%%%%%%%%%%%%%%%%%%%%%%%%%%%%%%%%

%12/8/88   SMG added a bunch of callall statements to the end of this file:
%
%%%%%%%%%%%%%%%%%%%%%%%%%%%%%%%%%%%%%%%%%%%%%%%%%%%%%%%%%%%%%%%%%%%%%%%%%%%%
%%
\catcode`@=11
\newcount\tagnumber\tagnumber=0

\immediate\newwrite\eqnfile
\newif\if@qnfile\@qnfilefalse
\def\write@qn#1{}
\def\writenew@qn#1{}
\def\w@rnwrite#1{\write@qn{#1}\message{#1}}
\def\@rrwrite#1{\write@qn{#1}\errmessage{#1}}

\def\taghead#1{\gdef\t@ghead{#1}\global\tagnumber=0}
\def\t@ghead{}

\expandafter\def\csname @qnnum-3\endcsname
  {{\t@ghead\advance\tagnumber by -3\relax\number\tagnumber}}
\expandafter\def\csname @qnnum-2\endcsname
  {{\t@ghead\advance\tagnumber by -2\relax\number\tagnumber}}
\expandafter\def\csname @qnnum-1\endcsname
  {{\t@ghead\advance\tagnumber by -1\relax\number\tagnumber}}
\expandafter\def\csname @qnnum0\endcsname
  {\t@ghead\number\tagnumber}
\expandafter\def\csname @qnnum+1\endcsname
  {{\t@ghead\advance\tagnumber by 1\relax\number\tagnumber}}
\expandafter\def\csname @qnnum+2\endcsname
  {{\t@ghead\advance\tagnumber by 2\relax\number\tagnumber}}
\expandafter\def\csname @qnnum+3\endcsname
  {{\t@ghead\advance\tagnumber by 3\relax\number\tagnumber}}

\def\equationfile{%
  \@qnfiletrue\immediate\openout\eqnfile=\jobname.eqn%
  \def\write@qn##1{\if@qnfile\immediate\write\eqnfile{##1}\fi}
  \def\writenew@qn##1{\if@qnfile\immediate\write\eqnfile
    {\noexpand\tag{##1} = (\t@ghead\number\tagnumber)}\fi}
}

\def\callall#1{\xdef#1##1{#1{\noexpand\call{##1}}}}
\def\call#1{\each@rg\callr@nge{#1}}

\def\each@rg#1#2{{\let\thecsname=#1\expandafter\first@rg#2,\end,}}
\def\first@rg#1,{\thecsname{#1}\apply@rg}
\def\apply@rg#1,{\ifx\end#1\let\next=\relax%
\else,\thecsname{#1}\let\next=\apply@rg\fi\next}

\def\callr@nge#1{\calldor@nge#1-\end-}
\def\callr@ngeat#1\end-{#1}
\def\calldor@nge#1-#2-{\ifx\end#2\@qneatspace#1 %
  \else\calll@@p{#1}{#2}\callr@ngeat\fi}
\def\calll@@p#1#2{\ifnum#1>#2{\@rrwrite{Equation range #1-#2\space is bad.}
\errhelp{If you call a series of equations by the notation M-N, then M and
N must be integers, and N must be greater than or equal to M.}}\else%
 {\count0=#1\count1=#2\advance\count1
 by1\relax\expandafter\@qncall\the\count0,%
  \loop\advance\count0 by1\relax%
    \ifnum\count0<\count1,\expandafter\@qncall\the\count0,%
  \repeat}\fi}

\def\@qneatspace#1#2 {\@qncall#1#2,}
\def\@qncall#1,{\ifunc@lled{#1}{\def\next{#1}\ifx\next\empty\else
  \w@rnwrite{Equation number \noexpand\(>>#1<<) has not been defined yet.}
  >>#1<<\fi}\else\csname @qnnum#1\endcsname\fi}

\let\eqnono=\eqno
\def\eqno(#1){\tag#1}
\def\tag#1$${\eqnono(\displayt@g#1 )$$}

\def\aligntag#1\endaligntag
  $${\gdef\tag##1\\{&(##1 )\cr}\eqalignno{#1\\}$$
  \gdef\tag##1$${\eqnono(\displayt@g##1 )$$}}

\let\eqalignnono=\eqalignno

\def\eqalignno#1{\displ@y \tabskip\centering
  \halign to\displaywidth{\hfil$\displaystyle{##}$\tabskip\z@skip
    &$\displaystyle{{}##}$\hfil\tabskip\centering
    &\llap{$\displayt@gpar##$}\tabskip\z@skip\crcr
    #1\crcr}}

\def\displayt@gpar(#1){(\displayt@g#1 )}

\def\displayt@g#1 {\rm\ifunc@lled{#1}\global\advance\tagnumber by1
        {\def\next{#1}\ifx\next\empty\else\expandafter
        \xdef\csname @qnnum#1\endcsname{\t@ghead\number\tagnumber}\fi}%
  \writenew@qn{#1}\t@ghead\number\tagnumber\else
        {\edef\next{\t@ghead\number\tagnumber}%
        \expandafter\ifx\csname @qnnum#1\endcsname\next\else
        \w@rnwrite{Equation \noexpand\tag{#1} is a duplicate number.}\fi}%
  \csname @qnnum#1\endcsname\fi}

\def\ifunc@lled#1{\expandafter\ifx\csname @qnnum#1\endcsname\relax}

\let\@qnend=\end\gdef\end{\if@qnfile
\immediate\write16{Equation numbers written on []\jobname.EQN.}\fi\@qnend}

\catcode`@=12
%%%%%%%%%%%%%%%%%%%%%%%%%%%%%%%%%%%%%%%%%%%%%%%%%%%%%%%%%%%%%%%%%%%%%%%%%%%
%  added by smg 12/8/88 per Mats suggestion
%  what you do is "as shown in \eq{fred}, the answer is..."
%%%%%%%%%%%%%%%order the equations: %%%%%%%%%%%%%%%%%%%%%%%%%%%%
\callall\Equation
\callall\Equations
\callall\Eq
\callall\eq
\callall\Eqs
\callall\eqs
%%%%%%%%%%%%%%%%%%%%%%%%%%%%%%%%%%%%%%%%%%%%%%%%%%%%%%%%%%%%%%%



%% DEBUG
%%\def\see#1 {\expandafter\show\csname#1\endcsname}
%Hope you enjoy it.
%Ciao.
%-Daniel
%%%%%%%%%%%%%%%%%%%%%%%%%%%%%%%%%%%%%%%%%%
