% This should be processed using LaTeX2e:
%
\documentclass[12pt]{book}
\usepackage{plenum}
\begin{document}
\chapter{PREPARING YOUR CAMERA-READY ARTICLE FOR OFFSET
REPRODUCTION IN \TeX}

\author{John Q. Editor,\refnote{1} Jane Doe,\refnote{2} and Richard
Smith\refnote{1}}

\affiliation{\affnote{1}Editing Department\\
\affnote{2}Production Department\\
Plenum Press\\
New York, NY, 10013}

\section{INTRODUCTION}

The purpose of this leaflet is twofold: (1) to furnish an example
showing just how the various elements of a technical paper should
be presented, and (2) to provide a brief summary of the rules
governing the preparation of camera-ready copy. It is of course not
possible to address all the problems that may arise in these few
pages. If you run into a problem that you cannot solve, call us at
(212) 620-8456, Fax (212) 463-0742.

We must distinguish between mechanical requirements for the
preparation of copy suitable for direct offset reproduction and
editorial guidelines concerning such matters as style, content, and
scope. Editorial guidelines are the responsibility of the volume
editor, and questions about such matters should be addressed to
him. Mechanical requirements are the province of the publisher and
constitute the principal subject of these guidelines.

\atable{Example of a Table.\refnote a The Heading Should Be in the
Text
Font\refnote b}
{\begin{tabular}{cccc}
\hline
&&\multicolumn{2}{c}{Straddle rule} \\ \cline{3-4}
Column 1 & Column 2 & Column 1 & Column 2 \\ \hline
Example \#1 & Example \#1$^\prime$ & Example \#4 & Example
\#4$^\prime$ \\
Example \#2 & Example \#2$^\prime$ & Example \#5 & Example
\#5$^\prime$ \\
Example \#3 & Example \#3$^\prime$ & Example \#6 & Example
\#6$^\prime$ \\ \hline
\end{tabular}}
{\tablenote{a}A top rule starts the table and a bottom rule ends
the table. A rule follows the column heads. If needed, straddle
rules establish hierarchies among headings.\\
\tablenote{b}The footnotes should be lettered instead of numbered.
Please use the {\tt \char92tablenote} macro for the letters.}

\section{PREPARING THE CAMERA-READY ARTICLE}

We prefer that the article be submitted in camera-ready form as
hard copy printed by a laser printer on A4 or legal-size paper as
well as files preferably in the form of {\AmSLaTeX} but \LaTeX,
\LaTeX2$\epsilon$, \AmSTeX, or plain {\TeX} will be acceptable. We
prefer equations with the equation numbers on the right-hand side.

The trim size of the printed volume will be $6\frac12 \times
9\frac78$ inches (ca.\ $165 \times 251$ mm), with a final type
area --- the area encompassing the text, figures, tables, etc. ---
of
$30\frac12 \times 61$ picas (ca.\ $5\frac1{16} \times 10\frac18$
inches or $128 \times 258$ mm). In order to improve the type
quality in the printed volume, the camera copy you will be
preparing will be designed to be reduced to $82\frac12\%$ of
its initial size during the printing process. Thus, the type area
of your camera-ready article should be $37 \times 58$ picas
(ca.\ $6\frac18 \times 9\frac58$ inches or $156 \times 245$ mm).

\begin{figure}[tbp]
%\centerfig{6.24in}{5.98in}{tex-fig}{bmp} %across,down,name,extension
\vspace*{5.98in}
\caption{{\TeX} articles must be prepared according to the
guidelines given in these pages. Poor artwork will be returned to
you for replacement and the entire article will be returned if it
exhibits gross violations of these guidelines. Please also note
that figure captions should be footnotesize and have a baselineskip
of 12pt.}
\end{figure}

\section{TEXT HEADINGS}

In most cases, only two values of subheadings should be employed.
The paragraphs below demonstrate the use of such subheadings. If a
heading is immediately followed by another one with no text in
between, you should include the line \verb+\vskip -1pc+ between
them.
Please note that a section heading should be on the same page as
the first
line of text. A section heading can be forced onto a new page by
inserting
the \verb+\pagebreak+ command before it.

\section{FIRST-VALUE SUBHEADINGS}

These headings should be printed in capital letters and preceded by
the \verb$\section$ command we provide.

\subsection{Second-Value Headings}

These headings should be printed in upper- and lower-case type,
with the initial letter of each major word capitalized, and
preceded by the \verb$\subsection$ command.

\subsubsection{Third-Value Subheadings.} Third-value subheadings,
if unavoidable, should follow the second-value subheadings in type
style but should be run into the text and preceded by the
\verb$\subsubsection$ command.

\section{REFERENCES}

The list of works cited should appear at the end of the article.
Establishing rules for styling the references is the province of
the editor, so questions concerning reference styling should be
addressed to the editor. If the editor has not chosen to establish
style guidelines for the references, please follow one of the
examples below:

\subsection{References by Name and Year}

If reference citation is by name and year, the text citation may
take one of the following forms: ``...as shown by Miller (1967),
the...'' or ``... has often been demonstrated (Smith and Jones,
1972; Brown et al., 1974) that....'' In this case, the reference
list at the end of the paper must be in alphabetical order by the
first authors' names and presented in the following style (please
note that the references should begin with \verb${\referencestyle$
and end with \verb$}$):

\begin{thebibliography}
\bibitem Brown, C. D., Green, M. P., and Robinson, S. A., 1974,
Article title with only the first word having an initial capital,
{\it
Journal Name Abbr.} 37:468.

\bibitem Miller, R. J., 1967, ``Book Title with the Initial Letter
of Each Major Word Capitalized,'' Publisher, City.

\bibitem Smith, A. B., and Jones, C. D., 1972, Article or chapter
title, {\it in}: ``Book Title,'' W. F. White, ed., Publisher, City.
\end{thebibliography}

\smallskip

The above references were produced by the following {\TeX}
commands:
\begin{verbatim}
\begin{thebibliography}
\bibitem Brown, C. D., Green, M. P., and Robinson, S. A., 1974,
Article title with only the first word having an initial capital,
{\it
Journal Name Abbr.} 37:468.

\bibitem Miller, R. J., 1967, ``Book Title with the Initial Letter
of Each Major Word Capitalized,'' Publisher, City.

\bibitem Smith, A. B., and Jones, C. D., 1972, Article or chapter
title, {\it in}: ``Book Title,'' W. F. White, ed., Publisher, City.
\end{thebibliography}
\end{verbatim}

\subsection{Numbered References}

If reference citation is by number, the text citation may take one
of the following forms: ``...as shown by Miller,\refnote{1}
the...,'' ``... has often been demonstrated\refnote{2,3} that....''
The superscript\refnote{1} used for the text citation is produced
by \verb+\refnote{1}+ or --- if you are using labels ---
\verb+\refnote{\cite{label}}+. In this case, the references are to
be numbered in the order in which they are cited in the text and
presented in the following style:

\begin{numbibliography}
\bibitem{label}R. J. Miller. ``Book Title with the Initial Letter
of
Each Major Word Capitalized,'' Publisher, City (1967).

\bibitem{anotherlabel}A. B. Smith and C. D. Jones, Article or
chapter
title with only the first word having an initial capital, {\it in}:
``Book Title,'' W. F. White, ed., Publisher, City (1972).

% Labels are not always needed
\bibitem{}C. D. Brown, M. P. Green, and S. A. Robinson, Article
title, {\it Journal Name Abbr.} 37:468 (1974).
\end{numbibliography}

\smallskip

The above references were produced by the following {\TeX}
commands:
\begin{verbatim}
\begin{numbibliography}
\bibitem{label}R. J. Miller. ``Book Title with the Initial Letter
of
Each Major Word Capitalized,'' Publisher, City (1967).

\bibitem{anotherlabel}A. B. Smith and C. D. Jones, Article or
chapter
title with only the first word having an initial capital, {\it in}:
``Book Title,'' W. F. White, ed., Publisher, City (1972).

% Labels are not always needed
\bibitem{}C. D. Brown, M. P. Green, and S. A. Robinson, Article
title, {\it Journal Name Abbr.} 37:468 (1974).
\end{numbibliography}
\end{verbatim}

\section{MISCELLANEOUS}

\vskip -1pc

\subsection{The Opening Page}

The opening page should follow the example shown in these
guidelines. The file should start with the lines:

\smallskip

If you are using \LaTeX 2.09:

\begin{verbatim}
\documentstyle[12pt]{book}
\input times.tex %This is optional and requires NFSS.
\input plenum.tex
\input update.tex
\begin{document}
\end{verbatim}

\smallskip

If you are using \AmSLaTeX:

\begin{verbatim}
\documentstyle[12pt,amstex,righttag]{book}
\input times.tex %This is optional
\input plenum.tex
\input update.tex
\begin{document}
\end{verbatim}

\smallskip

If you are using \LaTeX 2$\epsilon$:

\begin{verbatim}
\documentstyle[12pt]{book}
\input times2e.tex %This is optional
\input plenum.tex
\begin{document}
\end{verbatim}

\smallskip

The title should be printed in capital letters and preceded by
\verb$\chapter{$ or \verb$\title{$ and followed by \verb$}$. The
author names and addresses should be printed in upper- and
lower-case type and preceded by \verb$\author$. If there are
different affiliations for different authors, please use the
\verb+\refnote+ macro for the superscripts following the authors'
names. Please use the \verb+\affnote+ macro for the superscripts
preceding the affiliations. The text or first subheading should
start below the addresses, not on the following page.

\subsection{Equations}

The default {\TeX} format for most equations is acceptable. Please
note that the equations should be narrower than $30 \frac12$ picas.

\subsection{Footnotes and Endnotes}

Notes may appear at the bottom of the pertinent pages or gathered
together in a notes section immediately preceding the reference
section.\footnote{Notes should not be included with the literature
references. Footnotes should be footnotesize, but endnotes should
be set in ordinary type.}

\subsection{Acknowledgments}

Acknowledgments, if required, should appear in a section
immediately before the reference section, or the endnote section,
if there is one.

\subsection{Special Symbols}

Special symbols not yet available in your version of {\TeX} may be
hand-drawn using a fine black pen. Symbols drawn with pens of other
colors, with pencils, or with felt tip pens will not reproduce
well. Symbols should be drawn roughly the same size as the printed
characters surrounding them. To allow for mistakes in drawing
symbols and to allow the space needed for hand-drawn symbols to be
judged accurately, you should draw symbols on a separate sheet of
paper and pasted into the text. As an alternative, such symbols may
also be submitted as encapsulated PostScript.

\atable
{Suitability of artwork for reproduction\refnote a}
{\begin{tabular}{lp{2.6in}p{2.6in}}\hline
& \multicolumn{1}{c}{Line drawings} &
\multicolumn{1}{c}{Photographs to be reproduced as halftones}\\
\hline
Good &
{\begin{list}{$\bullet$}{\topsep 0in\parsep 0in\itemsep
0in\partopsep 0in}\raggedright
\item Original ink drawings
\item Crisp laser prints
%\item Bitmaps that produced the laser prints
\item Photographic prints made from clear originals
\item Pages taken from publications
\end{list}}
&
{\begin{list}{$\bullet$}{\topsep 0in\parsep 0in\itemsep
0in\partopsep 0in}\raggedright
\item Continuous-tone photographic prints
\item Continuous-tone negatives
\end{list}}
\\
Bad &
{\begin{list}{$\bullet$}{\topsep 0in\parsep 0in\itemsep
0in\partopsep 0in}\raggedright
\item Xerographic copies
\item Anything covered with transparent tape
\item Drawings containing blurred or faded lines, distorted
symbols, etc.
\item Drawings containing faint or damaged shading
\end{list}}
&
{\begin{list}{$\bullet$}{\topsep 0in\parsep 0in\itemsep
0in\partopsep 0in}\raggedright
\item Xerographic copies
\item Screened prints
\item Pages taken from publications (these are screened)
\end{list}}
\\ \hline
\end{tabular}}
{\tablenote{a}Providing ``good'' artwork will help to speed the
publication of the book.}

\section{FIGURES}

To introduce uniformity of appearance to the volume, Plenum Press
will scale all illustrations to be included in the book, produce
sized prints, and mount the sized prints in the camera copy.
Therefore, the original artwork or full-size photographic prints
made from the original artwork should be supplied rather than
reduced-size prints and should be supplied separately rather than
being mounted in the article. Illustrations should be identified by
author and figure number and the orientation should be indicated if
it is not obvious. The illustrations will be scaled so that the
capital letters of the bulk of the labels will measure about
$\frac3{32}$
inches (2.5 mm) in height. Please take this scaling into account
when calculating how much space to leave for each figure in the
text.

Figures should be numbered consecutively throughout the article
(1, 2, 3, ...). (If you are using our style in the {\LaTeX}
\verb+{figure}+ environment, that will be done automatically by the
\verb+\caption+ command. It will also adjust the font size and line
spacing.) In the text, figures should be identified by number
rather than by ``above,'' ``below,'' etc.

\subsection{Figure Captions}

A brief explanatory caption should appear below each figure. The
caption should be printed in footnotesize type. Captions of figures
that will be three-quarters page width or wider after scaling
should be printed page width, and captions of narrower figures
should be printed three-quarters page width. (The width of a
caption can be adjusted by putting the {\LaTeX} \verb+{minipage}+
or \verb+{parbox}+ environments inside the \verb+{figure}+
environment.)

In calculating how much space to leave for a figure, add in the
one-line space that will divide the figure and caption and the
two-line space that will separate the top of the figure from
whatever is above it. Of course, if the figure will be at the top
of a page, the two-line space should be omitted. Do not run text
next to a figure. Large figures may be placed sideways (landscape
orientation) on a page. All elements on a page containing a
landscape-oriented material must be oriented in the same manner,
and no text should be placed on such a page.

\subsection{Computer-Generated Illustrations}

Computer-generated illustrations should have clean, crisp lines,
not jagged lines (i.e., should be printed at a high resolution).
Half-tone screening should be coarse (less than 80 lines per inch),
since fine screens are hard to reproduce. All computer-generated
art should be output on a laser printer, not a dot-matrix printer.

\subsection{Line Drawings}

Line drawings supplied should be original ink drawings,
high-quality laser prints produced using appropriate software, the
PostScript files that produced such laser prints, or full-size
photographic prints made from such artwork (see Table 2). If
possible, drawings and laser prints should be prepared to be
reduced by about 50\% during the scaling process. Note that in
labeling figures it is important to avoid making subscripts and
superscripts too much smaller than the main characters; if the
subscripts and superscripts are too small in comparison with the
main type, they will not be readable when the figures are reduced.
If the original art cannot be obtained when reproducing a
previously-published figure, the appropriate page from the
publication involved or a photographic print made from the page may
be supplied instead. Please do not supply xerographic copies.

\subsection{Continuous-Tone Photographs}

Photographs should be supplied either in the form of good-quality
continuous-tone photographic prints made from the original
negatives or in the form of the negatives themselves.
Magnifications should be indicated by scale bars included in the
figures rather than stated in the captions.

Screened prints and xerographic copies will not reproduce well and
are not acceptable.

\subsection{Color Figures}

It is technically feasible to reproduce figures in color, but it is
very expensive. If a figure is to be reproduced in color, special
arrangements must be made to cover the cost of color reproduction
and specific instructions must be included indicating that color
reproduction will be subsidized. Figures originally in color but to
be reproduced in black and white in the book should be supplied in
the form of black and white continuous-tone photographs made from
the color originals. Such prints can be prepared by any
photographic department or photography shop. Providing the figures
in this form will give you the opportunity to see how well the
figures will reproduce in black and white, while obviating any
misunderstanding regarding how the figures are to be reproduced. If
you have a question about color reproduction, please get in touch
with us at the earliest possible time.

\section{TABLES}

Table placement, spacing, and numbering should follow the
guidelines given above for figures.

A table-width table title (caption), printed in type of the same
size and style as the text type, should be placed above the table.
(The width of a caption can be adjusted by using our \verb+\atable+
macro. The first argument of \verb+\atable+ is the caption text,
the second argument is the table itself (including
\verb+\begin{tabular}+ and \verb+\end{tabular}+), and the third
consists of the footnotes.

Large tables may be placed sideways on a page. (Such a table should
be typeset separately.)

Keeping the arrangement of tables simple will help to make the
tables easy to read. In most cases, table-width rules at the top
and bottom of the table body and below the column headings, if
there are any, are all the lines necessary. Short straddle rules
may be used to establish a hierarchy among column headings. Tables
that are basically lists should have the same rules as other
tables, viz., top and bottom rules, and rules below the column
headings, if any.

It is a good practice to set tables (not, however, the titles) in
footnotesize type smaller than the text type. This sets the tables
off from the text and helps to give a more uniform look to sets of
tables that include large tables that would have to be set in
smaller type anyway in order to fit on the page.
\end{document}
